\documentclass{article}

\usepackage{styles/arxiv}
\usepackage{styles/quiver}

\usepackage[utf8]{inputenc} % allow utf-8 input
\usepackage[T1]{fontenc}    % use 8-bit T1 fonts
\usepackage{hyperref}       % hyperlinks
\usepackage{url}            % simple URL typesetting
\usepackage{booktabs}       % professional-quality tables
\usepackage[english]{babel}
\usepackage{nicefrac}       % compact symbols for 1/2, etc.
\usepackage{microtype}      % microtypography
\usepackage{graphicx}
\usepackage{stmaryrd}

\usepackage{amsthm}
\usepackage{amssymb}
\usepackage{amsfonts}
\usepackage{amsmath}
\usepackage{mathtools}

\usepackage{tikz}
\usetikzlibrary{angles,fit,arrows,calc,math,intersections,through,backgrounds}
\usepackage{qtree}

\usepackage{listings}
\lstset{
  basicstyle=\itshape,
  xleftmargin=3em,
  literate={->}{$\rightarrow$}{2}
           {α}{$\alpha$}{1}
           {δ}{$\delta$}{1}
}

\usepackage{xstring}
\usepackage{stmaryrd}
\usepackage{wasysym}
\usepackage{textcomp}
\usepackage{blindtext}
\usepackage{subfiles}

\newtheorem{definition}{Definition}
\numberwithin{definition}{section}
\newtheorem{lemma}{Lemma}
\numberwithin{lemma}{section}
\newtheorem{proposition}{Proposition}
\numberwithin{proposition}{section}
\newtheorem{corollary}{Corollary}
\numberwithin{corollary}{section}
\newtheorem{theorem}{Theorem}
\numberwithin{theorem}{section}

\DeclareMathSymbol{\mathinvertedexclamationmark}{\mathclose}{operators}{'074}
\DeclareMathSymbol{\mathexclamationmark}{\mathclose}{operators}{'041}
\makeatletter
\newcommand{\raisedmathinvertedexclamationmark}{%
  \mathclose{\mathpalette\raised@mathinvertedexclamationmark\relax}%
}
\newcommand{\raised@mathinvertedexclamationmark}[2]{%
  \raisebox{\depth}{$\m@th#1\mathinvertedexclamationmark$}%
}
\begingroup\lccode`~=`! \lowercase{\endgroup
  \def~}{\@ifnextchar`{\raisedmathinvertedexclamationmark\@gobble}{\mathexclamationmark}}
\mathcode`!="8000
\makeatother

\newcommand{\intga}{\mathclap{\smash{\oplus}}{\int}}
\newcommand{\intgp}{\mathclap{\smash{\otimes}}{\int}}
\newcommand{\intgg}{\mathclap{\rightsquigarrow}\mathclap{\int}}

\DeclareMathOperator{\arcsinh}{arcsinh}

\title{Geometry of Arithmetic Expressions: I.\\ Basic Concepts and Unsolved Problems (Draft)}

%\date{Decmber 8, 2022}	% Here you can change the date presented in the paper title
%\date{} 				% Or removing it

\author{
  Mingli~Yuan \\
  AI Lab \\
  ColorfulClouds Tech.\\
  Beijing, 100083 \\
  \texttt{mingli.yuan@gmail.com}
}

% Uncomment to remove the date
%\date{}

% Uncomment to override  the `A preprint' in the header
\renewcommand{\headeright}{A preprint}
\renewcommand{\undertitle}{A preprint}

\begin{document}
\maketitle

\begin{abstract}
    TODO
\end{abstract}

\keywords{arithmetic expressions, hyperbolic geometry}

\setcounter{tocdepth}{2}
\tableofcontents
\newpage

\section{Introduction}\label{sec:introduction}

From the syntactic trees of arithmetic expressions to the hyperbolic geometry of modular forms, the interplay between discrete algebra and continuous geometry has long fascinated mathematicians. This work addresses a fundamental question: Can the evaluation dynamics of arithmetic expressions themselves form a geometric space? We demonstrate that threadlike arithmetic expressions naturally embed into hyperbolic surfaces through a novel flow equation.

\subsection{Key Definitions}

We formalize arithmetic expressions $a \in \mathbb{E}[\mathbb{Q}]$ using production rules that generate terms through addition, subtraction, multiplication, and division operations. The evaluation $\nu(a)$ of these expressions can be viewed from multiple perspectives:

\begin{itemize}
    \item \textit{Syntactically} as tree structures with branch nodes (operators) and leaf nodes (constants)
    \item \textit{Algebraically} as compositions of elementary operations
    \item \textit{Geometrically} as paths through a continuous space
\end{itemize}

Of particular importance are threadlike expressions, where all left nodes are leaf nodes. These expressions, analogous to paths in homotopy theory, provide a natural bridge between algebraic and geometric perspectives.

Through currying and path notation, we establish a formal framework for representing threadlike expressions as sequences of elementary operations:
\begin{equation}
x a_1 a_2 \cdots a_n \coloneqq a_n(a_{n-1}(\cdots a_2(a_1(x))\cdots))
\end{equation}

This representation reveals that the commutator of addition and multiplication operations exhibits a non-trivial torsion:
\begin{equation}
\tau = x \oplus_\mu \otimes_\lambda - x \otimes_\lambda \oplus_\mu = \mu(e^\lambda - 1)
\end{equation}

\subsection{Foundational Results}

The central insight of our approach is that arithmetic operations—specifically addition and multiplication—can be interpreted as movements along orthogonal directions in a properly constructed geometric space. This interpretation transforms arithmetic evaluation into geometric propagation, with expression values corresponding to points in a hyperbolic manifold.

The embedding of arithmetic expressions into geometry is governed by a flow equation:
\begin{equation}
\frac{da}{ds} = \mu \cos \theta + a \lambda \sin \theta
\end{equation}

This partial differential equation describes how assignment values propagate through space along directions with angle $\theta$. In its coordinate-free form:
\begin{equation}
\|\nabla a\| = \sqrt{\mu^2 + a^2\lambda^2}
\end{equation}

This is an Eikonal equation equivalent to a special Hamilton-Jacobi equation, connecting our construction to fundamental concepts in analytical mechanics and differential geometry.

We establish a specific realization of this framework in the first kind arithmetic expression space $\mathfrak{E}_1$, defined on the upper half-plane $\mathcal{B} = \{(x,y) \mid y > 0\}$ with a hyperbolic metric:
\begin{equation}
ds^2 = \frac{1}{y^2}\left(\frac{dx^2}{\mu^2} + \frac{dy^2}{\lambda^2}\right)
\end{equation}

In this space, the assignment function $a = -\frac{x}{y}$ satisfies the flow equation with parameters $\mu$ and $\lambda$. Moreover, $a$ is an eigenfunction of the Laplacian with eigenvalue 2, reinforcing the intrinsic geometric nature of this construction.

\subsection{Implications}

\subsection{Roadmap}

Our main results establish: (1) A flow equation governing arithmetic propagation in curved spaces; (2) The $\mathfrak{E}_1$ space as a universal geometric framework for expression evaluation;

Through this work, we demonstrate that arithmetic expressions do indeed form a geometric space—one with rich structure and deep connections to hyperbolic geometry, differential equations, and algebraic systems.

\newpage

\section{Basic concepts}\label{sec:concepts}

Historically, researchers such as Post\cite{Post1943FormalRO} and Chomsky\cite{Chomsky1956ThreeMF} linked formal grammars to automata and rewriting systems, providing the classic recipe for generating well-formed strings via production rules. Meanwhile, foundational work by Russell\cite{Russell1908MathematicalLA}, Church\cite{Church1940AFO}\cite{Church1941TheCO}, and later Martin-Löf\cite{MartinLf1975AnIT}\cite{MartinLf1984IntuitionisticTT} introduced type-theoretic frameworks to avoid paradoxes and to systematically handle partial operations, especially critical when dealing with real numbers and division by zero. These two trajectories—rewriting rules versus type discipline—ultimately converge on the need for both syntactic generativity and semantic rigor\cite{Howard1969TheFN}\cite{Oosten2014TheUF}.

In this paper, we begin by defining arithmetic expressions through basic production rules for addition, subtraction, multiplication, and division over $\mathbb{Q}$. We keep this first step lean, acknowledging that purely rewriting-based definitions can clash with hidden subtleties when extended to $\mathbb{R}$. The result is a fertile middle ground: a tree-structured syntax and partial evaluation that highlight geometric properties (e.g., “threadlike” paths), but also carry unresolved issues around singularities, infinite expansions, and semantic gaps. We postpone those deeper type-theoretic and topological remedies to the final sections, where we address how these complexities reveal new facets of “arithmetic torsion” and other structural phenomena in the geometry of expressions.

\subsection{Arithmetic expression}\label{subsec:arithmetic-expression}

In order to define arithmetic expressions involving real numbers  $\mathbb{R}$ in a rigorous way, we need to use a sophisticated type theory.
However, in order to keep things simple and maintain clarity, we will start by using only production rules, but with certain semantic restrictions.
We will also begin with rational numbers $\mathbb{Q}$ to avoid the difficulties inside real numbers $\mathbb{R}$ .

\begin{definition}\label{def:arithmetic-expression}
    An arithmetic expression $a$ over $\mathbb{Q}$ is a structure given by the following production rules:
\begin{equation}\label{eq:productionrule}
\begin{aligned}
a &\longleftarrow x\\
a &\longleftarrow ( a + a )\\
a &\longleftarrow ( a - a )\\
a &\longleftarrow ( a \times a )\\
a &\longleftarrow ( a \div a )
\end{aligned}
\end{equation}
    where $x \in \mathbb{Q}$, and we denote this as $a \in \mathbb{E} \left [\mathbb{Q} \right ]$.
\end{definition}

During the production process, we can obtain both a string representation and a tree representation of arithmetic expression $a$,
where the two representations are equivalent.
For instance, the string representation of $a$ might be:

\begin{equation}
(((((1 \times 2) \times 2) - 1) \times (2 + 1)) - 6)\label{eq:equation}
\end{equation}

and the parsed syntax tree is depicted in Figure~\ref{fig:syntaxtree}.

\begin{figure}[ht]
\centering
\resizebox{0.2\textheight}{!}{\includegraphics{images/02-example-expression-syntax-tree.pdf}}
\caption{a tree representation of an arithmetic expression}\label{fig:syntaxtree}\label{fig:figure}
\end{figure}

If we interpret the target as a string and the building processes in production rule~\eqref{eq:productionrule} as string building, we get the \emph{string representation}.
On the other hand, if the target is a tree, tree building leads to the \emph{tree representation}.
We can easily obtain the string representation of $a$ from its tree representation by performing a pre-order traversal.

The concept of a \emph{sub-expression} can also be derived from the concept of a subtree.
The branch nodes are all labeled with operators: $+$, $-$, $\times$, $\div$.
The leaf nodes are all labeled with numbers.

Evaluation $\nu$ is a partial function that operates on arithmetic expression $a \in \mathbb{E} \left [\mathbb{Q} \right ]$.
It is undefined only if division by zero occurs during the recursive evaluation process.

We can define evaluation $\nu(a)$ of $a$ recursively as follows:
\begin{itemize}
  \item Constant leaf: for any $x \in \mathbb{Q}$, $\nu(x) = x$.
  \item Compositional node by $+$: For any $(a + b)$, $\nu((a + b)) = \nu(a) + \nu(b)$.
  \item Compositional node by $-$: For any $(a - b)$, $\nu((a - b)) = \nu(a) - \nu(b)$.
  \item Compositional node by $\times$: For any $(a \times b)$, $\nu((a \times b)) = \nu(a) \nu(b)$.
  \item Compositional node by $\div$: For any $(a \div b)$, if $\nu(b) \neq 0$, then $\nu((a \div b)) = \nu(a) / \nu(b)$.
\end{itemize}

We say that an arithmetic expression $a$ is \emph{evaluable} if $\nu(a)$ is defined.
In the rest of this article, we will only consider evaluable arithmetic expressions unless stated otherwise.

Given an arithmetic expression $a$, whatever evaluable or not, we can obtain its tree representation.
If a node $l$ is a leaf node, its corresponding subexpression $s$ is a number, so we consider it to be already \("\)evaluated\("\).
If a node $b$ is a branch node, its corresponding subexpression $s$ is an expression, and we can apply $\nu$ to it to obtain a number $\nu(s)$.
During the recursive evaluation process, starting from the leaves and moving towards the root, the subexpressions are evaluated one after another.
However, the order of evaluations is generally not unique.

\begin{definition}
The evaluation order of an arithmetic expression $a$ is an ordering of branch nodes in the tree representation of $a$
such that every node (sub-expression) is evaluated before its parent.
\end{definition}

For example, the possible evaluation orders of the arithmetic expression in Figure~\ref{fig:syntaxtree} are:
\begin{itemize}
  \item $1 \times 2 \rightarrow \underline{2}; \underline{2} \times 2 \rightarrow \underline{4}; \underline{4} - 1 \rightarrow \underline{3}; 2 + 1 \rightarrow \underline{3}; \underline{3} \times \underline{3} \rightarrow \underline{9}; \underline{9} - 6 \rightarrow 3$
  \item $1 \times 2 \rightarrow \underline{2}; \underline{2} \times 2 \rightarrow \underline{4}; 2 + 1 \rightarrow \underline{3}; \underline{4} - 1 \rightarrow \underline{3}; \underline{3} \times \underline{3} \rightarrow \underline{9}; \underline{9} - 6 \rightarrow 3$
  \item $1 \times 2 \rightarrow \underline{2}; 2 + 1 \rightarrow \underline{3}; \underline{2} \times 2 \rightarrow \underline{4}; \underline{4} - 1 \rightarrow \underline{3}; \underline{3} \times \underline{3} \rightarrow \underline{9}; \underline{9} - 6 \rightarrow 3$
  \item $2 + 1 \rightarrow \underline{3}; 1 \times 2 \rightarrow \underline{2}; \underline{2} \times 2 \rightarrow \underline{4}; \underline{4} - 1 \rightarrow \underline{3}; \underline{3} \times \underline{3} \rightarrow \underline{9}; \underline{9} - 6 \rightarrow 3$
\end{itemize}

The underlined numbers are the numbers that are evaluated during the evaluation process.

Below are examples of expressions that have a unique evaluation order.
These include right-expanded, left-expanded,
and combinations of them, as shown in Figure~\ref{fig:leftright} and Figure~\ref{fig:combination}.

\begin{figure}[ht]
\centering
\resizebox{0.4\textheight}{!}{\includegraphics{images/03-example-expression-syntax-tree-left-right.pdf}}
\caption{right-expanded and left-expanded expressions}\label{fig:leftright}
\end{figure}

\begin{figure}[ht]
\centering
\resizebox{0.2\textheight}{!}{\includegraphics{images/04-example-expression-syntax-tree-combination}}
\caption{combinations of right-expanded and left-expanded expressions}\label{fig:combination}
\end{figure}

The evaluation order of an arithmetic expression is related to the topological order of its tree representation, but they are not the same.
The topological order of a tree is an ordering of nodes such that every node is visited before its parent\cite{Knuth1997TheAO}.
However, we are only interested in the ordering of branch nodes, as leaf nodes have already been evaluated and can be ignored.
Additionally, the topological order goes from parent to children, while the evaluation order goes from children to parent.

\begin{definition}
A threadlike expression is an arithmetic expression that all the left nodes in its tree representation are leaf nodes.
\end{definition}

So a threadlike expression is right-expanded and its evaluation order is unique.
One example of threadlike expressions is shown on the left side of Figure~\ref{fig:leftright}.

Threadlike expressions are significant here because they are analogous to the concept of paths in homotopy theory in geometry.
In a more general context, certain special types of threadlike expressions are also interesting:
for example, \emph{alternating threadlike expressions} are expressions in which the additional and multiplicative operators appear in an alternating manner.
In the field of computing, a hardware component called \emph{multiplier-accumulator} (MAC) unit has been implemented~\cite{Quinnell2007FloatingPointFM},
which is a special case of an alternating threadlike expression.
As a result, some numerical algorithms based on MAC units have been studied~\cite{Markstein2004SoftwareDA}.

\subsection{A scalar field and a mesh grid}\label{subsec:meshgrid}

Consider the upper half plane $\{\mathcal{H}: (x, y) | y > 0 \}$ equipped with an inner product and metrics defined as follows:

\[
\mathbf{a} \cdot \mathbf{b} = \begin{bmatrix} a_x & a_y \end{bmatrix} \begin{bmatrix} \frac{1}{y^2} & 0 \\ 0 & \frac{1}{y^2 \ln^2 2} \end{bmatrix} \begin{bmatrix} b_x \\ b_y \end{bmatrix}
\]

and

\[
ds^2 = \frac{1}{y^2} (dx^2 + \frac{dy^2}{\ln^2 2})
\]

We consider a scalar field satisfying

\begin{equation}
A = - \frac{x}{y}\label{eq:assignment}
\end{equation}

We call this field an \emph{assignment}.

Proper assignments allow us to establish a connection between paths in homotopy and threadlike arithmetic expressions,
and to incorporate function theory into the study of arithmetic expression geometry.

\begin{figure}[ht]
\centering
\resizebox{0.9\textwidth}{!}{\includegraphics{images/01-grid-example-1.pdf}}
\caption{An addition-multiplication grid by generators with $\mu=1$ and $\lambda=\ln 2$}\label{fig:gridex0}
\end{figure}

We can draw a grid on the scalar field $A$ and underlying upper half plane $\mathcal{H}$ as shown in Figure~\ref{fig:gridex0}.
The blue lines encode a $+ 1$ relationship, the green lines encode a $\times 2$ relationship,
and they are line families that are perpendicular to each other.
The length of the line segments between two neighboring crossing points are unit length(calculations in lemma~\ref{lem:regular}).
The red value at the crossing points is the value of the scalar field $A$ at that point.
Based on the relationships encoded by the lines, we can encode threadlike arithmetic expressions,
which will be introduced in the subsection~\ref{subsec:encoding}.

The addition-multiplication grid is also scale-invariant under the transformation
\[
\begin{cases}
x' = \alpha x\\
y' = \alpha y
\end{cases}
\]

where $\alpha = 2^k , k \in \mathbb{Z}$.

We can imagine if we make the grid finer and finer, the grid will become a continuous space.
This leads to a rigorous treatment of arithmetic expressions as a geometric space in section~\ref{sec:topology}.

\subsection{Encoding threadlike expressions on the addition-multiplication grid}\label{subsec:encoding}

If we interpret the horizontal blue lines as $+ 1$ and the vertical green lines as $\times 2$ in Figure~\ref{fig:gridex0},
we can encode threadlike expressions on the addition-multiplication grid.
For example, in Figure~\ref{fig:encoding}
we encode $((((1 \times 4) - 1) \times 2) - 3)$ as the bold black lines.

\begin{figure}[ht]
\centering
\resizebox{0.9\textwidth}{!}{\includegraphics{images/05-example-expression-embedding}}
\caption{encoding threadlike expression}\label{fig:encoding}
\end{figure}

The zigzag lines in Figure~\ref{fig:encoding} can be divided into four parts:
\begin{itemize}
\item the vertical line from $1$ to $4$: encoded as multiplication by $4$
\item the horizontal line from $4$ to $3$: encoded as subtraction by $1$
\item the vertical line from $3$ to $6$: encoded as multiplication by $2$
\item the horizontal line from $6$ to $3$: encoded as subtraction by $3$
\end{itemize}

\subsection{From a scalar field to a space of threadlike expressions}\label{subsec:from-field-to-space}

As shown in Figure~\ref{fig:canonicalform}, we have the following paths and expressions:
\begin{itemize}
\item the black path: $((1 \times 8) - 5) = 3$
\item the purple path: $((1 - \frac{5}{8}) \times 8) = 3$
\item the brown path: $((((((1 - \frac{1}{8}) \times 2) - \frac{1}{2}) \times 2) - 1) \times 2) = 3$
\item the orange path: infinite many addition-multiplication terms accumulated together, a special kind of integration
\end{itemize}

\begin{figure}[ht]
\centering
\resizebox{0.9\textwidth}{!}{\includegraphics{images/06-example-canonical-form}}
\caption{different encodings and their canonical form}\label{fig:canonicalform}
\end{figure}

All of the paths in Figure~\ref{fig:canonicalform} have the same source $1$ and same target $3$.
We will discuss a canonical form for these paths.

It is easy to see that the expressions can be transformed into each other by using the multiplication distributive law and by combining and decomposing terms.

Conversion form brown path to black path
\begin{align}
3 & = ((((((1 - \frac{1}{8}) \times 2) - \frac{1}{2}) \times 2) - 1) \times 2) \\
& = 1 \times 8 -  \frac{1}{8} \times 8 - \frac{1}{2} \times 4 - 1 \times 2 \\
& = ((1 \times 8) - 5)
\end{align}

Conversion form brown path to purple path
\begin{align}
3 & = ((((((1 - \frac{1}{8}) \times 2) - \frac{1}{2}) \times 2) - 1) \times 2) \\
& = (1 - \frac{1}{8}) \times 8 - \frac{1}{2} \times 4 - 1 \times 2 \\
& = (1 - \frac{1}{8}) \times 8 - \frac{1}{4} \times 8 -  \frac{1}{4} \times 8 \\
& = (1 - \frac{1}{8} - \frac{1}{4} - \frac{1}{4}) \times 8 \\
& = ((1 - \frac{5}{8}) \times 8)
\end{align}

Therefore, we can define the black and purple paths in Figure~\ref{fig:canonicalform} as a pair of canonical paths,
which represent all threadlike expressions connecting the source $1$ and the target $3$.

Once we have such canonical paths, we can determine the canonical form of the whole space relative to an arbitrary source point $O$ and any other target point $P$.
This allows us to define the space as a space of threadlike expressions.

\subsection{Currying and path notation}\label{subsec:currying}

Currying is a basic technique in functional programming\cite{Reynolds1972DefinitionalIF},
which is used to transform a function with multiple arguments into a sequence of functions with one argument.
By currying a threadlike arithmetic expression, we can obtain a sequence of functions that operate on an operand, which is the leftmost leaf node.

We introduce the following notation for currying a threadlike arithmetic expression:
\begin{itemize}
    \item initial operand: the leftmost leaf node
    \item operator: $\oplus_y: x \mapsto x + y$
    \item operator: $\ominus_y: x \mapsto x - y$
    \item operator: $\otimes_y: x \mapsto x \cdot e^y$
    \item operator: $\oslash_y: x \mapsto x \cdot e^{-y}$
\end{itemize}

For example, the threadlike arithmetic expression $(((((1 \times 2) \times 2) + 1) \times 3) + 6)$ can be curried as

\[\oplus_6(\otimes_{\ln 3}(\oplus_1(\otimes_{\ln 2}(\otimes_{\ln 2}(1)))))\]

Suppose we have a series of operators $a_1, a_2, \cdots a_{n-1}, a_n$, we introduce a \emph{path notation}.

\[x a_1 a_2 \cdots a_{n-1} a_n \coloneqq a\_n( a_{n-1}( \cdots a_2( a_1(x) ) \cdots ) )\]

So, the above example can be written as

\[1 \otimes_{\ln 2} \otimes_{\ln 2} \oplus_1 \otimes_{\ln 3} \oplus_6 \]

If a path begins with a number, we refer to it as a \emph{bounded path}.
If it does not, we refer to it as a \emph{free path}, similar to the concept of vectors from the origin versus vectors at arbitrary points.
a bounded path results in a number, while a free path results in a function.

Now we will verify that the operators within a path are associative.

\begin{lemma}\label{lemma:associative}
    The operators within a path are associative, i.e. we have \[a [b c] = [a b] c\]
\end{lemma}

\begin{proof}
We use normal typeface to express the path notation, and bold typeface to express the function notation.

For a free path, follow the definition, we have
\[a [b c] = [b c](\mathbf{a}) = \mathbf{c}(\mathbf{b}(\mathbf{a}))\]
\[[a b] c = \mathbf{c}([a b]) = \mathbf{c}(\mathbf{b}(\mathbf{a}))\]

hence, we have
\[a [b c] = [a b] c\]
is hold for a free path.

For a bounded path, we have
\[x a [b c] = [b c](\mathbf{a}(x)) = \mathbf{c}(\mathbf{b}(\mathbf{a}(x)))\]
\[x [a b] c = \mathbf{c}([a b](x)) = \mathbf{c}(\mathbf{b}(\mathbf{a}(x)))\]

hence, we have
\[a [b c] = [a b] c\]
is hold for a bounded path.

\end{proof}

\begin{definition}\label{definition:concatenate}
    The concatenation of paths $p_1 \cdot p_2$ is defined as the composite of functions:
    \[p_1 \cdot p_2 \coloneqq p_2 \circ p_1 \]
\end{definition}

When a sequence of paths is concatenated, and only the first path can be bounded.
If the first path is bounded, the concatenated result is a bounded path.
Otherwise, the concatenated result is a free path.

\subsection{Alternating threadlike expressions}\label{subsec:alternating}

Now we can define alternating threadlike expressions, which were mentioned in Section~\ref{sec:expression}, using the path notion.

\begin{equation}\label{eq:alternative}
    \alpha = a_1 b_1 a_2 b_2 \cdots a_l b_l, a_i = \otimes_{\lambda_i}, b_i = \oplus_{\mu_i}, \lambda_i, \mu_i \in \mathbb{R}
\end{equation}

where $\bigoplus$ and $\bigotimes$ denote addition and multiplication, respectively,
and the expression is a zigzag of alternating addition and multiplication operations.
$\alpha$ is a free path, and we can bind a number to it.

Since $0$ is the identity element for addition and $1$ is the identity element for multiplication,
it is straightforward to see that any arithmetic expression can be converted into an alternating threadlike expression
by introducing more $0$ and $1$ into the original expression.
So alternating threadlike expression is a kind of canonical form.

We can derive a formula for perturbations in alternating threadlike expressions.

Let us define the left-to-right accumulated sum of $\lambda_i$ as $\check{\lambda}_i$, such that:
\begin{equation}
\check{\lambda}_i = \sum_{j=1}^i \lambda_j, \check{\lambda}_0 = 0\label{eq:accsumlr}
\end{equation}

Then we also have right-to-left accumulated sum of $\lambda_i$
\begin{equation}
\hat{\lambda}_i = \check{\lambda}_l - \check{\lambda}_{l - i}, \hat{\lambda}_0 = 0\label{eq:accsumrl}
\end{equation}

Expanding equation~\eqref{eq:alternative} using the distributive law and the above notion at point $\mu_0$, we obtain:
\begin{align}
\alpha(\mu_0) & = e^{\lambda_l}(\cdots (e^{\lambda_2} (e^{\lambda_1} \mu_0 + \mu_1) + \mu_2) \cdots) + \mu_l \\
& = e^{\hat{\lambda}_l} \mu_0 + e^{\hat{\lambda}_{l - 1}} \mu_1  + e^{\hat{\lambda}_{l - 2}} \mu_2 + \cdots + e^{\hat{\lambda}_1} \mu_{l - 1} + e^{\hat{\lambda}_0} \mu_l
\end{align}

Next, at the starting point $\mu_0$, we introduce a perturbation $\tilde{\mu}_0 = e^{\eta_0} \mu_0 + \epsilon_0$,
where $\eta_0$ and $\epsilon_0$ are the disturbance terms added by the summation and multiplication operations, respectively. Then, we have:
\begin{align}
\alpha(\tilde{\mu}_0) & = e^{\hat{\lambda}_l} (\tilde{\mu}_0) + e^{\hat{\lambda}_{l - 1}} \mu_1  + e^{\hat{\lambda}_{l - 2}} \mu_2 + \cdots + e^{\hat{\lambda}_1} \mu_{l - 1} + e^{\hat{\lambda}_0} \mu_l \\
& = \alpha(\mu_0) + e^{\hat{\lambda}_l} (\tilde{\mu}_0 - \mu_0)
\end{align}

As a result, purely from an arithmetic perspective, without the need for limits, we can derive the following meaningful ratio:
\begin{equation}
\frac{\alpha(\tilde{\mu}_0) - \alpha(\mu_0)}{\tilde{\mu}_0 - \mu_0} = e^{\hat{\lambda}_l} = e^{\check{\lambda}_l}\label{eq:ratio}
\end{equation}

Now we extend this relationship from the starting point $\mu_0$ to the entire process, we define the recursive formula

\[
w_i = e^{\lambda_i} w_{i-1} + \mu_i, w_0 = 0
\]

and then we have

\begin{equation}
\frac{\tilde{w}_i - w_i}{\tilde{\mu}_0 - \mu_0} = e^{\check{\lambda}_i}, i \in \{1, ..., l\}\label{eq:perturbation1}
\end{equation}

So, we have

\[
\tilde{w}_i - w_i = e^{\check{\lambda}_i} (\tilde{\mu}_0 - \mu_0)
\]

and hence

\begin{equation}
\tilde{w}_i - w_i = e^{\lambda_i}(\tilde{w}_{i - 1} - w_{i - 1})\label{eq:perturbation2}
\end{equation}

That means the perturbation along the path is controlled by the multiplication terms of $e^{\lambda_i}$.

\subsection{Generated structure, commutator and arithmetic torsion}\label{subsec:generated-structure}

In order to study mesh grids like the one described in subsection~\ref{subsec:meshgrid},
we need to investigate the algebraic structure of the threadlike arithmetic expressions that are generated.

For real number $\mathbb{R}$ and elements $\mu, \lambda \in \mathbb{R}$, we consider all the arithmetical expressions
that are freely generated from
\begin{itemize}
    \item initial operand: $0$
    \item operator: $\oplus_\mu: x \mapsto x + \mu$
    \item operator: $\ominus_\mu: x \mapsto x - \mu$
    \item operator: $\otimes_\lambda: x \mapsto x \cdot e^\lambda$
    \item operator: $\oslash_\lambda: x \mapsto x \cdot e^{- \lambda}$
\end{itemize}

We denote these expressions as $E(\mu, \lambda)$, where $\mu$ is the additional generator and $e^\lambda$ is the multiplicative generator.
In cases where the context is clear, we may omit $\mu$ and $\lambda$ from the index.
Our goal is not to study only a single $E(\mu, \lambda)$, but rather to use a family of $E(\mu, \lambda)$ to approach a continuous space.

Since $\oplus_\mu$ and $\ominus_\mu$ are mutually inverse operations, it follows that $\otimes_\lambda$ and $\oslash_\lambda$ are also mutually inverse. This means that $E(\mu, \lambda)$ forms a group.
An intriguing observation is that the commutator of this group is not equal to identity generally,
especially the commutator of the generators.

\begin{equation}
x \oplus_\mu \otimes_\lambda \ominus_\mu \oslash_\lambda - x = \mu(1 - e^{-\lambda})\label{eq:commutator1}
\end{equation}
\begin{equation}
x \otimes_\lambda \oplus_\mu \oslash_\lambda \ominus_\mu - x = - \mu(1 - e^{-\lambda})\label{eq:commutator2}
\end{equation}

Or equivlently, we define below error $\tau$:

\begin{equation}
\tau = x \otimes_\lambda \oplus_\mu - x \oplus_\mu \otimes_\lambda = \mu(1 - e^\lambda)\label{eq:torsion}
\end{equation}

These errors are constant, indicating a type of torsion in the generated group.
And torsion $\tau$ is specifically referred to as the arithmetic torsion.

We will reveal that $\tau$ is related to the curvature of the surface in later sections.

\subsection{Problems on equality, singularity, symmetries}\label{subsec:problems-on-equality-singularity-symmetries}

From the perspective of computer science, it is useful to consider different levels of equality within freely generated structures.
\begin{itemize}
\item Literal equality: the finest level of equality, judged by the string representation of the expression
\item Syntactical equality: equality under certain syntactical rules
\begin{itemize}
\item When inverse operators exist, it forms a group
\item When the commutative and distributive laws exist, it can be considered an algebra
\end{itemize}
\item Semantic equality: the coarsest level of equality, judged by the evaluation of the expression
\end{itemize}

Literal equality is the strictest level of equality, and two different threadlike expressions are considered equal only if their string representations are exactly the same.
This level of equality may be too strict, as it may not be compatible with the evaluation of the expression.
However, under literal equality, the generated structure is the most rich and provides the base textures that can be woven into a space.

Semantic equality is the least strict level of equality, and two different threadlike expressions are considered equal if they evaluate to the same number.
This level of equality provides the total symmetrical resources of the space.

We can think of literal equality as the bottom and semantic equality as the top of a lattice,
with syntactical equality being a compromise between the two extremes.

To end this introduction part of the paper, we present several problems and speculations that drives our research.
These important problems arise from distance between syntactical and semantic structures.

\emph{Foundational problem}: A careful reader may have noticed that the definition~\ref{def:arithmetic-expression} is based on rational numbers $\mathbb{Q}$.
Why can't we use real numbers $\mathbb{R}$ instead?
The answer is that syntactically valid expressions may not be semantically valid.
Dividing by zero can lead to invalid expressions, and the evaluation of the expression cannot be defined in this situation.
Therefore, in real numbers, an expression may be syntactically valid but semantically not valid,
and there is no algorithm that can decide whether an expression is semantically valid or not.
How can we bridge this gap and provide a continuous geometry space?
We will attempt to partially solve this problem in some special cases in section~\ref{sec:topology}.

\emph{Singular point problem}: We have a very strong intuition that semantically invalid expressions lead to singular points.
The way we discussed in complex analysis may be borrowed here: essential singularities and poles.

\emph{Symmetry and classification problem}: We conjecture that the equality lattice may not only play a role in the construction of a space, but also determine the symmetry of that space.
We can imagine that, at certain levels of the lattice, we weave syntactically generated substructures into points to form a space,
and the weaving process uses up some symmetrical resources, leaving the rest to form a symmetry on the space.
The structure within the total symmetry may provide us with a systematic way of constructing spaces, and allow us to classify spaces based on their symmetries.

\newpage

\section{Flow equation and its conclusion}\label{sec:flowequation}

\subsection{Flow equation}\label{sec:equation}

Consider an infinitesimal generating process on a Riemannian surface $M$ using two generators:
one for an additional action $\mu$ and the other for a multiplicative action $e^\lambda$.
These two generators are perpendicular.
This generation process produces an assignment $A: M \to R$ over the surface.

For any point with an assignment $a_0$, if we consider a movement of distance $\epsilon$ in a direction with angle $\theta$
over a time period of $\delta$, we can establish the following:

\[
    a_{\delta} = (a_0 + \mu \epsilon \cos \theta)e^{\lambda \epsilon \sin \theta}
\]

or

\[
    a_{\delta} = a_0 e^{\lambda \epsilon \sin \theta} + \mu \epsilon \cos \theta
\]

Both formula can be simplified to the same result:

\[
    a_{\delta} = a_0 + \epsilon (a_0 \lambda \sin \theta + \mu \cos \theta)
\]

Then, we have the following equation:

\[
    \frac{1}{\delta} (a_{\delta} - a_0) = \frac{\epsilon}{\delta} (\mu \cos \theta + x_0 \lambda \sin \theta)
\]

When both $\delta$ and $\epsilon$ are towards zero, we get $da / dt$, and hence

\[
    \frac{da}{dt} = u (\mu \cos \theta + a \lambda \sin \theta)
\]

Or, we can change it to another form

\begin{equation}
    \frac{da}{ds} = \mu \cos \theta + a \lambda \sin \theta\label{eq:flow}
\end{equation}

We name this equation~\eqref{eq:flow} as the flow equation.

The left side of this equation is governed by the distance structure, while the right side is governed by the angle structure.
This leads to below theorem.

\begin{theorem}
Isometries keep the flow equation\eqref{eq:flow}
\label{thm:isometry}
\end{theorem}

\begin{proof}
  An isometry keeps both the distance structure and the angle structure, so it keeps the flow equation.
\end{proof}

We can also get a direct formal solution of the flow equation~\eqref{eq:flow}(details in Appendix~\ref{sec:directformalsolution}).

\begin{equation}
   a = (a_0 + \frac{\mu}{\lambda} \cot \theta) e^{\lambda s \sin \theta} - \frac{\mu}{\lambda} \cot \theta\label{eq:solution}
\end{equation}

Now, we can use this solution to derive (details in Appendix~\ref{sec:conformance}) the relationship between assignments at different vertices in a cell of a mesh grid.

Assuming the assignment at the center point of a cell is $a_0$, we move in different directions and obtain the following assignments:

\begin{itemize}
\item $\theta = 0$: $a_s = a_0 + \mu s$
\item $\theta = \frac{\pi}{2}$: $a_s = a_0 e^{\lambda s}$
\item $\theta = \pi$: $a_s = a_0 - \mu s$
\item $\theta = \frac{3 \pi}{2}$: $a_s = a_0 e^{- \lambda s} $
\end{itemize}

This result is straightforward, but it demonstrates that the infinitesimal generating process is consistent with the discrete mesh grid.

\subsection{The contour-gradient form of flow equation}\label{subsec:the-contour-gradient-form}

It is easy to derive the contour equation in the local coordinate

\begin{equation}
    \mu \cos \theta_c + a \lambda \sin \theta_c = 0\label{eq:contour}
\end{equation}

then we have

\begin{equation}
    \theta_c = - \arctan \frac{\mu}{a \lambda}\label{eq:contourangle}
\end{equation}

the contour and the gradient are perpendicular to each other

\begin{equation}
    \theta_g = \pm \frac{\pi}{2} - \arctan \frac{\mu}{a \lambda}\label{eq:gradientangle}
\end{equation}

then along $\theta_g$ we have

\begin{equation}
    \frac{da}{ds} = \mu \cos (\pm \frac{\pi}{2} - \arctan \frac{\mu}{a \lambda}) + a \lambda \sin (\pm \frac{\pi}{2} - \arctan \frac{\mu}{a \lambda})
    \label{eq:alonggradient}
\end{equation}

\begin{equation}
    \frac{da}{ds} = \pm \sqrt{\mu^2 + \lambda^2 a^2}\label{eq:grad}
\end{equation}

By introducing the right-hand rotation angle $\phi$ along the gradient direction, we can establish a local polar coordinate system based on the gradient and contour lines.
Then the growth rate of $a$ along the angle $\phi$ is

\begin{equation}
    \frac{da}{ds} = \mu \cos (\frac{\pi}{2} - \arctan \frac{\mu}{a \lambda} + \phi) + a \lambda \sin (\frac{\pi}{2} - \arctan \frac{\mu}{a \lambda} + \phi)
    \label{eq:fourfold}
\end{equation}

And the simplified equation is

\begin{equation}
    \frac{da}{ds} = \sqrt {\mu^2 + a^2 \lambda^2} \cos \phi\label{eq:contourgradient}
\end{equation}

or

\begin{equation}
    \frac{da_{\phi}}{ds_{\phi}} = \sqrt {\mu^2 + a^2 \lambda^2} \cos \phi\label{eq:contourgradient2}
\end{equation}

if we want to emphasize the path is along the angle $\phi$.

The equation~\eqref{eq:contourgradient} is the flow equation in the contour-gradient coordinate system.

Equation~\eqref{eq:contourgradient} is solvable, and we get the relation between $a$ and $s$:

\begin{equation}\label{eq:rel_a_s}
    \tanh(\lambda s \cos \phi - c) = \frac{\lambda a}{\sqrt{\mu^2 + \lambda^2 a^2}}
\end{equation}

we can further simplify the equation to

\begin{equation}
  a = \pm \frac{\mu}{\lambda} \sinh(\lambda s \cos \phi - c)\label{eq:gradevo}
\end{equation}

Under the initial condition $a = a_0$ when $s = 0$, we can get the following equation:

\begin{equation}
    a = \frac{\mu}{\lambda} \sinh(\lambda s \cos \phi + \arcsinh \frac{a_0 \lambda}{\mu})\label{eq:gradevo2}
\end{equation}

or

\begin{equation}
    a = - \frac{\mu}{\lambda} \sinh(\lambda s \cos \phi - \arcsinh \frac{a_0 \lambda}{\mu})\label{eq:gradevo3}
\end{equation}

In this coordinate system, the additional line and the multiplicative line are:

\begin{equation}
    \phi = \arccos \frac{\mu}{\sqrt {\mu^2 + a^2 \lambda^2}} \label{eq:additionalline}
\end{equation}

\begin{equation}
    \phi = \arcsin \frac{\mu}{\sqrt {\mu^2 + a^2 \lambda^2}}\label {eq:mulitiplcativeline}
\end{equation}

\subsection{Local Descartes coordinate and area formula}\label{subsec:descartes-coordinate}
We begin our exploration by examining the flow equation~\eqref{eq:flow} within the framework of a local polar coordinate system:

\begin{equation}
    \frac{da}{ds} = \mu \cos \theta + a \lambda \sin \theta
\end{equation}

In an effort to recontextualize this equation, we set $dx = \cos \theta ds$ and $dy = \sin \theta ds$,
thereby enabling us to express it in a different light:

\begin{equation}
    da = \mu dx + a \lambda dy
\end{equation}

Our attention now turns to the concept of arithmetic torsion, particularly at an infinitesimal level. Delving into the interplay between two infinitesimal generating processes, we observe that:

\begin{equation}
    d\tau = a_0 e^{\lambda dy} + \mu dx - (a_0 + \mu dx) e^{\lambda dy}
\end{equation}

From this relationship, we deduce:

\begin{equation}
    d\tau = \mu dx (1 - e^{\lambda dy})
\end{equation}

This leads us to an intriguing area formula, capturing the essence of this interaction:

\begin{equation}
    d\tau = - \mu \lambda dx dy \label{eq:area_formula}
\end{equation}

or in the form of a differential form:
\begin{equation}
    d\tau = - \mu \lambda dS \label{eq:area_formula2}
\end{equation}

where $dS$ is the area element.

This formula is interesting because it connects the area elements and arithmetic torsion which by nature are defined on edge of the area.

These kind of formulas are also meet in the study of classic analysis or differential geometry, for example, Stokes theorem or the Gauss-Bonnet theorem.
We will develop this formula further in the following section\ref{sec:curvature} to connect it with curvature and the Gauss-Bonnet theorem.

\subsection{Flow and function}\label{subsec:flow-and-function}

In this section, we aim to present novel insight into functions.
Namely, the treatment of functions as flows will be discussed.

\begin{definition}\label{def:projection}
Given a function $k$ on the real domain $R$, we can introduce a mapping $l$ on the arithmetic expression space $H$ such that the following diagram commutes.

\begin{center}
    \begin{tikzcd}
        H && H \\
        R && R
        \arrow["l", from=1-1, to=1-3]
        \arrow["\nu"', from=1-1, to=2-1]
        \arrow["\nu", from=1-3, to=2-3]
        \arrow["k"', from=2-1, to=2-3]
    \end{tikzcd}
\end{center}

where $\nu$ is the evaluation function of the expression. Then we call the mapping $l$ is the promotion of the function $k$,
or function $k$ is the projection of the mapping $l$.
\end{definition}

Giving an arithmetic expression space as definition at the beginning of the section\ref{subsec:meshgrid},
we will show examples of flows as functions.

\subsubsection{Example: Brachistochrone curve}\label{subsubsec:brachistochrone}

The brachistochrone curve is the curve between two points that minimizes the time taken by a frictionless bead sliding along a wire.
We assume the wire is a curve $\eta = f(\delta)$, and the bead starts at $(0, 0)$ and ends at $( \Delta, H )$ as time $t$ goes from $0$ to $T$.

TODO

\subsection{The existence theorems}\label{subsec:existence-theorems}

There are two important existence theorems related to the flow equation~\eqref{eq:flow}.

The first existence theorem states that if we have a 2D Riemannian manifold $M$, then there exists a function $a$ on $M$ that satisfies the flow equation~\eqref{eq:flow}. This theorem is proved in a later section, and we will not go into the details of the proof here.

\begin{theorem}
    Given a 2D Riemannian manifold $M$, there exists a function $a$ on $M$ satisfying the flow equation~\eqref{eq:flow}.
    \label{prop:existence1st}
\end{theorem}

The second existence theorem is also crucial.
It says that if we have a smooth surface $S$ and a function $a$ on $S$, then we can find a metric $g$ on $S$ that makes $a$ satisfy the flow equation~\eqref{eq:flow}.
We morph the metric $g$ to make $a$ satisfy the flow equation~\eqref{eq:flow}, this technique will be touched back and forth in the following sections.

\begin{theorem}(By Le Zhang)
    Given an oriented 2D compact Riemannian manifold $S$, and a function $a$ over $S$, there exists a metric $g$ on $S$ that makes $a$
    satisfying the flow equation~\eqref{eq:flow}.
    \label{prop:existence2nd}
\end{theorem}

\begin{proof}

    The proof is divided into two parts: local perspective and global perspective.

    \emph{Local perspective}:

    Consider a point \( p \) on the surface \( S \), and there is a neighborhood \( U \) around \( p \).
    In this area, we can find a local isothermal coordinate system in which the metric takes the form:
    \[ ds^2 = e^{2\rho}(du^2 + dv^2), \]
    where \( u \) and \( v \) are the coordinates of \( U \), and \( \rho \) is a function of \( u, v \) in \( U \).
    The gradient of \( a \) in this local isothermal coordinate system is expressed as:
    \[ \nabla a = \frac{\partial a}{\partial u} du + \frac{\partial a}{\partial v} dv. \]
    Using the definition of the directional derivative, we obtain:
    \[ \frac{da_{\psi}}{ds_{\psi}} = ||\nabla a|| \cos \psi, \]
    where \( ||\nabla a|| \) is the norm of \( \nabla a \), and \( \psi \) is the angle between \( \nabla a \) and the direction of movement.

    Now, considering the flow equation~\ref{eq:contourgradient2} in the gradient-contour coordinate system, we have:
    \[ \frac{da_{\phi}}{ds_{\phi}} = \sqrt{\mu^2 + a^2 \lambda^2} \cos \phi. \]

    Note that \( ||\nabla a|| \) is fixed for the given function \( a \) and the local coordinate system, and \( \sqrt{\mu^2 + a^2 \lambda^2} \) is also fixed for the given function \( a \).
    We can scale \( e^{2\rho} \) with a linear factor \( \kappa \) to make \( ||\nabla a|| \) match the fixed value of \( \sqrt{\mu^2 + a^2 \lambda^2} \),
    thus we have a morphing process controlled by \( \kappa \) that
    \begin{align}
    ds^2 &= \kappa e^{2 \rho}(du^2 + dv^2)\label{eq:morphing} \\
         &= e^{2 \rho + \ln \kappa}(du^2 + dv^2).
    \end{align}

    Under the morphing ratio \( \kappa \), we have:

    \begin{equation}
      ||\nabla_\kappa a|| = \kappa^{-1} ||\nabla a||,
    \end{equation}

    and when \( \kappa \) is set to the value of:

    \[ ||\nabla_\kappa a|| = \sqrt{\mu^2 + a^2 \lambda^2}, \]

    the flow equation~\eqref{eq:flow} is satisfied in the local coordinate system.

    The morphing ratio \( \kappa \) is calculated as follows:
    \begin{equation}
        \kappa = \frac{||\nabla a||}{\sqrt{\mu^2 + a^2 \lambda^2}}\label{eq:ratio}.
    \end{equation}

    \emph{Global perspective}:

    When we consider the broader scope of the surface \( S \), it's possible to extend the morphing process to every point,
    ensuring that the flow equation~\eqref{eq:flow} is satisfied on a global scale.
    However, this expansion necessitates a harmonious integration of the morphing process across neighboring locales.
    Specifically, this means that the morphing should not only preserve the circles centered at point \( p \) within its immediate local chart
    but also maintain the integrity of these circles within the adjacent charts of point \( p \).
    In essence, the morphing process must be seamlessly coordinated across the various local regions to achieve a unified global transformation.

    TODO

    \qedhere
\end{proof}

\newpage

\section{The first kind arithmetic expression space $\mathfrak{K}_1$}\label{sec:firstkind}

In this section, we introduce the first kind arithmetic expression space $\mathfrak{E}_1$, which constitutes a geometric framework for the systematic analysis of arithmetic expressions. We commence with foundational exemplars, establish a comprehensive theoretical framework, examine geometric propagation mechanisms, and investigate the relationship between grid structures and torsion within this space.

\subsection{Foundational exemplars}\label{subsec:motivexamples}

We present two analytically equivalent examples that belong to the class of spaces designated as the first kind arithmetic expression space $\mathfrak{E}_1$.

\subsubsection{Example 1: Upper Half Plane Model}

Consider the upper half plane ${\mathcal{H}: (x, y) \ | \ y > 0}$ equipped with the following inner product and metric tensor:

$$
\mathbf{a} \cdot \mathbf{b} = \begin{bmatrix} a_x & a_y \end{bmatrix} \begin{bmatrix} \frac{1}{y^2} & 0 \\ 0 & \frac{1}{y^2} \end{bmatrix} \begin{bmatrix} b_x \\ b_y \end{bmatrix}
$$

$$
ds^2 = \frac{1}{y^2} (dx^2 + dy^2)
$$

On this manifold, we define an assignment field $a$ as follows:

\begin{equation}\label{eq:exmp1}
a = - \frac{x}{y}
\end{equation}

\begin{theorem}\label{thm:exmp1}
The assignment $a$ defined by formula \eqref{eq:exmp1} satisfies the flow equation \eqref{eq:flow}.
\end{theorem}

\begin{proof}
We initiate with the differential of the assignment:
$$
da = d\left(-\frac{x}{y}\right) = \frac{xdy - ydx}{y^2} = -\frac{dx + ady}{y}
$$

The differential of arc length is given by:
$$
ds = \frac{\sqrt{dx^2 + dy^2}}{y}
$$

Therefore:
$$
\frac{da}{ds} = - \frac{dx + ady}{y} \cdot \frac{y}{\sqrt{dx^2 + dy^2}} = - \frac{dx + ady}{\sqrt{dx^2 + dy^2}}
$$

In the local coordinate system determined by $(-1, 0)$ and $(0, -1)$ under the right-hand rule, we have:
$$
\cos \theta = \frac{-dx}{\sqrt{dx^2 + dy^2}} \quad \text{and} \quad \sin \theta = \frac{-dy}{\sqrt{dx^2 + dy^2}}
$$

Substituting these values:
$$
\frac{da}{ds} = \cos \theta + a \sin \theta
$$

This precisely corresponds to the flow equation \eqref{eq:flow} with $\mu=1$ and $\lambda=1$.
\end{proof}

We can verify that $a$ constitutes an eigenfunction of the Laplacian operator:
$$
\Delta a = - y^2 \left(\frac{\partial^2 a}{\partial x^2} + \frac{\partial^2 a}{\partial y^2}\right) = y^2 \left(\frac{\partial}{\partial y} \left(\frac{\partial}{\partial y} \frac{x}{y}\right)\right) = 2a
$$

\subsubsection{Example 2: Horocycle-Based Coordinate System}

For our second exemplar, we introduce a horocycle-based coordinate system for hyperbolic surfaces. This global coordinate system comprises two orthogonal families of curves: horocycles sharing the same ideal point, and geodesics perpendicular to these horocycles.

\begin{figure}[ht]
\centering
\resizebox{0.5\textwidth}{!}{\includegraphics{images/11-horocyclebased}}
\caption{A horocycle-based coordinate system on the Poincaré disc. Blue curves represent horocycles tangent at ideal point $\Omega$, green lines depict perpendicular geodesics.}\label{fig:horocyclecoord}
\end{figure}

On the Poincaré disc $\mathcal{P}$, the coordinates of a point $P$ are denoted by $(u,v)$, where:
\begin{itemize}
\item $u$ represents the signed length of $OQ$
\item $v$ represents the signed length of $QP$
\item The sign conventions adhere to the right-hand rule and orientation relative to the ideal point $\Omega$
\end{itemize}

We equip this coordinate system with the inner product:
$$
\mathbf{a} \cdot \mathbf{b} = \begin{bmatrix} a_u & a_v \end{bmatrix} \begin{bmatrix} e^{-2v} & 0 \\ 0 & 1 \end{bmatrix} \begin{bmatrix} b_u \\ b_v \end{bmatrix}
$$

And the corresponding metric tensor:
$$
ds^2 = e^{-2v} du^2 + dv^2
$$

The Laplacian operator in this coordinate system is expressed as:
$$
\Delta = e^{2v} \frac{\partial^2}{{\partial u}^2} + \frac{\partial^2}{{\partial v}^2} - \frac{\partial}{\partial v}
$$

In this coordinate framework, we define an assignment:

\begin{equation}\label{eq:exmp2}
a = u e^{-v}
\end{equation}

\begin{theorem}\label{thm:exmp2}
The assignment $a$ defined by formula \eqref{eq:exmp2} satisfies the flow equation \eqref{eq:flow}.
\end{theorem}

\begin{proof}
We establish this result by demonstrating that examples 1 and 2 are equivalent through a Möbius transformation. Consider the complex representation of the upper half plane:
$$
z = x + yi
$$

The Möbius transformation mapping the upper half plane to the Poincaré disc is given by:
$$
z \mapsto \frac{z-i}{z+i}
$$

\begin{figure}[ht]
\centering
\resizebox{0.8\textwidth}{!}{\includegraphics{images/12-proofbymapping}}
\caption{Mapping between the upper half plane and Poincaré disc models}\label{fig:mapping}
\end{figure}

This conformal transformation maps horizontal lines in $\mathcal{H}$ to horocycles sharing the ideal point $\Omega = 1$ in $\mathcal{P}$, and vertical geodesics in $\mathcal{H}$ to perpendicular geodesics in $\mathcal{P}$.

Expressed in the target coordinate system, this transformation yields:
$$
\begin{cases}
x = u\\
y = e^v \\
\end{cases}
$$

Substituting into the assignment from Example 1:
$$
a = -\frac{x}{y} = -\frac{u}{e^v} = -u e^{-v}
$$

Since the Möbius transformation is conformal and preserves the flow equation, and accounting for the orientation change, we obtain $a = u e^{-v}$ satisfying the flow equation.
\end{proof}

As in Example 1, we can verify that $a$ constitutes an eigenfunction of the Laplacian:
$$
\Delta a = e^{2v} \frac{\partial^2(u e^{-v})}{{\partial u}^2} + \frac{\partial^2(u e^{-v})}{{\partial v}^2} - \frac{\partial(u e^{-v})}{\partial v} = 2a
$$

These two examples, emerging from the same geometric foundation but expressed in different coordinate systems, demonstrate the fundamental properties of the first kind arithmetic expression space.

\subsection{Theoretical framework of $\mathfrak{E}_1$ space}\label{subsec:generalframework}

Building upon the foundational exemplars, we now establish a comprehensive theoretical framework for the first kind arithmetic expression space $\mathfrak{E}_1$. 

Consider the upper half plane $\mathcal{B}$:
$$
\{\mathcal{B}: (x, y) | y > 0 \}
$$

equipped with an inner product and metric tensor parameterized by constants $\mu$ and $\lambda$:

$$
\mathbf{a} \cdot \mathbf{b} = \begin{bmatrix} a_x & a_y \end{bmatrix} \begin{bmatrix} \frac{1}{\mu^2 y^2} & 0 \\ 0 & \frac{1}{\lambda^2 y^2} \end{bmatrix} \begin{bmatrix} b_x \\ b_y \end{bmatrix}
$$

$$
ds^2 = \frac{1}{y^2}\left(\frac{dx^2}{\mu^2} + \frac{dy^2}{\lambda^2}\right)
$$

The assignment function in this generalized framework maintains the form:

\begin{equation}\label{eq:genassignment}
a = - \frac{x}{y}
\end{equation}

This defines the first kind arithmetic expression space $\mathfrak{E}_1$, characterized by the following theorem:

\begin{theorem}\label{thm:generalE1}
The assignment $a$ given by \eqref{eq:genassignment} satisfies the flow equation \eqref{eq:flow} with parameters $\mu$ and $\lambda$, independent of the specific values of these generators.
\end{theorem}

\begin{proof}
The differential of the assignment is given by:
$$
da = d\left(-\frac{x}{y}\right) = \frac{xdy - ydx}{y^2} = -\frac{dx + a dy}{y}
$$

The differential of arc length is expressed as:
$$
ds = \frac{1}{y}\sqrt{\frac{dx^2}{\mu^2} + \frac{dy^2}{\lambda^2}}
$$

Therefore:
$$
\frac{da}{ds} = - \frac{dx + a dy}{y} \cdot \frac{y}{\sqrt{\frac{dx^2}{\mu^2} + \frac{dy^2}{\lambda^2}}} = -\frac{dx + a dy}{\sqrt{\frac{dx^2}{\mu^2} + \frac{dy^2}{\lambda^2}}}
$$

In the local coordinate system determined by $(-1, 0)$ and $(0, -1)$ according to the right-hand rule:

$$
\cos \theta = \frac{-\frac{dx}{\mu}}{\sqrt{\frac{dx^2}{\mu^2} + \frac{dy^2}{\lambda^2}}} \quad \text{and} \quad \sin \theta = \frac{-\frac{dy}{\lambda}}{\sqrt{\frac{dx^2}{\mu^2} + \frac{dy^2}{\lambda^2}}}
$$

Substituting these values:
$$
\frac{da}{ds} = \mu \cos \theta + a \lambda \sin \theta
$$

This precisely corresponds to the flow equation \eqref{eq:flow} with the given parameters $\mu$ and $\lambda$.
\end{proof}

The $\mathfrak{E}_1$ space is distinguished by its intrinsic connection to hyperbolic geometry and the property that the assignment function $a = -x/y$ constitutes an eigenfunction of the Laplacian operator with eigenvalue 2. This space provides a natural geometric framework for analyzing arithmetic expressions, particularly those involving addition and multiplication operations.

\subsection{Geometric propagation mechanisms}\label{subsec:geompropagation}

The flow equation in the $\mathfrak{E}_1$ space can be interpreted as describing the propagation of arithmetic expressions. This interpretation extends the propagation method discussed in Section \ref{subsec:propagation-method}.

In the $\mathfrak{E}_1$ space, paths along constant $x$ (vertical geodesics in the upper half plane) correspond to multiplication operations, while paths along constant $y$ (horizontal geodesics) correspond to addition operations. The assignment value $a = -x/y$ propagates according to the flow equation as we traverse these paths.

The propagation can be visualized as wavefronts emanating from source points in the upper half plane. Points with identical assignment values form equipotential curves, and the flow equation governs the geometric evolution of these wavefronts.

From equation \eqref{eq:gradevo5}, we recall that along the gradient direction ($\phi=0$) starting from $a_0=0$:
\begin{equation}
a = \pm \frac{\mu}{\lambda} \sinh(\lambda s)
\end{equation}

This expression exhibits a structural similarity to the formula for the circumference of a circle in hyperbolic space with curvature $-\lambda^2$:
\begin{equation}
C = \frac{2\pi}{\lambda} \sinh(\lambda s)
\end{equation}

This correspondence suggests that the assignment $a$ propagates analogously to expanding concentric circles in hyperbolic space, with the zero assignment locus serving as the collection of centroids from which these circles emanate.

The dual perspective of propagation and evaluation provides a powerful framework for understanding arithmetic expressions geometrically:
1. Evaluation of an expression follows geodesic paths through the $\mathfrak{E}_1$ space
2. Different evaluation orders correspond to distinct paths with identical endpoints
3. The flow equation ensures that the final value remains invariant with respect to the path taken (for evaluable expressions)

\subsection{Grid structures}\label{subsec:grids}

A significant geometric characteristic of the first kind arithmetic expression space is the presence of two distinct yet interrelated grid structures, each encoding addition and multiplication operations in different ways. These dual grids reflect the geometric structure of the Baumslag–Solitar group, whose Cayley graph exhibits an anisotropic, hierarchical lattice with a natural correspondence to mixed additive-multiplicative expressions.

Both grid structures are constructed within the upper half-plane model. The first grid is rectilinear, consisting of horizontal lines encoding addition operations and vertical lines encoding multiplication operations. Specifically, this grid is constructed through iterative applications of these two operations to generate a lattice in the $(x,y)$ coordinates. Each horizontal displacement from $(x,y)$ to $(x+1,y)$ represents an addition, and each vertical displacement from $(x,y)$ to $(x,2y)$ represents a multiplication by 2. The grid vertices correspond to values of expressions constructed from repeated applications of addition and multiplication operations, typically originating from a rational base point, often designated as 1.

Notably, in this first grid, the scalar field $a$ remains invariant under variations of the metric parameters $\mu$ and $\lambda$. That is, while the geometric properties of the space (lengths and angles) vary with these parameters, the locus where $a=0$—specifically, the vertical line $x=0$—remains structurally invariant and unaffected by changes in $\mu$ and $\lambda$.

The second grid emerges through a conformal transformation, specifically the Möbius transformation acting within the upper half-plane. This transformation maps horizontal lines to semicircles centered on the real axis and vertical rays to orthogonal semicircles. Under this transformation, the roles of addition and multiplication are effectively interchanged. The addition-multiplication structure of the first grid transforms into a new system of curved geodesics: the images of addition steps now follow curved trajectories around the origin, while the multiplicative steps exhibit contraction and inversion properties.

Although the visual structure of the second grid exhibits greater complexity, it retains a profound arithmetic coherence. The grid vertices continue to correspond to expressions generated through repeated applications of addition and multiplication operations, albeit composed under the transformed geometry. This grid demonstrates enhanced flexibility: under the conformal transformation, the images of zero lines such as $x=0$ are no longer fixed but deform into dynamic arcs. Consequently, the second grid accommodates a richer family of zero structures, allowing for the possibility of curved, branching, or nested nodal sets that vary with $\mu$, $\lambda$, or the choice of conformal framing.

Each of these grid structures can be interpreted as a geometric realization of a Cayley graph:

\begin{itemize}
\item The rectilinear grid corresponds to the Cayley graph of the Baumslag–Solitar group $BS(1,2)$, where multiplication by 2 followed by addition by 1 follows the relation $b^{-1}ab = a^2$.
\item The transformed (dual) grid corresponds to the Cayley graph of the dual Baumslag–Solitar group, where the roles of addition and multiplication are inverted.
\end{itemize}

Thus, the two grid systems exhibit duality not only in geometric terms but also in group-theoretic structure. The Möbius transformation, acting within the upper half-plane, connects these dual groups by exchanging generators and inverting flow directions, reflecting an intrinsic duality in the geometric composition of arithmetic operations.

The coexistence of these dual grid structures—one linear, one curved—linked via conformal symmetry, suggests a profound underlying symmetry in the geometry of arithmetic expressions. This duality reflects how different evaluation paths or expression embeddings can be interpreted as projections from a common, more complex geometric source. Understanding this symmetry may provide a pathway to expressing arithmetic relations via modular or automorphic structures, particularly in the context of recursive expressions and iterative identities, where Baumslag–Solitar-like behavior naturally emerges.

We conjecture that this grid duality corresponds to a deeper expression-theoretic equivalence, and that the Möbius transformation connecting them manifests an underlying arithmetic symmetry in expression geometry.

\subsection{Torsion under scale transformation}\label{subsec:gridsandtorsion}

The addition-multiplication grid introduced in Section~\ref{subsec:meshgrid} has a natural embedding in the arithmetic expression space $\mathfrak{E}_1$. This grid consists of two orthogonal families of curves:

\begin{enumerate}
\item \textbf{Addition curves} (blue lines): horizontal geodesics along which $y$ remains constant, representing iterated additions.
\item \textbf{Multiplication curves} (green lines): vertical or logarithmically scaled geodesics where the ratio $x/y$ remains constant, representing multiplicative transformations.
\end{enumerate}

This grid structure facilitates the geometric analysis of \emph{arithmetic torsion}—a quantity arising from the non-commutativity of certain additive and multiplicative expression sequences. Specifically, torsion quantifies the discrepancy between two seemingly equivalent but differently ordered expressions.

\begin{figure}[ht]
\centering
\resizebox{0.8\textwidth}{!}{\includegraphics{images/17-area-formula}}
\caption{Illustration of the correspondence between hyperbolic area and arithmetic torsion}\label{fig:area-formula}
\end{figure}

Figure~\ref{fig:area-formula} illustrates the relationship between the area enclosed by expression paths in the grid and the resulting torsion. Consider the following expression identity comparisons:

\begin{itemize}
\item One-step case:
\begin{equation}
x \times 2 + 1 - (x + 1) \times 2 = -1
\end{equation}

\item Two-step case:
\begin{equation}
    x \times 4 + 2 - (x + 2) \times 4 = -6
\end{equation}

\item Three-step case:
\begin{equation}
    x \times 8 + 3 - (x + 3) \times 8 = -21
\end{equation}\end{itemize}

These differences correspond precisely to the hyperbolic areas enclosed between alternative evaluation paths:

\begin{itemize}
\item The region $ABCD$ encompasses 1 unit cell.
\item The region $AEFG$ encompasses 6 unit cells.
\item The region $AHIJ$ encompasses 21 unit cells.
\end{itemize}
Thus, arithmetic torsion accumulates proportionally to the area enclosed by the grid paths, indicating an intrinsic connection between algebraic non-commutativity and geometric surface area.

This relationship is expressed through a differential formulation:
\begin{equation}
d\tau = \mu \lambda\, du\, dv
\end{equation}

where $d\tau$ represents the infinitesimal arithmetic torsion, and $du\, dv$ denotes the area element in the $(u, v)$ coordinate system adapted to the grid.

The analogy with curvature in differential geometry becomes evident: just as Gaussian curvature encodes deviation from flatness, arithmetic torsion quantifies deviation from commutativity in arithmetic flow. In this sense, torsion constitutes a measure of the operational significance of evaluation order.

The $\mathfrak{E}_1$ space thus provides a mathematical framework where algebraic non-commutativity manifests as measurable geometric distortion—establishing a novel interpretation of arithmetic structure as a form of discrete curvature. This opens avenues for investigating further geometric invariants such as torsion density, torsion-induced flow bifurcation, and potentially a Gauss–Bonnet-type integral identity for arithmetic surfaces.


\subsection{Tube structure}\label{sec:tubestructure}

In preceding sections, we introduced the expression space $\mathfrak{E}_1$ as a geometric realization of arithmetic flow under fixed generator parameters $\mu$ and $\lambda$. However, more complex structures emerge when we consider the family of all such spaces parameterized by $\lambda$ (or both parameters) and analyze the behavior of expressions across this family. This leads naturally to the concept of a \emph{tube structure}.

\subsubsection{From slices to parametric families}

Each individual $\mathfrak{E}_1$ space may be conceptualized as a single "slice" within a parameterized family of expression spaces indexed by $\lambda$. Within each slice, arithmetic expressions are realized as geodesic paths, their evaluations governed by the scalar field $a$, and their flow by the local metric tensor.

Consider fixing an expression—for example, a polynomial of the form $P(x)$—and examining how its geometric representation evolves as $\lambda$ varies. This defines a \emph{fiber} that traverses through the slices of $\mathfrak{E}_1$ spaces, delineating a trajectory across a higher-dimensional space. The totality of such fibers over all values of $\lambda$ constitutes a new structure: a \emph{tube}.

\subsubsection{Tube structure as total space}

We define a \textbf{tube structure} $\mathcal{T}$ as the total space formed by the disjoint union of a continuous family of $\mathfrak{E}_1$ spaces:
\begin{equation}
\mathcal{T} = \bigsqcup_{\lambda > 0} \mathfrak{E}_1^{(\lambda)}
\end{equation}
endowed with a topology and bundle-like structure facilitating coherent traversal along the $\lambda$-direction.

In this structure:
\begin{itemize}
\item The base space is the parameter domain $\Lambda$ (e.g., $\mathbb{R}^+$).
\item Each fiber over $\lambda$ is a geometric expression space $\mathfrak{E}_1^{(\lambda)}$.
\item Expression paths can be lifted to continuous trajectories across fibers.
\item Polynomials $P(x)$, viewed as expressions in $\mathfrak{E}_1$, trace canonical sections through $\mathcal{T}$.
\end{itemize}

\subsubsection{Zero surfaces and nodal evolution}

A primary motivation for examining tube structures is to investigate how \emph{zero lines}—loci where an expression evaluates to zero—evolve across $\lambda$. In a single $\mathfrak{E}_1$ slice, these constitute 1-dimensional curves. As $\lambda$ varies, their union forms a 2-dimensional surface within $\mathcal{T}$, which we designate as a \emph{zero surface}.

These zero surfaces may exhibit several notable phenomena:
\begin{itemize}
\item \textbf{Bifurcation}: new zero lines may emerge or merge as $\lambda$ undergoes variation.
\item \textbf{Branching}: certain expressions may exhibit multi-valued zero loci across $\lambda$.
\item \textbf{Topology change}: zero surfaces may develop handles, pinch points, or singularities.
\end{itemize}

This perspective enables sophisticated analysis of expression dynamics—particularly when considering a family of expressions or differential equations involving $\lambda$.

\subsubsection{Canonical polynomials as fibers}

In practical applications, we frequently analyze a fixed expression such as a polynomial $P(x)$, evaluating it at $x=e^\lambda$. This yields a one-parameter family of values:
\begin{equation}
P(e^\lambda), \quad \lambda \in \mathbb{R}
\end{equation}
Each evaluation corresponds to a point in a fiber $\mathfrak{E}_1^{(\lambda)}$, and the totality defines a canonical section in the tube structure.

More generally, expressions involving both $x$ and $\lambda$ naturally trace \emph{embedded submanifolds} within $\mathcal{T}$. These can be utilized to analyze asymptotic behavior, resonance patterns, or nodal crossings across a family of arithmetic flows.

The tube structure formalism facilitates multiple investigative approaches:

\begin{itemize}
\item Examining global properties of zero surfaces (e.g., genus, curvature concentration)
\item Constructing flow equations across $\lambda$-families, potentially defining connections or transport laws
\item Defining moduli of expression geometries as structured bundles over parameter spaces
\end{itemize}

Ultimately, tube structures provide a mathematical framework in which arithmetic expression dynamics can be analyzed analogously to field theory, with expressions acting as structured sections and torsion/curvature defining local invariants.

\newpage

\section{Fundamental domains in $\mathfrak{K}_1$}\label{sec:fundamentaldomains}

In this section, we will explore the fundamental domain in $\mathfrak{K}_1$ space.


\newpage

\section{On order-4 apeirogonal tiling}\label{sec:tiling}

In the preceding section, we introduced the Accumulative Commutative Space (ACS) as an abstract parameter plane essential for quantifying global arithmetic torsion. While powerful for global analysis, the ACS lacks an intrinsic mechanism to describe the \textit{local}, non-commutative dynamics of arithmetic operations.

This section bridges that gap by endowing the ACS with a rich differential structure. We construct a three-dimensional \textbf{Arithmetic Contact Manifold}, whose two-dimensional base is precisely a local, differential version of the ACS. This framework provides the geometric engine for the arithmetic flow, unifying the global, commutative picture of torsion with the local, non-commutative nature of operations. It is here that the flow equation from Section~\ref{sec:flow_equation} finds its most natural home, and we develop the differential calculus necessary for its study.

\subsection{Core definitions and the contact structure}\label{subsec:core_definitions}

We begin by establishing our geometric space. Consider a 3-dimensional manifold with coordinates $(u,v,a) \in \mathbb{R}^3$, where the base plane spanned by $(u,v)$ is understood as the differential realization of the ACS. The geometry is built upon two fundamental 1-forms, defined using constant real parameters $\mu$ and $\lambda$:
\begin{equation}\label{eq:contact_forms_def}
\omega := \mu\,du + \lambda a\,dv, \qquad \alpha := da - \omega.
\end{equation}
The 1-form $\alpha$ is central to our construction. Its primary role is to define a ``horizontal'' plane at each point, known as the contact distribution. To see how it works, let's consider its action on an arbitrary vector field $X = x_u \partial_u + x_v \partial_v + x_a \partial_a$. Recalling that $da(X) = x_a$, $du(X) = x_u$, and $dv(X) = x_v$, the action is given by:
\[
\alpha(X) = da(X) - \mu\,du(X) - \lambda a\,dv(X) = x_a - \mu x_u - \lambda a x_v.
\]
The set of all vectors $X$ for which $\alpha(X)=0$ constitutes this horizontal plane.

Throughout this work, we use two types of differentials in parallel: (i) the standard de Rham exterior derivative $d$, which is always nilpotent ($d^2=0$); and (ii) the \textbf{expression differential} $\delta$, which is axiomatically defined in Section~\ref{ch:differential_calculus}.

\paragraph{The contact property.}
A key property of $\alpha$ is that it defines a contact structure on our 3D space. This is verified by checking if the 3-form $\alpha \wedge d\alpha$ is a volume form (i.e., non-zero everywhere). First, we compute the exterior derivative of $\omega$:
\[
d\omega = d(\mu\,du + \lambda a\,dv) = \lambda\,da \wedge dv.
\]
Then, the exterior derivative of $\alpha$ is simply $d\alpha = d(da - \omega) = -d\omega = -\lambda\,da \wedge dv$. Now, we can compute the wedge product:
\begin{align*}
\alpha \wedge d\alpha &= (da - \omega) \wedge (-\lambda\,da \wedge dv) \\
&= -da \wedge (\lambda\,da \wedge dv) + \omega \wedge (\lambda\,da \wedge dv) \\
&= 0 + (\mu\,du + \lambda a\,dv) \wedge (\lambda\,da \wedge dv) \\
&= \mu\lambda\,du \wedge da \wedge dv + \lambda^2 a\,\underbrace{dv \wedge da \wedge dv}_{=0} \\
&= \mu\lambda\,du \wedge da \wedge dv.
\end{align*}
Provided $\mu\lambda \neq 0$, this is a volume form, confirming that $\alpha$ is a \textbf{contact form}.

\paragraph{Normal form, Reeb field, and contact distribution.}
By introducing natural units $\tilde{u} = \mu u$ and $\tilde{v} = \lambda v$, the form $\alpha$ is reduced to its canonical form $\alpha_0 = da - d\tilde{u} - a\,d\tilde{v}$, which facilitates comparison with standard literature.

The Reeb vector field $R$, defined by $i_R d\alpha = 0$ and $\alpha(R) = 1$, is $R = -(1/\mu)\partial_u$. The contact distribution $\mathcal{H}$, as introduced earlier, is formally the kernel of $\alpha$:
\[
\mathcal{H} := \ker\alpha = \{X \in TM : \alpha(X) = 0\}.
\]
Viewing the tangent bundle as a composition of the $(u,v)$-base and the $a$-fiber, the form $\alpha = da - \omega$ intrinsically links the ``vertical'' change $da$ to the ``horizontal'' displacement defined by $\omega$.

\subsection{The geometry on the contact distribution \texorpdfstring{$\ker\alpha$}{ker(alpha)}} % Revised title for 6.2
\label{subsec:geometry_on_ker_alpha}

We now define two special vector fields that form a basis for the horizontal plane $\mathcal{H}$ at every point. These are the horizontal lifts of the base coordinate vectors, which we term the \textbf{expression directional derivatives}:
\begin{equation}\label{eq:directional_derivatives}
D_u := \partial_u + \mu\,\partial_a, \qquad D_v := \partial_v + \lambda a\,\partial_a.
\end{equation}
These fields are constructed specifically to be horizontal, a fact we can verify directly. For $D_u$, its coordinate components are $(x_u, x_v, x_a) = (1, 0, \mu)$. Applying the formula for $\alpha(X)$:
\[
\alpha(D_u) = x_a - \mu x_u - \lambda a x_v = \mu - \mu(1) - \lambda a(0) = 0.
\]
For $D_v$, its components are $(x_u, x_v, x_a) = (0, 1, \lambda a)$. Applying $\alpha$:
\[
\alpha(D_v) = x_a - \mu x_u - \lambda a x_v = \lambda a - \mu(0) - \lambda a(1) = 0.
\]
Since both $D_u$ and $D_v$ are annihilated by $\alpha$, and they are clearly linearly independent, they form a basis for the 2-dimensional contact distribution:
\[
\ker\alpha = \text{span}\{D_u, D_v\}.
\]

For any smooth scalar field $F(u,v,a)$, we define its expression differential $\delta F$ and the directional derivative $D_\theta$ as:
\begin{equation}\label{eq:3}\tag{3}
\delta F := (D_uF)\,du + (D_vF)\,dv, \qquad D_\theta := \cos\theta\,D_u + \sin\theta\,D_v.
\end{equation}
This construction effectively reduces the geometry from the three-dimensional space $(u,v,a)$ to the two-dimensional contact distribution $\mathcal{H}$.

\paragraph{Structural relation between d and $\delta$.}
For any scalar field $F$, the two differentials are related by a fundamental identity:
\[
\boxed{ \delta F = dF - (\partial_a F)\,\alpha }
\]
This identity can be interpreted as projecting $dF$ onto the horizontal distribution $\mathcal{H}$ by subtracting its vertical component along $\alpha$. Expanding this definition yields:
\[
\delta F = (F_u + \mu F_a)\,du + (F_v + \lambda a F_a)\,dv,
\]
which is consistent with Eq.~\eqref{eq:15}. In particular, we recover $\delta a = \omega$, $\delta u = du$, and $\delta v = dv$.

\subsection{Legendrian flow and rectification}

A curve $\gamma(s) = (u(s), v(s), a(s))$ is \textbf{Legendrian} if its tangent vector lies in the contact distribution, i.e., $\dot\gamma(s) \in \ker\alpha$. For such a curve, the evolution of $a$ is governed by the \textbf{flow equation}:
\begin{equation}\label{eq:4}\tag{4}
\frac{da}{ds} = D_\theta a = \mu\cos\theta + \lambda a\sin\theta,
\end{equation}
where $\theta$ parameterizes the angle of the tangent vector in the basis $\{D_u,D_v\}$. With respect to a given metric on the base manifold, this flow equation can be written in its Eikonal form:
\begin{equation}\label{eq:5}\tag{5}
\|\nabla a\| = \sqrt{\mu^2 + \lambda^2 a^2}.
\end{equation}
We introduce a \textbf{rectifying variable} $y$:
\begin{equation}\label{eq:6}\tag{6}
y = \arcsin\left(\frac{\lambda a}{\mu}\right) \quad\Rightarrow\quad \|\nabla y\| = \lambda.
\end{equation}
This rectification transforms the non-linear velocity field for $a$ into a constant-speed flow for $y$, which is advantageous for geometric constructions and enhances numerical stability.

\paragraph{Non-commutativity and curvature.}
The commutator of the horizontal vector fields yields a purely vertical vector, reflecting the "curvature" of the contact distribution. This phenomenon can be described as a "vertical return":
\begin{equation}\label{eq:7}\tag{7}
[D_u,D_v] = \mu\lambda\,\partial_a,\qquad \delta^2F = \mu\lambda(\partial_a F)\,du\wedge dv,\qquad \delta^2 a = \mu\lambda\,du\wedge dv.
\end{equation}
The circulation-area formula provides a tool for quantifying mesh singularities and global topological constraints:
\begin{equation}\label{eq:8}\tag{8}
\oint_{\partial\Sigma}\omega = \iint_\Sigma d\omega = \mu\lambda\iint_\Sigma du\wedge dv.
\end{equation}

\paragraph{Compatibility with de Rham cohomology.}
The exterior derivative of $\omega$ is $d\omega = \lambda\,da\wedge dv$. By pulling this 2-form back to a section where $\alpha=0$ (i.e., substituting $da=\omega$), we obtain:
\[
(d\omega)^* = \lambda\,\omega\wedge dv = \mu\lambda\,du\wedge dv,
\]
which is identical to $\delta^2 a$. This demonstrates the precise collaborative relationship between $d$ and $\delta$.

\paragraph{Example: Basis Flows.}
Flowing along $D_u$ implies $\dot{a} = \mu \Rightarrow a(s) = a_0 + \mu s$. Flowing along $D_v$ implies $\dot{a} = \lambda a \Rightarrow a(s) = a_0 e^{\lambda s}$. Linear combinations of these generate the general flow given by Eq.~\eqref{eq:4}.

\subsection{Classification and Significance of the AEG Lie Algebra}

The contact geometry framework developed in the preceding sections, particularly the horizontal vector fields \(D_u\) and \(D_v\), does more than provide a dynamical system for the arithmetic flow. Its algebraic structure is of fundamental importance. The vector fields, under the operation of the Lie bracket, form a Lie algebra. Classifying this algebra reveals the intrinsic nature of the structure generated by the interplay of addition and multiplication.

Let us define a basis for this Lie algebra, \(\mathfrak{g}\), using the vector fields central to our construction:
\begin{itemize}
    \item \(e_1 = D_u = \partial_u + \mu\partial_a\)
    \item \(e_2 = D_v = \partial_v + \lambda a \partial_a\)
    \item \(e_3 = \partial_a\)
\end{itemize}
From direct computation, their commutation relations are given by:
\begin{align}
    [e_1, e_2] &= \mu\lambda e_3 \\
    [e_1, e_3] &= [\partial_u + \mu\partial_a, \partial_a] = 0 \\
    [e_2, e_3] &= [\partial_v + \lambda a \partial_a, \partial_a] = -\lambda e_3
\end{align}

\subsubsection*{Verification of the Lie Algebra Structure}
For \(\mathfrak{g}\) to be a valid Lie algebra, the Jacobi identity, \([X, [Y, Z]] + [Y, [Z, X]] + [Z, [X, Y]] = 0\), must hold for all \(X, Y, Z \in \mathfrak{g}\). It is sufficient to verify this for the basis vectors \((e_1, e_2, e_3)\).

\begin{align*}
    [e_1, [e_2, e_3]] + [e_2, [e_3, e_1]] + [e_3, [e_1, e_2]] &= [e_1, -\lambda e_3] + [e_2, -[e_1, e_3]] + [e_3, \mu\lambda e_3] \\
    &= -\lambda [e_1, e_3] - [e_2, 0] + \mu\lambda [e_3, e_3] \\
    &= -\lambda(0) - 0 + \mu\lambda(0) \\
    &= 0
\end{align*}
The identity holds. This rigorously confirms that the kinematic structure of Arithmetic Expression Geometry is governed by a bona fide Lie algebra.

\subsubsection*{Classification}
To determine the position of this Lie algebra within the standard classification system, we examine its derived series and lower central series.

\paragraph{Solvability:} The derived series \(\mathcal{D}^i\mathfrak{g}\) is defined by \(\mathcal{D}^0\mathfrak{g} = \mathfrak{g}\) and \(\mathcal{D}^{i+1}\mathfrak{g} = [\mathcal{D}^i\mathfrak{g}, \mathcal{D}^i\mathfrak{g}]\).
\begin{itemize}
    \item \(\mathcal{D}^1\mathfrak{g} = [\mathfrak{g}, \mathfrak{g}] = \text{span}\{[e_1, e_2], [e_1, e_3], [e_2, e_3]\} = \text{span}\{e_3\}\).
    \item \(\mathcal{D}^2\mathfrak{g} = [\mathcal{D}^1\mathfrak{g}, \mathcal{D}^1\mathfrak{g}] = [\text{span}\{e_3\}, \text{span}\{e_3\}] = \{0\}\).
\end{itemize}
Since the derived series terminates at zero, \textbf{the Lie algebra is solvable}. This reflects a hierarchical nature of non-commutativity within the system.

\paragraph{Nilpotency:} The lower central series \(\mathcal{L}^i\mathfrak{g}\) is defined by \(\mathcal{L}^0\mathfrak{g} = \mathfrak{g}\) and \(\mathcal{L}^{i+1}\mathfrak{g} = [\mathfrak{g}, \mathcal{L}^i\mathfrak{g}]\).
\begin{itemize}
    \item \(\mathcal{L}^1\mathfrak{g} = [\mathfrak{g}, \mathfrak{g}] = \text{span}\{e_3\}\).
    \item \(\mathcal{L}^2\mathfrak{g} = [\mathfrak{g}, \mathcal{L}^1\mathfrak{g}] = [\mathfrak{g}, \text{span}\{e_3\}]\). This contains the non-zero bracket \([e_2, e_3] = -\lambda e_3\), thus \(\mathcal{L}^2\mathfrak{g} = \text{span}\{e_3\}\).
\end{itemize}
Since \(\mathcal{L}^2\mathfrak{g} = \mathcal{L}^1\mathfrak{g} \neq \{0\}\), the series becomes stable and never reaches zero. Therefore, \textbf{the Lie algebra is non-nilpotent}.

\subsubsection*{Significance}
The classification of the AEG algebra as a \textbf{three-dimensional, solvable, non-nilpotent real Lie algebra} is profoundly significant. The non-nilpotent character stems directly from the commutation relation \([D_v, \partial_a] = -\lambda\partial_a\), which reveals that the multiplicative flow \(D_v\) acts upon the emergent `value' field \(\partial_a\) and reproduces it. This self-referential action, inherent to scaling, contrasts sharply with the additive flow \(D_u\), whose `translation-invariant' nature is captured by \([D_u, \partial_a] = 0\).

This algebraic structure is closely related to the Lie algebra of the \textbf{1D affine group}, \(\mathfrak{aff}(1)\), which governs translations and scalings of the real line and serves as a prototype for solvable, non-nilpotent Lie algebras. The AEG algebra can thus be understood as a richer, three-dimensional realization of this fundamental affine structure, with \(D_u\) playing the role of translation and \(D_v\) a value-dependent scaling. This algebraic positioning solidifies the structural legitimacy of the AEG framework and opens avenues for leveraging the well-developed theory of solvable Lie algebras to further explore the geometry of arithmetic.


\newpage

\section{The second and third kind of arithmetic expression space $\mathfrak{K}_2$ and $\mathfrak{K}_3$}\label{sec:morekinds}

\subsection{Axiomatic Definition and Basic Formulas}

\paragraph{A1 (Definition of the Expression Differential)}
We define the action of $\delta$ on the coordinate functions as:
\begin{equation}\label{eq:9}\tag{9}
\delta a=\omega,\quad \delta u=du,\quad \delta v=dv,
\end{equation}
and extend its action to all scalar fields $F$ by linearity and the Leibniz rule.

\paragraph{A2 (Directional Derivatives)} The operators $D_u, D_v$ are defined as in Eq.~\eqref{eq:2}, with directional synthesis given by Eq.~\eqref{eq:3}.

\paragraph{A3–A4 (Chain Rules)}
For univariate and bivariate compositions, $\Phi(a)$ and $F(E_1, E_2)$, the following chain rules hold:
\begin{equation}\label{eq:10-11}\tag{10-11}
\delta\Phi(a)=\Phi'(a)\,\omega,\qquad
\delta F=\partial_1F\,\delta E_1+\partial_2F\,\delta E_2.
\end{equation}
These are consistent with the equivalent definition $\delta F=dF-(\partial_aF)\alpha$.

\paragraph{T1–T5 (Fundamental Theorems)}
The axiomatic system yields five fundamental theorems:
\begin{itemize}
    \item[T1] \textbf{Flow Equation:} Setting $F=a$ in the definition of $\delta$ immediately gives Eq.~\eqref{eq:4}.
    \item[T2] \textbf{Non-Commutativity:} $[D_u,D_v]=\mu\lambda\,\partial_a$.
    \item[T3] \textbf{Curvature / Covariant Second Differential:} $\delta^2F=\mu\lambda(\partial_aF)\,du\wedge dv$.
    \item[T4] \textbf{Compatibility with de Rham:} $(d\omega)^*=\mu\lambda\,du\wedge dv$ on the section $\alpha=0$.
    \item[T5] \textbf{Circulation-Area Theorem:} As stated in Eq.~\eqref{eq:8}.
\end{itemize}

\subsection{Quick Reference and Computation Rules}
Let $\omega=\mu\,du+\lambda a\,dv$. The following rules apply:
\begin{gather}
\delta(a^n)=n a^{n-1}\omega\ (n\in\mathbb{Z}),\quad
\delta(\ln a)=\frac{\omega}{a}\ (a>0),\quad
\delta(e^a)=e^a\,\omega, \label{eq:13} \tag{13} \\
\delta(\sin a)=\cos a\,\omega,\quad \delta(\cos a)=-\sin a\,\omega,\quad
\delta(\arcsin a)=\frac{\omega}{\sqrt{1+a^2}}. \label{eq:14} \tag{14}
\end{gather}
If $E=E(u,v,a)$, its expression derivatives and differential are:
\begin{equation}\label{eq:15}\tag{15}
D_uE=E_u+\mu E_a,\quad D_vE=E_v+\lambda a E_a,\quad
\delta E=(D_uE)\,du+(D_vE)\,dv.
\end{equation}
These identities provide a unified interface for automatic differentiation, optimization, and geometric modeling.

\subsection{Geometric Interpretation: An Ehresmann Connection}
If we view the projection $\pi:(u,v,a)\mapsto(u,v)$ as a fiber bundle, the condition $\alpha=0$ defines a horizontal distribution $\mathcal{H}$. The operators $D_u,D_v$ are the horizontal lifts of the base vector fields $\partial_u, \partial_v$. The curvature 2-form of this connection corresponds to $\delta^2 a$. This perspective rigorously explains why applying $\delta$ is equivalent to restricting $d$ to $\ker\alpha$ and reveals the geometric meaning of the non-commutativity factor $\mu\lambda$.

\subsection{Numerics and Scaling: Natural Units and Preconditioning}
In natural units where $\tilde u=\mu u$ and $\tilde v=\lambda v$, all formulas retain their form. For numerical optimization, working with the rectified variable $y$ (or, equivalently, preconditioning the gradients for $a$ with the pointwise factor $(\mu^2+\lambda^2 a^2)^{-1}$) can significantly smooth the loss landscape and improve convergence stability.

\subsection{Extending polynomials: the affine–Appell basis}
  Natural units: $\tilde u=\mu u,\ \tilde v=\lambda v$. Define scaled powers
  \[
    B_n(a,v):=e^{-n\tilde v}a^n\quad(n\in\mathbb{N}).
  \]
  Let
  \[
    \mathcal{B}:=\Big\{\sum_{n=0}^N P_n(u,v,e^{\tilde v})\,B_n(a,v)\ \text{(finite)}\Big\}.
  \]
  \textbf{Closure Theorem.} $\mathcal{B}$ is closed under the mixed calculus generated by
  \[
    D_u=\partial_u+\mu\,\partial_a,\qquad D_v=\partial_v+\lambda a\,\partial_a.
  \]
  Rules:
  \[
    D_u(PB_n)=(\partial_u P)B_n+\mu n\,e^{-\tilde v}P\,B_{n-1},\quad
    D_v(PB_n)=(\partial_v P)B_n.
  \]

\subsection{Antiderivatives and a finite upward sweep}
  If $P$ is independent of $u$,
  \[
    D_u^{-1}\!\big(P(v,e^{\tilde v})\,B_n\big)=\frac{e^{\tilde v}P(v,e^{\tilde v})}{\mu(n+1)}\,B_{n+1}.
  \]
  In general, define $Q^{(0)}_n=\dfrac{e^{\tilde v}}{\mu(n+1)}P_n$ and set $G^{(0)}=\sum_n Q^{(0)}_nB_{n+1}$.
  Then $F-\!D_uG^{(0)}=-\sum_n(\partial_uQ^{(0)}_n)B_{n+1}$. Repeat for finitely many steps.
  \vspace{0.6em}

  \textbf{Example.} For $F=a^3e^{\lambda v}=e^{4\tilde v}B_3$,
  \[
    D_u^{-1}F=\frac{e^{\tilde v}}{4\mu}B_4=\frac{e^{\lambda v}}{4\mu}a^4,\quad
    D_u\!\left(\frac{e^{\lambda v}}{4\mu}a^4\right)=a^3e^{\lambda v}.
  \]

\newpage

\section{Topological arithmetic expression space}\label{sec:topology}


\subsection{The construction of the grid $G_0$}\label{sec:construction-of-grids}

To construct the grid described in subsection \ref{sec:meshgrid} and Figure \ref{fig:gridex0},
we will use the number theory decomposition introduced by Victor Pambuccian in \cite{Pambuccian2016THEAO} and
Celia Schacht in \cite{Schacht2018ANOTHERAO}.

$$
n = \tau(n) \omega(n)
$$

where $\tau(n)$ is a power of 2 and $\omega(n)$ is an odd.

We can see directly from the Figure \ref{fig:gridex0} that the grid is constructed by the following rules:
\begin{itemize}
\item horizantal lines (blue, additional lines) satisfying $y = 2^k, k \in \mathbb{Z}$
\item vertical lines (green, multiplicative lines) satisfying
    \begin{itemize}
        \item the value of x satisfying $x = \frac{m}{2^l}, l \in \mathbb{Z}^+, m \in \mathbb{Z}$
        \item the assignment begin from $\omega(-m)$ and increase exponentially by power 2.
    \end{itemize}
\end{itemize}

\begin{figure}[ht]
\centering
\resizebox{0.9\textwidth}{!}{\includegraphics{images/07-grid-detail}}
\caption{$\omega$ gives the assignment at the start points of vertical lines in $G_0$}\label{fig:griddetail}
\end{figure}

The horizontal lines and the vertical lines divide the whole space into a mesh grid $G_0 = (V_0, E_0, F_0)$, where
$V_0$ is the set of crossing points, $E_0$ is the set of edges (segments in the horizontal and vertical lines,
it should be noted that only the vertical lines are geodesics, while the horizontal lines are horocycles) connecting
the crossing points, and $F_0$ is the set of cut cells. This mesh grid is generated by the additional generator $1$ and
the multiplicative generator $2$, and $V_0$, $E_0$, and $F_0$ are all countable sets.

We illustrate the construction schema of $G_0$ in Figure \ref{fig:gridschema}.

\begin{figure}[ht]
\centering
\resizebox{0.9\textwidth}{!}{\includegraphics{images/08-grid-schema}}
\caption{the construction schema of $G_0$}\label{fig:gridschema}
\end{figure}

We notice that $G_0$ is very regular, in fact, all edges are equidistant.

\begin{lemma}\label{lem:regular}
All edges in $G_0$ are equidistant.
\end{lemma}

\begin{proof}
Following the construction schema in Figure \ref{fig:gridschema}, we can calculate the length of segments are all equidistant.

Length of $AC$ and $BD$:

$$
\int ds = \int\limits_{2^{k}}^{2^{k+1}} \frac{1}{y \ln 2} dy = \frac{1}{\ln 2} \left( \ln 2^{k+1} - \ln 2^k \right) = 1
$$

Length of $AB$:

$$
\int ds = \int\limits_{-2^k(\omega + 1)}^{-2^k(\omega - 1)} \frac{1}{2^{k+1}} dx = \frac{1}{2^{k+1}} \left( 2^{k + 1} \right) = 1
$$

Length of $CE$:

$$
\int ds = \int\limits_{-2^k(\omega + 1)}^{-2^k \omega} \frac{1}{2^k} dx = \frac{1}{2^k} \left( 2^k \right) = 1
$$

Length of $ED$:

$$
\int ds = \int\limits_{-2^k \omega}^{-2^k(\omega - 1)} \frac{1}{2^k} dx = \frac{1}{2^k} \left( 2^k \right) = 1
$$
\end{proof}

\subsection{The construction of the grid $G_1$}\label{sec:construction-of-grids}

We can similarly construct the grid $G_1$ using the additional generator $\frac{1}{2}$ and the multiplicative generator
$\sqrt{2}$, the grid $G_2$ using the additional generator $\frac{1}{4}$ and the multiplicative generator $\sqrt[4]{2}$,
and so on. And each time the cell of the mesh grid is divided into smaller cells and the end points of the vertical lines
move upward.

\begin{figure}[ht]
\centering
\resizebox{0.9\textwidth}{!}{\includegraphics{images/09-grid-schema-g1}}
\caption{the construction schema of $G_1$}\label{fig:gridschemag1}
\end{figure}

\subsection{The construction of the grid $G_2$}\label{sec:construction-of-grids}


\subsection{The grid mesh is dense}\label{sec:construction-of-grids}

It is easy to see that there is a chain of inclusion relations:
$$
V_0 \subset V_1 \subset V_2 \subset \cdots V_i \subset \cdots
$$

Suppose $V = \bigcup_{i=1}^{\infty} V_i$, we have below lemma

\begin{lemma}
$V$ is a countable dense set.
\end{lemma}

\begin{proof}
Because $V_i$ is countable, and the union is over a countable index set, so $V$ is countable.

We can prove it is dense by contradiction. Suppose $V$ is not a dense set. Then there is a point $p$ in the space
neither belongs to $V$ nor is a limit point of $V$.

TODO...

\end{proof}

\subsection{Completeness and topology}\label{sec:topdef}

\subsection{As a special integral}\label{sec:integral}

\newpage

\newpage

\section{From arithmetic torsion to curvature}\label{sec:curvature}

from homotopy to homology

\begin{itemize}
\item 0-form: assignment a
\item 1-form: path elements
\item 2-form: area elements
\end{itemize}

Poincare duality on the meshgrid

\newpage

\section{Tube structure, complexification and fibration}\label{sec:morekinds}


\subsection{On integral theory}\label{sec:integral}

\subsection{On representation of function}\label{sec:function}

\subsection{Questions related with complexity?}\label{sec:complexity}

\newpage

\section{General discussion}\label{sec:function}


\subsection{Foundation questions}\label{sec:foundation}

\subsection{Classification problem}\label{sec:donaghey}

\subsection{Eigenfunction problem}\label{sec:eigenfunction}

\subsection{Tube structure}\label{sec:tube}

\subsection{Singular points and divergent series}\label{sec:singularity}

\subsection{Function and a new calculus?}\label{sec:caculus}

\subsection{Category of function theories?}\label{sec:function}

\subsection{Geometrilization of computation}\label{sec:computation}

\subsection{Geometrilization of logic}\label{sec:logic}
\subsubsection{irrationality of $\sqrt{17}$}

\newpage

\section{A glossary of unsolved problems}\label{sec:problems}


We can also get a direct formal solution of the flow equation (\eqref{eq:flow}) step by step:

$$
    \frac{da}{\mu \cos \theta + a \lambda \sin \theta} = ds
$$

$$
    \frac{1}{\lambda \sin \theta} \frac{d(\mu \cos \theta + a \lambda \sin \theta)}{\mu \cos \theta + a \lambda \sin \theta} = ds
$$

$$
    \frac{1}{\lambda \sin \theta} ln(\mu \cos \theta + a \lambda \sin \theta) = s + C
$$

$$
    \mu \cos \theta + a \lambda \sin \theta = e^{\lambda s \sin \theta} e^{C \lambda \sin \theta}
$$

Considering the initial condition
$$
    \mu \cos \theta + a_0 \lambda \sin \theta = e^{C \lambda \sin \theta}
$$

We have
$$
    \mu \cos \theta + a \lambda \sin \theta = e^{\lambda s \sin \theta} (\mu \cos \theta + a_0 \lambda \sin \theta)
$$

$$
   a = \frac{\mu \cos \theta + a_0 \lambda \sin \theta}{\lambda \sin \theta} e^{\lambda s \sin \theta} - \frac{\mu}{\lambda}\cot \theta
$$

$$
   a = (a_0 + \frac{\mu}{\lambda} \cot \theta) e^{\lambda s \sin \theta} - \frac{\mu}{\lambda} \cot \theta
$$

\begin{equation}
   a =  a_0 e^{\lambda s \sin \theta} + \frac{\mu}{\lambda} (e^{\lambda s \sin \theta} - 1) \cot \theta
\end{equation}

\begin{equation}\label{eq:directformalsolution}
   a =  a_0 e^{\lambda s \sin \theta} + \frac{\mu}{\lambda} (e^{\lambda s \sin \theta} - 1) \cot \theta
\end{equation}


\newpage

\bibliographystyle{plain}
\bibliography{aeg-paper.bib}

\appendix

\section{A direct formal solution of the flow equation}\label{sec:directformalsolution}


In order to verify the conformance, we expand the formula \eqref{eq:directformalsolution} in the following way:

\begin{equation}
   a =  a_0 e^{\lambda s \sin \theta} + \frac{\mu}{\lambda} [1 + \lambda s \sin \theta + \frac{1}{2!} (\lambda s \sin \theta)^2  + \frac{1}{3!} (\lambda s \sin \theta)^3 + \cdots - 1] \cot \theta
\end{equation}

\begin{equation}
   a = a_0 e^{\lambda s \sin \theta} + \mu s \cos \theta + \frac{\mu}{\lambda} \sin \theta \cos \theta (\frac{\lambda^2s^2}{2!} + \frac{\lambda^3s^3}{3!} \sin \theta + \frac{\lambda^4s^4}{4!} \sin^2 \theta + \cdots)
\end{equation}

\begin{equation}
   a = a_0 e^{\lambda s \sin \theta} + \mu s \cos \theta + \frac{\mu}{2\lambda} \sin 2\theta (\frac{\lambda^2s^2}{2!} + \frac{\lambda^3s^3}{3!} \sin \theta + \frac{\lambda^4s^4}{4!} \sin^2 \theta + \cdots)
\end{equation}

\begin{equation}
   a = a_0 e^{\lambda s \sin \theta} + \mu s \cos \theta + \frac{\mu}{2\lambda} \Psi(s) \sin 2\theta
\end{equation}

When $\theta = \frac{k \pi}{2}, k = 0, 1, 2, 3\cdots, s = 0, 1, 2, 3\cdots$, we have

\begin{equation}
    a = a_0 e^{\lambda s \sin \theta} + \mu s \cos \theta
\end{equation}

Especially, we have

\begin{equation}
    a = a_0 + \mu s, s = 0, 1, 2, 3\cdots, k = 0, 1, 2, 3\cdots, \theta = 2k\pi
\end{equation}

\begin{equation}
    a = x_0e^{\lambda s}, s = 0, 1, 2, 3\cdots, k = 0, 1, 2, 3\cdots, \theta = 2k\pi + \frac{\pi}{2}
\end{equation}

\begin{equation}
    a = a_0 - \mu s, s = 0, 1, 2, 3\cdots, k = 0, 1, 2, 3\cdots, \theta = 2k\pi + \pi
\end{equation}

\begin{equation}
    a = a_0 e^{- \lambda s}, s = 0, 1, 2, 3\cdots, k = 0, 1, 2, 3\cdots, \theta = 2k\pi + \frac{3 \pi}{2}
\end{equation}

which gives the conformance.


\newpage

\section{Conformance between infinitesimal generating process and discrete mesh grid}\label{sec:conformance}


\subsection{LISP and combinators}\label{sec:donaghey}

\subsection{Applicative and concatenative}\label{sec:donaghey}

\subsection{Donaghey transformation}\label{sec:donaghey}

\newpage

\newpage

\section{Arithmetic expression, combinators and transformation over trees}\label{sec:expressions}


\subsection{Curvature}\label{sec:curvature}

\subsection{Laplacian}\label{sec:laplacian}

\newpage

\newpage

\end{document}
