\documentclass{article}

\usepackage{styles/arxiv}

\usepackage[utf8]{inputenc} % allow utf-8 input
\usepackage[T1]{fontenc}    % use 8-bit T1 fonts
\usepackage{hyperref}       % hyperlinks
\usepackage{url}            % simple URL typesetting
\usepackage{booktabs}       % professional-quality tables
\usepackage[english]{babel}
\usepackage{nicefrac}       % compact symbols for 1/2, etc.
\usepackage{microtype}      % microtypography
\usepackage{graphicx}
\usepackage{stmaryrd}

\usepackage{amsthm}
\usepackage{amssymb}
\usepackage{amsfonts}
\usepackage{amsmath}
\usepackage{mathtools}

\usepackage{tikz}
\usepackage{qtree}

\newtheorem{definition}{Definition}
\numberwithin{definition}{section}
\newtheorem{lemma}{Lemma}
\numberwithin{lemma}{section}
\newtheorem{proposition}{Proposition}
\numberwithin{proposition}{section}
\newtheorem{corollary}{Corollary}
\numberwithin{corollary}{section}
\newtheorem{theorem}{Theorem}
\numberwithin{theorem}{section}

\DeclareMathSymbol{\mathinvertedexclamationmark}{\mathclose}{operators}{'074}
\DeclareMathSymbol{\mathexclamationmark}{\mathclose}{operators}{'041}
\makeatletter
\newcommand{\raisedmathinvertedexclamationmark}{%
  \mathclose{\mathpalette\raised@mathinvertedexclamationmark\relax}%
}
\newcommand{\raised@mathinvertedexclamationmark}[2]{%
  \raisebox{\depth}{$\m@th#1\mathinvertedexclamationmark$}%
}
\begingroup\lccode`~=`! \lowercase{\endgroup
  \def~}{\@ifnextchar`{\raisedmathinvertedexclamationmark\@gobble}{\mathexclamationmark}}
\mathcode`!="8000
\makeatother

\newcommand{\intga}{\mathclap{\smash{\oplus}}{\int}}
\newcommand{\intgp}{\mathclap{\smash{\otimes}}{\int}}
\newcommand{\intgg}{\mathclap{\rightsquigarrow}\mathclap{\int}}

\title{Geometry of arithmetic expressions}

%\date{September 9, 1985}	% Here you can change the date presented in the paper title
%\date{} 					% Or removing it

\author{
  Mingli~Yuan \\
  AI Lab \\
  ColorfulClouds Tech.\\
  Beijing, 100083 \\
  \texttt{mingli.yuan@gmail.com}
  \And
  Le~Zhang \\
  %% Affiliation \\
  %% Address \\
  %% \texttt{email} \\
  \And
  Wenfeng~Jiang \\
  %% Affiliation \\
  %% Address \\
  %% \texttt{email} \\
}

% Uncomment to remove the date
%\date{}

% Uncomment to override  the `A preprint' in the header
\renewcommand{\headeright}{A preprint}
\renewcommand{\undertitle}{A preprint}

\begin{document}
\maketitle

\begin{abstract}
    TODO
\end{abstract}

\keywords{arithmetic expressions, hyperbolic geometry}

\setcounter{tocdepth}{2}
\tableofcontents
\newpage

\section{Introduction}\label{sec:introduction}

Can arithmetic expressions form a geometry space? What properties do these spaces hold? Can they apply to other domains
of mathematics? They are the central problems of this article.


\section{Arithmetic expressions space}\label{sec:space}

\subsection{Arithmetic expression}\label{sec:expression}

\subsection{Topological arithmetic expression space}\label{sec:topological}

\subsection{Arithmetic expression surface}\label{sec:surface}

\subsection{Flow equation}\label{sec:equation}

\section{Examples and related questions}\label{sec:examples}

\subsection{The first kind}\label{sec:firstkind}

\subsection{The second kind}\label{sec:sencondkind}

\subsection{The third kind}\label{sec:thirdkind}

\subsection{Tube structure}\label{sec:tube}



\section{Two spaces that can encode arithmetic expressions}\label{sec:examples}

\section{Flow equation and its application}\label{sec:equation}

\section{Arithmetic expression, LISP and transformation on tree}\label{sec:expressions}

\subsection{Donaghey transformation}\label{sec:donaghey}



\begin{thebibliography}{9}

\bibitem{Giuseppe2019}
    Giuseppe Negro,
    \textit{Laplacian on Poincaré upper half plane},
    Mathematics Stack Exchange,
    2019.

\bibitem{Costa2001ADO}
  S. Costa,
  \textit{A description of several coordinate systems for hyperbolic spaces},
  arXiv: Mathematical Physics,
  2001.


\end{thebibliography}


\end{document}
