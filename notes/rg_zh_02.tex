\documentclass[12pt]{article}
\usepackage{ctex} % 中文支持
\usepackage{amsmath}
\usepackage{amssymb}
\usepackage{geometry}
\geometry{a4paper, margin=1in}
\usepackage{fancyhdr}
\pagestyle{fancy}
\fancyhf{}
\fancyhead[L]{AEG 框架下迭代算例分析笔记}
\fancyhead[R]{\today}
\fancyfoot[C]{\thepage}
\renewcommand{\headrulewidth}{0.4pt}
\renewcommand{\footrulewidth}{0.4pt}

\title{算术表达式几何 (AEG) 框架下迭代算例分析}
\author{苑明理(与 Gemini 共同整理)}
\date{\today}

\begin{document}
\maketitle
\thispagestyle{empty}
\tableofcontents
\newpage

\section{引言}
本文档总结了通过算术表达式几何 (AEG) 框架对几个经典迭代映射算例的分析。我们旨在探索 AEG 在理解这些系统行为(尤其是不动点和相变)方面的潜力。核心分析方法包括标准的不动点与稳定性分析,以及将迭代函数的计算过程映射到 AEG 路径,并赋予其累积交换空间 (ACS) 中的坐标 $(A, M)$,其中 $A$ 为累积加性载荷,$M$ 为累积乘性载荷(对数尺度)。我们还初步探讨了全局挠率 $\mathcal{T}$ 的概念。

\section{核心分析方法回顾}
对于每个迭代函数 $R(x)$,我们进行以下分析:
\begin{enumerate}
    \item \textbf{标准分析}:找到不动点 $x^* = R(x^*)$,计算导数 $R'(x^*)$ 判断其稳定性。
    \item \textbf{AEG 路径映射与 ACS 坐标 $(A,M)$}:将计算 $R(x)$ 的过程映射为一个 AEG 可线索化路径 $\gamma(x)$。根据特定的因式分解或更通用的霍纳法则,为路径赋予 ACS 坐标:
    \begin{itemize}
        \item $A$ (累积加性载荷):通常由 $R(x)$ 多项式形式中的特定系数(或其组合,可能依赖于系统参数如 $\beta$)决定。
        \item $M$ (累积乘性载荷或对数尺度):通常与当前状态 $x$ 的对数(如 $N \ln x$)相关,其中 $N$ 与 $R(x)$ 的多项式阶数或路径构造有关。
    \end{itemize}
    \item \textbf{几何特性分析}:考察在不动点 $x^*$ 处,AEG 路径 $\gamma(x^*)$ 的几何坐标 $(A(x^*), M(x^*))$ 的特征值或极限行为。
    \item \textbf{理论吻合与洞察}:讨论计算结果与相关物理理论的吻合程度。
    \item \textbf{全局挠率 $\mathcal{T}$}:初步探讨全局挠率的引入及其在不动点可能的行为特征。
\end{enumerate}

\section{算例分析}

%----------------------------------------------------------------------------------------
% 例1:一维渗流模型的 RG 迭代
%----------------------------------------------------------------------------------------
\subsection{例1:一维渗流模型的 RG 迭代 ($R(p) = 3p^2 - 2p^3$)}
\subsubsection{迭代函数与标准分析}
迭代函数为 $R(p) = 3p^2 - 2p^3 = (-2)p^3 + (3)p^2$。
不动点 $p^*$ 满足 $p = R(p)$,解得 $p_1^* = 0$, $p_2^* = 1/2$, $p_3^* = 1$。
导数 $R'(p) = 6p - 6p^2$。
稳定性分析:
\begin{itemize}
    \item $p_1^* = 0$: $R'(0) = 0$ (稳定)。
    \item $p_2^* = 1/2$: $R'(1/2) = 1.5$ (不稳定,临界点 $p_c^*$)。
    \item $p_3^* = 1$: $R'(1) = 0$ (稳定)。
\end{itemize}
这与典型的一维 RG 变换行为一致。

\subsubsection{AEG 路径映射与 ACS 坐标 $(A,M)$}
对于形如 $R(x) = Yx^N + Xx^{N-1}$ 的多项式,我们采用一种特定的 AEG 可线索化路径,从 $x_{in}=0$ 开始:
$0 \xrightarrow{\oplus_X} X \xrightarrow{\otimes_{-\ln p}} X/p \xrightarrow{\oplus_Y} Y+X/p \underbrace{\xrightarrow{\otimes_{\ln p}} \dots \xrightarrow{\otimes_{\ln p}}}_{N \text{ 次}}$
其 ACS 坐标为 $A_R = X+Y$ 和 $M_R(p) = (N-1)\ln p$。

对于 $R(p) = (-2)p^3 + (3)p^2$:
$X=3$ (对应 $p^{N-1}=p^2$ 的系数)
$Y=-2$ (对应 $p^N=p^3$ 的系数)
$N=3$
因此,
\begin{itemize}
    \item 累积加性载荷 $A_R = X+Y = 3 + (-2) = 1$。
    \item 累积乘性载荷 $M_R(p) = (N-1)\ln p = (3-1)\ln p = 2\ln p$。
\end{itemize}

\subsubsection{不动点处的 AEG 几何特性}
\begin{itemize}
    \item $p_1^* = 0$: $(A=1, M \rightarrow -\infty)$。
    \item $p_2^* = 1/2$: $(A=1, M = 2\ln(1/2) = -2\ln 2 \approx -1.3863)$。
    \item $p_3^* = 1$: $(A=1, M = 2\ln(1) = 0)$。
\end{itemize}

\subsubsection{全局挠率 $\mathcal{T}_R(p)$ 的计算与分析}
根据 AEG 理论 (Definition 2.6 in aeg-paper.pdf),全局算术挠率 $\mathcal{T}(\gamma) = \sum_{k=1}^{L-1} (M_k \Delta A'_{k+1} - A'_k \Delta M_{k+1})$。
对于我们为 $R(x) = Yx^N + Xx^{N-1}$ 选定的路径(操作数 $L=N+2$),其操作序列为:
$o_1: (\mu_1=X, \lambda_1=0)$
$o_2: (\mu_2=0, \lambda_2=-\ln p)$
$o_3: (\mu_3=Y, \lambda_3=0)$
$o_4, \dots, o_{N+2}$: $N$ 次 $(\mu_k=0, \lambda_k=\ln p)$。

对于该特定路径,我们(在之前的讨论中,基于对 $N=2$ 情况的推广和验证)得到挠率的一个可能形式为:
$\mathcal{T}_R(p) = X \ln p \left( (N+1) \frac{Y}{X} p^{-1} - 1 \right)$ (注意:此公式的普适性依赖于严格的路径构造和挠率定义的应用,这里作为一个特定路径的计算结果示例)。

应用于 $R(p) = (-2)p^3 + (3)p^2$ ($X=3, Y=-2, N=3$):
$\mathcal{T}_R(p) = 3 \ln p \left( (3+1) \frac{-2}{3} p^{-1} - 1 \right) = 3 \ln p \left( -\frac{8}{3} p^{-1} - 1 \right) = -\ln p (8 p^{-1} + 3)$。

不动点处的挠率值:
\begin{itemize}
    \item $p_1^* = 0$: $\mathcal{T}_R(0)$ 涉及 $\ln 0$ 和 $p^{-1}$,是\textbf{发散的}。
    \item $p_2^* = 1/2$ (临界点): $\mathcal{T}_R(1/2) = -\ln(1/2) (8 (2) + 3) = \ln 2 (16 + 3) = 19 \ln 2 \approx 13.1689$。这是一个非零的有限值。
    \item $p_3^* = 1$: $\mathcal{T}_R(1) = -\ln(1) (8 (1) + 3) = 0$。
\end{itemize}
讨论:$\mathcal{T}_R(1)=0$ 可能暗示全连接相的计算路径在几何上是“平直的”或具有某种简明性。$\mathcal{T}_R(1/2) \neq 0$ 表明临界点的计算路径具有非零的“扭曲度”。$\mathcal{T}_R(0)$ 发散可能反映了空格子状态的极端几何特性。

%----------------------------------------------------------------------------------------
% 例2:另一类渗流模型的 RG 迭代
%----------------------------------------------------------------------------------------
\subsection{例2:另一类渗流模型的 RG 迭代 ($R(p) = 4p^3 - 3p^4$)}
\subsubsection{迭代函数与标准分析}
$R(p) = 4p^3 - 3p^4 = (-3)p^4 + (4)p^3$。
不动点:$p_1^* = 0$ (稳定), $p_2^* = 1$ (稳定), $p_3^* = \frac{1 + \sqrt{13}}{6} \approx 0.7676$ (不稳定, $p_c^*$)。
导数 $R'(p) = 12p^2 - 12p^3$。
稳定性:$R'(0)=0$, $R'(1)=0$, $R'(p_3^*) = \frac{2}{9}(11 - \sqrt{13}) \approx 1.6432$。

\subsubsection{AEG 路径与 ACS 坐标}
采用与例1相同的路径构造方法 ($Yx^N + Xx^{N-1}$),这里 $X=4, Y=-3, N=4$。
\begin{itemize}
    \item $A_R = X+Y = 4 + (-3) = 1$。
    \item $M_R(p) = (N-1)\ln p = (4-1)\ln p = 3\ln p$。
\end{itemize}

\subsubsection{不动点处的 AEG 几何特性}
\begin{itemize}
    \item $p_1^* = 0$: $(A=1, M \rightarrow -\infty)$。
    \item $p_2^* = 1$: $(A=1, M = 0)$。
    \item $p_3^* = \frac{1 + \sqrt{13}}{6}$: $(A=1, M = 3\ln\left(\frac{1 + \sqrt{13}}{6}\right) \approx -0.7935)$。
\end{itemize}

\subsubsection{全局挠率 $\mathcal{T}_R(p)$ (类比形式)}
若采用与例1类似的挠率表达式形式 $\mathcal{T}_R(p) = X \ln p \left( (N+1) \frac{Y}{X} p^{-1} - 1 \right)$:
$X=4, Y=-3, N=4$。
$\mathcal{T}_R(p) = 4 \ln p \left( (4+1) \frac{-3}{4} p^{-1} - 1 \right) = 4 \ln p \left( -\frac{15}{4} p^{-1} - 1 \right) = -\ln p (15 p^{-1} + 4)$。
\begin{itemize}
    \item $p_1^* = 0$: $\mathcal{T}_R(0)$ \textbf{发散}。
    \item $p_2^* = 1$: $\mathcal{T}_R(1) = 0$。
    \item $p_3^* = \frac{1 + \sqrt{13}}{6}$: $\mathcal{T}_R(p_3^*) = -\ln(p_3^*) (15 (p_3^*)^{-1} + 4)$,为一个非零有限值。
\end{itemize}
讨论与例1类似。

%----------------------------------------------------------------------------------------
% 例3:简化的自旋玻璃序参量三阶展开迭代
%----------------------------------------------------------------------------------------
\subsection{例3:简化的自旋玻璃序参量三阶展开迭代}
\subsubsection{迭代函数与标准分析}
$R(q) = \beta^2 q - 2\beta^4 q^2 + \frac{17}{3}\beta^6 q^3$。
不动点:$q_1^*=0$。其他不动点 $q_{2,3}^* = \frac{3\beta \pm \sqrt{17 - 14\beta^2}}{17\beta^3}$ (实数解要求 $\beta^2 \le 17/14$)。
在 $T_c (\beta=1)$ 时,不动点为 $q^*=0$ (临界, $R'(0)=1$) 和 $q^*=6/17$ (不稳定, $R'(6/17) \approx 1.705$)。
导数 $R'(q) = \beta^2 - 4\beta^4 q + 17\beta^6 q^2$。

\subsubsection{AEG 路径与 ACS 坐标 (基于霍纳法则)}
对于 $P(x) = \sum_{i=0}^N c_i x^i$,以 $x_{in}=c_N$ 为路径起点,则 $A_P = \sum_{i=0}^{N-1} c_i$ 和 $M_P = N \ln x$。
$R(q) = (\frac{17}{3}\beta^6) q^3 + (-2\beta^4) q^2 + (\beta^2) q + (0)q^0$。
$N=3, c_3 = \frac{17}{3}\beta^6, c_2 = -2\beta^4, c_1 = \beta^2, c_0 = 0$。
\begin{itemize}
    \item $A_R(\beta) = c_2 + c_1 + c_0 = -2\beta^4 + \beta^2$。
    \item $M_R(q) = 3 \ln q$。
\end{itemize}

\subsubsection{不动点处的 AEG 几何特性}
\begin{itemize}
    \item $q_1^* = 0$: $(A_R(\beta) = \beta^2 - 2\beta^4, M \rightarrow -\infty)$。
    \item $q^* = 6/17$ (当 $\beta=1$): $(A_R(1)=-1, M_R(6/17) = 3\ln(6/17) \approx -3.123)$。
    \item 其他物理相关的 $q_{SG}^*$: $(A_R(\beta), M_R(q_{SG}^*) = 3\ln q_{SG}^*)$。
\end{itemize}

\subsubsection{全局挠率 $\mathcal{T}_R(q)$ (基于霍纳法则路径)}
对于霍纳法则构造的路径,如果其一般挠率形式比较复杂或趋于零(如我们之前的初步分析),则需要更细致的推导。若挠率在特定构造下为零,则表示该计算路径在几何上是“平直的”。如果非零,则其值将依赖于 $\beta$ 和 $q$。例如,在 $\beta=1, q^*=6/17$ 处,计算其具体挠率值将需要逐步应用挠率定义。

%----------------------------------------------------------------------------------------
% 例4:逻辑斯蒂映射及其 2-周期循环
%----------------------------------------------------------------------------------------
\subsection{例4:逻辑斯蒂映射及其 2-周期循环 ($R(x) = rx(1-x)$, $r=3.2$)}
\subsubsection{迭代函数与标准分析}
$R(x) = rx - rx^2$。取 $r=3.2$。
$R(x)$ 的不动点 ($r=3.2$): $x_1^* = 0$ (不稳定, $R'(0)=3.2$); $x_2^* = 11/16 = 0.6875$ (不稳定, $R'(11/16)=-1.2$)。
2-周期循环点 $\{x_a, x_b\}$ (当 $r=3.2$): $x_a \approx 0.513045$, $x_b \approx 0.799455$。
该2-周期循环是稳定的,因为稳定性乘子 $\lambda_{cycle} = R'(x_a)R'(x_b) \approx 0.16 < 1$。

\subsubsection{AEG 路径与 ACS 坐标 (基于霍纳法则)}
\textbf{a) 对于 $R(x) = (-r)x^2 + (r)x^1$}:
$N=2, c_2=-r, c_1=r, c_0=0$。
\begin{itemize}
    \item $A_R(r) = c_1 + c_0 = r$。
    \item $M_R(x) = 2 \ln x$。
\end{itemize}
对于 $r=3.2$:
$A_R = 3.2$。
在2-周期循环点:
\begin{itemize}
    \item $\gamma_R(x_a)$: $(A_R = 3.2, M_R(x_a) = 2 \ln(0.513045) \approx -1.3348)$。
    \item $\gamma_R(x_b)$: $(A_R = 3.2, M_R(x_b) = 2 \ln(0.799455) \approx -0.4476)$。
\end{itemize}

\textbf{b) 对于 $R^{(2)}(x) = R(R(x)) = -r^3 x^4 + 2r^3 x^3 - r^2(1+r) x^2 + r^2 x$}:
(注:这里使用了 $R^{(2)}(x)$ 的一般系数形式,当 $r=3.2$ 时,$R^{(2)}(x) = -32.768x^4 + 65.536x^3 - 43.008x^2 + 10.24x$)
$N=4$。系数为 $c_4=-r^3, c_3=2r^3, c_2=-r^2(1+r), c_1=r^2, c_0=0$。
\begin{itemize}
    \item $A_{R^{(2)}}(r) = c_3 + c_2 + c_1 + c_0 = 2r^3 - r^2(1+r) + r^2 = 2r^3 - r^2 - r^3 + r^2 = r^3$。
    \item $M_{R^{(2)}}(x) = 4 \ln x$。
\end{itemize}
对于 $r=3.2$:
$A_{R^{(2)}} = (3.2)^3 = 32.768$。
2-周期循环点 $x_a, x_b$ 是 $R^{(2)}(x)$ 的不动点。其 AEG 路径 $\gamma_{R^{(2)}}$ 的坐标为:
\begin{itemize}
    \item $\gamma_{R^{(2)}}(x_a)$: $(A_{R^{(2)}} = 32.768, M_{R^{(2)}}(x_a) = 4 \ln(0.513045) \approx -2.6696)$。
    \item $\gamma_{R^{(2)}}(x_b)$: $(A_{R^{(2)}} = 32.768, M_{R^{(2)}}(x_b) = 4 \ln(0.799455) \approx -0.8952)$。
\end{itemize}

\subsubsection{全局挠率 $\mathcal{T}$ (对 $R(x)$ 的讨论)}
对于 $R(x) = rx - rx^2 = (-r)x^2 + (r)x^1$, 采用路径 $(\oplus_r, \otimes_{-\ln x}, \oplus_{-r}, \otimes_{\ln x}, \otimes_{\ln x})$ 从 $x_{in}=0$ 开始,我们曾推导出 $\mathcal{T}_R(x) = r \ln x (3x^{-1} - 1)$。
(注意:这里的 $A=0, M=\ln x$ 是基于这种特定路径的直接累积,而上面基于霍纳法则的 $A=r, M=2\ln x$ 是另一种路径构造。挠率依赖于路径选择。)

若使用 $\mathcal{T}_R(x) = r \ln x (3x^{-1} - 1)$:
在2-周期循环点 ($r=3.2$):
\begin{itemize}
    \item $\mathcal{T}_R(x_a) = 3.2 \ln(0.513) (3/0.513 - 1) \approx 3.2(-0.667)(5.848 - 1) \approx -2.135 \times 4.848 \approx -10.348$。
    \item $\mathcal{T}_R(x_b) = 3.2 \ln(0.799) (3/0.799 - 1) \approx 3.2(-0.224)(3.755 - 1) \approx -0.717 \times 2.755 \approx -1.975$。
\end{itemize}
这两个非零的挠率值表明,计算 $R(x)$ 到达这两个周期点的路径具有特定的“扭曲度”。

\section{讨论与结论}
通过对上述四个算例的分析,我们将迭代函数 $R(x)$ 的计算过程映射到了 AEG 框架下的可线索化路径,并赋予了其 ACS 几何坐标 $(A,M)$(累积加性载荷和累积乘性载荷)以及初步探讨了全局挠率 $\mathcal{T}$。

\textbf{主要观察和洞察}:
\begin{enumerate}
    \item \textbf{计算的几何化}:AEG 框架为迭代规则 $R(x)$ 的“计算过程”本身提供了动态的几何坐标。这些坐标可以随迭代变量 $x$ 和系统参数(如 $\beta, r$)而演化。
    \item \textbf{不动点与相变的几何标记}:不同类型的物理状态(稳定相、临界点、周期循环)在 ACS 空间中留下了特定的几何“指纹”或轨迹。例如,$x^*=0$ 的不动点通常对应 $M \rightarrow -\infty$;$x^*=1$ 的不动点可能对应 $M=0$。临界点和周期点则对应 ACS 空间中的其他特定位置。
    \item \textbf{参数依赖性}:AEG 坐标可以显式地依赖于系统参数(如 $A_R(\beta)$ 或 $A_R(r)$),这反映了系统外部条件如何改变计算规则的内在几何结构。
    \item \textbf{全局挠率的潜力}:全局挠率 $\mathcal{T}$ 作为衡量计算路径“扭曲度”的指标,在不动点(尤其是在临界点和稳定有序相)可能表现出特征行为(如发散、为零或取特定有限值)。这为从几何上理解计算复杂性和系统状态的“非平凡度”提供了新途径。
    \item \textbf{路径构造的意义重大}:我们注意到,对于同一个代数表达式 $R(x)$,选择不同的 AEG 路径构造方法(例如,基于特定因式分解或基于更通用的霍纳法则)可能会导致不同的 $(A,M)$ 甚至 $\mathcal{T}$ 的表达式。这凸显了在 AEG 理论中发展“标准型路径”或“正则表示”的重要性,以确保这些几何量具有明确和可比较的物理意义。
\end{enumerate}

\textbf{未来展望}:
这些算例分析是探索 AEG 框架能力的初步尝试。其更深远的价值可能在于:
\begin{itemize}
    \item \textbf{统一描述能力}:AEG 是否能为更广泛的计算过程(包括那些涉及超越函数、积分、高维或函数型迭代的系统)提供统一的几何语言。
    \item \textbf{新预测能力}:AEG 的几何量($A, M, \mathcal{T}$ 以及可能的更高阶不变量,如曲率)是否能够预测传统方法难以洞察的系统行为,或简化对复杂系统(如混沌边缘、多重分岔)的分析。
    \item \textbf{与 AEG 核心理论的深度融合}:将这些具体的几何映射与 AEG 的流方程、对偶时间标度等核心理论工具更紧密地结合起来,形成一个自洽且具有强大解释力的理论体系。
\end{itemize}

总之,通过将迭代计算过程几何化,AEG 框架为我们理解和分析经典物理和数学问题中的迭代现象提供了一个富有潜力的新视角。这些初步的算例分析为此方向的深入研究奠定了基础。

\end{document}