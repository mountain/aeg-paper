\documentclass[12pt, a4paper]{article}
\usepackage{amsmath}
\usepackage{amssymb}
\usepackage{amsthm}
\usepackage[utf8]{inputenc}
\usepackage[T1]{fontenc}
\usepackage{geometry}
\usepackage{hyperref}

\geometry{a4paper, margin=1in}
\hypersetup{
    colorlinks=true,
    linkcolor=blue,
    urlcolor=cyan,
}

\title{Note on First Principles in Arithmetic Expression Geometry: $F_2$, Canonical Forms, and Condensation}
\author{Mingli Yuan and Gemini AI}
\date{May 27, 2025}

\theoremstyle{definition}
\newtheorem{definition}{Definition}[section]
\newtheorem{theorem}{Theorem}[section]
\newtheorem{proposition}{Proposition}[section]
\newtheorem{lemma}{Lemma}[section]
\newtheorem{corollary}{Corollary}[section]
\newtheorem{remark}{Remark}[section]


\begin{document}
\maketitle
\begin{abstract}
This note summarizes a late-night discussion on foundational, "first principles" thinking for Arithmetic Expression Geometry (AEG). The core idea posits the free group on two generators, $F_2$, as the origin of computational complexity. It proposes that the introduction of canonical forms for arithmetic expressions acts as a "seed for spatiality," with these forms "weaving" a spatial grid, defining a "geometrization condition." Computation is then conceptualized as flow on this grid, and computational complexity as the length of this flow. A significant extension to this first principle is the idea that canonical forms, viewed as "points" in this space, encapsulate "computational energy," thereby possessing "computational mass" and inducing "spatial curvature," drawing an analogy to general relativity. The process of "condensation" is identified with the formation of a quotient group from $F_2$ by a normal subgroup $N$, where $N$ consists of the canonical forms of the relations defining the target group (e.g., a Baumslag-Solitar group $BS(m,n)$). The Baumslag-Solitar group $BS(1,1)$ (the free abelian group $\mathbb{Z}^2$) is analyzed as the simplest case, where the condensed canonical form is the commutator. The global arithmetic torsion of the commutator $[x,y]$ (where $x = \oplus_\mu, y = \otimes_\lambda$) is algebraically calculated as $\tau_{\text{alg}}([x,y]) = \mu(e^{-\lambda} - e^\lambda)$, highlighting a distinction from locally defined torsion measures.
\end{abstract}

\tableofcontents
\newpage

\section{Introduction: Seeking First Principles}

This note captures the essence of a discussion aimed at unearthing "first principles" to underpin the framework of Arithmetic Expression Geometry (AEG). The driving motivation is to move beyond constructing various AEG spaces પાણી ad hoc and instead to understand how such geometric structures might naturally emerge from fundamental algebraic and computational concepts. The discussion explored the role of the free group $F_2$, the significance of canonical forms, the concept of "condensation" in forming specific group structures, and a novel analogy to general relativity where computational entities might induce geometric curvature.

\section{Core Tenets of the First Principle}

The proposed first principle thinking can be summarized through several interconnected ideas:

\begin{enumerate}
    \item \textbf{$F_2$ as the Primordial Source of Complexity}: The free group on two generators, $F_2 = \langle \mathbf{x}, \mathbf{y} \rangle$, is posited as the fundamental origin of both time complexity and space complexity in arithmetic processes. $\mathbf{x}$ and $\mathbf{y}$ represent abstract elementary operations (e.g., a unit addition and a unit multiplication).
    \item \textbf{Canonical Forms as Seeds of Spatiality and the Geometrization Condition}: The act of choosing or defining a \textit{canonical form} for arithmetic expressions (elements of $F_2$ or related structures) is not merely a representational convenience. Instead, it is a foundational step that "implants a seed of spatiality." The collection of these canonical forms, and the way they relate to each other (how they "join hands"), is what \textit{weaves the fabric of a geometric space}. This process itself can be understood as a "geometrization condition."
    \item \textbf{Computation as Flow on a Geometrized Grid}: Once this geometric space (or grid) is established by the interlinking of canonical forms, any specific computation (i.e., the evaluation of an arithmetic expression) can be visualized as a \textit{flow} or a path on this grid.
    \item \textbf{Complexity as Path Length}: The \textit{length} of this flow path, measured according to a metric inherent to the geometrization, corresponds to the computational complexity (e.g., time complexity) of the process.
    \item \textbf{Canonical Forms as the Basis for Spacetime Transformation in Computation}: The chosen canonical forms are fundamental in establishing the relationship and potential transformations between the temporal aspect of computation (sequence of operations) and its spatial realization (geometric path or location).
\end{enumerate}

\section{Extension: Canonical Forms, Computational Energy, and Spatial Curvature}

A significant and stimulating extension to the above first principles emerged:

\begin{enumerate}
    \item \textbf{Canonical Forms as Points with "Computational Energy"}: Each canonical form, now viewed as a "point" in the emergent geometric space, is considered to encapsulate a certain amount of "computational energy." This energy could be related to the complexity of the canonical form itself (e.g., its length in $F_2$, or the resources needed to construct or represent it).
    \item \textbf{"Computational Mass" from Energy}: Drawing an analogy with $E=mc^2$, this computational energy endows the canonical form (the point) with a "computational mass." This is not a physical mass but a measure of its "computational inertia" or its capacity to influence the structure of the space.
    \item \textbf{Spatial Curvature Induced by Computational Mass}: This is the most profound analogy, linking to General Relativity. The "computational mass" of these canonical-form-points dictates how the arithmetic expression space itself is curved. Regions davranışsal dense with "massive" canonical forms, or containing forms of very high "mass," would exhibit significant curvature.
    \item \textbf{Space Complexity and Curvature}: The initial idea that space complexity might be related to curvature finds a more concrete basis here. High curvature regions might correspond to areas of high computational difficulty or resource demand, reflecting a higher intrinsic spatial complexity.
\end{enumerate}
This perspective shifts AEG from describing arithmetic processes within a pre-defined (often hyperbolic) geometry to a framework where the geometry itself is an emergent property, shaped by the "matter" (canonical forms with computational energy) of arithmetic.

\section{Condensation of Canonical Forms and Group Structure}
The discussion then focused on how specific algebraic structures, such as Baumslag-Solitar groups $BS(m,n)$, arise from $F_2$ through this lens.

\begin{definition}[Condensation]
The process of "condensation" refers to the formation of a specific group $G$ (e.g., $BS(m,n)$) from $F_2$ by imposing relations. The canonical forms of these relations, which are elements of $F_2$, are "condensed" into the identity element (a single "point") in the resulting quotient group $G \cong F_2/N$. The normal subgroup $N = \ker(\phi)$ (where $\phi: F_2 \twoheadrightarrow G$ is an epimorphism) consists of all elements of $F_2$ that become trivial in $G$, and it is these canonical forms within $N$ that are considered "condensed."
\end{definition}

The way these condensed canonical forms (relations) "join hands" is dictated by the algebraic rules of the target group $G$. The stable letter in an HNN extension (of which $BS(m,n)$ is an example) plays a crucial role in defining these relations and thus orchestrating the condensation.

\subsection{The Case of $BS(1,1) \cong \mathbb{Z}^2$}
$BS(1,1)$ is the free abelian group on two generators, $\mathbb{Z}^2$. It is obtained from $F_2 = \langle \mathbf{x}, \mathbf{y} \rangle$ by adding the commutation relation $[\mathbf{x},\mathbf{y}] = \mathbf{x}\mathbf{y}\mathbf{x}^{-1}\mathbf{y}^{-1} = 1$.
\begin{itemize}
    \item \textbf{Condensed Canonical Form}: The commutator $[\mathbf{x},\mathbf{y}]$.
    \item \textbf{Algebraic Effect}: Operations corresponding to $\mathbf{x}$ and $\mathbf{y}$ become commutative in $BS(1,1)$.
    \item \textbf{Geometric Effect}: The Cayley graph of $F_2$ (an infinite 4-valent tree) "collapses" or "folds" into the standard square grid of $\mathbb{Z}^2$. All loops defined by the commutator relation become contractible.
    \item \textbf{AEG Interpretation}:
        \begin{itemize}
            \item The Accumulative Commutative Space (ACS) with coordinates $(A_\gamma, M_\gamma)$ directly corresponds to this $\mathbb{Z}^2$ grid (or its realification).
            \item The global arithmetic torsion (related to path order dependence) for any path in a geometry faithfully representing $BS(1,1)$ should effectively be zero, as the underlying operations commute in their net effect.
            \item An AEG Expression Space ($\mathfrak{E}$) realizing $BS(1,1)$ would be expected to be flat (zero curvature).
        \end{itemize}
\end{itemize}

\section{Global Arithmetic Torsion of the Commutator}
A detailed calculation of the global arithmetic torsion for the commutator path $\gamma_c = [\mathbf{x},\mathbf{y}] = \mathbf{x}\mathbf{y}\mathbf{x}^{-1}\mathbf{y}^{-1}$ was undertaken, where $\mathbf{x}$ represents an additive operation $\oplus_\mu$ (with parameter $\mu$) and $\mathbf{y}$ represents a multiplicative operation $\otimes_\lambda$ (multiplication by $e^\lambda$, with logarithmic parameter $\lambda$).

The global arithmetic torsion is defined algebraically as $\tau_{\text{alg}}(\gamma) = \nu(\gamma)(v_0) - \nu(\bar{\gamma})(v_0)$, where $\nu(\gamma)(v_0)$ is the evaluation of path $\gamma$ acting on an initial value $v_0$, and $\bar{\gamma}$ is the path obtained by applying the literal sequence of operations from $\gamma$ in the exact reverse order to $v_0$.

Let $x(v) = v + \mu$, $x^{-1}(v) = v - \mu$, $y(v) = v e^\lambda$, $y^{-1}(v) = v e^{-\lambda}$.
\begin{itemize}
    \item Evaluation of $\gamma_c = \mathbf{x}\mathbf{y}\mathbf{x}^{-1}\mathbf{y}^{-1}$ (operations applied from right to left):
    $\nu(\gamma_c)(v_0) = x(y(x^{-1}(y^{-1}(v_0)))) = (v_0 - \mu e^\lambda) + \mu$.
    \item The reversed-sequence path $\bar{\gamma}_c$ has the literal operation sequence $\mathbf{y}^{-1}, \mathbf{x}^{-1}, \mathbf{y}, \mathbf{x}$.
    Evaluation of $\bar{\gamma}_c$:
    $\nu(\bar{\gamma}_c)(v_0) = y^{-1}(x^{-1}(y(x(v_0)))) = (v_0 + \mu) - \mu e^{-\lambda}$.
\end{itemize}
Thus, the global arithmetic torsion for the commutator is:
\begin{align*}
\tau_{\text{alg}}([\mathbf{x},\mathbf{y}]) &= \nu(\gamma_c)(v_0) - \nu(\bar{\gamma}_c)(v_0) \\
&= (v_0 - \mu e^\lambda + \mu) - (v_0 + \mu - \mu e^{-\lambda}) \\
&= -\mu e^\lambda + \mu e^{-\lambda} \\
&= \mu (e^{-\lambda} - e^\lambda)
\end{align*}
This result is distinct from the locally defined arithmetic torsion $\tau_{local} = \mu(e^\lambda - 1)$, which arises from comparing $\nu(\mathbf{y}\mathbf{x})(v_0)$ with $\nu(\mathbf{x}\mathbf{y})(v_0)$. This highlights the importance of precise definitions when relating algebraic torsion calculations to their geometric counterparts in the ACS via the Triple Identity. The geometric integral forms (area or boundary) must correspond to $\tau_{\text{alg}}(\gamma)$ for the identity to hold.

\section{Concluding Remarks}
The exploration of these first principles, particularly the idea of canonical forms with "computational energy/mass" shaping the geometry of arithmetic expression spaces, offers a potentially transformative direction for AEG. It suggests a path towards a theory where geometry is not merely a descriptive backdrop but an emergent consequence of the fundamental algebraic structure of computation ($F_2$) and the act of imposing relations (condensation of canonical forms). The precise calculation of torsion for fundamental paths like commutators serves as a crucial step in ensuring the internal consistency and interpretative power of the AEG framework. Future work will need to rigorously define "computational energy/mass" and explore the mechanisms by which these induce curvature, starting from simple condensed structures like $BS(1,1)$ and progressing towards more complex groups and their AEG realizations (e.g., $BS(2,1)$ in $\mathfrak{E}_1$).

% Add references from aeg-paper.bib if specific citations were used and available
% \bibliographystyle{alpha}
% \bibliography{aeg-paper}

\end{document}