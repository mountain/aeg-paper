\documentclass[12pt, a4paper]{article}
\usepackage{amsmath, amssymb, amsthm}
\usepackage[margin=1in]{geometry}
\usepackage{graphicx}
% Placeholder for CTeX if Chinese characters were needed, but sticking to English for this note.
% \usepackage[UTF8]{ctex}

\title{Summary of Discussions on Arithmetic Expression Geometry (AEG) for the $4_1$ Knot}
\author{Mingli Yuan (and Gemini AI)}
\date{May 17, 2025} % Updated to current date as per user request context, though content is from prior discussions.

\newtheorem{theorem}{Theorem}
\newtheorem{lemma}{Lemma}
\newtheorem{proposition}{Proposition}
\newtheorem{corollary}{Corollary}
\newtheorem{definition}{Definition}
\newtheorem{remark}{Remark}
\newtheorem{conjecture}{Conjecture}

\begin{document}
\maketitle

\section{Objective and Starting Point}
The primary objective of our discussions was to deepen the understanding and explore constructive methods for a non-trivial Arithmetic Expression Geometry (AEG) space, specifically one associated with the $4_1$ knot, denoted $E_{4_1}$. This exploration aims to move beyond the foundational $\mathfrak{E}_1$ space, whose fixed assignment function $a(x,y) = -x/y$ and simple zero-locus might be insufficient for capturing the complexities of knot groups and their invariants.

\section{The Geometric Arena and Assignment via ``Fire-Burning''}
\subsection{Universal Cover and Initial Zero-Surface}
We converged on the idea of using the universal cover of the $4_1$ knot complement, $S^3 \setminus 4_1$, which is hyperbolic 3-space $\mathbb{H}^3$, as the geometric background for $E_{4_1}$.

A key proposal for defining the assignment function $a(P)$ in $E_{4_1}$ is the ``fire-burning method'' (geometric propagation). This involves:
\begin{enumerate}
    \item Selecting an initial ``zero-surface'' where $a(P)=0$.
    \item Propagating the assignment value $a(P)$ throughout $\mathbb{H}^3$ according to the AEG flow equation, likely in its Eikonal form: $||\nabla a|| = \sqrt{\mu^2 + a^2 \lambda^2}$.
\end{enumerate}
The choice of the initial zero-surface was refined from a face of an ideal tetrahedron (part of the fundamental domain of $S^3 \setminus 4_1$) to a more topologically relevant choice:
\begin{itemize}
    \item \textbf{A lift of the fiber ($F_{lift}$)}: Since $4_1$ is a fibered knot with fiber $F$ (a punctured torus), a lift of $F$ to $\mathbb{H}^3$ (which is an embedded $\mathbb{H}^2$) was proposed as the $a=0$ surface. This choice is appealing due to its direct connection to the HNN presentation of the knot group $G(4_1)$.
\end{itemize}

\subsection{Crucial AEG Parameters: $\mu=1, \lambda=\left(\frac{1+\sqrt{5}}{2}\right)^2$}
A significant specification was made for the AEG flow equation parameters:
\begin{itemize}
    \item $\mu = 1$
    \item $\lambda = \left(\frac{1+\sqrt{5}}{2}\right)^2 = \frac{3+\sqrt{5}}{2}$ (where $\frac{1+\sqrt{5}}{2}$ is the golden ratio)
\end{itemize}
This choice is profound because $\left(\frac{1+\sqrt{5}}{2}\right)^2$ is a root of the Alexander polynomial of the $4_1$ knot, $\Delta_{4_1}(t) = t^2 - 3t + 1 = 0$. It is also the stretching factor of the monodromy $h_*$ of the $4_1$ knot fibration.
Setting $\lambda = \left(\frac{1+\sqrt{5}}{2}\right)^2$ in the AEG space $E_{4_1}$ means that the intrinsic multiplicative response of the space is tuned to a fundamental dynamical invariant of the $4_1$ knot. This is hypothesized to lead to a ``perfect matching'' between the propagation of assignment values (governed by $\lambda=\left(\frac{1+\sqrt{5}}{2}\right)^2$) and the geometric action of the monodromy (whose stretching is $\left(\frac{1+\sqrt{5}}{2}\right)^2$).

\section{Geometric Correspondences and Algebraic Structures}
\subsection{Tiling of the Fiber Surface}
We discussed that a lift of the fiber surface $F_{lift}$ (an $\mathbb{H}^2$ embedded in $\mathbb{H}^3$) would be tiled by its intersection with the copies of the 3D fundamental domain $\mathcal{D}$ of $S^3 \setminus 4_1$ (the ``double tetrahedron cell'').
\begin{itemize}
    \item The region $P_F = F_{lift} \cap \mathcal{D}$ forms a fundamental tile for the action of $\pi_1(F)$ on $F_{lift}$.
    \item The boundary of this tile $P_F$ consists of geodesic arcs on $F_{lift}$. These arcs are identified in pairs by the generators $u, v$ (and their inverses) of $\pi_1(F)$, which appear in the HNN presentation $G_{HNN}(4_1) = \langle u,v,t_{HNN} \mid t_{HNN}^{-1}ut_{HNN}=h_*(u), t_{HNN}^{-1}vt_{HNN}=h_*(v) \rangle$.
    \item Thus, the side-pairing transformations of this 2D tiling on $F_{lift}$ are literally described by the generators $u,v$ from the HNN presentation.
\end{itemize}

\subsection{The ``AEG-Topology Dictionary'': A Key Insight}
A crucial development in our understanding was the establishment of a conceptual ``dictionary'' linking AEG constructs to fundamental concepts in algebraic topology:
\begin{itemize}
    \item \textbf{AEG Arithmetic Expression Paths} (non-commutative sequences of operations $\oplus, \otimes$) $\longleftrightarrow$ \textbf{Elements of the Fundamental Group $\pi_1(F)$} (capturing homotopy information, where path order and composition are crucial).
    \item \textbf{Accumulative Commutative Space (ACS)} (mapping paths to commutative, additive coordinates $(A_\gamma, M_\gamma) \in \mathbb{R}^2$) $\longleftrightarrow$ \textbf{First Homology Group $H_1(F)$} (the abelianization of $\pi_1(F)$, $H_1(F;\mathbb{R}) \cong \mathbb{R}^2$, capturing net ``winding numbers'').
\end{itemize}
This correspondence re-anchors AEG concepts within a broader mathematical context, potentially explaining phenomena like the torsion-area formula in ACS as a manifestation of information lost during abelianization.

\section{Arithmetic Interpretation of HNN Structure in AEG}
\subsection{Traceable Expressions as Group Elements}
We clarified that the study of HNN relations should be conducted directly within the AEG framework using ``traceable expressions'' (sequences of AEG operations, denoted $\mathcal{E}(\cdot)$ for an operation sequence corresponding to a group element). These expressions form a group, where the inverse of a sequence is the reversed sequence of inverse operations (e.g., $(\oplus_\mu)^{-1} = \oplus_{-\mu}$, $(\otimes_{e^\lambda})^{-1} = \otimes_{e^{-\lambda}}$).

\subsection{HNN Relations as Identities of Traceable Expressions}
The HNN relation, e.g., $t_{HNN}^{-1}ut_{HNN} = h_*(u)$ (for $4_1$, $h_*(u)=uv$), translates into an identity between traceable expressions:
$$ \mathcal{E}(t_{HNN}^{-1}) \circ \mathcal{E}(u) \circ \mathcal{E}(t_{HNN}) \equiv \mathcal{E}(u) \circ \mathcal{E}(v) $$
This identity must hold for any input $x_0$ and any embedded parameters (like $t_{param}$ for the Alexander polynomial, primarily associated with $\mathcal{E}(t_{HNN})$). This imposes strong algebraic constraints on the structure of the traceable expressions $\mathcal{E}(u)$ and $\mathcal{E}(v)$, given a definition for $\mathcal{E}(t_{HNN})$.

\subsection{Emergence of the Alexander Polynomial $\Delta_{4_1}(t_{param})$}
The Alexander polynomial $\Delta_{4_1}(t_{param})$ is intrinsically linked to the monodromy $h_*$'s action on $H_1(F)$.
\begin{itemize}
    \item In AEG, when a full relator $R$ of $G(4_1)$ (derived from HNN or another presentation) is arithmetized into a path $\mathcal{E}(R)$, the condition for this path to act as the identity (i.e., $\nu(\mathcal{E}(R))(x_0; t_{param}) = x_0$) is expected to yield an equation involving $\Delta_{4_1}(t_{param})=0$.
    \item The global arithmetic torsion $\tau(\mathcal{E}(R))(t_{param})$ is expected to factor as $\Delta_{4_1}(t_{param})(t_{param}^K-1) / (\text{denominator})$.
    \item This occurs because the arithmetization of the knot group relations, when projected to the ACS (the $H_1$ analogue), must reflect the characteristic behavior of the monodromy on homology.
    \item The choice of $\lambda_{AEG} = \left(\frac{1+\sqrt{5}}{2}\right)^2$ for the background AEG space $E_{4_1}$ is particularly relevant. If the arithmetic interpretation for $\mathcal{E}(t_{HNN})$ uses $t_{param} = \left(\frac{1+\sqrt{5}}{2}\right)^2$ (the space's intrinsic multiplicative strength), then $\Delta_{4_1}(\left(\frac{1+\sqrt{5}}{2}\right)^2)=0$, implying that the torsion $\tau(\mathcal{E}(R))(\left(\frac{1+\sqrt{5}}{2}\right)^2)$ should vanish. The ``fire-burning'' construction of $E_{4_1}$ with these parameters aims to realize this zero-torsion condition geometrically.
\end{itemize}

\subsection{Cyclotomic Polynomials}
The $t_{param}^K-1$ term in the torsion, decomposable into cyclotomic polynomials $\Phi_d(t_{param})$, arises from periodicities. In AEG, this suggests that for $t_{param}$ being certain roots of unity, the action of $\mathcal{E}(t_{HNN})$ (monodromy) might exhibit finite order behavior when projected to the ACS/$H_1$ level, leading to simplifications or vanishing torsion under specific conditions.

\section{Conclusion and Future Directions}
Our discussions have significantly advanced the conceptual framework for constructing a non-trivial AEG space $E_{4_1}$ for the $4_1$ knot. Key developments include:
\begin{enumerate}
    \item A concrete geometric proposal: using $\mathbb{H}^3$ with a lift of the fiber $F_{lift}$ as the initial zero-surface for the ``fire-burning'' method, with AEG parameters $\mu=1, \lambda=\left(\frac{1+\sqrt{5}}{2}\right)^2$.
    \item A deeper understanding of the AEG-topology dictionary: $\pi_1 \leftrightarrow$ AEG Paths (traceable expressions), $H_1 \leftrightarrow$ ACS.
    \item A clearer path for analyzing HNN relations: as identities between traceable expressions, imposing algebraic constraints on the arithmetic interpretation of fiber group generators $u,v$ under the action of the monodromy generator $t_{HNN}$.
\end{enumerate}
The challenge of finding explicit word relationships between different presentations of $G(4_1)$ remains, but the current conceptual framework allows for a more principled investigation of how knot invariants like the Alexander polynomial emerge within AEG. Future work could involve attempting to find traceable expression structures for $\mathcal{E}(u), \mathcal{E}(v), \mathcal{E}(t_{HNN})$ that satisfy the HNN identities and exploring the geometric properties of the resulting $E_{4_1}$ space.

\end{document}