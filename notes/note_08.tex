\documentclass{article}
\usepackage{amsmath,amssymb,amsfonts}
\usepackage{graphicx}
\usepackage{hyperref}
\usepackage{xcolor} % For highlighting changes, remove in final

\title{Mathematical Description of the Zeroth Kind of Arithmetic Expression Space ($\mathfrak{E}_0$): Diffusion from a Single Zero Point}
\author{Mingli Yuan, Gemini AI} % Please replace with actual author
\date{\today}

\begin{document}
\maketitle

\begin{abstract}
This note details the mathematical construction of a zeroth kind of Arithmetic Expression Geometry space, denoted $\mathfrak{E}_0$.

$\mathfrak{E}_0$ is characterized by an assignment function $a$ emanating from a single singular point (a zero point) in a hyperbolic plane, modeled here by the Poincaré disk.

Its evolution, under ideal conditions (propagation along the gradient of $a$), exhibits high symmetry, analogous to a diffusion circle.
We explore its geometric properties, including the assignment function itself, characteristic lines (similarity, popularity, additive, and multiplicative lines), and its basic topological nature.
This model serves as a foundational case for understanding AEG spaces with point-like singularities, contrasting with the line-like singularities of the first kind space $\mathfrak{E}_1$.
\end{abstract}

\section{Introduction}

This note introduces a specific model for a zeroth kind of Arithmetic Expression Geometry (AEG) space, which we will denote by $\mathfrak{E}_0$.

Unlike the first kind of AEG space ($\mathfrak{E}_1$), which is typically built upon the upper half-plane and features a line of zero points (e.g., the imaginary axis for $a(z) = -\text{Re}(z)/\text{Im}(z)$), $\mathfrak{E}_0$ is conceptualized within the Poincaré disk with a single, central zero point acting as a source of 'diffusion' for the assignment function $a$.


The motivation for studying $\mathfrak{E}_0$ is to understand the behavior of AEG principles in a highly symmetric setting originating from a point singularity. This provides a contrast and a building block for more complex AEG constructions. We assume the curvature of this space is $K = -\lambda^2$, and the assignment function $a(s)$ is derived from the AEG flow equation $da/ds = \sqrt{\mu^2 + (\lambda a)^2}$, where $s$ is the hyperbolic distance in this $K=-\lambda^2$ space.


\section{Geometric Setting and Coordinates}

The geometric setting for $\mathfrak{E}_0$ is a hyperbolic plane with constant Gaussian curvature $K = -\lambda^2$, where $\lambda > 0$ is the multiplicative parameter from the arithmetic gene $(\mu, \lambda)$. We model this space using the Poincaré disk $\mathbb{D}^2 = \{w \in \mathbb{C} : |w| < 1 \}$.

The center of the disk, $w=0$, is chosen as the unique zero point of the assignment function $a$, so $a(w=0)=0$. This point acts as the source or singularity.

The path parameter $s$ is the hyperbolic distance from this central zero point $w=0$ to any point $w$ in the disk. If $r_e = |w|$ is the Euclidean distance from the origin in the Poincaré disk model, the relationship between $s$ and $r_e$ for a space with curvature $K = -\lambda^2$ is:
$$ s(r_e) = \frac{2}{\lambda} \text{arctanh}(r_e) $$
This means $\lambda s = 2 \text{arctanh}(r_e)$. As $r_e \to 1$ (approaching the boundary of the disk), $s \to \infty$.

\section{Assignment Function $a(s)$ and $a(r_e)$}

The assignment function $a(s)$ depends only on the hyperbolic distance $s$ from the origin, reflecting the radial symmetry of the construction. It is the solution to the AEG flow equation $\frac{da}{ds} = \sqrt{\mu^2 + (\lambda a)^2}$ with the initial condition $a(s=0)=0$.
The solution is:
$$ a(s) = \frac{\mu}{\lambda} \sinh(\lambda s) $$
where $\mu$ is the additive generator strength and $\lambda$ is the multiplicative generator strength from the arithmetic gene $(\mu, \lambda)$. We assume $\mu > 0$ and $\lambda > 0$.

Substituting $s(r_e) = \frac{2}{\lambda} \text{arctanh}(r_e)$ into $a(s)$, we get $a$ as a function of the Euclidean radius $r_e$:
$$ \lambda s = \lambda \left( \frac{2}{\lambda} \text{arctanh}(r_e) \right) = 2 \text{arctanh}(r_e) $$
So,
$$ a(r_e) = \frac{\mu}{\lambda} \sinh(2 \text{arctanh}(r_e)) $$
Using the identity $\sinh(2y) = \frac{2\tanh y}{1-\tanh^2 y}$, and letting $y = \text{arctanh}(r_e)$ (so $\tanh y = r_e$), we have:
$$ \sinh(2 \text{arctanh}(r_e)) = \frac{2r_e}{1-r_e^2} $$
Thus, the assignment function in terms of Euclidean radius is:
$$ a(r_e) = \frac{\mu}{\lambda} \left( \frac{2r_e}{1-r_e^2} \right) $$
This function is zero at $r_e=0$ and tends to infinity as $r_e \to 1$.

\section{Characteristic Geometric Lines}

The structure of $\mathfrak{E}_0$ can be further understood by examining its characteristic lines.

\subsection{Contour Lines of $a(r_e)$ (Similarity Lines)}
Contour lines are defined by $a(r_e) = \text{constant}$. Since $a(r_e)$ depends only on $r_e$ (for fixed $\mu, \lambda$), and $\frac{2r_e}{1-r_e^2}$ is monotonic for $r_e \in [0,1)$, the contour lines are curves where $r_e = \text{constant}$. These are Euclidean circles centered at the origin $w=0$ in the Poincaré disk model. They represent states with the same "process richness" or assignment value.

\subsection{Gradient Lines of $a(r_e)$ (Popularity Lines)}
The gradient $\nabla a(r_e)$ indicates the direction of the most rapid change in $a$. Due to the radial symmetry, $\nabla a(r_e)$ points radially outward (or inward if $\mu/\lambda < 0$, but we assume $\mu/\lambda > 0$). These gradient lines are Euclidean straight line segments passing through the origin (radii of the disk).

\subsection{Additive and Multiplicative Lines}
The directions of additive and multiplicative lines are determined by the angle $\alpha(a)$, defined as $\alpha(a) = \arctan\left(\frac{\mu}{\lambda a}\right)$.
Substituting $a(r_e) = \frac{\mu}{\lambda} \left( \frac{2r_e}{1-r_e^2} \right)$:
$$ \lambda a(r_e) = \mu \left( \frac{2r_e}{1-r_e^2} \right) $$
So,
$$ \alpha(r_e) = \arctan\left(\frac{\mu}{\mu \left( \frac{2r_e}{1-r_e^2} \right)}\right) = \arctan\left(\frac{1-r_e^2}{2r_e}\right) $$
This angle $\alpha(r_e)$ remarkably depends only on the Euclidean radius $r_e$ and is independent of $\mu$ and $\lambda$.
Let $\vec{g}$ be the unit radial vector (gradient direction) and $\vec{c}$ be the unit tangential vector (contour direction).
\begin{itemize}
    \item \textbf{Additive Line Direction ($\vec{d}_{\text{add}}$)}: $\vec{d}_{\text{add}} \propto \cos(\alpha(r_e)) \vec{g} + \sin(\alpha(r_e)) \vec{c}$.
    \item \textbf{Multiplicative Line Direction ($\vec{d}_{\text{mult}}$)}: $\vec{d}_{\text{mult}} \propto -\sin(\alpha(r_e)) \vec{g} + \cos(\alpha(r_e)) \vec{c}$ (orthogonal to $\vec{d}_{\text{add}}$).
\end{itemize}
Behavior of $\alpha(r_e)$:
\begin{itemize}
    \item Near the zero point ($r_e \to 0$): $2r_e \to 0^+$, $1-r_e^2 \to 1$. So, $\alpha(r_e) \to \arctan(+\infty) = \pi/2$.
    Additive lines are primarily tangential (circulating the origin), and multiplicative lines are primarily radial.
    \item Near the boundary ($r_e \to 1$): $1-r_e^2 \to 0^+$, $2r_e \to 2$. So, $\alpha(r_e) \to \arctan(0^+) = 0$.
    Additive lines are primarily radial, and multiplicative lines are primarily tangential.
\end{itemize}
The zero point $w=0$ acts as an "additive singularity" (source of spiraling additive lines), while the boundary $r_e=1$ acts as an "effective multiplicative singularity" where multiplicative behavior (lines becoming parallel to contours) dominates.

\section{Topological Property}
If we consider the space $\mathfrak{E}_0$ with its central singularity (the zero point $w=0$) removed, i.e., $\mathbb{D}^2 \setminus \{0\}$, its first homology group is $H_1(\mathbb{D}^2 \setminus \{0\}) \cong \mathbb{Z}$. This indicates the presence of non-trivial loops that can be drawn around the singularity. This is analogous to a cylinder or a punctured hyperbolic plane. This topological feature may be relevant for understanding global properties such as arithmetic torsion when mapping to/from the Accumulative Commutative Space (ACS) for paths that encircle this singularity.

\section{Conclusion}
The Zeroth Kind of Arithmetic Expression Space, $\mathfrak{E}_0$, defined by a single diffusing zero point in a hyperbolic plane of curvature $K=-\lambda^2$, exhibits a highly symmetric structure. Its assignment function $a(r_e)$ and characteristic angle $\alpha(r_e)$ take on forms where $\lambda$ primarily influences the curvature and overall scaling, while the geometric pattern of lines in the Poincaré disk model (dictated by $\alpha(r_e)$) becomes canonical. This model provides a fundamental and analytically tractable example of an AEG space, serving as a valuable reference for more complex scenarios.

\end{document}