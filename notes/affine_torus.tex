
\documentclass[11pt]{article}

\usepackage[margin=1in]{geometry}
\usepackage{amsmath,amssymb,amsthm}
\usepackage{booktabs}
\usepackage{array}
\usepackage{tabularx}
\usepackage{mathtools}
\usepackage{graphicx}
\usepackage{hyperref}
\usepackage{tikz}
\usepackage{enumitem}

\hypersetup{
  colorlinks=true,
  linkcolor=blue,
  urlcolor=blue,
  citecolor=blue
}

% --- Notation for basic affine operators on x ---
\newcommand{\AddOne}{\oplus 1}
\newcommand{\SubOne}{\ominus 1}
\newcommand{\MulTwo}{\otimes 2}
\newcommand{\DivTwo}{\oslash 2}
\newcommand{\SubTwo}{\ominus 2}
\newcommand{\AddTwo}{\oplus 2}

\newcommand{\Aff}{\mathrm{Aff}(1)}
\newcommand{\Id}{\mathrm{Id}}

\title{A $4\times 4$ Affine Torus from a Nested Piecewise Map\\
and a Discrete Torsion--Offset Correspondence}
\author{Mingli Yuan (exercise note, drafted with ChatGPT)}
\date{December 2025}

\begin{document}
\maketitle

\begin{abstract}
We consider a piecewise affine map $a(x,s)$ on $s\in [0,5)$ whose four parameter intervals encode the elementary
arithmetic operations ``$+1$'', ``$\times 2$'', ``$-2$'', ``$\div 2$''.
By gluing the endpoints $0\sim 5$, the parameter becomes a circle. For two parameters $(p,q)$ we study the nested map
$A(x;p,q)=a(a(x,p),q)$.
The $4\times 4$ case table for $(p,q)$ yields a natural CW--decomposition of a torus with
$16$ vertices, $32$ edges, and $16$ faces, equipped with a discrete $\Aff$--valued connection.
We define a face holonomy (a commutator) as a discrete torsion constant $c$ and list the resulting torsion table.
Finally, we work out four representative loops, one in each of the four classes
(contractible/noncontractible) $\times$ (identity/non-identity),
and illustrate how a \emph{weighted} torsion integral recovers the translation offset of a loop.
This is a toy model for the Stokes-type torsion identities in the AEG framework.
\end{abstract}

\section{The piecewise map $a(x,s)$ and the circle parameter}

Define the map $a(x,s)$ for $x\in \mathbb{R}$ and $s\in [0,5)$ by
\begin{equation}\label{eq:def-a}
a(x,s)=
\begin{cases}
x+s, & 0\le s<1,\\[2pt]
2^{\,s-1}\,(x+1), & 1\le s<2,\\[2pt]
2x - s + 4, & 2\le s<4,\\[2pt]
2^{\,5-s}\,x, & 4\le s<5.
\end{cases}
\end{equation}
A direct check at the breakpoints $s=1,2,4$ shows continuity, and moreover
\[
a(x,0)=x,\qquad \lim_{s\to 5^-}a(x,s)=x.
\]
Hence the endpoints can be glued to obtain a continuous map $a(x,\bar s)$ on a circle parameter
$\bar s\in \mathbb{R}/5\mathbb{Z}\cong S^1$.

\paragraph{Operational reading.}
The four parameter intervals encode four basic operations on $x$:
\begin{center}
\begin{tabular}{@{}ll@{}}
$0\to 1$: & $x\mapsto x+1$ (write $\AddOne$),\\
$1\to 2$: & $x\mapsto 2x$ (write $\MulTwo$),\\
$2\to 4$: & $x\mapsto x-2$ (write $\SubTwo$),\\
$4\to 0$: & $x\mapsto x/2$ (write $\DivTwo$),
\end{tabular}
\end{center}
and reversing an oriented step produces the inverse operator (e.g.\ $\SubOne$ is the inverse of $\AddOne$).

\section{The $4\times 4$ case table for $A(x;p,q)=a(a(x,p),q)$}

Let $p,q\in [0,5)$ (again with $0\sim 5$ understood), and define
\[
A(x;p,q) \;:=\; a(a(x,p),q).
\]
Split each parameter circle into the four half-open intervals
\[
I_0=[0,1),\quad I_1=[1,2),\quad I_2=[2,4),\quad I_3=[4,5).
\]
On each rectangle $I_i\times I_j$ the function $A(x;p,q)$ is an explicit affine expression in $x$.
The full $4\times 4$ table (16 analytic pieces) is:

\begin{table}[h!]
\centering
\setlength{\tabcolsep}{6pt}
\renewcommand{\arraystretch}{1.45}
\small
\begin{tabular}{@{}l>{\centering\arraybackslash}p{0.20\linewidth}
                >{\centering\arraybackslash}p{0.22\linewidth}
                >{\centering\arraybackslash}p{0.22\linewidth}
                >{\centering\arraybackslash}p{0.22\linewidth}@{}}
\toprule
$p\backslash q$
& $0\le q<1$
& $1\le q<2$
& $2\le q<4$
& $4\le q<5$\\
\midrule
$0\le p<1$
& $x+p+q$
& $2^{q-1}(x+p+1)$
& $2x+2p-q+4$
& $2^{5-q}(x+p)$\\
$1\le p<2$
& $2^{p-1}(x+1)+q$
& $2^{q-1}\!\bigl(2^{p-1}(x+1)+1\bigr)$
& $2^{p}(x+1)-q+4$
& $2^{p+4-q}(x+1)$\\
$2\le p<4$
& $2x-p+4+q$
& $2^{q-1}(2x-p+5)$
& $4x-2p-q+12$
& $2^{5-q}(2x-p+4)$\\
$4\le p<5$
& $2^{5-p}x+q$
& $2^{q-1}\!\bigl(2^{5-p}x+1\bigr)$
& $2^{6-p}x-q+4$
& $2^{10-p-q}x$\\
\bottomrule
\end{tabular}
\caption{The 16 analytic pieces of $A(x;p,q)$ on $I_i\times I_j$. Each entry is valid only on its corresponding rectangle.}
\label{tab:case-table}
\end{table}

\section{A torus CW--complex with an $\Aff$--valued edge labelling}

\subsection{The $16$ vertices, $32$ edges, and $16$ faces}

Mark the four breakpoints $\{0,1,2,4\}\subset [0,5)$ on each parameter circle.
After gluing $0\sim 5$, each of $p$ and $q$ becomes a circle subdivided into four directed edges:
\[
0\to 1,\quad 1\to 2,\quad 2\to 4,\quad 4\to 0.
\]
Taking the product of the two subdivided circles produces a $4\times 4$ grid on the square fundamental domain
(with opposite sides identified). The resulting CW--complex has:
\begin{itemize}[leftmargin=18pt]
\item $V=4\cdot 4=16$ vertices: $(p,q)\in\{0,1,2,4\}^2$,
\item $E=16$ horizontal edges (in the $q$--direction) $+\,16$ vertical edges (in the $p$--direction), hence $E=32$,
\item $F=4\cdot 4=16$ square faces, corresponding to the rectangles $I_i\times I_j$ of Table~\ref{tab:case-table}.
\end{itemize}
Topologically, this is the standard CW--decomposition of a torus $T^2=S^1\times S^1$.

\subsection{Edge labels as affine maps and the induced path holonomy}

Each oriented edge is labelled by one of the four elementary affine operators on $x$:
\[
\AddOne:\;x\mapsto x+1,\qquad
\MulTwo:\;x\mapsto 2x,\qquad
\SubTwo:\;x\mapsto x-2,\qquad
\DivTwo:\;x\mapsto \frac{x}{2},
\]
and reversing an edge direction applies the inverse operator:
\[
\SubOne:\;x\mapsto x-1,\qquad
\DivTwo^{-1}=\MulTwo,\qquad
\SubTwo^{-1}=\AddTwo:\;x\mapsto x+2.
\]
Thus any oriented path $\gamma$ in the $1$--skeleton (a concatenation of edges) evaluates to an affine map
\begin{equation}\label{eq:holonomy}
\rho(\gamma):\; x\longmapsto \Phi_\gamma\,x + A_\gamma,
\qquad (\Phi_\gamma\in 2^{\mathbb{Z}},\; A_\gamma\in \tfrac12\mathbb{Z}).
\end{equation}
We call $\rho(\gamma)$ the \emph{holonomy} of the path.

This is the same affine viewpoint used in the AEG framework:
the group generated by $\oplus_\mu$ and $\otimes_\lambda$ embeds faithfully into $\Aff$
via $x\mapsto \Phi x + a$ matrices, and commutators become nontrivial holonomy in $\Aff$
(see Section~2.8--2.9 in the AEG paper). 

\section{Face holonomy constants and discrete torsion}

\subsection{Face commutators}

Fix a face whose lower edge in the $p$--direction is of type $P$ and whose left edge in the $q$--direction is of type $Q$,
where $P,Q\in\{\AddOne,\MulTwo,\SubTwo,\DivTwo\}$.
Define the face holonomy (a commutator) by
\begin{equation}\label{eq:commutator}
H(P,Q)\;:=\;P\,Q\,P^{-1}\,Q^{-1}.
\end{equation}
Because $\Aff(1)\cong \mathbb{R}\rtimes \mathbb{R}^\times$ is a semidirect product,
the commutator of a translation with a scaling is a \emph{translation}. Concretely,
\[
H(P,Q)(x)=x+c(P,Q)
\]
for some constant $c(P,Q)\in \tfrac12\mathbb{Z}$. We interpret $c(P,Q)$ as a discrete torsion (or curvature density)
assigned to that face type.
This is directly analogous to the AEG observation that
\[
x\,\oplus_\mu\,\otimes_\lambda\,\ominus_\mu\,\oslash_\lambda-x=\mu\,(1-e^{-\lambda})
\]
is a constant measuring non-commutativity (AEG Eq.\ (22)). 

\subsection{The torsion table}

Arranging the four $p$--edge types by rows and the four $q$--edge types by columns gives the $4\times 4$ table below:

\begin{table}[h!]
\centering
\renewcommand{\arraystretch}{1.4}
\begin{tabular}{@{}lcccc@{}}
\toprule
$p\backslash q$ 
& $0\to 1:\;\AddOne$ 
& $1\to 2:\;\MulTwo$ 
& $2\to 4:\;\SubTwo$ 
& $4\to 0:\;\DivTwo$ \\
\midrule
$0\to 1:\;\AddOne$ & $0$ & $+\tfrac12$ & $0$ & $-1$\\
$1\to 2:\;\MulTwo$ & $-\tfrac12$ & $0$ & $+1$ & $0$\\
$2\to 4:\;\SubTwo$ & $0$ & $-1$ & $0$ & $+2$\\
$4\to 0:\;\DivTwo$ & $+1$ & $0$ & $-2$ & $0$\\
\bottomrule
\end{tabular}
\caption{Face holonomy constants $c(P,Q)$ defined by $H(P,Q)(x)=x+c(P,Q)$. Eight faces have $c=0$ (commuting pairs), and eight faces have $c\ne 0$ (non-commuting pairs).}
\label{tab:torsion-table}
\end{table}

\section{A weighted torsion integral and the loop offset}

\subsection{Offset of a loop}

For a closed path (loop) $\gamma$ based at a vertex $v_0$, the holonomy \eqref{eq:holonomy} is an affine map.
If $\Phi_\gamma=1$ then $\rho(\gamma)$ is a pure translation $x\mapsto x+A_\gamma$.
\begin{itemize}[leftmargin=18pt]
\item If $A_\gamma=0$, then $\gamma$ produces an \emph{arithmetic identity} (operator-level loop).
\item If $A_\gamma\ne 0$, then $\gamma$ produces a non-identity affine shift.
\end{itemize}
The quantity $A_\gamma$ will be called the \emph{offset} of the loop.

\subsection{Why a weight is necessary}

A key structural point is that translations are \emph{not} central in $\Aff(1)$:
if $g(x)=\Phi x + a$ and $T_c(x)=x+c$ then
\begin{equation}\label{eq:conjugation}
g^{-1}\,T_c\,g \;=\; T_{c/\Phi}.
\end{equation}
Therefore, when one ``adds up'' face translations from different places on the torus,
each face constant must be transported to a common basepoint, producing a scale-dependent weight.

\subsection{Discrete Stokes formula (toy model)}

Let $\Sigma$ be an oriented union of faces (a $2$--chain) in the torus CW--complex, and suppose its boundary is the loop
$\partial\Sigma=\gamma$ (so $\gamma$ is contractible).
Choose a base vertex $v_0$ and, for each face $f\subset \Sigma$, choose an \emph{anchor vertex} $v_f$
(e.g.\ the lower-left corner of the face in a chosen fundamental domain),
together with a path $\pi_f$ from $v_0$ to $v_f$ in a spanning tree.
Write $\rho(\pi_f)(x)=\Phi_f x + A_f$.

Then the boundary holonomy is the product of conjugated face holonomies,
and using \eqref{eq:conjugation} (together with the fact that translations commute) one obtains the discrete Stokes law:
\begin{equation}\label{eq:discrete-stokes}
\rho(\gamma)(x)=x+\sum_{f\subset \Sigma}\epsilon_f\,\frac{c(f)}{\Phi_f},
\end{equation}
where $\epsilon_f=\pm 1$ records whether the face orientation agrees with $\Sigma$.

This is the discrete analogue of the AEG ``triple identity'' in the accumulative commutative space,
where the torsion is recovered by an $e^M$--weighted area integral and an equivalent boundary integral
(AEG Eq.\ (74)). 

\section{Four representative examples}

We now exhibit four explicit loops, one in each of the four classes:
\[
\text{(contractible / noncontractible)} \times \text{(identity / non-identity)}.
\]
We represent vertices by indices $(i,j)\in(\mathbb{Z}/4\mathbb{Z})^2$,
where $i$ tracks the $p$--circle breakpoints in order $0,1,2,4$ and similarly for $j$ and $q$.
A positive step $i\to i+1$ uses the $p$--edge type in the order
$\AddOne,\MulTwo,\SubTwo,\DivTwo$, and similarly for $j\to j+1$ in the $q$--direction.

\subsection{Example A: contractible \& identity (a $1\times 3$ strip)}

Let $\Sigma_A$ be the $1\times 3$ strip consisting of the three faces with anchors $(0,1)$, $(0,2)$, $(0,3)$.
Its boundary loop $\gamma_A=\partial\Sigma_A$ has edge word
\[
\gamma_A:\quad
\MulTwo\,\SubTwo\,\DivTwo\,\AddOne\,\MulTwo\,\AddTwo\,\DivTwo\,\SubOne,
\]
and one checks directly that $\rho(\gamma_A)=\Id$.

The three face types are $(\AddOne,\MulTwo)$, $(\AddOne,\SubTwo)$, $(\AddOne,\DivTwo)$,
so $c=\tfrac12,0,-1$. Using the basepoint at $(0,1)$, the anchors have scales
$\Phi=1,2,2$, hence weights $1,\tfrac12,\tfrac12$.
Therefore
\[
A_{\gamma_A}=\frac{\tfrac12}{1}+\frac{0}{2}+\frac{-1}{2}=0,
\]
in agreement with $\rho(\gamma_A)=\Id$.

\subsection{Example B: contractible \& non-identity (a $2\times 2$ block)}

Let $\Sigma_B$ be the $2\times 2$ block consisting of four faces with anchors $(0,1),(0,2),(1,1),(1,2)$.
Its boundary loop $\gamma_B=\partial\Sigma_B$ has holonomy
\[
\rho(\gamma_B)(x)=x+1.
\]
Only two of the four faces have nonzero torsion:
$c(\AddOne,\MulTwo)=\tfrac12$ at anchor $(0,1)$ with $\Phi=1$, and
$c(\MulTwo,\SubTwo)=1$ at anchor $(1,2)$ with $\Phi=2$.
Hence the weighted torsion sum gives
\[
A_{\gamma_B}=\frac{\tfrac12}{1}+\frac{1}{2}=1,
\]
again matching the boundary offset.

\subsection{Example C: noncontractible \& identity (a meridian loop)}

Consider the meridian loop $\gamma_C$ going once around the $q$--circle at fixed $p$:
\[
\gamma_C:\quad \AddOne\,\MulTwo\,\SubTwo\,\DivTwo.
\]
As an arithmetic expression this is exactly
\[
x \xmapsto{\AddOne} x+1 \xmapsto{\MulTwo} 2(x+1)
\xmapsto{\SubTwo} 2(x+1)-2 \xmapsto{\DivTwo} \frac{2(x+1)-2}{2}=x,
\]
so $\rho(\gamma_C)=\Id$.
Topologically, $\gamma_C$ represents a generator of $\pi_1(T^2)$.

\subsection{Example D: noncontractible \& non-identity (a longitude with a scaled bubble)}

Start at a basepoint, move to the line $q=2$, traverse once around the $p$--circle (a longitude),
and insert a small contractible ``bubble'' around the face of type $(\MulTwo,\SubTwo)$ at scale $2$.
The resulting loop $\gamma_D$ is homotopic to the longitude (hence noncontractible),
but its holonomy is
\[
\rho(\gamma_D)(x)=x+\tfrac12.
\]
Indeed, the inserted bubble has face constant $c(\MulTwo,\SubTwo)=1$,
and since it occurs at scale $\Phi=2$ relative to the basepoint,
its transported contribution is $1/2$ by \eqref{eq:conjugation}.
Equivalently, $\gamma_D$ differs from the identity longitude by a weighted torsion flux through that single face.

\section{Summary and outlook}

This $4\times 4$ torus is a compact ``finite state'' model of the affine non-commutativity phenomena
central to arithmetic expression geometry:
\begin{itemize}[leftmargin=18pt]
\item The $1$--skeleton paths form a path groupoid whose evaluation lands in an affine group, mirroring the AEG setup of
threadlike expressions and $\Aff(1)$ holonomy.
\item The face constants $c(P,Q)$ behave as a discrete torsion density measuring pairwise non-commutativity of generators.
\item The semidirect product structure forces a \emph{weight} (a scale transport) in any Stokes-type sum:
offsets are not plain sums of face constants, but conjugated sums.
\end{itemize}
Two natural next questions, now phrased purely in this toy model, are:
(i)~how to classify loops simultaneously by homotopy class in $T^2$ and by affine holonomy $(\Phi_\gamma,A_\gamma)$,
and (ii)~how these invariants interact with the ``commutativity vs.\ noncommutativity'' pattern (zero vs.\ nonzero faces)
encoded by Table~\ref{tab:torsion-table}.

\end{document}
