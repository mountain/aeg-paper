\documentclass[11pt]{article}
\usepackage{amsmath,amssymb,amsthm}
\usepackage{hyperref}

\title{From Algorithmic Thermodynamics to Arithmetic Expression Geometry:\\
An example of contactomorphism}

\author{Mingli Yuan\thanks{Draft note; comments welcome.}}

\date{\today}

\begin{document}
\maketitle

\section*{Algorithmic thermodynamics and arithmetic expression geometry}

In this note we spell out a concrete contactomorphism between the
algorithmic thermodynamics of Baez--Stay and the contact geometry
arising from arithmetic expression geometry (AEG).  We also explain
how a natural ``weight'' function
\[
  W(A,M) = A + \alpha M
\]
on arithmetic expressions leads to a partition function that is a
special case of the Baez--Stay partition function, once we restrict to
a suitable family of prefix-free programs.

\subsection*{1. Algorithmic thermodynamics}

Fix a universal prefix-free Turing machine $U$, and let
$X = \mathrm{dom}(U)$ be the set of halting programs.  Baez and Stay
treat $X$ as a set of microstates and consider the following
observables on $X$:
\begin{itemize}
  \item $E(x)$: the logarithm of the runtime of program $x$,
  \item $V(x)$: the length $|x|$ of $x$,
  \item $N(x)$: the output of $x$ as a natural number.
\end{itemize}
These play the roles of energy, volume, and particle number,
respectively.\footnote{See \cite{BaezStay} for details.}

For parameters $T$ (algorithmic temperature), $P$ (algorithmic
pressure), and $\mu$ (algorithmic chemical potential), the Gibbs
ensemble is defined by the partition function
\begin{equation}
  Z(T,P,\mu)
  = \sum_{x \in X}
    \exp\!\left(
      -\frac{1}{T}\bigl(E(x) + P V(x) - \mu N(x)\bigr)
    \right).
  \label{eq:alg-partition}
\end{equation}
Equivalently, in terms of the conjugate variables
\[
  \beta = \frac{1}{T},\qquad
  \gamma = \frac{P}{T},\qquad
  \delta = -\frac{\mu}{T},
\]
one can write
\[
  Z(\beta,\gamma,\delta)
    = \sum_{x\in X} e^{-\beta E(x) - \gamma V(x) - \delta N(x)}.
\]

Let $p(x)$ be the corresponding Gibbs probability measure.
Its entropy
\[
  S = - \sum_{x\in X} p(x)\,\ln p(x)
\]
is a function of the macroscopic variables $(E,V,N)$ (or
equivalently of $(T,P,\mu)$).  A standard calculation in statistical
mechanics yields the thermodynamic relation
\begin{equation}
  dS
  = \frac{1}{T}\,dE + \frac{P}{T}\,dV - \frac{\mu}{T}\,dN,
\end{equation}
or in the more familiar form
\begin{equation}
  dE = T\,dS - P\,dV + \mu\,dN.
  \label{eq:alg-first-law}
\end{equation}
This suggests considering the \emph{algorithmic thermodynamic contact
form}
\begin{equation}
  \alpha_{\mathrm{alg}} = dE - T\,dS + P\,dV - \mu\,dN,
\end{equation}
whose vanishing encodes the first law \eqref{eq:alg-first-law} along
equilibrium submanifolds.

\subsection*{2. AEG contact geometry and ordinary thermodynamics}

Arithmetic expression geometry (AEG) is built on a 3--dimensional
manifold with coordinates $(u,v,a)$ and contact form
\begin{equation}
  \alpha_{\mathrm{AEG}} = da - \mu\,du - \lambda a\,dv,
  \label{eq:aeg-contact}
\end{equation}
where $\lambda$ and $\mu$ are fixed real parameters that encode
additive and multiplicative generators.\footnote{See
  \cite{YuanThermal} for a detailed development.}

In the associated thermodynamic correspondence, one introduces a
standard thermodynamic phase space with coordinates
$(S,V;U,T,p)$ and contact form
\begin{equation}
  \alpha_{\mathrm{TD}} := dU + p\,dV - T\,dS.
\end{equation}
The dictionary
\begin{equation}
  (S,V;U,T,p) = (v,-u;\,a,\lambda a,\mu)
  \label{eq:aeg-thermo-dict}
\end{equation}
defines a map
\[
  \Phi\colon (u,v,a) \longmapsto (S,V;U,T,p)
          = (v,-u;\,a,\lambda a,\mu),
\]
for which one checks
\begin{align}
  \Phi^*(\alpha_{\mathrm{TD}})
   &= d a + \mu\,d(-u) - (\lambda a)\,dv \\
   &= da - \mu\,du - \lambda a\,dv \\
   &= \alpha_{\mathrm{AEG}}.
\end{align}
Thus $(\mathbb{R}^3,\alpha_{\mathrm{AEG}})$ is contactomorphic to the
thermodynamic contact manifold restricted to the image of
\eqref{eq:aeg-thermo-dict}.

\subsection*{3. A contactomorphism to algorithmic thermodynamics}

To connect with algorithmic thermodynamics, we restrict to a sector of
fixed $N$ and suppress the $\mu\,dN$ term.  On the submanifold where
$N$ is constant, the algorithmic contact form reduces to
\begin{equation}
  \alpha_{\mathrm{alg}}|_{dN=0} = dE - T\,dS + P\,dV.
\end{equation}
This is identical in shape to $\alpha_{\mathrm{TD}}$, with the obvious
identifications $U \leftrightarrow E$ and $p \leftrightarrow P$.
Therefore the same dictionary \eqref{eq:aeg-thermo-dict} yields a
contactomorphism
\[
  \Phi_{\mathrm{alg}}\colon (u,v,a)
    \longmapsto (S,V;E,T,P)
    = (v,-u;\,a,\lambda a,\mu)
\]
from $(\mathbb{R}^3,\alpha_{\mathrm{AEG}})$ onto the 3--dimensional
submanifold of the algorithmic thermodynamic phase space with fixed
$N$:
\begin{align}
  \Phi_{\mathrm{alg}}^*(\alpha_{\mathrm{alg}}|_{dN=0})
    &= d a - (\lambda a)\,dS + P\,dV \\
    &= d a - (\lambda a)\,dv + \mu\,d(-u) \\
    &= da - \mu\,du - \lambda a\,dv \\
    &= \alpha_{\mathrm{AEG}}.
\end{align}
In this sense, the AEG contact manifold is naturally realized as a
contact submanifold of the algorithmic thermodynamic contact manifold.

\subsection*{4. Weights, first-hit expressions, and prefix-free codes}

We now explain how a natural choice of microstates and a weight
function $W$ on AEG leads to a partition function that is a special
case of \eqref{eq:alg-partition}.

Each arithmetic expression carries a pair of ``charges''
\[
  (A,M) = (\text{additive count},\ \text{multiplicative count}),
\]
counting the number of additive and multiplicative steps in its
discrete generation.  Fix a positive slope parameter $\alpha>0$ and
define the weight
\begin{equation}
  W(A,M) = A + \alpha M.
\end{equation}
Geometrically, $W$ defines a foliation of the $(A,M)$--lattice by
affine lines of slope $-\alpha$.

For a threshold $C\in\mathbb{R}$, define the \emph{first-hit set}
$F_C$ as the set of expressions $e$ such that
\[
  W(e) \ge C
  \quad\text{and}\quad
  W(e') < C
  \quad\text{for every proper prefix }e'\prec e.
\]
Here $e'\prec e$ means that $e'$ is obtained from $e$ by cutting off
some trailing generators in the discrete construction process.  By
construction, $F_C$ is prefix-free with respect to this prefix order:
no element of $F_C$ is a prefix of another.  Indeed, if $e\in F_C$ and
$e\prec e''$, then any path from the root to $e''$ must pass through
$e$, at which point $W$ has already crossed the threshold.

In a binary branching model where each generating step increases $W$
by one unit, the first-hit set $F_C$ satisfies the Kraft inequality
\[
  \sum_{e\in F_C} 2^{-W(e)} \le 1,
\]
so $W(e)$ plays the role of a codeword length.  Consequently there
exists a prefix-free Turing machine (a Chaitin machine) whose domain
is in bijection with $F_C$, with the codeword length of the program
corresponding to $e$ equal to $W(e)$ (up to an additive constant).

More concretely, given any Turing machine $M$ with domain
$\mathrm{dom}(M)$, one can form a blank-endmarker machine $M'$ whose
domain consists of words of the form $x\diamond$ for $x\in\mathrm{dom}(M)$;
this domain is prefix-free, and the blank symbol $\diamond$ serves as
a self-delimiting terminator \cite{StayConcrete}.  Applying such
constructions to $F_C$ yields a universal prefix-free machine whose
halting programs realize the same multiset of code lengths as
$\{W(e):e\in F_C\}$.

\subsection*{5. Partition functions and the collapse of $E$ and $V$}

Restrict attention to the microstates $e\in F_C$.  For each such $e$,
let $x_e$ be the corresponding program on our chosen prefix-free
machine.  We can then define the algorithmic observables
\begin{equation}
  E(x_e) = c_E\,W(e),\qquad
  V(x_e) = c_V\,W(e),\qquad
  N(x_e) = 0,
\end{equation}
for some fixed positive constants $c_E,c_V$ encoding the choice of
units.  (Operationally, one may choose the runtime of $x_e$ to grow
exponentially in $W(e)$, so that $\log\text{runtime}\propto W(e)$.)

The Baez--Stay partition function, restricted to these microstates and
with $\delta=0$, becomes
\begin{align}
  Z_{\mathrm{alg}}(\beta,\gamma)
    &= \sum_{e\in F_C}
         \exp\!\bigl(-\beta E(x_e)-\gamma V(x_e)\bigr) \\
    &= \sum_{e\in F_C}
         \exp\!\bigl(-(\beta c_E + \gamma c_V) W(e)\bigr).
\end{align}
If we now introduce the effective inverse temperature
\[
  \kappa := \beta c_E + \gamma c_V
\]
and define the AEG partition function
\begin{equation}
  Z_{\mathrm{AEG}}(\beta_W)
    := \sum_{e\in F_C} e^{-\beta_W W(e)},
\end{equation}
then on the restricted ensemble we have the identity
\begin{equation}
  Z_{\mathrm{alg}}(\beta,\gamma)
    = Z_{\mathrm{AEG}}(\beta_W)
    \quad\text{whenever}\quad
    \beta_W = \kappa.
\end{equation}
In other words, the AEG partition function is obtained from the
algorithmic thermodynamic partition function by restricting to a
subsystem where the energy $E$ and program length $V$ are both linear
functions of the AEG weight $W$.

In this toy model, the microscopic information carried by $W(e)$ is
fully preserved---each microstate $e\in F_C$ remains distinct.  What
is lost, from the point of view of algorithmic thermodynamics, is one
macroscopic degree of freedom: the ability to vary the tradeoff
between log-runtime and program length independently.  This loss is
reflected in the fact that the pair $(\beta,\gamma)$ only enters the
restricted partition function through the single combination
$\kappa = \beta c_E + \gamma c_V$.

From the contact-geometric viewpoint, the AEG contact manifold
describes a 3--dimensional contact submanifold of the full
algorithmic thermodynamic contact manifold, obtained by fixing $N$ and
imposing the relations
\[
  E = a,\qquad V = -u,\qquad S = v.
\]
The contactomorphism $\Phi_{\mathrm{alg}}$ makes this identification
explicit.

\begin{thebibliography}{9}

\bibitem{BaezStay}
J.~C.~Baez and M.~Stay,
\newblock Algorithmic Thermodynamics,
\newblock \emph{arXiv:1010.2067}.

\bibitem{YuanThermal}
M.~Yuan,
\newblock Arithmetic Expression Geometry and its Thermodynamics Correspondence,
\newblock draft manuscript.

\bibitem{StayConcrete}
M.~Stay,
\newblock Very Simple Chaitin Machines for Concrete AIT,
\newblock \emph{Fundamenta Informaticae}, arXiv:cs/0508056.

\end{thebibliography}

\end{document}
