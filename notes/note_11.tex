\documentclass{article}
\usepackage[utf8]{inputenc}
\usepackage{amsmath, amssymb, amsthm}
\usepackage{geometry}
\usepackage{xcolor}
\usepackage{hyperref}
\usepackage{booktabs}

\geometry{a4paper, margin=1in}

\newtheorem{theorem}{Theorem}
\newtheorem{definition}{Definition}
\newtheorem{example}{Example}
\newtheorem{proposition}{Proposition}
\newtheorem{remark}{Remark}

\title{Notes on the Correspondence between Fox Calculus, \\ Alexander Modules, and Arithmetic Expression Geometry}
\author{Mingli Yuan (with Gemini as Sparring Partner)}
\date{November 26, 2025}

\begin{document}

\maketitle

\begin{abstract}
These notes summarize the theoretical bridge established between the algebraic topology of knot groups (Fox calculus, Alexander modules) and the geometric framework of Arithmetic Expression Geometry (AEG). We demonstrate that the \textit{arithmetic torsion} defined in AEG is the geometric realization of the Fox derivative in the group ring. Furthermore, we show that the \textit{Transformed Grid} structure in AEG corresponds precisely to the Alexander polynomial of the Baumslag-Solitar group $BS(1,2)$. Finally, we unify these concepts using the matrix representation of the affine group $\text{Aff}(1)$, rigorously verifying the theory with the Figure-Eight knot ($4_1$).
\end{abstract}

\section{Introduction}
In the developing theory of Arithmetic Expression Geometry (AEG), a central concept is the flow equation and the resulting \textit{arithmetic torsion}, which arises from the non-commutativity of addition and multiplication operations.

This note formalizes the intuition that the ``Leibniz-like'' behavior of the AEG flow equation is not merely analogous to, but isomorphic to, the free differential calculus (Fox Calculus) on group rings. By mapping arithmetic expressions to elements of a group ring $\mathbb{Z}[G]$, we recover geometric invariants from algebraic properties.

\section{Fox Calculus and the Geometric Area Formula}

\subsection{Algebraic Setup}
Let $F_2 = \langle x, y \rangle$ be the free group where $x$ represents the additive generator (corresponding to displacement $\mu$) and $y$ represents the multiplicative generator (corresponding to scaling $e^\lambda$). The Fox derivative $\frac{\partial}{\partial x}: \mathbb{Z}[F_2] \to \mathbb{Z}[F_2]$ satisfies the Leibniz rule:
\begin{equation}
    \frac{\partial (uv)}{\partial x} = \frac{\partial u}{\partial x} + u \frac{\partial v}{\partial x}
\end{equation}

\subsection{The Fundamental Commutator Calculation}
Consider the commutator word $w$ representing the simplest closed loop in the AEG grid: ``move right, move up, move left, move down.'' In standard path composition notation ($p_1 \cdot p_2$ means $p_2 \circ p_1$), this corresponds to the word $w = x y x^{-1} y^{-1}$.

Computing the Fox derivative with respect to $x$:
\begin{align*}
    \frac{\partial w}{\partial x} &= \frac{\partial (x y x^{-1} y^{-1})}{\partial x} \\
    &= 1 + x \frac{\partial y}{\partial x} + xy \frac{\partial x^{-1}}{\partial x} + xyx^{-1} \frac{\partial y^{-1}}{\partial x} \\
    &= 1 + 0 + xy(-x^{-1}) + 0 \\
    &= 1 - xyx^{-1}
\end{align*}

\subsection{Geometric Mapping (Abelianization)}
To bridge this result to the \textit{Accumulative Commutative Space} (ACS) defined in AEG, we apply the abelianization map $\psi: F_2 \to \mathbb{Z} \oplus \mathbb{Z}$. In the geometric limit, $x$ and $y$ commute locally.
\begin{equation}
    \psi\left( \frac{\partial w}{\partial x} \right) = 1 - y
\end{equation}
In the AEG context, the operator $y$ corresponds to multiplication by $e^\lambda$. For an infinitesimal step, $y \approx 1 + \lambda$. Thus, the defect coefficient is:
\begin{equation}
    1 - (1 + \lambda) = -\lambda
\end{equation}
Since this defect is attached to the $x$-generator (which carries magnitude $\mu$), the total geometric displacement (torsion) is:
\begin{equation}
    \tau \approx \mu \cdot (-\lambda) = -\mu\lambda
\end{equation}
This perfectly recovers the differential area formula derived in the AEG paper via geometric arguments:
\begin{equation}
    d\tau = \mu\lambda \, du \, dv
\end{equation}
(Note: The sign depends on the orientation convention of the path traversal).

\section{The Baumslag-Solitar Correspondence: $BS(1,2)$}

\subsection{The Geometric Observation}
In the AEG paper, the ``Transformed Grid'' is constructed via the conformal map $w = -1/z$. It is noted that to preserve chirality, the effective multiplication factor must be inverted to $1/2$. This structure is identified as a geometric realization of the Baumslag-Solitar group $BS(1,2)$.

\subsection{The Alexander Polynomial Calculation}
We verify this geometric intuition algebraically. The group presentation is:
\begin{equation}
    BS(1, 2) = \langle x, y \mid y^{-1} x y = x^2 \rangle
\end{equation}
Let the relator be $r = y^{-1} x y x^{-2}$. We compute the Jacobian element $\frac{\partial r}{\partial x}$:
\begin{align*}
    \frac{\partial (y^{-1} x y x^{-2})}{\partial x} &= y^{-1} \left( \frac{\partial x}{\partial x} + x \frac{\partial y}{\partial x} + xy \frac{\partial x^{-2}}{\partial x} \right) \\
    &= y^{-1} ( 1 + 0 + xy(-x^{-1} - x^{-2}) ) \\
    &= y^{-1} ( 1 - xyx^{-1} - xyx^{-2} )
\end{align*}
Using the group relation $y^{-1}xy = x^2$:
\begin{equation}
    \frac{\partial r}{\partial x} = y^{-1} - x - 1
\end{equation}
Applying the abelianization map $\psi$ (where $x \to 1$ and $y \to t$):
\begin{equation}
    \psi\left( \frac{\partial r}{\partial x} \right) = t^{-1} - 1 - 1 = t^{-1} - 2
\end{equation}
This yields the Alexander polynomial (normalized):
\begin{equation}
    \Delta(t) \doteq 1 - 2t
\end{equation}

\subsection{Conclusion}
The root of the Alexander polynomial is $t = 1/2$. This matches exactly the ``effective multiplication factor'' of $1/2$ derived geometrically in the AEG paper. This confirms that the geometric scaling factor in AEG is an intrinsic topological invariant of the underlying group structure.

\section{Case Study: The Figure-Eight Knot ($4_1$)}

The correspondence extends beyond toy models to non-trivial knot topologies. We apply the Affine Group representation theory to the Figure-Eight knot ($4_1$), verifying the AEG calculation results.

\subsection{Representation and Relator}
Based on the AEG mapping defined for $4_1$, we define a representation $\rho: G(4_1) \to \text{Aff}(1)$ mapping generators to $2 \times 2$ matrices over $\mathbb{C}[t, t^{-1}]$:
\begin{itemize}
    \item Multiplicative generator ($a \mapsto \otimes_t$): $M_a = \begin{pmatrix} t & 0 \\ 0 & 1 \end{pmatrix}$
    \item Additive generator ($b \mapsto \oplus_1$): $M_b = \begin{pmatrix} 1 & 1 \\ 0 & 1 \end{pmatrix}$
\end{itemize}
The standard relator for $4_1$ is $r = a b b b a B A A B$, where $A=a^{-1}, B=b^{-1}$.

\subsection{Matrix Computation of Arithmetic Torsion}
We compute the matrix product $\rho(r) = M_a M_b^3 M_a M_B M_A^2 M_B$.

\noindent \textbf{Step 1: Inner Segment ($M_B M_A^2 M_B$)}
Corresponds to the arithmetic operation sequence $B(A(A(B(x))))$.
\begin{equation}
    M_B (M_A^2 M_B) = \begin{pmatrix} 1 & -1 \\ 0 & 1 \end{pmatrix} \begin{pmatrix} t^{-2} & -t^{-2} \\ 0 & 1 \end{pmatrix} = \begin{pmatrix} t^{-2} & -t^{-2} - 1 \\ 0 & 1 \end{pmatrix}
\end{equation}

\noindent \textbf{Step 2: Middle Segment (Apply $M_a$)}
\begin{equation}
    M_a \begin{pmatrix} t^{-2} & -t^{-2} - 1 \\ 0 & 1 \end{pmatrix} = \begin{pmatrix} t^{-1} & -t^{-1} - t \\ 0 & 1 \end{pmatrix}
\end{equation}

\noindent \textbf{Step 3: Outer Segment (Apply $M_a M_b^3$)}
Adding 3 to the translation term, then scaling by $t$:
\begin{equation}
    \rho(r) = \begin{pmatrix} t & 0 \\ 0 & 1 \end{pmatrix} \begin{pmatrix} 1 & 3 \\ 0 & 1 \end{pmatrix} \begin{pmatrix} t^{-1} & -t^{-1} - t \\ 0 & 1 \end{pmatrix}
    = \begin{pmatrix} 1 & t(-t^{-1} - t + 3) \\ 0 & 1 \end{pmatrix}
\end{equation}

\subsection{Verification}
The translational component (upper-right element) represents the arithmetic torsion $D(r)$:
\begin{equation}
    D(r) = -1 - t^2 + 3t = -(t^2 - 3t + 1) = -\Delta_{4_1}(t)
\end{equation}
This derivation rigorously confirms that the arithmetic torsion calculated via AEG flow is exactly the Alexander polynomial, arising as the cohomological obstruction in the $\text{Aff}(1)$ representation.

\section{Synthesis: The $\text{Aff}(1)$ Representation and Cohomology}

To unify the ``flow equation'' and ``torsion'' concepts, we utilize the matrix representation in the affine group $\text{Aff}(1)$.

\subsection{Magnus Embedding}
Let $\rho: F_2 \to \text{Aff}(1)$ be a representation mapping a group element $g$ to an upper triangular matrix:
\begin{equation}
    \rho(g) = \begin{pmatrix} \phi(g) & D(g) \\ 0 & 1 \end{pmatrix}
\end{equation}
Here, $\phi(g)$ represents the multiplicative scaling (Abelianized), and $D(g)$ represents the additive translation.

\subsection{Recovering the Leibniz Rule}
Consider the product of two elements $uv$:
\begin{equation}
    \rho(uv) = \rho(u)\rho(v) = \begin{pmatrix} \phi(u) & D(u) \\ 0 & 1 \end{pmatrix} \begin{pmatrix} \phi(v) & D(v) \\ 0 & 1 \end{pmatrix} = \begin{pmatrix} \phi(u)\phi(v) & \phi(u)D(v) + D(u) \\ 0 & 1 \end{pmatrix}
\end{equation}
The term in the upper-right corner gives the composition law for the translation component:
\begin{equation}
    D(uv) = D(u) + \phi(u)D(v)
\end{equation}
This is exactly the Fox calculus Leibniz rule used in AEG.

\subsection{Arithmetic Holonomy and $H^1$}
The function $D: G \to M$ is a 1-cocycle in group cohomology $H^1(G, M)$, where $G$ acts on the module $M$ via $\phi$.
\begin{itemize}
    \item \textbf{Arithmetic Torsion} corresponds to the non-triviality of this 1-cocycle around closed loops.
    \item \textbf{Arithmetic Holonomy} can be defined as the matrix $\rho(\gamma)$ for a closed path $\gamma$. If $\rho(\gamma) = \begin{pmatrix} 1 & \tau \\ 0 & 1 \end{pmatrix}$, then $\tau$ is the accumulated torsion.
\end{itemize}

\section{Summary}
The correspondence establishes a rigorous dictionary:
\begin{center}
\begin{tabular}{|c|c|}
\hline
\textbf{Arithmetic Expression Geometry (AEG)} & \textbf{Algebraic Topology / Group Theory} \\
\hline
Flow Equation / Leibniz-like Rule & Fox Derivative / 1-Cocycle condition \\
\hline
Effective Multiplication Factor ($1/2$) & Root of Alexander Polynomial ($\Delta(t) = 1-2t$) \\
\hline
Arithmetic Torsion $\tau$ & Element of $H^1$ / Translational part of $\text{Aff}(1)$ \\
\hline
Accumulative Commutative Space & Abelianized Homology Group / Alexander Module \\
\hline
\end{tabular}
\end{center}

\end{document}