\documentclass{article}

\usepackage{amsmath}
\usepackage{amsthm}
\usepackage{amssymb}
\usepackage{amsfonts}
\usepackage{mathtools}

\usepackage{arxiv/arxiv}
\usepackage{styles/quiver}

\usepackage[utf8]{inputenc} % allow utf-8 input
\usepackage[T1]{fontenc}    % use 8-bit T1 fonts
\usepackage{hyperref}       % hyperlinks
\usepackage{url}            % simple URL typesetting
\usepackage{booktabs}       % professional-quality tables
\usepackage[english]{babel}
\usepackage{nicefrac}       % compact symbols for 1/2, etc.
\usepackage{microtype}      % microtypography
\usepackage{graphicx}
\usepackage{stmaryrd}

\usepackage{tikz}
\usetikzlibrary{angles,fit,arrows,calc,math,intersections,through,backgrounds}
\usepackage{qtree}

\usepackage{listings}
\lstset{
  basicstyle=\itshape,
  xleftmargin=3em,
  literate={->}{$\rightarrow$}{2}
           {α}{$\alpha$}{1}
           {δ}{$\delta$}{1}
}

\usepackage{xstring}
\usepackage{stmaryrd}
\usepackage{wasysym}
\usepackage{textcomp}
\usepackage{blindtext}
\usepackage{subfiles}

\newtheorem{definition}{Definition}
\numberwithin{definition}{section}
\newtheorem{lemma}{Lemma}
\numberwithin{lemma}{section}
\newtheorem{proposition}{Proposition}
\numberwithin{proposition}{section}
\newtheorem{corollary}{Corollary}
\numberwithin{corollary}{section}
\newtheorem{theorem}{Theorem}
\numberwithin{theorem}{section}
\newtheorem{remark}{Remark}
\numberwithin{remark}{section}

\DeclareMathSymbol{\mathinvertedexclamationmark}{\mathclose}{operators}{'074}
\DeclareMathSymbol{\mathexclamationmark}{\mathclose}{operators}{'041}
\makeatletter
\newcommand{\raisedmathinvertedexclamationmark}{%
  \mathclose{\mathpalette\raised@mathinvertedexclamationmark\relax}%
}
\newcommand{\raised@mathinvertedexclamationmark}[2]{%
  \raisebox{\depth}{$\m@th#1\mathinvertedexclamationmark$}%
}
\begingroup\lccode`~=`! \lowercase{\endgroup
  \def~}{\@ifnextchar`{\raisedmathinvertedexclamationmark\@gobble}{\mathexclamationmark}}
\mathcode`!="8000
\makeatother

\DeclareMathOperator{\arcsinh}{arcsinh}
\DeclareMathOperator{\Tr}{Tr} % Example, if trace was needed

\title{Arithmetic expression geometry \\ and its thermodynamics correspondence}

%\date{Decmber 8, 2022}	% Here you can change the date presented in the paper title
%\date{} 				% Or removing it

\author{
Mingli Yuan \\
\texttt{mingli.yuan@gmail.com}
}

% Uncomment to remove the date
%\date{}

% Uncomment to override  the `A preprint' in the header
\renewcommand{\headeright}{A draft}
\renewcommand{\undertitle}{A draft}

\begin{document}
\maketitle

\begin{abstract}
This paper explores the geometric structure of arithmetic expressions and establishes a correspondence with thermodynamic systems.
\end{abstract}

\keywords{arithmetic expressions, hyperbolic geometry, thermodynamics, contact structure}

\setcounter{tocdepth}{2}
\tableofcontents

\section{Arithmetic expressions space}\label{sec:aes}

Arithmetic expressions space (AES) is a geometric structure that encapsulates the properties and operations of arithmetic expressions.
It serves as a foundational framework for understanding the relationships between different expressions and their evaluations,
each point in AES represents a unique arithmetic expression, and their evaluations are mapped to geometric flows.
In the central part of AES, we define a differential equation that allows us to explore the infinitesimal changes in expressions and their evaluations.


\subsection{Derivation of the flow equation}\label{sec:equation}

Consider an infinitesimal generating process on a Riemannian surface $M$ using two generators:
one for an additional action $\mu$ and the other for a multiplicative action $e^\lambda$.
These two generators are perpendicular.
This generation process produces an assignment $A: M \to R$ over the surface.

For any point with an assignment $a_0$, if we consider a movement of distance $\epsilon$ in a direction with angle $\theta$
over a time period of $\delta$, we can establish the following:

\[
    a_{\delta} = (a_0 + \mu \epsilon \cos \theta)e^{\lambda \epsilon \sin \theta}
\]

or

\[
    a_{\delta} = a_0 e^{\lambda \epsilon \sin \theta} + \mu \epsilon \cos \theta
\]

Both formula can be simplified to the same result:

\[
    a_{\delta} = a_0 + \epsilon (a_0 \lambda \sin \theta + \mu \cos \theta)
\]

Then, we have the following equation:

\[
    \frac{1}{\delta} (a_{\delta} - a_0) = \frac{\epsilon}{\delta} (\mu \cos \theta + x_0 \lambda \sin \theta)
\]

When both $\delta$ and $\epsilon$ are towards zero, we get $da / dt$, and hence

\[
    \frac{da}{dt} = u (\mu \cos \theta + a \lambda \sin \theta)
\]

Or, we can change it to another form

\begin{equation}
    \frac{da}{ds} = \mu \cos \theta + a \lambda \sin \theta\label{eq:flow}
\end{equation}

We name this equation~\eqref{eq:flow} as the flow equation.

The left side of this equation is governed by the distance structure, while the right side is governed by the angle structure.
So that the isometrics of the surface keep the flow equation~\eqref{eq:flow}.

We can also get a direct formal solution of the flow equation~\eqref{eq:flow}(details in Appendix~\ref{sec:directformalsolution}).

\begin{equation}
   a = (a_0 + \frac{\mu}{\lambda} \cot \theta) e^{\lambda s \sin \theta} - \frac{\mu}{\lambda} \cot \theta\label{eq:solution}
\end{equation}

\subsection{Discrete generating}\label{subsec:discrete-generating}

In section~\ref{subsec:generated-structure}, we have discussed a discrete generating process.
Since flow equation governs an infinitesimal generating process,
we will show the above discrete generating process can be emerged from the solution of the flow equation~\eqref{eq:solution} naturally.
We expand the formula by the Taylor series:

\[
    a =  a_0 e^{\lambda s \sin \theta} + \frac{\mu}{\lambda} [1 + \lambda s \sin \theta + \frac{1}{2!} (\lambda s \sin \theta)^2  + \frac{1}{3!} (\lambda s \sin \theta)^3 + \cdots - 1] \cot \theta
\]

Change the formula slightly:
\[
    a = a_0 e^{\lambda s \sin \theta} + \mu s \cos \theta + \frac{\mu}{\lambda} \sin \theta \cos \theta (\frac{\lambda^2s^2}{2!} + \frac{\lambda^3s^3}{3!} \sin \theta + \frac{\lambda^4s^4}{4!} \sin^2 \theta + \cdots)
\]

By the formula of double angle, we have
\[
    a = a_0 e^{\lambda s \sin \theta} + \mu s \cos \theta + \frac{\mu}{2\lambda} \sin 2\theta (\frac{\lambda^2s^2}{2!} + \frac{\lambda^3s^3}{3!} \sin \theta + \frac{\lambda^4s^4}{4!} \sin^2 \theta + \cdots)
\]

We denote
\begin{equation}
    \Psi(s) = \frac{1}{2!} + \frac{\lambda s}{3!} \sin \theta + \frac{\lambda^2 s^2}{4!} \sin^2 \theta + \cdots
\end{equation}

Then we have
\begin{equation}
    a = a_0 e^{\lambda s \sin \theta} + \mu s \cos \theta + \frac{\mu\lambda}{2} s^2 \Psi(s) \sin 2\theta
\end{equation}

This formula gives the discrete generating process, when $\theta = \frac{k \pi}{2}, k = 0, 1, 2, 3\cdots, s = 0, 1, 2, 3\cdots$, we have

\begin{equation}
    a = a_0 e^{\lambda s \sin \theta} + \mu s \cos \theta
\end{equation}

Especially, we have the following four cases:
\begin{itemize}
\item $\theta = 0$: $a_s = a_0 + \mu s$
\item $\theta = \frac{\pi}{2}$: $a_s = a_0 e^{\lambda s}$
\item $\theta = \pi$: $a_s = a_0 - \mu s$
\item $\theta = \frac{3 \pi}{2}$: $a_s = a_0 e^{- \lambda s} $
\end{itemize}

This result is straightforward, but it demonstrates that the infinitesimal generating process is consistent with the discrete generating process.
And this expands our toolset, enabling us to explore the interplay between discrete and infinitesimal generating processes.

\subsection{The contour-gradient form of flow equation}\label{subsec:the-contour-gradient-form}

It is easy to derive the contour equation in the local coordinate

\begin{equation}
    \mu \cos \theta_c + a \lambda \sin \theta_c = 0\label{eq:contour}
\end{equation}

then we have

\begin{equation}
    \theta_c = - \arctan \frac{\mu}{a \lambda}\label{eq:contourangle}
\end{equation}

the contour and the gradient are perpendicular to each other

\begin{equation}
    \theta_g = \pm \frac{\pi}{2} - \arctan \frac{\mu}{a \lambda}\label{eq:gradientangle}
\end{equation}

then along $\theta_g$ we have

\begin{equation}
    \frac{da}{ds} = \mu \cos (\pm \frac{\pi}{2} - \arctan \frac{\mu}{a \lambda}) + a \lambda \sin (\pm \frac{\pi}{2} - \arctan \frac{\mu}{a \lambda})
    \label{eq:alonggradient}
\end{equation}

\begin{equation}
    \frac{da}{ds} = \pm \sqrt{\mu^2 + \lambda^2 a^2}\label{eq:grad}
\end{equation}

By introducing the right-hand rotation angle $\phi$ along the gradient direction, we can establish a local polar coordinate system based on the gradient and contour lines.
Then the growth rate of $a$ along the angle $\phi$ is

\begin{equation}
    \frac{da}{ds} = \mu \cos (\frac{\pi}{2} - \arctan \frac{\mu}{a \lambda} + \phi) + a \lambda \sin (\frac{\pi}{2} - \arctan \frac{\mu}{a \lambda} + \phi)
    \label{eq:fourfold}
\end{equation}

And the simplified equation is

\begin{equation}
    \frac{da}{ds} = \sqrt {\mu^2 + a^2 \lambda^2} \cos \phi\label{eq:contourgradient}
\end{equation}

or

\begin{equation}
    \frac{da_{\phi}}{ds_{\phi}} = \sqrt {\mu^2 + a^2 \lambda^2} \cos \phi\label{eq:contourgradient2}
\end{equation}

if we want to emphasize the path is along the angle $\phi$.

The equation~\eqref{eq:contourgradient} is the flow equation in the contour-gradient coordinate system.

Equation~\eqref{eq:contourgradient} is solvable, and we get the relation between $a$ and $s$:

\begin{equation}\label{eq:rel_a_s}
    \tanh(\lambda s \cos \phi - c) = \frac{\lambda a}{\sqrt{\mu^2 + \lambda^2 a^2}}
\end{equation}

we can further simplify the equation to

\begin{equation}
  a = \pm \frac{\mu}{\lambda} \sinh(\lambda s \cos \phi - c)\label{eq:gradevo}
\end{equation}

Under the initial condition $a = a_0$ when $s = 0$, we can get the following equation:

\begin{equation}
    a = \frac{\mu}{\lambda} \sinh(\lambda s \cos \phi + \arcsinh \frac{a_0 \lambda}{\mu})\label{eq:gradevo2}
\end{equation}

or

\begin{equation}
    a = - \frac{\mu}{\lambda} \sinh(\lambda s \cos \phi - \arcsinh \frac{a_0 \lambda}{\mu})\label{eq:gradevo3}
\end{equation}

In this coordinate system, the additional line and the multiplicative line are:

\begin{equation}
    \phi = \arccos \frac{\mu}{\sqrt {\mu^2 + a^2 \lambda^2}} \label{eq:additionalline}
\end{equation}

\begin{equation}
    \phi = \arcsin \frac{\mu}{\sqrt {\mu^2 + a^2 \lambda^2}}\label {eq:mulitiplcativeline}
\end{equation}

\subsection{Arithmetic coordinate and area formula}\label{subsec:descartes-coordinate}
We begin our exploration by examining the flow equation~\eqref{eq:flow} within the framework of a local polar coordinate system:

\begin{equation}
    \frac{da}{ds} = \mu \cos \theta + a \lambda \sin \theta
\end{equation}

In an effort to re-contextualize this equation, we set $du = \cos \theta ds$ and $dv = \sin \theta ds$,
where $du$ and $dv$ are perpendicular infinitesimal movements.
We can use these movements to construct a local Descartes coordinate system, and the first fundamental form of this system is:

\begin{equation}
    ds^2 = A^2 du^2 + B^2 dv^2
\end{equation}

Thereby this enables us to express the flow equation in a different light:

\begin{equation}
    da = \mu du + a \lambda dv
\end{equation}

Our attention now turns to the concept of arithmetic torsion, particularly at an infinitesimal level.
Delving into the interplay between two infinitesimal generating processes, we observe that:

\begin{equation}
    d\tau = (a_0 + \mu du) e^{\lambda dv} - (a_0 e^{\lambda dv} + \mu du)
\end{equation}

From this relationship, we deduce:

\begin{equation}
    d\tau = \mu du (e^{\lambda dv} - 1)
\end{equation}

This leads us to an area formula, capturing the essence of this interaction:

\begin{equation}
    d\tau = \mu \lambda du dv \label{eq:area_formula}
\end{equation}

and because the area element have a form

\begin{equation}
    dS = |AB| du dv \label{eq:area_element}
\end{equation}

Then we have
\begin{equation}
    \frac{d\tau}{\mu \lambda} = \frac{dS}{|AB|}\label{eq:area_formula2}
\end{equation}

This formula is compelling as it establishes a link between area elements and arithmetic torsion.
Such formulations find parallels in the realms of classic analysis and differential geometry.
For instance, they resonate with concepts akin to Stokes' theorem or the Gauss-Bonnet theorem.
We intend to expand upon this formula in the ensuing section\ref{sec:curvature},
aiming to forge a connection with curvature and delve into the intricacies of the Gauss-Bonnet theorem.

It's noteworthy to emphasize the distinctiveness of the local Descartes coordinate system.
This system, by integrating the assignment, lays the foundation for a theoretical framework.
We refer to this as the \emph{arithmetic coordinate system}, given its unique properties and alignment with arithmetic principles.

\subsection{The coordinate-free form of flow equation}\label{subsec:coordinate-free}

From the contour-gradient form of the flow equation~\eqref{eq:contourgradient}, we can derive a coordinate-free form of the flow equation.
Let's consider the direction of \(\phi = 0\) in the contour-gradient coordinate system, and we have

\[
    \frac{da}{ds}|_{\phi = 0} = \sqrt{\mu^2 + a^2 \lambda^2} \cos 0
\]

Notice the gradient of \(a\) is not dependent on the coordinate system, and we have the coordinate-free form of the flow equation:

\begin{equation}\label{eq:coordinate-free}
||\nabla a|| = \sqrt{\mu^2 + a^2 \lambda^2}
\end{equation}

It should be noted that the coordinate-free form of the flow equation~\eqref{eq:coordinate-free} is an Eikonal equation,
and can be viewed as a special Hamilton–Jacobi equation
\[
H(x, a, \nabla a) = 0
\]

where the Hamiltonian is

\begin{equation}\label{eq:hamiltonian}
    H(x, a, p) = ||p|| - \sqrt{\mu^2 + a^2 \lambda^2}
\end{equation}

\subsection{Propagation and Rectification}\label{subsec:propagation}

While Equation \eqref{eq:contourgradient} can be solved directly, its structure suggests a simplifying change of variables that "rectifies" the flow. Let us define a new variable $f$ by
\[
f = \operatorname{arcsinh}\left(\frac{\lambda a}{\mu}\right).
\]
This choice is motivated by the desire to find a variable whose gradient has a constant magnitude. Using the chain rule,
\[
\nabla f = \frac{df}{da}\nabla a = \frac{\lambda/\mu}{\sqrt{1+(\lambda a/\mu)^2}}\nabla a = \frac{\lambda}{\sqrt{\mu^2+\lambda^2 a^2}}\nabla a.
\]
Taking the norm of both sides and using the Eikonal form \eqref{eq:coordinate-free}, we find
\[
\|\nabla f\|=\frac{\lambda}{\sqrt{\mu^2+\lambda^2 a^2}}\|\nabla a\|=\lambda.
\]
The rectified variable $f$ has a gradient of constant magnitude. Consequently, its rate of change along a direction at an angle $\phi$ to the gradient is simply
\[
\frac{df}{ds}=\langle \nabla f, T \rangle = \|\nabla f\|\cos\phi = \lambda\cos\phi.
\]
This linearized equation is trivial to integrate: $f(s) = f_0 + \lambda s \cos\phi$. Substituting back for $a$ gives $a = (\mu/\lambda)\sinh(f)$, which yields the general solution for propagation from an initial value $f_0$:
\begin{equation}\label{eq:gradevo4}
a=\frac{\mu}{\lambda}\sinh\!\Big(\lambda s\cos\phi+\operatorname{arcsinh}\frac{\lambda a_0}{\mu}\Big).
\end{equation}
In the case of propagation along the gradient ($\phi=0$) from a zero-value contour ($a_0=0$), this simplifies to
\[
a=\pm\frac{\mu}{\lambda}\sinh(\lambda s).
\]
This form is reminiscent of the circumference of a hyperbolic circle of radius $s$, $C \propto \sinh(\sqrt{-k}s)$, suggesting that $a$ can be interpreted as a propagating wavefront in a space of constant negative curvature.

\section{A contact structure}\label{sec:contact_structure}
In this section, we develop another differential geometry framework related with AES, focusing on its contact structure and the associated differential calculus.
This framework provides a robust mathematical foundation for expression derivatives and their geometric interpretations.

\subsection{Core definitions and the contact structure}\label{subsec:core_definitions}

We begin by establishing our geometric space. Consider a 3-dimensional manifold with coordinates $(u,v,a) \in \mathbb{R}^3$. The geometry is built upon two fundamental 1-forms, defined using constant real parameters $\mu$ and $\lambda$:
\begin{equation}\label{eq:contact_forms_def}
\omega := \mu\,du + \lambda a\,dv, \qquad \alpha := da - \omega.
\end{equation}
The 1-form $\alpha$ is central to our construction. Its primary role is to define a ``horizontal'' plane at each point, known as the contact distribution. To see how it works, let's consider its action on an arbitrary vector field $X = x_u \partial_u + x_v \partial_v + x_a \partial_a$. Recalling that $da(X) = x_a$, $du(X) = x_u$, and $dv(X) = x_v$, the action is given by:
\[
\alpha(X) = da(X) - \mu\,du(X) - \lambda a\,dv(X) = x_a - \mu x_u - \lambda a x_v.
\]
The set of all vectors $X$ for which $\alpha(X)=0$ constitutes this horizontal plane.

Throughout this work, we use two types of differentials in parallel: (i) the standard de Rham exterior derivative $d$, which is always nilpotent ($d^2=0$); and (ii) the \textbf{expression differential} $\delta$, which is axiomatically defined in Section~\ref{ch:differential_calculus}.

\paragraph{The contact property.}
A key property of $\alpha$ is that it defines a contact structure on our 3D space. This is verified by checking if the 3-form $\alpha \wedge d\alpha$ is a volume form (i.e., non-zero everywhere). First, we compute the exterior derivative of $\omega$:
\[
d\omega = d(\mu\,du + \lambda a\,dv) = \lambda\,da \wedge dv.
\]
Then, the exterior derivative of $\alpha$ is simply $d\alpha = d(da - \omega) = -d\omega = -\lambda\,da \wedge dv$. Now, we can compute the wedge product:
\begin{align*}
\alpha \wedge d\alpha &= (da - \omega) \wedge (-\lambda\,da \wedge dv) \\
&= -da \wedge (\lambda\,da \wedge dv) + \omega \wedge (\lambda\,da \wedge dv) \\
&= 0 + (\mu\,du + \lambda a\,dv) \wedge (\lambda\,da \wedge dv) \\
&= \mu\lambda\,du \wedge da \wedge dv + \lambda^2 a\,\underbrace{dv \wedge da \wedge dv}_{=0} \\
&= \mu\lambda\,du \wedge da \wedge dv.
\end{align*}
Provided $\mu\lambda \neq 0$, this is a volume form, confirming that $\alpha$ is a \textbf{contact form}.

\paragraph{Normal form, Reeb field, and contact distribution.}
By introducing natural units $\tilde{u} = \mu u$ and $\tilde{v} = \lambda v$, the form $\alpha$ is reduced to its canonical form $\alpha_0 = da - d\tilde{u} - a\,d\tilde{v}$, which facilitates comparison with standard literature.

The Reeb vector field $R$, defined by $i_R d\alpha = 0$ and $\alpha(R) = 1$, is $R = -(1/\mu)\partial_u$. The contact distribution $\mathcal{H}$, as introduced earlier, is formally the kernel of $\alpha$:
\[
\mathcal{H} := \ker\alpha = \{X \in TM : \alpha(X) = 0\}.
\]
Viewing the tangent bundle as a composition of the $(u,v)$-base and the $a$-fiber, the form $\alpha = da - \omega$ intrinsically links the ``vertical'' change $da$ to the ``horizontal'' displacement defined by $\omega$.

\subsection{The geometry on the contact distribution \texorpdfstring{$\ker\alpha$}{ker(alpha)}} % Revised title for 6.2
\label{subsec:geometry_on_ker_alpha}

We now define two special vector fields that form a basis for the horizontal plane $\mathcal{H}$ at every point. These are the horizontal lifts of the base coordinate vectors, which we term the \textbf{expression directional derivatives}:
\begin{equation}\label{eq:directional_derivatives}
D_u := \partial_u + \mu\,\partial_a, \qquad D_v := \partial_v + \lambda a\,\partial_a.
\end{equation}
These fields are constructed specifically to be horizontal, a fact we can verify directly. For $D_u$, its coordinate components are $(x_u, x_v, x_a) = (1, 0, \mu)$. Applying the formula for $\alpha(X)$:
\[
\alpha(D_u) = x_a - \mu x_u - \lambda a x_v = \mu - \mu(1) - \lambda a(0) = 0.
\]
For $D_v$, its components are $(x_u, x_v, x_a) = (0, 1, \lambda a)$. Applying $\alpha$:
\[
\alpha(D_v) = x_a - \mu x_u - \lambda a x_v = \lambda a - \mu(0) - \lambda a(1) = 0.
\]
Since both $D_u$ and $D_v$ are annihilated by $\alpha$, and they are clearly linearly independent, they form a basis for the 2-dimensional contact distribution:
\[
\ker\alpha = \text{span}\{D_u, D_v\}.
\]

For any smooth scalar field $F(u,v,a)$, we define its expression differential $\delta F$ and the directional derivative $D_\theta$ as:
\begin{equation}\label{eq:3}\tag{3}
\delta F := (D_uF)\,du + (D_vF)\,dv, \qquad D_\theta := \cos\theta\,D_u + \sin\theta\,D_v.
\end{equation}
This construction effectively reduces the geometry from the three-dimensional space $(u,v,a)$ to the two-dimensional contact distribution $\mathcal{H}$.

\paragraph{Structural relation between d and $\delta$.}
For any scalar field $F$, the two differentials are related by a fundamental identity:
\[
\boxed{ \delta F = dF - (\partial_a F)\,\alpha }
\]
This identity can be interpreted as projecting $dF$ onto the horizontal distribution $\mathcal{H}$ by subtracting its vertical component along $\alpha$. Expanding this definition yields:
\[
\delta F = (F_u + \mu F_a)\,du + (F_v + \lambda a F_a)\,dv,
\]
which is consistent with Eq.~\eqref{eq:15}. In particular, we recover $\delta a = \omega$, $\delta u = du$, and $\delta v = dv$.

\subsection{Legendrian flow and rectification}

A curve $\gamma(s) = (u(s), v(s), a(s))$ is \textbf{Legendrian} if its tangent vector lies in the contact distribution, i.e., $\dot\gamma(s) \in \ker\alpha$. For such a curve, the evolution of $a$ is governed by the \textbf{flow equation}:
\begin{equation}\label{eq:4}\tag{4}
\frac{da}{ds} = D_\theta a = \mu\cos\theta + \lambda a\sin\theta,
\end{equation}
where $\theta$ parameterizes the angle of the tangent vector in the basis $\{D_u,D_v\}$. With respect to a given metric on the base manifold, this flow equation can be written in its Eikonal form:
\begin{equation}\label{eq:5}\tag{5}
\|\nabla a\| = \sqrt{\mu^2 + \lambda^2 a^2}.
\end{equation}
We introduce a \textbf{rectifying variable} $y$:
\begin{equation}\label{eq:6}\tag{6}
y = \arcsin\left(\frac{\lambda a}{\mu}\right) \quad\Rightarrow\quad \|\nabla y\| = \lambda.
\end{equation}
This rectification transforms the non-linear velocity field for $a$ into a constant-speed flow for $y$, which is advantageous for geometric constructions and enhances numerical stability.

\paragraph{Non-commutativity and curvature.}
The commutator of the horizontal vector fields yields a purely vertical vector, reflecting the "curvature" of the contact distribution. This phenomenon can be described as a "vertical return":
\begin{equation}\label{eq:7}\tag{7}
[D_u,D_v] = \mu\lambda\,\partial_a,\qquad \delta^2F = \mu\lambda(\partial_a F)\,du\wedge dv,\qquad \delta^2 a = \mu\lambda\,du\wedge dv.
\end{equation}
The circulation-area formula provides a tool for quantifying mesh singularities and global topological constraints:
\begin{equation}\label{eq:8}\tag{8}
\oint_{\partial\Sigma}\omega = \iint_\Sigma d\omega = \mu\lambda\iint_\Sigma du\wedge dv.
\end{equation}

\section{Thermodynamic correspondence}\label{sec:thermodynamic}

This section places the arithmetic contact form of Section~\ref{subsec:core_definitions} into direct correspondence with the standard contact geometry of equilibrium thermodynamics.

\subsection{Dictionary and a contactomorphism}\label{subsec:dictionary}

Let the thermodynamic phase space be coordinatized by $(S,V;U,T,p)$ and equipped with the standard 1-form
\begin{equation}\label{eq:td1form}
  \alpha_{\mathrm{TD}} := dU + p\,dV - T\,dS .
\end{equation}
(Other sign conventions are common; \eqref{eq:td1form} matches the first law $dU=T\,dS-p\,dV$ along equilibria.)

\begin{proposition}[AEG--Thermo contact correspondence]\label{prop:contactomorphism}
Consider the map
\begin{equation}\label{eq:dictionary}
  \Phi:\ (u,v,a)\ \longmapsto\ (S,V;U,T,p)=(v,\,-u;\,a,\,\lambda a,\,\mu).
\end{equation}
Then the pullback of the thermodynamic 1-form coincides with the AEG contact form \eqref{eq:contact_forms_def}:
\[
  \Phi^*(\alpha_{\mathrm{TD}})=d a + \mu\,d(-u) - (\lambda a)\,dv
  = d a - \mu\,du - \lambda a\,dv \;=\; \alpha .
\]
Hence $\Phi$ is a contactomorphism between the AEG contact manifold $(\mathbb{R}^3,\alpha)$ and the thermodynamic contact manifold $(\mathbb{R}^5,\alpha_{\mathrm{TD}})$ restricted to the image of~$\Phi$.
\end{proposition}

\begin{remark}[Conformal Massieu representation]\label{rem:massieu}
Introduce a dimensionless energy potential $Y:=\ln(a/E_\ast)$ using a fixed energy scale $E_\ast>0$, and define
\[
  T^\flat:=\lambda,\qquad p^\flat:=\frac{\mu}{a}.
\]
Then
\begin{equation}\label{eq:conformal_massieu}
  \alpha \;=\; d a - \mu\,du - \lambda a\,dv
          \;=\; a\Big(dY + p^\flat\,dV - T^\flat\,dS\Big),
\end{equation}
i.e., $\alpha$ is conformally equivalent (by the nowhere-vanishing factor $a$) to the Massieu-type form
$\alpha^\flat := dY + p^\flat dV - T^\flat dS$. Therefore the induced contact structure is unchanged, while multiplicative information of $a$ is linearized into $Y$.
\end{remark}

\subsection{Equilibria (Legendrian submanifolds) and Maxwell relations}\label{subsec:equilibria}

An \emph{equilibrium} (Legendrian) submanifold $L$ is characterized by the constraint $\alpha|_L=0$. Under the dictionary \eqref{eq:dictionary}, this condition is equivalent to the first law:
\begin{equation}\label{eq:firstlaw}
  \alpha_{\mathrm{TD}}|_L=0\quad\Longleftrightarrow\quad
  dU = T\,dS - p\,dV
  \quad\Longleftrightarrow\quad
  d a = \mu\,du + \lambda a\,dv .
\end{equation}
Consequently, along $L$ one may take $U$ as a fundamental equation $U=U(S,V)$ and identify the intensities as
\begin{equation}\label{eq:intensities}
  T=\Bigl(\frac{\partial U}{\partial S}\Bigr)_{V},\qquad
  p=-\Bigl(\frac{\partial U}{\partial V}\Bigr)_{S}.
\end{equation}
Translating back to AEG variables $(u,v,a)$ via $(S,V)=(v,-u)$ yields
\begin{equation}\label{eq:intensities-aeg}
  T=\lambda a=\Bigl(\frac{\partial a}{\partial v}\Bigr)_{u},\qquad
  p=\mu=\Bigl(\frac{\partial a}{\partial u}\Bigr)_{v}.
\end{equation}

\begin{proposition}[Maxwell relation in AEG variables]\label{prop:maxwell}
On an equilibrium manifold $L$, the classical Maxwell relation
\begin{equation}\label{eq:maxwell-classic}
  \Bigl(\frac{\partial T}{\partial V}\Bigr)_{S}
  \;=\; -\,\Bigl(\frac{\partial p}{\partial S}\Bigr)_{V}
\end{equation}
is equivalent to
\begin{equation}\label{eq:maxwell-aeg}
  \frac{\partial(\lambda a)}{\partial u}\Big|_{v}
  \;=\;
  \frac{\partial \mu}{\partial v}\Big|_{u}.
\end{equation}
\emph{Proof.} Apply the change of variables $(S,V)=(v,-u)$ in \eqref{eq:maxwell-classic} and use \eqref{eq:intensities-aeg}.
\qed
\end{proposition}

\begin{remark}[Compatibility of the fundamental equation]\label{rem:compat}
Suppose one prescribes functions $(\lambda,\mu)$ on the state space and seeks $U(S,V)$ solving
\[
\partial_S U=\lambda(S,V)\,U,\qquad \partial_V U=-\mu(S,V).
\]
A necessary and sufficient integrability condition is precisely the Maxwell compatibility
$\partial_V(\lambda U) = -\partial_S\mu$, which is equivalent to \eqref{eq:maxwell-classic}. In particular, the choice of constant \(\lambda\) and constant \(\mu\neq 0\) is \emph{not} compatible with a nontrivial two-variable fundamental equation $U(S,V)$; at least one of them must vary with state.
\end{remark}

\subsection{Thermodynamic potentials and Legendre transforms}\label{subsec:potentials}

On $L$ the usual potentials arise by Legendre transforms of $U(S,V)$:
\begin{align}
F(T,V) &:= U-TS,          & dF &= -S\,dT - p\,dV, \label{eq:helmholtz}\\
H(S,p) &:= U+pV,          & dH &= \ \ T\,dS + V\,dp, \label{eq:enthalpy}\\
G(T,p) &:= U+pV-TS,       & dG &= -S\,dT + V\,dp. \label{eq:gibbs}
\end{align}
In the Massieu representation of Remark~\ref{rem:massieu}, define the \emph{Massieu (free-entropy) potential}
\begin{equation}\label{eq:massieu}
  \Psi(T^\flat,p^\flat) := Y - T^\flat S - p^\flat V,
  \qquad d\Psi = -S\,dT^\flat - V\,dp^\flat ,
\end{equation}
which is contact-equivalent to \eqref{eq:helmholtz}--\eqref{eq:gibbs} under the conformal factor $a$ in \eqref{eq:conformal_massieu}. This form is convenient for information-geometric and convex-analytic arguments because $Y=\ln(a/E_\ast)$ is dimensionless and multiplicative effects become additive.

\subsection{Response functions and convexity}\label{subsec:response}

Assuming sufficient smoothness of the fundamental equation $U(S,V)$, the standard response functions are
\[
  C_V := \Bigl(\frac{\partial U}{\partial T}\Bigr)_{V},\qquad
  \kappa_T := -\frac{1}{V}\Bigl(\frac{\partial V}{\partial p}\Bigr)_{T},\qquad
  \alpha_p := \frac{1}{V}\Bigl(\frac{\partial V}{\partial T}\Bigr)_{p}.
\]
Stability in the energy representation is encoded by convexity of $U$ in $(S,V)$:
\[
  \frac{\partial^2 U}{\partial S^2}\ge 0,\qquad
  \det\!\begin{pmatrix}
    U_{SS} & U_{SV}\\ U_{VS} & U_{VV}
  \end{pmatrix}\ge 0,
\]
with the equalities characterizing marginal stability. Equivalently, in the Massieu representation one works with the convexity of $-\,\Psi$ in $(T^\flat,p^\flat)$.
In AEG variables, these conditions pull back via \eqref{eq:dictionary} and \eqref{eq:intensities-aeg}, and can be expressed in terms of the Jacobian and Hessians of $a(u,v)$; for instance
\[
  U_{SS}=\Bigl(\frac{\partial T}{\partial S}\Bigr)_V
  = \Bigl(\frac{\partial (\lambda a)}{\partial v}\Bigr)_{u},
  \qquad
  -U_{VV}=\Bigl(\frac{\partial p}{\partial V}\Bigr)_S
  = \Bigl(\frac{\partial \mu}{\partial u}\Bigr)_{v}.
\]

\subsection{Units and normalization}\label{subsec:units}

Under the SI assignment $[U]=[E]$, $[S]=[E][\Theta]^{-1}$, $[V]=[L]^3$, the dictionary \eqref{eq:dictionary} gives
\[
  [p]=[E][V]^{-1},\quad [T]=[\Theta],\quad [\lambda]=[S]^{-1},\quad [\mu]=[p],
\]
and each term in $\alpha_{\mathrm{TD}}$ and $\alpha$ has dimension $[E]$. In the Massieu representation, $Y$ and the intensive pair $(T^\flat,p^\flat)$ are dimensionless provided one fixes reference scales $E_\ast$ and $S_\ast$ (e.g.\ $S_\ast=k_B$) and writes $\lambda=T^\flat/S_\ast$, $\mu=p^\flat\,E_\ast$.

\subsection{A solvable family (illustration)}\label{subsec:example}

Let $\lambda=\lambda(S)$ be a given smooth function and set
\[
  U(S,V)=C(V)\,\exp\!\Big(\textstyle\int^S \lambda(\sigma)\,d\sigma\Big).
\]
Then $T=\partial_S U=\lambda(S)\,U$, and $p=-\partial_V U=-C'(V)\exp(\int^S\lambda)$, so the Maxwell compatibility is automatic. The corresponding AEG data along equilibria are
\[
  a=U,\qquad \mu=p=-C'(V)\,e^{\int^S\lambda},\qquad \lambda=\lambda(S),\qquad
  \text{with }(S,V)=(v,-u).
\]
This toy family exhibits the two canonical AEG modes (multiplicative in $v=S$ and additive in $u=-V$) while remaining globally integrable. Non-degeneracy of response functions requires $C''(V)\neq 0$ and $\lambda'(S)\neq 0$.

\medskip

In summary, the arithmetic contact form $\alpha=da-\mu\,du-\lambda a\,dv$ is not merely analogous to thermodynamic geometry: via the dictionary \eqref{eq:dictionary} it \emph{is} the thermodynamic contact form \eqref{eq:td1form}. Maxwell relations, Legendre transforms, stability/metric structures, and dimension bookkeeping all transfer verbatim; the Massieu (logarithmic) representation \eqref{eq:conformal_massieu} provides a conformally equivalent, dimensionless chart that linearizes multiplicative phenomena.

\end{document}
