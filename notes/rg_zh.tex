\documentclass[UTF8]{ctexart}
\usepackage{amsmath}
\usepackage{amssymb}
\usepackage{geometry}
\usepackage{hyperref}

\geometry{a4paper, margin=1in}

\title{算术表达式几何 (AEG):流、累积、双重时间标度与重整化的极简实现}
\author{苑明理}
\date{\today}

\begin{document}

\maketitle

\section{引言}
算术表达式几何 (Arithmetic Expression Geometry, AEG) 理论 \cite{YuanAEG} 为算术过程构建了一个内禀的几何框架,其核心在于将算术表达式的生成与演化理解为在特定几何结构中的运动与变形。此理论不仅揭示了算术运算背后隐藏的动力学与演化特性,更值得注意的是,AEG框架通过其对“尺度”的显式处理和“演化时间”的概念,为理解和实现重整化群 (Renormalization Group, RG) 的核心思想提供了一个出人意料的极简版本。

\section{AEG 框架简述}

AEG 理论引入了以下关键概念:

\begin{itemize}
    \item \textbf{赋值函数 ($a$)}: 代表算术表达式在特定演化阶段和动力学状态下的核心度量,可以理解为一种广义的“值”、“能量”或“信息承载状态”。
    \item \textbf{流方程}: 描述了赋值函数 $a$ 如何沿着几何路径 $s$ 在参数化的算术操作(加法和乘法)驱动下进行演化。其核心形式为 \cite[Sec 3.1, Eq. 30]{YuanAEG}:
    $$\frac{da}{ds} = \mu \cos\theta + a\lambda \sin\theta$$
    其中,$\mu$ 和 $\lambda$ 分别是加法和乘法生成元的强度参数,$\theta$ 是路径方向与加法主轴的夹角。此方程揭示了赋值函数的变化率同时受到加法性驱动($\mu \cos\theta$)和乘法性、尺度依赖性驱动($a\lambda \sin\theta$)的影响。该流方程具有坐标无关形式 $||\nabla a|| = \sqrt{\mu^2 + a^2\lambda^2}$,这是一个Eikonal方程,与物理学中的波前传播和经典力学的Hamilton-Jacobi方程具有深刻联系 \cite[Sec 3.5, Eq. 57, 58]{YuanAEG}。
    \item \textbf{累积交换空间 (ACS)}: 为了量化算术操作序列的全局非交换效应(算术挠率),AEG引入了累积交换空间。对于任一算术路径 $\gamma$,其在ACS中的坐标 $(A_\gamma, M_\gamma)$ 分别代表路径中所有加法操作参数 $\mu_k$ 的总和与所有(对数)乘法操作参数 $\lambda_k$ 的总和 \cite[Sec 5.1]{YuanAEG}:
    $$A_\gamma = \sum \mu_k$$
    $$M_\gamma = \sum \lambda_k$$
    ACS提供了一个交换性的参考平面,使得不同路径之间的比较和几何量的定义(如算术挠率的面积分形式 $\tau(\gamma) = \iint_{\Sigma_\gamma} e^M dA \wedge dM$ \cite[Sec 5.1, Eq. 75]{YuanAEG})成为可能。
\end{itemize}

\section{扰动(误差)传递与乘法尺度的作用:一个实例分析}

为阐明乘法尺度在AEG框架中的核心作用,考察在“\textbf{交错式可线索化表达式}”(Alternative threadlike expression) 中扰动如何传播的例子 \cite[Sec 2.6]{YuanAEG}。考虑一个由交错的乘法和加法操作构成的路径作用于初始值 $\mu_0$。设 $w_i$ 是经过第 $i$ 对乘法 $\otimes_{\lambda_i}$ (即乘以 $e^{\lambda_i}$) 和加法 $\oplus_{\mu_i}$ 操作后的中间值,即 $w_i = e^{\lambda_i} w_{i-1} + \mu_i$,其中 $w_0 = \mu_0$。

若初始值 $\mu_0$ 受到一个微小扰动,变为 $\tilde{\mu}_0$,则在第 $i$ 步产生的扰动 $\Delta w_i = \tilde{w}_i - w_i$ 与初始扰动 $\Delta \mu_0 = \tilde{\mu}_0 - \mu_0$ 之间的关系为 \cite[Sec 2.6, Eq. 25]{YuanAEG}:
$$\frac{\tilde{w}_i - w_i}{\tilde{\mu}_0 - \mu_0} = e^{\sum_{j=1}^i \lambda_j} = e^{\check{\lambda}_i}$$
其中 $\check{\lambda}_i = \sum_{j=1}^i \lambda_j$ 是到第 $i$ 步为止累积的对数乘法因子。

这个结果清晰地展示了\textbf{乘法尺度的放大(或缩小)作用}:
初始的扰动 $\Delta \mu_0$ 在沿着算术路径传播时,其绝对大小会被路径上经历的\textbf{累积乘法因子 $e^{\check{\lambda}_i}$} 所调制。

\begin{itemize}
    \item 如果累积的对数乘法因子 $\check{\lambda}_i > 0$,则 $e^{\check{\lambda}_i} > 1$,初始扰动被放大。这意味着系统对初始条件的敏感性增强。
    \item 如果累积的对数乘法因子 $\check{\lambda}_i < 0$,则 $0 < e^{\check{\lambda}_i} < 1$,初始扰动被缩小。这意味着系统对初始扰动具有一定的鲁棒性或衰减效应。
    \item 如果 $\check{\lambda}_i = 0$,则扰动以其原始大小传播(仅考虑乘法效应时)。
\end{itemize}
这直接说明了乘法操作不仅仅是数值上的改变,它\textbf{设定了信息(在此例中是扰动/误差)传播的“尺度”或“增益”}。每一步乘法操作 $\otimes_{\lambda_j}$ 都会贡献一个因子 $e^{\lambda_j}$ 到这个累积的尺度变换中。因此,算术表达式的乘法结构(即“演化历史”)深刻地影响着其内部“动力学过程”(如扰动传播)的特性。

\section{双重时间标度的哲学思考与尺度在其中的核心地位}

基于AEG框架及其揭示的算术过程特性,可以将“加法是动力学时间,乘法是演化时间”的观点提升到更具哲学性的层面进行讨论,其中“尺度”扮演着核心的联结与调节角色。

\begin{itemize}
    \item \textbf{动力学时间(加法)的局域性与演化时间(乘法)的全局性}:
    加法操作,作为动力学时间的体现,其影响往往是局域的、线性的,作用于系统在某一既定“尺度”下的状态。它描述了系统在一个稳定或缓慢变化的“背景”下的内部演化。
    乘法操作,作为演化时间的体现,其影响则是全局性的、结构性的。它直接改变了这个“背景”本身,即系统的内禀“尺度”。每一次乘法操作都可能将系统推入一个全新的行为区域,其中动力学规则(或其参数)可能发生质变。这个“尺度”是由累积的乘法历史(即演化时间 $M_\gamma$)所决定的。

    \item \textbf{尺度的层级与涌现}:
    演化时间(乘法)的累积不仅改变数值大小,更重要的是它建立了\textbf{尺度的层级 (hierarchy of scales)}。一个复杂的算术表达式,通过其乘法结构,可以被看作是在不同尺度层级上嵌套和组织的动力学过程的集合。
    高层次的尺度(由较早或较强的乘法操作设定)为低层次的尺度和动力学过程提供了“边界条件”或“宏观约束”。这种尺度的层级性是理解复杂系统(如大型语言模型LLM的损失景观或AEG自身的几何结构)如何从简单的算术规则中涌现出复杂行为的关键。例如,在LLM训练的河谷模型中,“河床”(慢过程,对应演化时间)决定了“山谷”的形态(快过程的环境,对应动力学时间)\cite{Liu2025NTL},这正是尺度层级调控作用的一个体现。

    \item \textbf{几何作为复杂性的表征与尺度的作用}:
    AEG的一个核心目标是寻求算术过程复杂性的几何度量。双重时间标度为此提供了基础:
    \begin{itemize}
        \item \textbf{动力学复杂性}:可以通过在特定尺度下,动力学路径的长度、遍历的“相空间体积”(可能对应于AEG中特定子流形的体积,如双曲体积)或信息熵来度量。这个复杂性是当前“演化阶段”或“尺度”的函数。
        \item \textbf{演化复杂性}:可以通过描述尺度本身变化的路径(例如 $M_\gamma$ 的轨迹)的复杂性,或者不同演化阶段之间的“几何距离”或“结构差异”(可能通过算术挠率等概念捕捉)来度量。
    \end{itemize}
    “尺度”在这里起到了关键的桥梁作用:它不仅是演化时间的产物,也是动力学复杂性发生的舞台。算术表达式的总复杂性,是其在所有相关尺度上演化和动力学行为的综合体现。AEG中的几何不变量,如与算术挠率相关的 $e^M$ 因子,明确地将演化尺度 $M$ 编织到动力学过程的几何度量中。

    \item \textbf{时间的不可逆性与对称性破缺的根源}:
    在纯粹的动力学时间(可逆的加减法)中,系统可能展现出高度的对称性。然而,演化时间(乘法)的引入,特别是当乘法因子不为1时,往往会打破这种对称性,并可能引入时间的不可逆性。例如,一个持续的乘法放大过程($\lambda > 0$)会使得系统远离初始状态,而一个持续的乘法缩小过程($\lambda < 0$)则可能导致信息丢失或系统退化到某种平凡状态。
    “尺度”的演化方向和速率,决定了系统对称性如何被打破以及新的结构如何形成。从这个意义上说,乘法(演化时间)是创造和塑造复杂算术结构(及其几何对应物)的主要驱动力。
\end{itemize}

\section{结论}
综上所述,AEG框架通过流方程和ACS等核心工具,为“加法是动力学时间,乘法是演化时间/尺度”这一核心论点提供了坚实的数学基础。扰动传递的例子直观地展示了乘法尺度对动力学过程的调制作用。从更深层次看,这种双重时间标度和尺度的核心地位,不仅揭示了算术过程内在的层级结构和演化机制,也为我们从几何和复杂性的视角理解和分析更广泛的计算和自然过程提供了富有启发的理论视角。\textbf{值得强调的是,通过将乘法运算直接映射为尺度的变换,并将这种尺度变换内禀地整合到流方程和几何结构中,AEG框架实质上提供了一种对重整化思想的极简但功能完备的实现。它允许我们观察在不同“演化时间”(即不同尺度累积)下,系统动力学(由加法驱动)的有效行为,这正是重整化分析的核心目标。这种内在的、基于基本算术运算的重整化视角,是AEG理论最具创新和潜力的方面之一。}

\begin{thebibliography}{9}
    \bibitem{YuanAEG} Yuan, M. Geometry of Arithmetic Expressions: I. Basic Concepts and Unsolved Problems (Draft). \textit{Manuscript in aeg-paper repository}.
    \bibitem{Liu2025NTL} Liu, Z., Liu, Y., Gore, J., \& Tegmark, M. (2025). Neural Thermodynamic Laws for Large Language Model Training. \textit{arXiv preprint arXiv:2505.10559}.
\end{thebibliography}

\end{document}