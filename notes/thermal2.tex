\documentclass[11pt]{article}
\usepackage{amsmath,amssymb,amsthm}
\usepackage{hyperref}

\title{From Algorithmic Thermodynamics to Arithmetic Expression Geometry:\\
A Contactomorphism via First-Hitting Expressions}

\author{Mingli Yuan\thanks{Draft note; comments welcome.}}

\date{\today}

\begin{document}
\maketitle

\begin{abstract}
Baez and Stay developed an ``algorithmic thermodynamics'' in which
prefix-free programs of a universal Turing machine play the role of
microstates, and the logarithm of runtime, program length, and output
are observables analogous to energy $E$, volume $V$, and particle number
$N$. In this note we explain how a restricted form of their
thermodynamic formalism admits a natural contactomorphism with the
contact geometry of arithmetic expressions (AEG). The key steps are:
(1) replace prefix-free programs by \emph{first-hitting expression paths}
in the accumulative commutative space (ACS), selected by a linear
functional $W(A,M)$; (2) use these paths to define an algorithmic
partition function in AEG variables; and (3) show that, after fixing
$N$ and imposing a simple equation of state $T=\lambda E$, $P=\mu_{\rm add}$,
the Baez--Stay contact form reduces to the AEG contact form
$\alpha_{\rm AEG}=da-(\mu_{\rm add}\,du+\lambda a\,dv)$ via an explicit
contactomorphism. This provides a precise geometric bridge between
algorithmic thermodynamics and the geometry of arithmetic expressions.
\end{abstract}

\section{Baez--Stay algorithmic thermodynamics}

Let $U$ be a universal prefix-free Turing machine and let
\[
  X = \mathrm{dom}(U)
\]
be the set of halting programs for $U$. Following Baez and Stay, each
program $x\in X$ is treated as a microstate, and any real-valued
function on $X$ is an observable.

We focus on the three observables
\begin{align*}
  E(x) &:= \log(\text{runtime of }x),\\
  V(x) &:= \text{length of }x,\\
  N(x) &:= \text{output of }x\in\mathbb{N},
\end{align*}
which play the roles of energy, volume, and particle number,
respectively.

For inverse temperature $\beta$, and conjugate parameters
$\gamma,\delta\in\mathbb{R}$, the \emph{partition function} is
\begin{equation}
  Z(\beta,\gamma,\delta)
  = \sum_{x\in X}
    \exp\!\bigl(-\beta E(x)-\gamma V(x)-\delta N(x)\bigr).
  \label{eq:Baez-partition}
\end{equation}
Whenever $Z$ converges, it defines a Gibbs ensemble
\[
  p(x)
  = \frac{1}{Z(\beta,\gamma,\delta)}
    \exp\!\bigl(-\beta E(x)-\gamma V(x)-\delta N(x)\bigr)
\]
which maximizes Shannon entropy $S = -\sum_x p(x)\log p(x)$ under
constraints on the mean values
\[
  E = \mathbb{E}_p[E(x)],\quad
  V = \mathbb{E}_p[V(x)],\quad
  N = \mathbb{E}_p[N(x)].
\]

Baez and Stay then introduce conjugate intensive variables
\[
  T = \frac{1}{\beta},\qquad
  P = \frac{\gamma}{\beta},\qquad
  \mu = -\frac{\delta}{\beta},
\]
and show that the usual thermodynamic differential relation holds:
\begin{equation}
  dE = T\,dS - P\,dV + \mu\,dN.
  \label{eq:Baez-fundamental}
\end{equation}
From the standpoint of contact geometry, \eqref{eq:Baez-fundamental}
suggests the \emph{algorithmic contact form}
\begin{equation}
  \alpha_{\rm alg} = dE - T\,dS + P\,dV - \mu\,dN
\end{equation}
on the $(2n+1)$-dimensional space with coordinates
$(S,V,N;E,T,P,\mu)$, whose equilibrium submanifold is the locus
$\alpha_{\rm alg}=0$.

\section{Arithmetic expression geometry and contact structure}

Arithmetic expression geometry (AEG) starts from threadlike arithmetic
expressions built from additive and multiplicative generators. These
expressions can be encoded as paths in the \emph{accumulative
commutative space} (ACS), a plane with coordinates
\[
  (A,M)\in\mathbb{R}^2
\]
recording total additive and (logarithmic) multiplicative charge along
a path. The non-commutativity of ``add then multiply'' vs.\ ``multiply
then add'' is measured by an \emph{arithmetic torsion} which can be
expressed as an $e^M$-weighted area in ACS.

To capture local dynamics, AEG introduces a $3$-dimensional contact
manifold with coordinates $(u,v,a)\in\mathbb{R}^3$ and 1-form
\begin{equation}
  \alpha_{\rm AEG}
  = da - (\mu_{\rm add}\,du + \lambda a\,dv),
  \label{eq:AEG-contact}
\end{equation}
where $\mu_{\rm add}$ and $\lambda$ are fixed real parameters. The base
coordinates $(u,v)$ are a differential realization of ACS, while $a$ is
the \emph{assignment value} propagated along expression paths. The
kernel $\ker\alpha_{\rm AEG}$ is spanned by the horizontal vector fields
\[
  D_u = \partial_u + \mu_{\rm add}\partial_a,\qquad
  D_v = \partial_v + \lambda a\,\partial_a,
\]
and the \emph{Legendrian curves} (curves tangent to $\ker\alpha$) obey
the flow equation
\[
  \frac{da}{ds}
  = \mu_{\rm add}\cos\theta + \lambda a\sin\theta,
\]
which is the infinitesimal form of the discrete alternation of addition
and multiplication.

It is convenient to view \eqref{eq:AEG-contact} as the restriction of a
thermodynamic contact form. In a separate note, a contactomorphism
\[
  \Phi:(u,v,a)\mapsto
  (S,V;U,T,p) = (v,-u; a,\lambda a,\mu_{\rm add})
\]
was exhibited, under which the standard 1-form
$\alpha_{\rm TD}=dU+p\,dV-T\,dS$ pulls back to $\alpha_{\rm AEG}$.
Thus the AEG contact manifold is contactomorphic to a thermodynamic
phase space restricted by $U=a$, $T=\lambda a$, $p=\mu_{\rm add}$.

\section{First-hitting expressions and prefix-free families}

Baez's formalism fundamentally relies on a prefix-free set of programs:
$\mathrm{dom}(U)$ is prefix-free, and this guarantees the convergence and
information-theoretic interpretation of the partition function
\eqref{eq:Baez-partition}.

On the AEG side, there is no explicit prefix-free constraint in the
definition of threadlike expressions. However, a natural prefix-like
structure appears as soon as we introduce a ``cost functional'' on ACS
and insist on \emph{first-hitting} paths.

\subsection{A linear cost functional on ACS}

Let $(A,M)\in\mathbb{R}^2$ denote the total additive and multiplicative
charges of a threadlike expression. Fix a positive parameter
$\alpha>0$ and define a linear functional
\begin{equation}
  W(A,M) := A + \alpha M.
  \label{eq:W-def}
\end{equation}
Geometrically, $W$ picks out a direction in ACS with slope
$\alpha$ and provides a notion of ``weighted length'' or ``work'' along
an expression path. For a discrete path $\sigma$ starting at $(0,0)$ and
consisting of unit steps in the $A$- and $M$-directions, we write
\[
  W(\sigma) := W\bigl(A(\sigma),M(\sigma)\bigr)
\]
for the cost of its endpoint.

\subsection{First-hitting paths at level $C$}

Fix a threshold $C>0$. A threadlike expression path $\sigma$ from
$(0,0)$ is said to be \emph{first-hitting at level $C$} if:
\begin{enumerate}
\item $W(\sigma)\ge C$, and
\item for each proper initial segment $\tau\prec\sigma$,
      we have $W(\tau)<C$.
\end{enumerate}
Let $S_C$ denote the set of all such first-hitting paths.

Two observations are immediate:
\begin{itemize}
\item If we encode each path $\sigma$ by its sequence of additive and
      multiplicative steps (for example using an alphabet $\{+,\times\}$),
      then $S_C$ is \emph{prefix-free}: no $\sigma\in S_C$ is a prefix of
      any other $\sigma'\in S_C$. Indeed, if $\sigma$ were a prefix of
      $\sigma'$ and both satisfied (1), then $\sigma$ would violate (2).

\item Under mild regularity assumptions on the allowed steps (e.g. unit
      steps, or a finite step set with bounded increments), the family
      $\{S_C\}_{C>0}$ behaves like a Kraft family of prefix codes. For
      discrete choices of $C$ and $\alpha$ one can check that the Kraft
      sum $\sum_{\sigma\in S_C}2^{-|\sigma|}$ is close to $1$, with
      $|\sigma|$ the number of steps of the path.
\end{itemize}

Thus $S_C$ provides a canonical prefix-free subset of threadlike
expressions at ``cost level'' $C$. It is natural to regard elements of
$S_C$ as AEG analogues of halting programs.

\subsection{Observables on first-hitting expressions}

Given $\sigma\in S_C$, we define three observables mirroring the Baez
triple $(E,V,N)$:
\begin{align*}
  W(\sigma) &:= A(\sigma)+\alpha M(\sigma)
              &&\text{(weighted cost, ``log runtime''),}\\
  L(\sigma) &:= A(\sigma)+M(\sigma)
              &&\text{(unweighted length, ``program length''),}\\
  a(\sigma) &:= \text{assignment value at the endpoint}
              &&\text{(output / particle number).}
\end{align*}
The assignment $a(\sigma)$ is defined by propagating the initial value
along the path via the addition/multiplication dynamics, as in the
original AEG construction.

\section{An AEG partition function}

Using the first-hitting set $S_C$ and the observables
$(W,L,a)$, we can define an AEG partition function that is formally
identical to \eqref{eq:Baez-partition}:
\begin{equation}
  Z_C(\beta,\gamma,\delta)
  := \sum_{\sigma\in S_C}
     \exp\!\bigl(-\beta W(\sigma)-\gamma L(\sigma)-\delta a(\sigma)\bigr).
  \label{eq:AEG-partition}
\end{equation}
Whenever $Z_C$ converges, it yields a Gibbs probability measure
\[
  p_C(\sigma)
  = \frac{1}{Z_C(\beta,\gamma,\delta)}
    \exp\!\bigl(-\beta W(\sigma)-\gamma L(\sigma)-\delta a(\sigma)\bigr),
\]
and we can define mean values
\[
  E := \mathbb{E}_{p_C}[W(\sigma)],\quad
  V := \mathbb{E}_{p_C}[L(\sigma)],\quad
  N := \mathbb{E}_{p_C}[a(\sigma)].
\]
Note that we have deliberately \emph{reused} the symbols $E,V,N$ for the
mean observables, as this allows us to apply Baez--Stay's formalism
essentially verbatim.

With the same definitions of
\[
  T = \frac{1}{\beta},\quad
  P = \frac{\gamma}{\beta},\quad
  \mu = -\frac{\delta}{\beta},
\]
the usual thermodynamic derivation applies and we again obtain
\[
  dE = T\,dS - P\,dV + \mu\,dN,
\]
where $S$ is the Shannon entropy of $p_C$. Thus the AEG ensemble
\eqref{eq:AEG-partition} carries the same contact structure as the
Baez--Stay ensemble \eqref{eq:Baez-partition}, provided we adopt the
identification
\[
  E \leftrightarrow \mathbb{E}[W],\quad
  V \leftrightarrow \mathbb{E}[L],\quad
  N \leftrightarrow \mathbb{E}[a].
\]

\section{The contactomorphism}

We now make precise the relation between the Baez--Stay contact form
and the AEG contact form \eqref{eq:AEG-contact}.

\subsection{Reduction of the algorithmic contact form}

Start from the algorithmic contact form
\[
  \alpha_{\rm alg}
  = dE - T\,dS + P\,dV - \mu\,dN.
\]
We perform two reductions.

\paragraph{Fixing $N$.} Restrict to a level set $N=N_0$. On this
4-dimensional submanifold, $dN=0$ and
\[
  \alpha'_{\rm alg} = dE - T\,dS + P\,dV.
\]

\paragraph{Imposing a simple equation of state.}
On $\{N=N_0\}$, choose an ``equation of state''
\begin{equation}
  T = \lambda E,\qquad P = \mu_{\rm add},
  \label{eq:EOS}
\end{equation}
with constants $\lambda,\mu_{\rm add}\in\mathbb{R}$.
This is the analogue of restricting to a family of ensembles where
temperature is proportional to energy and pressure is fixed. Substituting
\eqref{eq:EOS} into $\alpha'_{\rm alg}$ we obtain a 3-dimensional
contact form on $(S,V,E)$:
\begin{equation}
  \alpha_{\rm red}
  = dE - \lambda E\,dS + \mu_{\rm add}\,dV.
  \label{eq:alpha-red}
\end{equation}

\subsection{Explicit contactomorphism to AEG}

Consider the AEG contact manifold $(u,v,a)$ with
\[
  \alpha_{\rm AEG}
  = da - (\mu_{\rm add}\,du + \lambda a\,dv).
\]
Define a smooth map
\begin{equation}
  \Psi:(u,v,a)\longmapsto (S,V,E) = (v,-u,a).
  \label{eq:Psi}
\end{equation}
Compute the pullback of $\alpha_{\rm red}$:
\begin{align*}
  \Psi^*(\alpha_{\rm red})
  &= \Psi^*\bigl(dE - \lambda E\,dS + \mu_{\rm add}\,dV\bigr)\\
  &= da - \lambda a\,dv + \mu_{\rm add}\,d(-u)\\
  &= da - \lambda a\,dv - \mu_{\rm add}\,du\\
  &= da - (\mu_{\rm add}\,du + \lambda a\,dv)\\
  &= \alpha_{\rm AEG}.
\end{align*}
Thus $\Psi$ is a \emph{strict contactomorphism} between the AEG contact
manifold $(\mathbb{R}^3,\alpha_{\rm AEG})$ and the reduced algorithmic
contact manifold $(\mathbb{R}^3,\alpha_{\rm red})$ obtained from
$\alpha_{\rm alg}$ by fixing $N$ and imposing the equation of state
\eqref{eq:EOS}.

In words: after identifying
\[
  a \leftrightarrow E,\qquad
  v \leftrightarrow S,\qquad
  -u \leftrightarrow V,
\]
and fixing the functional dependencies $T=\lambda E$, $P=\mu_{\rm add}$,
the algorithmic thermodynamics of Baez--Stay and the AEG contact
geometry describe the same contact structure.

\section{On the naturality of $W$ and of the reduction}

The linear cost functional $W(A,M)=A+\alpha M$ plays two roles
simultaneously:
\begin{itemize}
\item It defines a family of first-hitting sets $S_C$ which are
      automatically prefix-free in the step-encoding of expressions.
      Thus $W$ provides the AEG analogue of the prefix-free domain of a
      universal Turing machine.

\item Its expectation $\mathbb{E}_{p_C}[W]$ serves as the energy variable
      $E$ in the Baez--Stay formalism, allowing the fundamental relation
      \eqref{eq:Baez-fundamental} to be imported into AEG.
\end{itemize}

The reduction $T=\lambda E$, $P=\mu_{\rm add}$ used in
\eqref{eq:EOS} is not forced by Baez--Stay's theory; it represents a
particularly simple family of algorithmic equations of state which
matches the AEG contact form. From the AEG perspective, it is natural
because:
\begin{enumerate}
\item The contact form $\alpha_{\rm AEG}$ already singles out constants
      $\lambda$ and $\mu_{\rm add}$ as the coefficients governing
      multiplicative and additive propagation in the Legendrian flow.

\item The mapping $T=\lambda E$ is consistent with the rectified flow
      picture in AEG, where the gradient of the transformed variable
      $y=\mathrm{arcsinh}(\lambda a/\mu_{\rm add})$ has constant norm
      proportional to $\lambda$.

\item The constant-pressure condition $P=\mu_{\rm add}$ reflects the
      fact that the additive generator has fixed ``strength'' in ACS:
      one additive step always increases $A$ by the same amount.
\end{enumerate}

One should nevertheless keep in mind that this identifies only a
\emph{restricted} sector of the full algorithmic thermodynamics with
AEG. Allowing more general equations of state on the Baez--Stay side
would correspond, on the AEG side, to deforming the contact form
\eqref{eq:AEG-contact} or enlarging the state space (for example by
treating $N$ as an additional continuous coordinate). Exploring such
deformations may be a natural way to incorporate richer algorithmic
observables into arithmetic expression geometry.

\section*{Outlook}

This note shows that the following diagram commutes at the level of
contact geometry:
\[
  \text{prefix-free programs}
   \;\longleftrightarrow\;
  \text{first-hitting expression paths}
   \;\longrightarrow\;
  \text{AEG contact manifold},
\]
where the left arrow is mediated by the observables $(E,V,N)$ vs.
$(W,L,a)$ and the right arrow by the contactomorphism $\Psi$. Several
directions suggest themselves:
\begin{itemize}
\item make the prefix-coding aspect of $S_C$ precise, including exact
      Kraft equalities and the relation to Chaitin's $\Omega$ for
      suitable choices of parameters and thresholds;

\item analyze convergence and computability properties of the AEG
      partition function $Z_C$ in analogy with Baez--Stay and Tadaki;

\item lift the present 3-dimensional contactomorphism to a higher
      dimensional picture where $N$ (or other expression-theoretic
      invariants) become additional contact coordinates.
\end{itemize}
In this way, algorithmic thermodynamics and arithmetic expression
geometry may be seen as two complementary coordinatizations of the same
underlying geometric structure.
\bigskip

\bibliographystyle{plain}
\begin{thebibliography}{9}

\bibitem{BaezStay}
J.~C.~Baez and M.~Stay,
\newblock Algorithmic Thermodynamics,
\newblock \emph{arXiv:1010.2067}, 2013.

\bibitem{YuanAEG}
M.~Yuan,
\newblock Geometry of Arithmetic Expressions: I. Basic Concepts and Unsolved Problems,
\newblock preprint, 2025.

\bibitem{YuanThermo}
M.~Yuan,
\newblock Arithmetic Expression Geometry and its Thermodynamics Correspondence,
\newblock draft note, 2025.

\end{thebibliography}

\end{document}
