\documentclass[11pt]{article}

\usepackage[margin=1in]{geometry}
\usepackage{amsmath,amssymb,amsthm,mathtools}
\usepackage{enumitem}
\usepackage{hyperref}
\usepackage{graphicx}
\usepackage{microtype}

\title{\textbf{Resource Geometry of Turing Machines}\\
and a Functorial Bridge to Arithmetic Expression Geometry (AEG)}
\author{Draft research proposal}
\date{\today}

% ---------- Macros ----------
\newcommand{\R}{\mathbb{R}}
\newcommand{\Z}{\mathbb{Z}}
\newcommand{\N}{\mathbb{N}}
\newcommand{\blank}{\sqcup}
\newcommand{\TM}{\ensuremath{\mathrm{TM}}}
\newcommand{\ACS}{\ensuremath{\mathrm{ACS}}}
\newcommand{\AEG}{\ensuremath{\mathrm{AEG}}}
\newcommand{\Conf}{\ensuremath{\mathrm{Conf}}}
\newcommand{\Path}{\ensuremath{\mathrm{Path}}}
\newcommand{\Hom}{\ensuremath{\mathrm{Hom}}}
\newcommand{\id}{\ensuremath{\mathrm{id}}}
\newcommand{\e}{\mathrm{e}}

\begin{document}
\maketitle

\begin{abstract}
This proposal studies a minimal and ``hard'' mathematical test-bed for a broader
structure/flow program: \emph{resource geometry} of Turing-machine computation.
We represent a computation as a path in a two-dimensional commutative parameter plane
spanned by \emph{time advance} and \emph{space unfolding} (tape information),
and we define closed-loop invariants by comparing alternative orderings/schedules that share
the same commutative totals.

The second goal is to build an explicit, functor-like correspondence from this resource geometry
to \AEG, a geometric framework where non-commutativity produces a torsion quantity that admits
an area formula on an accumulative commutative space.
The intended outcome is a reusable calculus: use \AEG-style area/torsion/contact tools to compute
and reason about the geometry of time--space tradeoffs for concrete machines.
\end{abstract}

\section{Motivation: from container constraints to structure/flow accounting}
Many ``boundedness'' discussions still resemble a container metaphor: a fixed capacity on one side
and a boundary flux on the other.
A more abstract and mechanically useful stance is to treat the system as a \emph{structure}
maintained by \emph{flows} (of operations, signals, negentropy, or resources).
In computation, this tension is visible in its simplest form: a Turing machine maintains and modifies
a structured external memory (tape) by a sequential flow of state transitions.

The proposal aims to turn this stance into a concrete tool by focusing on a case where the mathematics
is fully explicit: the Turing machine and its time/space resources.
We will define a \emph{resource geometry} in which:
(i) the two generating directions are time advance and space unfolding,
(ii) closed loops are defined by alternative orderings that preserve commutative totals,
and (iii) an ``area'' becomes a scalar that combines time and space into a single complexity-like quantity.

\section{Background}
\subsection{Turing machines and resource measures}
Fix a (single-tape) deterministic Turing machine
\[
M = (Q,\Gamma,\blank,\Sigma,\delta,q_0,q_{\mathrm{halt}}).
\]
A configuration $c\in \Conf(M)$ records the internal state, head position, and tape content.
A run on input $x\in \Sigma^\ast$ is a finite sequence of configurations
\[
c_0 \to c_1 \to \cdots \to c_T,
\]
where $T=T_M(x)$ is the (discrete) running time.

We will use two notions of \emph{space}:
\begin{itemize}[leftmargin=*]
\item \textbf{Geometric space:} $S_{\mathrm{geom}}(x)$ is the maximal number of tape cells in the
visited support (e.g.\ maximal head-span or number of non-blank cells), an integer.
\item \textbf{Information space:} $S_{\mathrm{info}}(x)$ is the tape information (in bits or nats)
associated with the visited support and alphabet.
A simple baseline is $S_{\mathrm{info}}(x) := (\log |\Gamma|)\, S_{\mathrm{geom}}(x)$ (in nats),
so that $\exp(S_{\mathrm{info}})=|\Gamma|^{S_{\mathrm{geom}}}$ is the number of tape microstates.
\end{itemize}

\subsection{\AEG: torsion as area on an accumulative commutative space}
In \AEG, one studies evaluation dynamics generated by two transversal directions:
addition $x\mapsto x+\mu$ and multiplication $x\mapsto x\,\e^\lambda$.
These generate a non-commutative structure with a torsion quantity measuring the discrepancy between
different orderings (e.g.\ ``add then multiply'' vs ``multiply then add'').

A key device is the \emph{Accumulative Commutative Space} $\ACS$:
a commutative plane with coordinates $(A,M)$ that record the accumulated additive and logarithmic
multiplicative charges of a path $\gamma$.
In this plane, a path $\gamma$ and its reversed-sequence counterpart $\bar\gamma$ share endpoints
and enclose a region $\Sigma_\gamma$.
A central theorem establishes a triple identity expressing global torsion as
(i) an algebraic evaluation difference,
(ii) an $\e^M$-weighted area integral over $\Sigma_\gamma$,
and (iii) a boundary integral along $\partial\Sigma_\gamma$ (a Stokes-like theorem).

This proposal treats the \AEG mechanism as a template: once we can express a resource process in terms
of a commutative accumulation plane and a curvature 2-form, we inherit a geometric calculus for
closed-loop invariants.

\section{Resource geometry: directions, paths, and closed loops}
\subsection{The resource plane \texorpdfstring{$\ACS_{\TM}$}{ACS\_TM}}
We define a resource accumulative commutative space $\ACS_{\TM}$ as a plane with coordinates
\[
(A,M)\in \R^2,
\]
where the intended semantics are:
\begin{itemize}[leftmargin=*]
\item $A$ (\textbf{time advance}) is accumulated discrete steps, so $\Delta A = 1$ per machine step.
\item $M$ (\textbf{space unfolding}) is accumulated \emph{log-space}, a commutative record of tape-information increments.
A natural choice is $M = S_{\mathrm{info}}$ (nats), so $\e^M$ equals the number of tape microstates on the visited support.
\end{itemize}

Given a run $c_0\to\cdots\to c_T$, we define an induced discrete path in $\ACS_{\TM}$:
\[
\gamma_M(x):\ (A_0,M_0) \to (A_1,M_1)\to\cdots\to(A_T,M_T),
\quad A_k:=k,\quad M_k:=S_{\mathrm{info}}(c_k),
\]
with piecewise-linear interpolation if we want a continuous curve.

\subsection{How to define a closed loop}
To obtain a loop, we need two curves in $\ACS_{\TM}$ with the same endpoints.
We will use the same principle as \AEG: compare a path with an \emph{order-reversed schedule} that preserves
commutative totals.

Given the increments $\Delta M_k := M_k-M_{k-1}$, we can form a ``reordered'' path $\bar\gamma$
by permuting the multiset $\{\Delta M_1,\dots,\Delta M_T\}$ while keeping all $\Delta A_k=1$.
Two extreme normal forms are:
\begin{itemize}[leftmargin=*]
\item \textbf{Front-loaded space:} apply positive $\Delta M$ as early as possible (allocate/expand early),
and negative $\Delta M$ as late as possible.
\item \textbf{Back-loaded space:} defer positive $\Delta M$ as late as possible (allocate late),
and apply negative $\Delta M$ as early as possible.
\end{itemize}
Both schedules end at the same $(A_T,M_T)$, hence the pair defines a closed boundary and an enclosed region
$\Sigma_\gamma$.

A core question is \emph{which reorderings correspond to realizable program transformations for the same function}.
The initial stage of the project treats reorderings as a geometric probe; later stages classify realizable
reorderings and the induced equivalence relations.

\section{A resource torsion and its area law}
\subsection{A canonical 1-form and a complexity-like functional}
On $\ACS_{\TM}$ we define a 1-form
\[
\omega := \e^{M}\, dA.
\]
This is the direct analogue of the \AEG boundary integrand.
We then define the \emph{resource action} (or ``space--time volume'' in microstate units) of a path $\gamma$ by
\[
\mathcal{W}(\gamma) := \int_{\gamma} \omega
\quad\text{(discrete: }\mathcal{W}(\gamma)\approx \sum_{k=1}^{T} \e^{M_{k-1}}\Delta A_k
= \sum_{k=0}^{T-1} \e^{M_k}\text{)}.
\]
Interpretation: $\e^{M}$ is the size of the tape microstate space, and $\mathcal{W}$ integrates it over time steps.
A more conservative variant replaces $\e^M$ by $M$ or by $S_{\mathrm{geom}}$ to recover the usual space--time product.
One goal of the proposal is to test which choice yields the most stable and meaningful invariants.

\subsection{Torsion as ordering sensitivity}
Given a path $\gamma$ and an endpoint-preserving reordering $\bar\gamma$, define the \emph{resource torsion}
\[
\tau_{\TM}(\gamma) := \mathcal{W}(\bar\gamma)-\mathcal{W}(\gamma).
\]
This quantifies how much the cumulative resource action depends on \emph{when} space is unfolded relative to time.

\subsection{The area identity (Stokes/Green)}
Since $d\omega = \e^{M}\, dM\wedge dA$, Stokes' theorem gives
\[
\tau_{\TM}(\gamma)
= \int_{\bar\gamma} \omega - \int_{\gamma}\omega
= \int_{\partial\Sigma_\gamma}\omega
= \iint_{\Sigma_\gamma} d\omega
= \iint_{\Sigma_\gamma} \e^{M}\, dM\wedge dA.
\]
Thus, the torsion equals an $\e^M$-weighted area of the enclosed region.
This identity is the resource-geometry analogue of the \AEG triple identity: a loop-sensitive scalar becomes an area.

\paragraph{Devil's advocate.}
At the level of abstract Turing machines, the step cost is constant and does not depend on $M$,
so why weight by $\e^M$?
The answer is: \emph{we are not analyzing the standard TM cost model; we are defining a geometric invariant of a run}.
The proposal will test several weightings and justify them via:
(i) physical interpretations (maintenance of stored bits, energy/heat),
(ii) algorithmic interpretations (configuration-space volume, search effort),
and (iii) invariance desiderata under program transformations.

\section{A functorial bridge to \AEG}
\subsection{Path groupoids and resource projection}
Define the path groupoid $\mathcal{G}_M$ of a Turing machine $M$:
\begin{itemize}[leftmargin=*]
\item objects: configurations $c\in \Conf(M)$,
\item morphisms: finite runs $c\to c'$ (with composition by concatenation).
\end{itemize}
There is a natural \emph{resource projection} sending a run to its path in $\ACS_{\TM}$:
\[
\mathcal{R}:\ \mathcal{G}_M \longrightarrow \Path(\ACS_{\TM}),\qquad
(c_0\to\cdots\to c_T)\ \mapsto\ \gamma_M(x).
\]

\subsection{A dictionary: time/space generators as affine generators}
\AEG is built from two basic generators acting affinely:
translation $x\mapsto x+\mu$ and dilation $x\mapsto x\,\e^\lambda$.
Their affine-matrix representation lives in $\mathrm{Aff}(1)$, and torsion is the commutator effect.

We propose a resource-expression calculus with the same algebra:
\begin{itemize}[leftmargin=*]
\item a \textbf{time generator} $T_{\mu}$: add a (possibly $M$-dependent) unit cost $\mu$ to an accumulator,
\item a \textbf{space generator} $S_{\lambda}$: scale the accumulator by a factor $\e^\lambda$ (e.g.\ memory doubling).
\end{itemize}
The commutator $T_{\mu}S_{\lambda}T_{-\mu}S_{-\lambda}$ measures ordering sensitivity and matches \AEG torsion.
This yields a direct group isomorphism between the generated affine subgroup in resource calculus and the
\AEG generator group.

\subsection{Functor-like correspondence}
Let $\mathcal{G}_{\AEG}$ denote the \AEG path groupoid (objects are points in an expression space; morphisms are
threadlike paths generated by the two operations).
We aim to define a structure-preserving map (a proto-functor)
\[
\mathcal{F}:\ \Path(\ACS_{\TM}) \longrightarrow \Path(\ACS_{\AEG})
\]
by the identification
\[
(A,M)_{\TM} \longleftrightarrow (A,M)_{\AEG},
\qquad \omega_{\TM}=\e^M dA \longleftrightarrow \omega_{\AEG}=\e^M dA.
\]
Under this identification, the torsion/area functional computed for a Turing-machine resource path is
literally the same integral expression as in \AEG.
The additional work is \emph{semantic}: interpret which TM transformations correspond to \AEG's reversed-sequence
operation, and what invariants are meaningful for complexity and for structure/flow modeling.

\section{Research questions}
We organize the project around six concrete questions.

\begin{enumerate}[leftmargin=*]
\item \textbf{Canonical resource coordinates.}
Which definition of $M$ (log-space) is most appropriate: geometric support, Shannon entropy, or a microstate count?
What choices make $\tau_{\TM}$ stable under minor implementation details?

\item \textbf{Normal forms and realizable reorderings.}
Given a run, which reorderings of $\Delta M$ can be realized by program transformations without changing the
computed function (or without changing the final tape content)?
Can we classify a family of reorderings that corresponds to known time--space tradeoffs?

\item \textbf{Resource torsion as an algorithmic invariant.}
When does $\tau_{\TM}$ distinguish two implementations with the same $(T,S)$ asymptotics but different memory schedules?
Can $\tau_{\TM}$ serve as a refined measure beyond the usual time and space?

\item \textbf{From curvature to tradeoff laws.}
The 2-form $\Omega := d\omega = \e^M dM\wedge dA$ plays the role of a curvature/area density.
Can we derive constraints (inequalities) that look like ``Gauss--Bonnet'' statements for families of computations?

\item \textbf{Concrete computations.}
Compute $\tau_{\TM}$ for small, explicit machines (write/erase machines; binary incrementers; simple transducers),
and build a library of worked examples with diagrams.

\item \textbf{Bridge to \AEG contact geometry.}
Can we lift $(A,M)$ to a 3D contact manifold $(A,M,w)$ with contact form $\alpha = dw-\e^M dA$,
develop an ``expression differential'' calculus for resources,
and map it to the \AEG contact picture in a structurally faithful way?
\end{enumerate}

\section{Methodology and work plan}
\subsection*{Phase I: formalization and toy models (weeks 1--4)}
\begin{itemize}[leftmargin=*]
\item Formal definitions of $\ACS_{\TM}$, resource paths, reorderings, $\omega$, $\mathcal{W}$, and $\tau_{\TM}$.
\item Prove basic properties: additivity under concatenation, sign under reversal, scaling under reparameterization.
\item Construct two families of toy machines and compute $\tau_{\TM}$ by hand.
\end{itemize}

\subsection*{Phase II: realizable reorderings and invariance (weeks 5--10)}
\begin{itemize}[leftmargin=*]
\item Define an equivalence relation on runs induced by semantics-preserving transformations.
\item Identify which reorderings correspond to delaying allocation, streaming, checkpointing, or recomputation.
\item Study invariance of $\tau_{\TM}$ under these transformations; propose a ``reduced torsion'' that is invariant
on equivalence classes.
\end{itemize}

\subsection*{Phase III: computational experiments and the \AEG bridge (weeks 11--16)}
\begin{itemize}[leftmargin=*]
\item Build a simulator that logs $(A_k,M_k)$ and computes $\tau_{\TM}$ via boundary sums and area estimates.
\item Implement the mapping $\mathcal{F}$ to \AEG-style formulas, verifying consistency on examples.
\item Explore the contact lift $(A,M,w)$ and compare with the \AEG contact form paradigm.
\end{itemize}

\section{Expected outcomes}
\begin{itemize}[leftmargin=*]
\item A precise definition of \emph{resource geometry} for Turing-machine runs as paths in $\ACS_{\TM}$,
with a closed-loop invariant $\tau_{\TM}$ and an explicit area law.
\item A set of canonical toy examples where $\tau_{\TM}$ can be computed exactly and interpreted as an ordering sensitivity.
\item A functor-like bridge that reuses \AEG's torsion/area machinery to compute resource invariants.
\item A small open-source codebase that computes $\tau_{\TM}$ for explicit machines and visualizes the associated regions.
\end{itemize}

\section{Risks and falsifiability}
\begin{itemize}[leftmargin=*]
\item \textbf{Risk:} $\tau_{\TM}$ depends too much on arbitrary choices of $M$.
\ \textbf{Mitigation:} test multiple definitions (support size, entropy, microstate count) and seek equivalence classes.

\item \textbf{Risk:} reorderings are rarely realizable, making torsion ``purely formal''.
\ \textbf{Mitigation:} treat torsion first as a diagnostic; then identify a minimal transformation set that yields
non-trivial realizable loops.

\item \textbf{Falsifiability:} if for broad classes of nontrivial computations all realizable reorderings yield
$\tau_{\TM}=0$ (or trivial bounds), the approach fails to add information beyond $(T,S)$.
\end{itemize}

\section{References}
\begin{enumerate}[leftmargin=*]
\item A.~M.~Turing, ``On Computable Numbers, with an Application to the Entscheidungsproblem,'' 1936.
\item M.~Yuan, \emph{Geometry of Arithmetic Expressions: I. Basic Concepts and Unsolved Problems (Draft)}, 2025.
\end{enumerate}

\end{document}
