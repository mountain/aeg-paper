\documentclass{article}
\usepackage[utf8]{inputenc}
\usepackage{amsmath}
\usepackage{amssymb}
\usepackage{amsfonts}
\usepackage{graphicx}
\usepackage{tikz}
\usetikzlibrary{angles,fit,arrows,calc,math,matrix,intersections,through,backgrounds,cd}
\usepackage{pgfplots}
\pgfplotsset{compat=1.18}
\usepackage{geometry}
\geometry{a4paper, margin=1in}
\usepackage{booktabs} % For professional looking tables

\title{Experimental Calibration of AEG Parameter $t$ via $4_1$ Knot Pseudo-Anosov Dynamics}
\author{Manus}
\date{\today}

\begin{document}
\maketitle

\section{Introduction}
This document details the third proposed research direction: the experimental calibration of the Arithmetic Expression Geometry (AEG) parameter $t$ by leveraging the pseudo-Anosov dynamics associated with the figure-eight knot ($4_1$). The core idea is to connect the stretching factor $\lambda_{4_1}$ of the $4_1$ knot's monodromy to the parameter $t$ in AEG and observe the consequences for arithmetic torsion and AEG flow dynamics.

The figure-eight knot is a fibered knot, and its complement admits a pseudo-Anosov map as its monodromy. The stretching factor of this map is a key topological invariant.

\begin{figure}[h!]
    \centering
    \documentclass{standalone}
\usepackage{amsthm}
\usepackage{amssymb}
\usepackage{amsfonts}
\usepackage{amsmath}
\usepackage{mathtools}

\usepackage{pgf}
\usepgflibrary{fpu}
\usepackage{pgfplots}
\usepackage{tikz}
\usetikzlibrary{angles,fit,arrows,calc,math,matrix,intersections,through,backgrounds,cd}
\usepackage{tkz-euclide}
\usepackage{tkz-graph}
\usepackage{graphicx}
\pgfplotsset{compat=1.18}

\begin{document}

        \tikzmath{
                \one = 1;
                \base = 2.618033988749;
                \offset = 15.8888888;
                \valofpi = 3.1415926;
                \anglei = 3.1415926;
                \angleo = 3.1415926;
        }

        \begin{tikzpicture}[scale=1.0]
                % 1. 绘制坐标轴
                \draw[black, line width=0.6pt, ->]
                (\offset,0) to[out=90,in=270] (\offset,15.5)
                node [anchor=south] {y};

                \draw[black, line width=0.6pt, ->]
                (-7.5,0) to[out=0,in=180] (18,0)
                node [anchor=west] {x};

                % 2. 绘制 x 和 y 坐标轴刻度
                \foreach \x in {-25,...,2} {
                        \node [anchor=north] at (\x/9*8 + \offset, 0) {\x};
                }
                \foreach \y in {1,...,17} {
                        \node [anchor=-135] at (18, \y/9*8) {\y};
                }

                % 3. 浅灰色水平网格线
                \foreach \t in {17,...,1} {
                        \draw [lightgray, line width=0.6pt]
                        (-7.5,\t/9*8)
                        to[out=0,in=180]
                        (18,\t/9*8);
                }

                % 4. 浅灰色竖直网格线
                \foreach \t in {-26,...,2} {
                        \draw [lightgray, line width=0.6pt]
                        (\t/9*8 + \offset, 0)
                        to[out=90,in=270]
                        (\t/9*8 + \offset, 15.5);
                }




        \end{tikzpicture}
\end{document}

    \caption{The Figure-Eight Knot ($4_1$).}
    \label{fig:knot_4_1_q3}
\end{figure}

\section{Pseudo-Anosov Dynamics and the Parameter $t$}

\subsection{The Stretching Factor $\lambda_{4_1}$}
For the figure-eight knot ($4_1$), the monodromy of its fibration is a pseudo-Anosov diffeomorphism of a punctured torus. The stretching factor (or dilatation) of this map is denoted $\lambda_{4_1}$. This value is the largest root of the Alexander polynomial $\Delta_{4_1}(t) = t^2 - 3t + 1 = 0$. 
Specifically, $\lambda_{4_1} = \frac{3 + \sqrt{5}}{2}$, which is the square of the golden ratio, $\phi^2$, where $\phi = \frac{1+\sqrt{5}}{2}$. Its inverse, $\lambda_{4_1}^{-1} = \frac{3 - \sqrt{5}}{2}$, is the other root of the Alexander polynomial.

\begin{figure}[h!]
    \centering
    \begin{tikzpicture}
        \begin{axis}[
            axis lines=middle,
            xlabel=$t$,
            ylabel=$\Delta_{4_1}(t)$,
            ymin=-1.5, ymax=2.5,
            xmin=-0.5, xmax=3.5,
            xtick={0, 0.382, 1, 2.618, 3},
            xticklabels={$0$, $\lambda_{4_1}^{-1} \approx 0.382$, $1$, $\lambda_{4_1} \approx 2.618$, $3$},
            grid=major,
            width=10cm,
            height=7cm,
            legend_pos=outer north east
        ]
        \addplot[domain=-0.5:3.5, samples=100, color=blue, thick] {x^2 - 3*x + 1};
        \addlegendentry{$t^2 - 3t + 1$}
        \addplot[color=red, mark=*, mark size=2pt] coordinates { (0.381966,0) (2.618034,0) };
        \end{axis}
    \end{tikzpicture}
    \caption{The Alexander Polynomial $\Delta_{4_1}(t) = t^2 - 3t + 1$ for the $4_1$ knot, showing its roots $\lambda_{4_1}^{-1}$ and $\lambda_{4_1}$.}
    \label{fig:alex_poly_41}
\end{figure}


\subsection{Calibrating AEG Parameter $t$}
The proposal is to set the multiplicative parameter $t$ within the AEG framework (e.g., in an arithmetic interpretation like $a \mapsto \otimes_t, b \mapsto \oplus_1$ as suggested in Notes 2) to one of these dynamically significant values, i.e., $t = \lambda_{4_1}$ or $t = \lambda_{4_1}^{-1}$.

\section{Experimental Investigation}

\subsection{Effect on Arithmetic Torsion}
With $t$ set to a root of $\Delta_{4_1}(t)$, the Alexander polynomial term in the global arithmetic torsion formula $\mathcal{T}(S) = \Delta_{4_1}(t)(t^K-1)$ becomes zero. Therefore, for any path $S$ and any integer $K$, the arithmetic torsion $\mathcal{T}(S)$ should vanish: $\mathcal{T}(S) = 0 \cdot (t^K-1) = 0$.

This provides a direct experimental test:
\begin{enumerate}
    \item Implement the arithmetic interpretation for $G(4_1)$ with $t = \lambda_{4_1}$ (or $t = \lambda_{4_1}^{-1}$).
    \item Compute the value of various relator paths $S$ in $G(4_1)$ under this interpretation.
    \item Verify that the resulting values, which correspond to $\mathcal{T}(S)$, are indeed zero (or numerically very close to zero, accounting for potential floating-point inaccuracies if numerical methods are used).
\end{enumerate}
This experiment would confirm a fundamental consistency between the AEG framework and the known properties of the $4_1$ knot's Alexander polynomial and its relation to pseudo-Anosov dynamics.

\subsection{Exploring AEG Flow Dynamics}
A more speculative but potentially fruitful direction is to explore the AEG flow equation:
\[
\frac{da}{ds} = \mu \cos \theta + a\lambda \sin \theta
\]
Here, $a$ is the assignment function, $s$ is a path parameter, and $\mu, \theta, \lambda$ are parameters of the flow. The question is whether setting the AEG parameter $\lambda$ (distinct from $\lambda_{4_1}$ used for $t$) in relation to $t = \lambda_{4_1}$ could reveal interesting dynamics.

For instance:
\begin{itemize}
    \item Could specific choices of $\lambda$ (perhaps $\lambda = \ln \lambda_{4_1}$ or a related value) lead to AEG paths in the $(U,V)$ space that exhibit characteristics reminiscent of the stable or unstable foliations of the pseudo-Anosov map on the fiber surface of the $4_1$ knot complement?
    \item Does the AEG space, when probed with this calibrated $t$, naturally exhibit structures or symmetries that align with the hyperbolic geometry of $S^3 \setminus 4_1$ or its fundamental group $G(4_1)$?
\end{itemize}
This part of the investigation is more open-ended and aims to see if the dynamical properties inherent in the $4_1$ knot (via its monodromy) can be reflected or modeled within the AEG framework by a careful choice of its parameters.

\section{Expected Outcomes}
This experimental calibration is expected to:
\begin{itemize}
    \item Verify the consistency of the AEG framework with known properties of the $4_1$ knot, specifically that arithmetic torsion vanishes when $t$ is a root of the Alexander polynomial.
    \item Provide a concrete, dynamically motivated choice for the parameter $t$ in AEG when studying the $4_1$ knot.
    \item Potentially uncover deeper connections between AEG flow dynamics and the geometric/topological structures of the $4_1$ knot complement, such as its fibration and pseudo-Anosov monodromy.
\end{itemize}

Successfully linking the AEG parameter $t$ to the intrinsic dynamics of the knot itself would be a significant step in establishing AEG as a relevant tool for knot theory.

\end{document}

