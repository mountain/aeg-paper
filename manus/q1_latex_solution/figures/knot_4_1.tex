\documentclass{standalone}
\usepackage{amsthm}
\usepackage{amssymb}
\usepackage{amsfonts}
\usepackage{amsmath}
\usepackage{mathtools}

\usepackage{pgf}
\usepgflibrary{fpu}
\usepackage{pgfplots}
\usepackage{tikz}
\usetikzlibrary{angles,fit,arrows,calc,math,matrix,intersections,through,backgrounds,cd}
\usepackage{tkz-euclide}
\usepackage{tkz-graph}
\usepackage{graphicx}
\pgfplotsset{compat=1.18}

\begin{document}

        \tikzmath{
                \one = 1;
                \base = 2.618033988749;
                \offset = 15.8888888;
                \valofpi = 3.1415926;
                \anglei = 3.1415926;
                \angleo = 3.1415926;
        }

        \begin{tikzpicture}[scale=1.0]
                % 1. 绘制坐标轴
                \draw[black, line width=0.6pt, ->]
                (\offset,0) to[out=90,in=270] (\offset,15.5)
                node [anchor=south] {y};

                \draw[black, line width=0.6pt, ->]
                (-7.5,0) to[out=0,in=180] (18,0)
                node [anchor=west] {x};

                % 2. 绘制 x 和 y 坐标轴刻度
                \foreach \x in {-25,...,2} {
                        \node [anchor=north] at (\x/9*8 + \offset, 0) {\x};
                }
                \foreach \y in {1,...,17} {
                        \node [anchor=-135] at (18, \y/9*8) {\y};
                }

                % 3. 浅灰色水平网格线
                \foreach \t in {17,...,1} {
                        \draw [lightgray, line width=0.6pt]
                        (-7.5,\t/9*8)
                        to[out=0,in=180]
                        (18,\t/9*8);
                }

                % 4. 浅灰色竖直网格线
                \foreach \t in {-26,...,2} {
                        \draw [lightgray, line width=0.6pt]
                        (\t/9*8 + \offset, 0)
                        to[out=90,in=270]
                        (\t/9*8 + \offset, 15.5);
                }




        \end{tikzpicture}
\end{document}
