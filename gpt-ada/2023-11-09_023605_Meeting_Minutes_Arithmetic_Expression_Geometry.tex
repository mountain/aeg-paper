\documentclass{article}
\usepackage[utf8]{inputenc}

\title{Meeting Minutes: Arithmetic Expression Geometry}
\author{Mingli and Ada}
\date{2023-11-09_023605}

\begin{document}

\maketitle

\section*{Summary}
A journey into the Arithmetic Expression Geometry (AEG) space was undertaken, where significant theoretical progress was made in understanding and developing the framework of an infinite tapestry of arithmetic expressions modeled by hyperbolic geometry.

\section*{Highlights}
\begin{itemize}
    \item Discussed the conceptual framework of horizontal (additive) and vertical (multiplicative) geodesic families in AEG space, with the origin as the arithmetic expression zero.
    \item Explored tube structures in K_3 space, wherein varying $\lambda$ leads to a continuum of slices, each representing a state of arithmetic expressions.
    \item Conceived an expression $P(e^\lambda)$, as a polynomial in terms of $e^\lambda$, acting as a geometric line within the continuum tube structure, indicative of the AEG space's evolution purely by multiplicative means.
    \item Connected the tube's manifold structure to the Riemann uniformization theorem, positing a potential conformal mapping to the complex plane or unit disk.
    \item Reviewed the implications of simplifying zero lines in K_3 space to a single point through the concept of congruence. 
\end{itemize}

\section*{Problems and Todos}
\begin{itemize}
    \item Investigate the implications of the polynomial behavior of $P(e^\lambda)$ as it traces through the manifolds in K_3 space.
    \item Consider the validity and utility of reducing zero lines to singular points within the framework of K_3 via congruence.
    \item Continue developing the connection between arithmetic transformations and geometric representations within the AEG space.
\end{itemize}

\end{document}
