\documentclass[11pt]{article}
\usepackage[margin=1in]{geometry}
\usepackage{amsmath,amssymb,amsthm}
\usepackage{hyperref}
\usepackage{enumitem}

\newtheorem{theorem}{Theorem}
\newtheorem{lemma}{Lemma}
\newtheorem{prop}{Proposition}
\newtheorem{cor}{Corollary}
\newtheorem{definition}{Definition}
\theoremstyle{remark}
\newtheorem{remark}{Remark}

\begin{document}

    \title{Arithmetic Expression Spaces and 3D Manifold Surgery:\\
    Loops, Coverings, and Fundamental Groups}
    \author{}
    \date{}

    \maketitle

    \section*{Introduction}

\section*{Arithmetic Expression Spaces as Curved Geometric Structures}
An arithmetic expression space is an abstract geometric object whose points correspond to arithmetic expressions, and whose structure is designed so that performing algebraic operations equates to moving within the space. In such a space, the binary operations of addition and multiplication act like independent “directions” of motion. For example, starting at a point representing some value $x$, moving along one generator direction could correspond to adding a fixed number (e.g. $+1$), while moving along another corresponds to multiplying by a certain factor (e.g. $\times 2$). Formally, one can imagine a Cayley graph or manifold associated with a presentation having two kinds of generators: one for addition and one for multiplication (and their inverses if defined). The resulting space has a natural non-Euclidean geometry because the two operations do not commute trivially and can produce exponential branching in the structure. In fact, a tree-like structure emerges if we treat different sequences of operations as distinct paths. This suggests equipping the space with negative curvature (as in a hyperbolic metric) to accommodate the rapid “spreading out” of paths created by multiplication.

\href{https://globberingmattress.wordpress.com/2017/12/26/deck-transformations-revisted/#:~:text=are%20quite%20unnatural,means%20forgetting%20about%20the%20geometry}{globberingmattress.wordpress.com}
\href{https://globberingmattress.wordpress.com/2017/12/26/deck-transformations-revisted/#:~:text=Image%3A%20cayley_graph_of_f2}{globberingmattress.wordpress.com}
For a simple illustration, consider the free group on two generators (analogous to an expression space with two basic operations but no relations). Its Cayley graph is an infinite 4-valent tree, which expands exponentially from any starting vertex. Such a tree cannot be embedded isometrically in a flat plane without distortion, but it fits naturally inside the hyperbolic plane. In fact, one can embed the 4-valent tree in the Poincaré disk model of the hyperbolic plane, with each edge a geodesic arc\href{https://globberingmattress.wordpress.com/2017/12/26/deck-transformations-revisted/#:~:text=are%20quite%20unnatural,means%20forgetting%20about%20the%20geometry}{globberingmattress.wordpress.com}
. The negative curvature of the disk “opens up” enough room for the exponentially branching graph to sit without overlaps\href{https://globberingmattress.wordpress.com/2017/12/26/deck-transformations-revisted/#:~:text=Image%3A%20cayley_graph_of_f2}{globberingmattress.wordpress.com}
. \textit{This exemplifies how a curved geometry can realize an arithmetic expression space:} addition and multiplication generate paths that diverge exponentially (much like a free group), and hyperbolic space provides the curvature to accommodate this growth. Essentially, addition could be viewed as a translation along one geodesic direction, and multiplication as a different motion (a hyperbolic rotation or dilation) – both are isometries in an appropriate metric. By constructing the expression space with a hyperbolic or otherwise curved metric, these operations become geodesic motions.

More concretely, one can model a simplified arithmetic expression space as a group of transformations. For instance, consider the group of affine transformations of the real line of the form $x \mapsto ax+b$. Here $b$ (translation) plays the role of addition and $a$ (scaling) the role of multiplication. This affine group can be given a geometric structure (it is in fact a 2-dimensional Lie group) which one can analyze for curvature. In general, if the operations are independent (unconstrained by algebraic relations), the structure is essentially a \textit{free group or free monoid} on those operations and is hyperbolic (in the sense of Gromov)

\href{https://www3.nd.edu/~gszekely/rtg/GTS/www3.nd.edu/_jquigle2/GSTS%20FA18/Week3.pdf#:~:text=Definition%204,2}{www3.nd.edu}
. Free groups of rank $n\ge 2$ have Cayley graphs that are trees (0-curvature), hence they are word-hyperbolic groups\href{https://www3.nd.edu/~gszekely/rtg/GTS/www3.nd.edu/_jquigle2/GSTS%20FA18/Week3.pdf#:~:text=Definition%204,2}{www3.nd.edu}
. By contrast, if addition and multiplication commute or satisfy significant relations, the space gains flat substructures. For example, if one allowed arbitrary addition and multiplication that commute (ignoring the usual distributive law structure), one would effectively get a $\mathbb{Z}\times \mathbb{Z}$ grid as a subspace (one axis for repeated $+1$, another for repeated $\times p$). A $\mathbb{Z}^2$ in the Cayley graph introduces a Euclidean plane sector, which is not hyperbolic (Euclidean planes fail the thin-triangle condition)\href{https://www3.nd.edu/~gszekely/rtg/GTS/www3.nd.edu/_jquigle2/GSTS%20FA18/Week3.pdf#:~:text=%28b%29%20Finite%20groups%20are%20hyperbolic,Z%202%20is%20not%20hyperbolic}{www3.nd.edu}
. Thus, an arithmetic expression space that respects the usual arithmetic identities will have loops (relations) that partially “flatten” the geometry. The classic distributive law, $a*(b+c) = a\textit{b + a}c$, for instance, introduces a loop in the space: one path applies addition then multiplication, another applies multiplication then addition, arriving at the same point. Such loops are analogous to flat quadrilaterals in the geometry, indicating zero curvature along that cycle.

To construct a rigorous model, one can start with a universal cover of an expression complex: take a node for each distinct formal expression and connect it to nodes representing the results of adding or multiplying by allowed constants or variables. The result is a rooted tree if no identifications are made. Imposing the usual arithmetic equivalences (commutativity, distributivity, etc.) means quotienting this tree by certain identifications, yielding a quotient space with non-trivial loops corresponding to those algebraic laws. One can view this quotient as a sort of CW-complex: for example, distributivity adds a 2-dimensional face gluing the path $a*(b+c)$ to the path $a\textit{b + a}c$. The \textit{arithmetic expression space} is this quotient complex. It inherits a path metric from the tree (each operation step has unit length, say). If the original tree was given a hyperbolic embedding, the quotient will locally remain negatively curved except along the identified loops (where curvature may go to zero or positive, depending on how the identification is done). In many cases, the quotient still admits a hyperbolic metric in a coarse sense – for instance, many groups defined by “reasonable” relations remain Gromov-hyperbolic. (If the relations introduce a $\mathbb{Z}^2$, though, true hyperbolicity is lost.) This construction aligns with ideas from geometric group theory: we are essentially studying the geometry of a group (or group-like structure) presented by generators (the operations) and relations (the arithmetic identities).

\section*{Loops in Expression Spaces as Covering Transformations}
In any geometric space, a loop (closed path) represents an element of the fundamental group $\pi_1$. In an arithmetic expression space, loops arise from sequences of operations that return to the starting expression. For example, starting at an expression $E$, follow a path of additions and multiplications that eventually produces an expression algebraically equal to $E$ again – this closed path is a loop in the space corresponding to an arithmetic identity. These loops have a dual interpretation: they generate \textit{symmetries of the universal covering space}. Formally, if $X$ is our expression space (which may be a complicated quotient with loops), let $\widetilde{X}$ be its universal cover. $\widetilde{X}$ is simply connected (all loops lifted to $\widetilde{X}$ unravel to paths), and it covers $X$. Any loop in $X$ lifts to a path in $\widetilde{X}$ that starts and ends at possibly different points – the end point is a \textit{deck transformation} of the cover applied to the start point. In fact, there is a well-known correspondence: the deck transformation group of the universal cover is isomorphic to the fundamental group of the base space

\href{https://www3.cs.stonybrook.edu/~gu/lectures/2022/Lecture_1/2022_Summer_CCG_Lecture_1.pdf#:~:text=Definition%20,3%2C%202022%2044%20%2F%2059}{www3.cs.stonybrook.edu}
\href{https://www3.cs.stonybrook.edu/~gu/lectures/2022/Lecture_1/2022_Summer_CCG_Lecture_1.pdf#:~:text=The%20quotient%20group%20of%20%CF%801,deck%20transformation%20group%20of%20S%CB%9C}{www3.cs.stonybrook.edu}
. Each homotopy class of loop in $X$ gives an automorphism of $\widetilde{X}$ (moving a chosen base lift to another lift). In the context of our arithmetic space, \textit{each loop acts as an isometry (symmetry) of the universal covering space}. This action is by necessity an isometry if we equip $\widetilde{X}$ with the lifted metric from $X$ – deck transformations always preserve the metric and structure of the cover by definition.

To make this concrete, consider a simple case: $X = S^1$, a circle, which we can think of as an expression space for “adding $1$ modulo $\mathbb{Z}$.” The universal cover is $\widetilde{X} = \mathbb{R}$, the real line, and each loop around $S^1$ (the generator of $\pi_1(S^1)=\mathbb{Z}$) corresponds to a translation of $\mathbb{R}$ by one period. Indeed, the deck transformation group of $\mathbb{R}\to S^1$ is the infinite cyclic group of integer translations, which is isomorphic to $\pi_1(S^1)$

math.uchicago.edu
. The translation is an isometry of $\mathbb{R}$ (with the standard metric) and represents how the fundamental group acts on the cover. Now, for a more intricate example, suppose $X$ is a hyperbolic 3-manifold (imagine our expression space managed to be a closed hyperbolic manifold). Then $\widetilde{X}$ is hyperbolic 3-space $ \mathbb{H}^3$, and a closed loop in $X$ (especially a geodesic loop) corresponds to a deck transformation of $\mathbb{H}^3$. In hyperbolic geometry, nontrivial deck transformations are isometries of $\mathbb{H}^3$ – specifically, a loop that is homotopic to a closed geodesic corresponds to a hyperbolic isometry that translates along the geodesic’s axis in $\widetilde{X}$ (and possibly rotates)\href{https://math.rice.edu/~ar99/LLR_v4.pdf#:~:text=match%20at%20L320%20loxo%02dromic%20element,and%20commutes%20with%20a}{math.rice.edu}
. In group terms, the loop is an element $\gamma$ of $\pi_1(X)$, and in the standard Kleinian group model, $\gamma$ is realized as a loxodromic isometry of hyperbolic space with a fixed invariant geodesic (its axis)\href{https://math.rice.edu/~ar99/LLR_v4.pdf#:~:text=match%20at%20L320%20loxo%02dromic%20element,and%20commutes%20with%20a}{math.rice.edu}
. Thus \textit{loops in the base expression space manifest as isometries of the covering space.} This is precisely the concept of monodromy: going around a loop applies a certain symmetry to the fiber (here the fiber \textit{is} the universal cover).

If our arithmetic expression space $X$ was initially constructed as a quotient of a tree (as discussed above), then its universal cover $\widetilde{X}$ may essentially be that infinite tree (before quotienting by arithmetic relations). In that case, loops in $X$ correspond to deck transformations of the tree. For instance, the distributivity loop $a*(b+c) = a\textit{b + a}c$ would correspond to a deck transformation on the covering tree that swaps one branch for another – an isometry of the infinite tree graph mapping the path for $a*(b+c)$ to the path for $a\textit{b + a}c$. The tree (being homogeneous) has many symmetries; the ones coming from these algebraic loops are just those needed to quotient the tree down to the expression space. Topologically, we can say: \textit{the arithmetic expression space $X$ is the quotient of its simply-connected cover $\widetilde{X}$ by the action of the fundamental group}, $X = \widetilde{X} / \pi_1(X)$

\href{https://www3.cs.stonybrook.edu/~gu/lectures/2022/Lecture_1/2022_Summer_CCG_Lecture_1.pdf#:~:text=Definition%20,3%2C%202022%2044%20%2F%2059}{www3.cs.stonybrook.edu}
\href{https://www3.cs.stonybrook.edu/~gu/lectures/2022/Lecture_1/2022_Summer_CCG_Lecture_1.pdf#:~:text=The%20quotient%20group%20of%20%CF%801,deck%20transformation%20group%20of%20S%CB%9C}{www3.cs.stonybrook.edu}
. Each $\pi_1$ element (loop) acts by a deck transformation on $\widetilde{X}$, moving points along that symmetry until the entire orbit of a point under $\pi_1$ corresponds to all the identical expressions identified by that loop. The quotient glues those points together, reproducing the loop in the base space. This viewpoint is powerful because it means we can study complicated loops in $X$ by looking at isometries in $\widetilde{X}$, which is often a much nicer space (like a tree or $\mathbb{H}^n$). The length or shape of a loop can be understood via the corresponding isometry (for example, the length of a closed geodesic in $X$ is related to the translation length of the isometry in the cover).

In summary, loops in the arithmetic space act as deck transformations (isometries) of the universal cover. This leverages the fundamental theorem of covering spaces: $\pi_1(X)$ is isomorphic to the group of deck transformations of $\widetilde{X}$

\href{https://www3.cs.stonybrook.edu/~gu/lectures/2022/Lecture_1/2022_Summer_CCG_Lecture_1.pdf#:~:text=Definition%20,3%2C%202022%2044%20%2F%2059}{www3.cs.stonybrook.edu}
\href{https://www3.cs.stonybrook.edu/~gu/lectures/2022/Lecture_1/2022_Summer_CCG_Lecture_1.pdf#:~:text=The%20quotient%20group%20of%20%CF%801,deck%20transformation%20group%20of%20S%CB%9C}{www3.cs.stonybrook.edu}
. By studying these deck transformations, we gain insight into the algebraic structure of the loops. In practice, this means an identity like $E_1 = E_2$ (which forms a loop in $X$ by traveling from $E_1$ to $E_2$ and then back along the identified path) corresponds to a symmetry of $\widetilde{X}$ mapping one lift of $E_1$ to another lift of $E_2$. If the space has a hyperbolic or metric structure, these symmetries are rigorously isometries of $\widetilde{X}$ – for instance, translations, rotations, or hyperbolic motions – ensuring that the correspondence between algebra (loops) and geometry (motions) is exact.

\section*{3D Surgery: Cutting, Gluing, and Induced Covering Spaces}
3D manifold surgery is a process of cutting a 3-dimensional manifold along a submanifold (often a torus or loop) and re-gluing it in a different way to produce a new manifold. A quintessential example is Dehn surgery on a knot in a 3-manifold: one removes a tubular neighborhood of the knot (which is topologically $S^1 \times D^2$) and then glues it back so that a different loop on the torus boundary is identified as the meridian of the solid torus being glued in. This operation has a profound effect on the topology – in particular, on the fundamental group. From the fundamental group perspective, performing surgery along a loop adds a new relation that \textit{kills} that loop (or a combination of loops) in the group. The loop that gets filled in (the meridian of the removed torus) becomes contractible after surgery, so its homotopy class is trivial in the new manifold. As a result, the fundamental group of the new manifold is the old fundamental group modulo the normal subgroup generated by that loop relation

\href{https://math.stackexchange.com/questions/2477228/fundamental-group-of-3-manifold-obtained-by-surgery-description#:~:text=The%20fundamental%20group%20of%20the,to%20the%20following%20additional%20relations}{math.stackexchange.com}
.

Seifert–van Kampen’s theorem provides a method to compute this. In Dehn surgery, we glue two pieces along a torus boundary. The fundamental group of the glued manifold is an amalgam of the fundamental groups of the pieces, with an added relation identifying the peripheral subgroup loops according to the gluing map. In practical terms, one can write $\pi_1(\text{new}) = \pi_1(\text{exterior}) *_{\pi_1(\text{torus})} \pi_1(\text{solid torus})$. The solid torus’s fundamental group is $\mathbb{Z}$ (generated by its meridian $m$), and the gluing map sends $m$ to a curve $p\mu + q\lambda$ on the knot exterior’s torus (where $\mu$ is the meridian of the knot and $\lambda$ the longitude of the knot in the original exterior). Thus the combined fundamental group is the free product modulo the relation $m = \mu^p\lambda^q$

\href{https://math.stackexchange.com/questions/2477228/fundamental-group-of-3-manifold-obtained-by-surgery-description#:~:text=The%20fundamental%20group%20of%20the,to%20the%20following%20additional%20relations}{math.stackexchange.com}
. But $m$ also dies (because the solid torus interior kills it), so effectively $\mu^p \lambda^q = 1$ in the new manifold\href{https://math.stackexchange.com/questions/2477228/fundamental-group-of-3-manifold-obtained-by-surgery-description#:~:text=The%20fundamental%20group%20of%20the,to%20the%20following%20additional%20relations}{math.stackexchange.com}
. In summary, \textit{the new fundamental group is the old knot group with the additional relation $\mu^p \lambda^q = 1$.} For example, a $p$-fold Dehn surgery (when $q=1$ without loss of generality) kills $\mu^p$ in the knot group. If $p=1$ (meridian filled trivially), the knot itself becomes homotopically trivial in the new space (one recovers the original $S^3$ if starting from a knot exterior). If $p>1$, the new manifold’s fundamental group is a quotient where some power of the meridian is trivial. This is how lens spaces arise: performing a $p/q$ surgery on the unknot in $S^3$ yields a lens space whose fundamental group is $\mathbb{Z}/p\mathbb{Z}$ (since $\mu^p=1$ yields a finite cyclic group)\href{https://math.stackexchange.com/questions/2477228/fundamental-group-of-3-manifold-obtained-by-surgery-description#:~:text=The%20second%20relation%20says%20that,is%20almost%20what%20you%20said}{math.stackexchange.com}
. The process of surgery thus directly corresponds to \textit{adding a relation to the fundamental group}. In algebraic terms, it’s akin to taking a presentation and modding out by the normal closure of a word (the word representing the loop we glued). Geometrically, we are attaching a 2-dimensional disk along that loop (filling the loop in), which precisely kills it in $\pi_1$.

Now, how does this relate to covering spaces? Consider an initial space $X$ and a loop $\gamma$ in it along which we perform surgery to obtain $Y$. Let $G = \pi_1(X)$ and $N = \langle!\langle \gamma \rangle!\rangle$ (the normal closure of $\gamma$) in $G$. Then $\pi_1(Y) \cong G/N$. Intuitively, the universal cover of $Y$ can be obtained from the universal cover of $X$ by \textit{attaching cells corresponding to those new relations} so that $\gamma$’s lifts become null-homotopic. In the covering picture, cutting along a loop corresponds to excising a tube in the cover, and gluing corresponds to identifying the sheets in a new way. More formally, if $\widetilde{X}$ is the universal cover of $X$ (with deck group $G$), then the subgroup $N$ corresponds to a certain covering of $X$ – in fact $N$ is the kernel of the surjection $G \to G/N$, so $N$ is a normal subgroup. The quotient $\widetilde{X}/N$ can be shown to be (isomorphic to) the universal cover of $Y$. One way to see this is that $\widetilde{X}/N$ is a covering of $X/N = Y$ with deck transformation group $G/N$ (by the First Isomorphism Theorem, $G/N$ acts freely on $\widetilde{X}/N$ with that quotient)

\href{https://www3.cs.stonybrook.edu/~gu/lectures/2022/Lecture_1/2022_Summer_CCG_Lecture_1.pdf#:~:text=Definition%20,3%2C%202022%2044%20%2F%2059}{www3.cs.stonybrook.edu}
\href{https://www3.cs.stonybrook.edu/~gu/lectures/2022/Lecture_1/2022_Summer_CCG_Lecture_1.pdf#:~:text=The%20quotient%20group%20of%20%CF%801,deck%20transformation%20group%20of%20S%CB%9C}{www3.cs.stonybrook.edu}
. Thus $\widetilde{X}/N$ is a connected cover of $Y$ whose fundamental group is $N$; but $N$ was exactly the kernel making $G/N$ the full $\pi_1(Y)$, so $N$ is trivial as a subgroup of $\pi_1(Y)$ – meaning $\widetilde{X}/N$ is simply connected. Hence $\widetilde{X}/N$ is the universal cover of $Y$. In plainer terms: \textit{to construct the new universal cover, one can start with the old universal cover and identify any point that differed only by the now-killed loop}. All copies that were previously related by the loop’s deck transformation collapse together when that loop is killed.

From a topological surgery viewpoint, what we have done is attach a 2-cell in $X$ along the loop $\gamma$ (that’s the filling). In the universal cover, there were many lifts of $\gamma$; attaching a 2-cell along each lift of $\gamma$ in $\widetilde{X}$ (equivariantly for all deck transformations) produces a simply connected space $\widetilde{Y}$. In effect, you fill in all the tunnels that corresponded to $\gamma$ in $\widetilde{X}$, so that $\gamma$ can no longer produce a nontrivial loop. This \textit{extends the universal cover} by removing the “gap” that $\gamma$ represented. One can visualize it in an example: Let $X$ be a knot complement in $S^3$ (which is a manifold with torus boundary). Its universal cover $\widetilde{X}$ is an infinite 3-space with an infinite pattern of “tubular holes” corresponding to the lifts of the torus boundary. When we glue a solid torus into $X$ (Dehn fill), in the cover we glue infinitely many copies of the universal cover of the solid torus (which is a tube that extends to $\infty$) along each of those boundary lifts. The resulting space $\widetilde{Y}$ is simply connected (one can show it ends up homeomorphic to $\mathbb{R}^3$ in many cases). In particular, when $X$ is hyperbolic, $Y$ is often also hyperbolic (by Thurston’s hyperbolic Dehn surgery theorem), so $\widetilde{Y} \cong \mathbb{H}^3$. In those cases, the universal cover before and after surgery is (topologically) the same $\mathbb{R}^3$, just the group of deck transformations has changed. For example, the figure-8 knot complement has $\widetilde{X} = \mathbb{H}^3$ (since it’s hyperbolic) and $G = \pi_1(X)$ an arithmetic Kleinian group. After an appropriate filling, one obtains a closed hyperbolic $Y$ whose $\widetilde{Y} = \mathbb{H}^3$ as well, but now with a smaller deck group $G/N$. In general, most 3-manifolds (except certain exotic cases) have universal cover $\mathbb{R}^3$

\href{https://mathoverflow.net/questions/321640/homology-sphere-with-mathbbr3-as-the-universal-cover#:~:text=Homology%20sphere%20with%20,there%20do%20exist%20integer}{mathoverflow.net}
, so surgery typically doesn’t change the \textit{homeomorphism type} of the universal cover – it remains simply connected $\mathbb{R}^3$ – but it changes how the fundamental group (deck group) wraps that cover up to form the manifold.

Importantly, performing surgery can sometimes produce new covering relations between manifolds. If a loop $\gamma$ of $X$ is of finite order in $\pi_1(Y)$ (say $\mu^p=1$ after surgery yields a finite cyclic quotient), then the covering space corresponding to the subgroup that avoids that identification can be interpreted as an \textit{unfilled version covering the filled one}. For instance, $S^3$ with an unknot ($X$) covers the lens space $L(p,1)$ ($Y$) after $p$-surgery in a branched sense; or more concretely, one can view $L(p,q)$ as the result of gluing, which is also the $\mathbb{Z}/p$-quotient of the knot complement’s $p$-fold cyclic cover. In any case, the theme is that cutting and gluing along loops modifies the fundamental group in a controlled way, and hence one can track how \textit{covering spaces} (which correspond to subgroups) change. A subgroup of $\pi_1(Y)$ corresponds to some subgroup of $\pi_1(X)$ that contains $N$. Thus any covering of the new manifold $Y$ lifts to a covering of $X$ (because if $H<\pi_1(Y)$, its preimage in $G$ is a subgroup of $G$ containing $N$, giving a cover of $X$ which then factors through to $Y$). This is the algebraic “reading-off” of how new covers relate to old ones.

In summary, 3D surgery adds relations that kill loops, altering the fundamental group by a quotient, and thereby inducing changes in the universal cover and other covering spaces. The loop along which we cut/glue goes from being nontrivial in $\pi_1(X)$ to trivial in $\pi_1(Y)$. The universal cover of $Y$ is obtained from that of $X$ by identifying (or filling in) all lifts of the loop, yielding a simply connected space. From the covering space perspective, one can say the original manifold $X$ (before surgery) is often a \textit{cover} of the new manifold $Y$ in a loose sense: there’s a surjection $\pi_1(X) \twoheadrightarrow \pi_1(Y)$, though this need not correspond to an actual topological covering unless the kernel is subgroup of $\pi_1(X)$ (here it’s a normal subgroup generated by one element, which typically is not of finite index). However, one can analyze finite-sheeted coverings of $Y$ by lifting them to (possibly infinite-sheeted) coverings of $X$. This technique is frequently used in 3-manifold theory to understand how properties (like Haken-ness or hyperbolicity) survive under Dehn filling by examining covers: e.g. if a certain cover of $X$ has a property, often some cover of $Y$ will inherit a related property if the subgroup relationships line up.

\section*{Universal Covers and Fundamental Groups after Surgery}
The universal cover of a space is tightly linked to its fundamental group, as we’ve discussed: $\pi_1(X)$ is the group of deck transformations of $\widetilde{X}$. When we modify a space by surgery or other operations, we correspondingly change its universal cover and fundamental group. Let us formalize this connection in the context of the preceding discussion.

If $Y$ is obtained from $X$ by adding a relation in $\pi_1$ (such as killing a loop $\gamma$), then $\pi_1(Y) = \pi_1(X)/\langle!\langle \gamma\rangle!\rangle$. The universal cover $\widetilde{Y}$ can be thought of as $\widetilde{X}/\langle!\langle \widetilde{\gamma}\rangle!\rangle$, where $\widetilde{\gamma}$ denotes the set of all lifts of $\gamma$ in $\widetilde{X}$. Intuitively, $\widetilde{Y}$ is constructed by taking $\widetilde{X}$ and “glueing together” or identifying points that were related by the now-trivial loop. In practice, one attaches a cell along each lifted loop, making those loops contractible. The result is that $\widetilde{Y}$ is simply connected, and its group of deck transformations is isomorphic to $\pi_1(Y)$. This matches the algebraic quotient: one has an exact sequence $1 \to \pi_1(\widetilde{X}) (=1) \to \pi_1(X) \to \pi_1(Y)\to 1$, and covering space theory tells us $\widetilde{Y} = \widetilde{X}/\pi_1(\widetilde{X})$ (trivial) is $\widetilde{X}$ with the action of $\ker(\pi_1(X)\to\pi_1(Y))$ (which is $\langle!\langle\gamma\rangle!\rangle$) collapsed. In effect, the universal cover functor commutes with this kind of quotient: $\widetilde{Y} \cong \widetilde{X}/N$ for $N=\ker(\pi_1(X)\to\pi_1(Y))$.

A concrete outcome of this is that the fundamental group of $Y$ acts on $\widetilde{Y}$ as deck transformations, just as $\pi_1(X)$ acted on $\widetilde{X}$. But because $\pi_1(Y)$ is $\pi_1(X)$ modulo a relation, the action of $\pi_1(X)$ on $\widetilde{X}$ factors through to an action of $\pi_1(Y)$ on $\widetilde{Y}$. Geometrically, any loop in $Y$ (which is a coset of a loop in $X$ that wasn’t killed) still lifts to a deck transformation of $\widetilde{Y}$. Meanwhile, the particular loop that was killed now lifts to \textit{nothing} (or more precisely, its lifts become boundaries of attached disks rather than loops around holes). For example, in a $p/q$ Dehn surgery, the meridian $\mu$ of the knot satisfies $\mu^p \lambda^q = 1$ in $Y$. In the universal cover, this means if you follow $p$ lifts of $\mu$ and $q$ lifts of $\lambda$, you end up in the same place – those deck transformations composed now give the identity. So a certain combination of deck transformations in $\widetilde{X}$ becomes trivial in $\widetilde{Y}$.

The relationship between universal covers, fundamental groups, and deck transformations is beautifully exemplified in computational topology software. For instance, SnapPy, a tool for studying 3-manifolds, can explicitly compute how Dehn filling affects the fundamental group. When you specify a Dehn filling on a cusp (torus boundary) in SnapPy, the software “kills” the corresponding peripheral loop in the fundamental group presentation

\href{https://snappy.computop.org/manifold.html#:~:text=Return%20a%20HolonomyGroup%20representing%20the,corresponding%20peripheral%20elements%20are%20killed}{snappy.computop.org}
. The output fundamental group is literally the quotient by that relation, and SnapPy’s internal representation (the \textit{holonomy group}) reflects that the peripheral element is now trivial\href{https://snappy.computop.org/manifold.html#:~:text=Return%20a%20HolonomyGroup%20representing%20the,corresponding%20peripheral%20elements%20are%20killed}{snappy.computop.org}
. This is essentially the software enacting the theoretical result that $\pi_1$ gets quotiented. The universal cover of the filled manifold is not constructed explicitly by SnapPy (since it’s infinite), but one infers that it has one fewer “cusp” direction – all lifts of the filled loop are capped off.

Crucially, universal covers provide the link to the fundamental group’s geometry. In many cases, especially for \textit{word-hyperbolic groups} (which include fundamental groups of hyperbolic manifolds), the universal cover can be endowed with a geometric structure of constant curvature. For example, if $X$ is a closed hyperbolic 3-manifold, $\widetilde{X}\cong \mathbb{H}^3$ and $\pi_1(X)$ is a discrete group of isometries of $\mathbb{H}^3$. If we do a surgery to get $Y$ which is also hyperbolic, then $\widetilde{Y}\cong \mathbb{H}^3$ as well, and $\pi_1(Y)$ is a subgroup of $\pi_1(X)$ of finite index or a quotient by some geometric relation. In this scenario, the universal covers are \textit{the same space (H³) but with different group actions}. More generally, even if the covers are not literally the same, they are often quasi-isometric (having similar large-scale geometry) because surgery only changes the group by a “small” piece. Universal covers also tie into the concept of \textit{universal objects} in other contexts: for instance, in algebraic geometry, one has the notion of an \textit{étale fundamental group} and a \textit{universal pro-cover}, drawing a deep analogy between topological covers and field extensions in number theory.

One particularly beautiful connection is with arithmetic topology (the analogy between number fields and 3-manifolds). In this analogy, the universal cover of a 3-manifold (with its deck transformation group) is akin to the \textit{universal pro-finite extension} of a number field (with Galois group acting). The fundamental group of a knot complement is conjecturally analogous to the absolute Galois group of $\mathbb{Q}$ in various ways

\href{https://math.stackexchange.com/questions/38936/analogies-between-prime-ideals-and-knots#:~:text=,between%20prime%20ideals%20and%20knots}{math.stackexchange.com}
. Just as the universal cover of a 3-manifold corresponds to all finite-sheeted covers (through its deck group’s subgroups), the \textit{maximal unramified extension} of a number field corresponds to all finite extensions unramified at certain places (through its Galois group’s subgroups). The fundamental group thus plays a role analogous to a Galois group, and the universal cover corresponds to a sort of \textit{universal field extension}. This analogy is made precise in arithmetic topology: loops in a 3-manifold (like a meridian around a knot) correspond to prime ideals in a number ring, and covers of the manifold correspond to field extensions of the number field\href{https://old.maa.org/press/maa-reviews/knots-and-primes-an-introduction-to-arithmetic-topology#:~:text=All%20right%2C%20then%2C%20what%20does,manifolds%20and%20%E2%80%9Cnumber%20rings.%E2%80%9D}{old.maa.org}
. In particular, the \textit{universal cover} (pro-finite) of a 3-manifold corresponds to the \textit{arithmetic universal cover} (the field obtained by joining all algebraic extensions) of a number field. The fundamental group acting on the universal cover aligns with the absolute Galois group acting on the algebraic closure\href{https://old.maa.org/press/maa-reviews/knots-and-primes-an-introduction-to-arithmetic-topology#:~:text=All%20right%2C%20then%2C%20what%20does,manifolds%20and%20%E2%80%9Cnumber%20rings.%E2%80%9D}{old.maa.org}
. Thus, understanding $\widetilde{X}$ and $\pi_1(X)$ for a manifold can shed light on analogous structures in arithmetic, and vice versa.

In conclusion, universal covers serve as the canvas on which the fundamental group draws the manifold. After a surgery, the canvas remains simply connected, but the way the fundamental group tiles it changes. The new fundamental group is a quotient of the old, and its action on the universal cover is by a subgroup of the old action (if one views $\widetilde{Y}$ inside $\widetilde{X}$ appropriately). This perspective is powerful: many questions about the new manifold’s fundamental group (like is it word-hyperbolic? is it finite? does it act freely on some space?) can be answered by understanding how the addition of relations modifies the group action on the universal cover. For example, adding the relation $\mu^p=1$ can break hyperbolicity if it forces a $\mathbb{Z}/p$ torsion cycle that lifts to a rotational symmetry – but in most hyperbolic Dehn fillings, the resulting group remains torsion-free and hyperbolic for all but finitely many slopes, so the universal cover stays as $\mathbb{H}^3$. All these considerations show the interplay between cutting/gluing (surgery), group quotients, and covering space theory.

\section*{Connections to Knot Theory and 3-Manifold Topology}
The ideas of arithmetic expression spaces, loops, and covering transformations have rich parallels and applications in knot theory and 3-manifold topology. In knot theory, one studies knots via their complements in $S^3$; the fundamental group of a knot complement is called the knot group. This group can be thought of as generated by loops going around portions of the knot (meridians and others) with relations coming from the knot diagram (like Wirtinger relations). It is not a coincidence that many knot groups resemble the kinds of groups we’ve been discussing – they often have a presentation with one relation for each crossing (which is somewhat analogous to an arithmetic identity). For example, the trefoil knot group has a presentation $\langle a,b \mid a^2 = b^3 \rangle$, which is reminiscent of a relation mixing two generators in a nontrivial way (in fact, it’s isomorphic to the Baumslag–Solitar group $BS(1,2)$ in that case). The Baumslag–Solitar groups $BS(m,n)=\langle a,b \mid a^{-1}b^m a = b^n\rangle$ provide a family of groups that explicitly capture an “arithmetic-like” operation: here $b$ can be seen as an analog of an addition generator (each power $b^m$ is like repeated addition), and conjugation by $a$ scales $b$ by $n$ (like a multiplicative dilation)

\href{https://encyclopediaofmath.org/wiki/Baumslag-Solitar_group#:~:text=non,group%20defined%20by%20the%20presentation}{encyclopediaofmath.org}
. These groups arise in topology as fundamental groups of certain 2-dimensional complexes and as subgroups of knot groups. They are interesting because depending on $m,n$, their geometry changes: $BS(1,1)$ is a torus group ($\mathbb{Z}^2$), which is Euclidean; $BS(1,n>1)$ contains a $\mathbb{Z}^2$ subgroup and is a solvable group not far from the structure of a knot complement’s group (trefoil as noted); many $BS(m,n)$ are not hyperbolic groups because they contain abelian subgroups of rank $>1$\href{https://www3.nd.edu/~gszekely/rtg/GTS/www3.nd.edu/_jquigle2/GSTS%20FA18/Week3.pdf#:~:text=%28b%29%20Finite%20groups%20are%20hyperbolic,2}{www3.nd.edu}
. In geometric group theory, they serve as boundary cases for hyperbolicity. Thus, knot theory provides concrete realizations of these abstract “expression groups” – the relations in a knot group can be viewed as equations (like $a^{-1}b^2a=b^3$ for trefoil) relating loops (operations). Studying how loops in the knot group project to covering spaces is central to knot theory: for example, the \textit{infinite cyclic cover} of a knot (covering corresponding to the commutator subgroup of the knot group) is topologically a string of copies of the knot complement and is used to define the Alexander polynomial. This is analogous to studying an expression space's cover where a particular loop (the longitude, which commutes in the abelianization) is made infinite – essentially analyzing the additive structure alone.

Performing Dehn surgeries on knots is a principal way to generate new 3-manifolds, and it mirrors the discussion of adding relations. Many famous 3-manifolds are surgeries on knots; for instance, lens spaces arise from surgeries on the unknot or other simple knots. The effect on fundamental groups is well-understood (adding a relation $\mu^p\lambda^q=1$ as described). In knot theory, one might ask: for which knots does a given surgery yield a particular manifold? This can often be translated to a question about the knot group having a quotient isomorphic to the target manifold’s group. For example, Poincaré homology sphere is a closed 3-manifold whose fundamental group is the binary icosahedral group (order 120). It is known to be obtainable by $+1$ surgery on the trefoil knot. That means the trefoil’s group $\langle a,b \mid a^{-1}b^2a=b^3\rangle$ has a quotient of order 120 when we add the relation corresponding to that surgery slope. Indeed it does – this is an algebraic verification that the surgery yields that manifold. So, analyzing loops in the knot group and adding relations (surgery) helps classify which manifolds you can get (this is part of Kirby’s surgery theory and the Lickorish–Wallace theorem that any closed orientable 3-manifold comes from surgery on some link).

Another connection is via universal covers and knot complements: A knot complement’s universal cover is an infinite 3-space where each deck transformation corresponds to an element of the knot group. An important concept is the universal abelian cover, which corresponds to the commutator subgroup $[\pi_1,\pi_1]$. Its deck transformation group is the abelianization of the knot group (which is always $\mathbb{Z}$ for a knot, generated by the meridian). That cover is essentially $\mathbb{R}^2$ times a line (it has a $\mathbb{Z}$ of decks, producing an infinite chain of complements). This cover is used to define the Alexander invariants. We see here that loops (commutators) in the knot space project to deck transformations in this infinite cyclic cover, and one can compute homology to get the Alexander polynomial, which encodes how those loops interrelate. In more advanced terms, some knot invariants (like \textit{Casson’s invariant or AJ conjecture relating the A-polynomial and Jones polynomial}) involve analyzing certain covering spaces or $\mathrm{SL}_2(\mathbb{C})$ representations of the knot group – effectively studying how loops act as isometries of various model spaces (hyperbolic space, in the case of $\mathrm{SL}_2(\mathbb{C})$ representations via Thurston’s work).

Arithmetic topology, hinted above, directly links knot complements to number theory. In this analogy, a knot in $S^3$ corresponds to a prime in $\mathrm{Spec}(\mathbb{Z})$. The knot group’s profinite completion is analogous to the absolute Galois group of $\mathbb{Q}$. Loops in the arithmetic expression space vs loops around a prime: A small loop encircling a knot in $S^3$ (the meridian) corresponds in analogy to a small loop around a prime in an arithmetic scheme (which gives rise to a Frobenius element in the Galois group). The covering spaces of the knot complement (like its finite-sheeted covers) correspond to finite extensions of number fields (unramified outside a set of primes). This analogy has been developed by Mazur, Morishita, and others

\href{https://math.stackexchange.com/questions/38936/analogies-between-prime-ideals-and-knots#:~:text=,between%20prime%20ideals%20and%20knots}{math.stackexchange.com}
\href{https://old.maa.org/press/maa-reviews/knots-and-primes-an-introduction-to-arithmetic-topology#:~:text=All%20right%2C%20then%2C%20what%20does,manifolds%20and%20%E2%80%9Cnumber%20rings.%E2%80%9D}{old.maa.org}
. They show, for instance, that the Alexander polynomial of a knot is analogous to the characteristic polynomial of Frobenius acting on étale cohomology in number theory, and that linking numbers of knots correspond to quadratic residue symbols in number fields\href{https://old.maa.org/press/maa-reviews/knots-and-primes-an-introduction-to-arithmetic-topology#:~:text=Morishita%20emphatically%20takes%20this%20position,there%20is%20indeed%20a%20close}{old.maa.org}
\href{https://old.maa.org/press/maa-reviews/knots-and-primes-an-introduction-to-arithmetic-topology#:~:text=All%20right%2C%20then%2C%20what%20does,manifolds%20and%20%E2%80%9Cnumber%20rings.%E2%80%9D}{old.maa.org}
. In Morishita’s work, as reviewed in\href{https://old.maa.org/press/maa-reviews/knots-and-primes-an-introduction-to-arithmetic-topology#:~:text=All%20right%2C%20then%2C%20what%20does,manifolds%20and%20%E2%80%9Cnumber%20rings.%E2%80%9D}{old.maa.org}
, the fundamental group of a 3-manifold (knot complement) and the Galois group of a number ring are put side by side, and many dictionary entries are established (e.g., $S^1$ corresponds to $\mathrm{Spec}(\mathbb{F}_q)$, a knot complement corresponds to $\mathrm{Spec}$ of an integer ring in a number field, etc., and covering correspondences match up). Thus, not only is this theory conceptually elegant, it suggests that techniques can sometimes transfer: for example, studying covers of 3-manifolds might inspire analogous constructions in algebraic number theory and vice versa.

In 3-manifold topology more broadly, covering space techniques are indispensable. Many deep results involve finding a finite cover of a given manifold with nicer properties. For instance, if an arithmetic expression space (or any space) has a fundamental group that is large (contains a free group of finite index), then it has many covers that are richer. In recent breakthroughs, Agol showed that every hyperbolic 3-manifold is virtually Haken and virtually fibered – meaning a finite cover exists that contains an essential surface or is a surface bundle. These results often rely on subgroup separability – the ability to separate loops by going to appropriate covers. Loops in the fundamental group that one might want to “get rid of” or make primitive can often be separated into distinct lifts in a finite-index subgroup (if the group is residually finite or subgroup separable). For example, if a 3-manifold’s fundamental group is residually finite (which is true for most 3-manifold groups, including all hyperbolic ones

\href{https://old.maa.org/press/maa-reviews/knots-and-primes-an-introduction-to-arithmetic-topology#:~:text=All%20right%2C%20then%2C%20what%20does,manifolds%20and%20%E2%80%9Cnumber%20rings.%E2%80%9D}{old.maa.org}
 indirectly via Mal’cev rigidity and arithmeticity in some cases), then given any nontrivial loop, there is a finite cover where that loop lifts to a disjoint copy (or is trivial in the cover’s group). This is analogous to solving an equation by going to a field extension in number theory that makes the solution apparent.

To tie back to arithmetic expression spaces: the loop structures in those spaces can be complicated (just as knot groups are complicated). But one strategy to understand them is to pass to a convenient covering space where the loop becomes simpler. In a computational setting, one might enumerate cosets of the subgroup normal closure to see the quotient; in a theoretical setting, one might use covering space existence to impose conditions (like finding a cover where a given loop \textit{splits} into multiple loops, perhaps simplifying the expression). The powerful tools developed in 3-manifold theory – JSJ decompositions (cutting along tori to isolate “atomic” pieces), hyperbolization (assigning negative curvature metrics), and virtual properties (finding covers with desirable features) – all have analogues in the world of group theory and could conceptually apply to any space where loops represent relations. An “arithmetic expression 3-manifold” could hypothetically be decomposed into simpler pieces (like separating the additive part and multiplicative part) similar to how a 3-manifold is cut along tori into geometric pieces. In fact, a number field’s analog of JSJ is the factorization into primes (each prime corresponds to a place, akin to a torus boundary in a 3-manifold), and one can study each part separately in arithmetic topology.

In summary, knot theory and 3-manifold topology offer both examples and analogies for arithmetic expression spaces. The fundamental group viewpoint treats algebraic operations as generators and algebraic identities as relations, just like a presentation of a knot group. Covering spaces are used in knot theory to define invariants and to classify possible surgeries (very much as covering and quotient considerations classify which relations can be added without contradiction). And through the lense of arithmetic topology, these ideas transcend into number theory, suggesting that perhaps the “space of arithmetic expressions” for actual numbers might be something like an arithmetic three-dimensional manifold whose fundamental group is the absolute Galois group of $\mathbb{Q}$ – a deep and largely still speculative idea, but one that continues to inspire research

\href{https://math.stackexchange.com/questions/38936/analogies-between-prime-ideals-and-knots#:~:text=,between%20prime%20ideals%20and%20knots}{math.stackexchange.com}
\href{https://old.maa.org/press/maa-reviews/knots-and-primes-an-introduction-to-arithmetic-topology#:~:text=All%20right%2C%20then%2C%20what%20does,manifolds%20and%20%E2%80%9Cnumber%20rings.%E2%80%9D}{old.maa.org}
.

\section*{Geometric Group Theory Perspectives}
From a geometric group theory perspective, an arithmetic expression space can be seen as a group or complex that encodes the allowed operations (as generators) and their relations. We have already mentioned free groups (no relations) and how adding commutativity or distributivity introduces relations. Geometric group theory classifies groups by the large-scale geometry of their Cayley graphs or associated spaces. A key class is the word-hyperbolic groups (negative curvature in the large) versus groups containing $\mathbb{Z}^2$ subgroups (which are “flat” planes in the Cayley graph). Expression spaces provide natural examples on both sides:

\begin{itemize}
\item If we allow only non-commuting operations (like formal addition $a$ and multiplication $m$ that do not interact), we essentially get a free product of groups (or a free group if considering one operation as a single generator). Free groups are hyperbolic

\href{https://www3.nd.edu/~gszekely/rtg/GTS/www3.nd.edu/_jquigle2/GSTS%20FA18/Week3.pdf#:~:text=Definition%204,2}{www3.nd.edu}
, meaning their Cayley graph satisfies a slim-triangle condition characteristic of negative curvature. In this case, the arithmetic expression space’s loops all come from trivial relations (none except the ones needed to have a group like inverses), so it’s as curved as possible (no “flat” pockets). Any two distinct geodesic paths diverge exponentially, much like in a tree.


\item If we impose significant algebraic relations, the group can lose hyperbolicity. A prime example is the Baumslag–Solitar group $BS(1,n)=\langle a,b \mid a^{-1} b a = b^n \rangle$

\href{https://encyclopediaofmath.org/wiki/Baumslag-Solitar_group#:~:text=non,group%20defined%20by%20the%20presentation}{encyclopediaofmath.org}
. This group has one relation tying $a$ and $b$, which causes the Cayley graph to contain a distorted plane. Indeed, it’s known that $BS(1,n)$ is \textit{not} word-hyperbolic for $n\neq \pm1$ (it is instead a solvable group with a warped plane geometry). Intuitively, one can find in its Cayley graph a sequence of loops that increasingly “approach” an infinite flat strip. (One rigorous criterion is that $BS(1,n)$ contains an isomorphic copy of $\mathbb{Z}[\frac{1}{n}] \rtimes \mathbb{Z}$, which has a $\mathbb{Z}\times\mathbb{Z}$ subgroup.) Geometrically, what happens is that the relation $a^{-1} b^m a = b^n$ allows one to walk in a large rectangle: travel $m$ steps in $b$-direction, one step in $a$-direction, $n$ steps back in $b$-direction, and one step back in $a$ – if $m=n$, this would literally be a flat rectangle loop. For $m\neq n$, it’s a loop that isn’t geodesic but still creates a \textit{larger and larger nearly flat annulus} in the Cayley graph when $b^m$ and $b^n$ powers are iterated. This is reflected by $BS(1,n)$ containing an exponentially distorted $\mathbb{Z}^2$. This shows how a single arithmetic-like relation (here resembling $a b^m = b^n a$) can spoil curvature. Many expression spaces corresponding to rings (with full distributive and commutative laws) will similarly contain $\mathbb{Z}^2$ substructures (for example, the integers under $+$ and $\times$ itself have a huge abelian part under addition and a multiplicative part – not a finitely generated group, but any finitely generated piece will see some commuting behavior). Thus, the presence or absence of hyperbolicity in an arithmetic expression space’s fundamental group is determined by the algebraic independence of addition and multiplication. If they generate a nonabelian free subgroup, we get hyperbolicity; if they commute or satisfy a relation implying a $\mathbb{Z}^2$, we get Euclidean flats.


\end{itemize}
One can apply the tools of geometric group theory to study these spaces. For instance, automatic group algorithms could be used to solve the word problem in a given arithmetic expression group (word problem: decide if a given loop is trivial). If the group is word-hyperbolic, there is a linear-time solution via Dehn’s algorithm: repeatedly “cut off” rims of big loops (which correspond to relations). If we have a specific arithmetic identity to verify (like a complicated equality of two expressions), it translates to checking if a certain loop in the expression space is null-homotopic. In a hyperbolic setting, one would quickly find a contradiction if it were not null-homotopic (the loop would tighten to a unique geodesic representative). However, if the space is not hyperbolic (say it has a Euclidean sector), the word problem might be more subtle (potentially solvable by other means, but not as quickly). The Todd–Coxeter algorithm from combinatorial group theory can enumerate cosets of a subgroup in the group, which effectively explores finite-sheeted coverings. For example, if one suspects a certain expression identity holds only modulo some relation, one could use Todd–Coxeter to impose that relation and see the quotient structure.

Another fruitful viewpoint is to embed these groups into known geometric spaces. We saw the tree embedding in hyperbolic plane for a free group. One could ask: can an expression space that includes distributivity be embedded in some higher-dimensional hyperbolic space or product of trees? There is a concept of quasi-isometric embeddings: even if the group is not strictly hyperbolic, maybe it’s \textit{relatively hyperbolic} (hyperbolic except for some flat subspace). Indeed, a ring of integers $\mathbb{Z}$ under $(+, \times)$ is not finitely generated as a group, but if one restricts to a finitely generated sub-semigroup (like $\langle +1, \times 2\rangle$ acting on 1), the structure is that of the semidirect product $\mathbb{Z} \rtimes \mathbb{N}$ (which gives $BS(1,2)$, as noted). That group is relatively hyperbolic (hyperbolic relative to a peripheral subgroup isomorphic to $\mathbb{Z}[\frac{1}{2}]$ perhaps). Techniques from relatively hyperbolic groups then apply (these are groups that are hyperbolic except for “tubes” that correspond to parabolic subgroups like $\mathbb{Z}^2$ or $\mathbb{Q}$).

Finally, geometric group theory encourages the use of computer experiments and visualizations. We can attempt to \textit{construct the Cayley graph} of a given small arithmetic group and inspect it. Modern tools allow us to visualize graphs in hyperbolic space. As noted in the blog excerpt, a project called \textit{Walrus} visualizes large graphs using hyperbolic geometry

\href{https://globberingmattress.wordpress.com/2017/12/26/deck-transformations-revisted/#:~:text=However%2C%20there%20is%20one%20setting,also%20an%20exponentially%20growing%20graph}{globberingmattress.wordpress.com}
. One could input the generation rules of an arithmetic expression space (like nodes = numbers up to some limit, edges = +1 or ×2 operations) and see a pattern in the hyperbolic disk. The hyperbolic visualization will clearly show a tree-like expansion if the structure is free, or reveal grid-like patterns if abelian relations appear. For instance, the blog visualized the Cayley graph of the free group $F_2$ as a tessellation in the Poincaré disk\href{https://globberingmattress.wordpress.com/2017/12/26/deck-transformations-revisted/#:~:text=The%20covering%20space%20looked%20like,the%20nodes%20that%20connect%20them}{globberingmattress.wordpress.com}
 (the image we included earlier). If we similarly visualized $BS(1,2)$, we’d see a tree with a cyclic structure “spiraling” through it, indicating the solvable nature. These visual aids can guide conjectures about the geometry (e.g., seeing a flaring in one direction but not the other).

In terms of applications, understanding the geometry of these expression spaces could help in designing better algorithms for simplifying arithmetic expressions. If the space is hyperbolic, any two distinct ways to write the same expression will have a “central” form to which they all collapse (analogous to a geodesic normal form in a hyperbolic group). If not, there might be infinitely many distinct minimal forms (like how $\mathbb{Z}\times \mathbb{Z}$ has infinitely many geodesics between the same two points due to flats). This connects to rewrite theory in computer algebra: one wants convergence of term rewriting systems. A hyperbolic-like property might ensure confluence.

\section*{Computational Approaches to Constructing Such Spaces}
Constructing and exploring arithmetic expression spaces can be approached with computational tools from algebraic topology and group theory. Here are several strategies:

\begin{itemize}
\item Cayley Graph Enumeration: One straightforward approach is to explicitly generate the Cayley graph of the expression group up to a certain radius. For example, start from the identity expression (say the number 1 or an empty expression) and apply operations (like $+1$, $-1$, $\times 2$, etc.) in all possible ways up to a given length. This builds a finite portion of the infinite graph. By analyzing this portion, one can detect loops (when two different sequences yield the same result). Algorithms for finding cycles in a graph can then enumerate the fundamental group’s relations. This is essentially how the Todd–Coxeter procedure or Knuth–Bendix completion work, by discovering relations when different paths converge to the same state. The challenge is the state space may be huge or infinite, but clever pruning can be used (for instance, if certain combinations obviously explode to large numbers, maybe impose bounds or consider classes modulo some equivalence).


\item Solving the Word Problem: As mentioned, determining if two expressions are equal is the word problem in the expression group. This can be attempted via rewriting algorithms. One can input the basic axioms (commutativity $a+b = b+a$, distributivity $a*(b+c)=a\textit{b + a}c$, etc.) into a computer algebra system and attempt a \textit{confluent rewrite system}. Systems like Mathematica or Prover9 can attempt to reduce any expression using those rewrite rules to a canonical form. If they succeed (i.e. the system is confluent and terminating), one can decide equality of expressions by comparing the normal forms. This is essentially computing in the quotient group defined by those relations.


\item SnapPy for 3-Manifolds: On the 3-manifold side, programs like SnapPy can be used to explore surgeries and covering spaces. One can specify a manifold by giving SnapPy a triangulation or a link (for instance, the figure-8 knot is m004 in SnapPy’s census). Then one can perform Dehn surgeries by specifying filling coefficients. SnapPy will compute the resulting manifold’s invariants, including a presentation of its fundamental group and often the hyperbolic structure if it exists

\href{https://snappy.computop.org/manifold.html#:~:text=A%20Manifold%20is%20a%20Triangulation,manifolds.%20Here%E2%80%99s%20a%20quick%20example}{snappy.computop.org}
\href{https://snappy.computop.org/manifold.html#:~:text=A%20Manifold%20can%20be%20specified,g}{snappy.computop.org}
. For example, SnapPy can confirm that doing a $(p,q)$ filling on a cusp adds the relation $\mu^p\lambda^q=1$ in the group\href{https://snappy.computop.org/manifold.html#:~:text=Return%20a%20HolonomyGroup%20representing%20the,corresponding%20peripheral%20elements%20are%20killed}{snappy.computop.org}
. It can then attempt to simplify the presentation of the new fundamental group. Moreover, SnapPy can compute finite-sheeted covers of many manifolds by searching for subgroups of given index. For instance, one could ask SnapPy to find a double cover of a surgered manifold (if it exists) – this corresponds to finding an index-2 subgroup of the fundamental group. SnapPy uses algorithms related to coset enumeration to do this. Using such tools, one can experiment with specific cases: e.g., take the trefoil knot group, add a relation for surgery, and see the resulting group structure; or attempt to find a cover where a certain loop (maybe a longitude) lifts to two separate loops, illustrating subgroup separability.


\item Visualization: As noted, visualization tools can help qualitatively. Beyond the hyperbolic disk drawings, one might use software like Cayley graph plotters (some exist in SageMath or Magma for small groups). If one restricts to, say, addition by 1 and multiplication by 2 and 3 as generators on positive integers, one can generate a few hundred nodes and attempt to plot the graph. This can reveal patterns like a tree with loops connecting branches (coming from equalities like $2\textit{3 = 3}2$). If the graph is too large, focusing on a quotient or a specialized case (like just the free semigroup without commutativity) will simplify it.


\item Automatic Structures: Some arithmetic groups might be automatic or biautomatic, meaning there’s a regular language describing all words in normal form. Computer algorithms exist to find automatic structures. If one suspects the expression group is automatic, one could attempt to use the \textit{Walnut} prover or others to find a finite state automaton recognizing the equality of two expressions. For example, the theory of Presburger arithmetic (addition only) is well-known to be decidable by automata. Multiplication alone (as a free semigroup) is trivial to handle. The combination is much harder (in fact, $(\mathbb{N},+,*)$ is known to be undecidable in general – Matiyasevich’s theorem related to Hilbert’s 10th problem), but restricted versions might be automatic.


\end{itemize}
In practice, hybrid approaches work well. For instance, use SnapPy to get a presentation of a fundamental group after surgery, then use a group-theory package to analyze that presentation (compute its abelianization, its finite quotients, etc.). Or use a custom script to generate part of an expression Cayley graph, then feed the relations discovered into a theorem prover to see if they generate all relations.

As a concrete computational illustration: consider the small arithmetic space generated by ${+1, \times 2}$ acting on the value 1. A quick manual or programmed search finds paths: $1 \xrightarrow{+1} 2 \xrightarrow{\times2} 4$ and $1 \xrightarrow{\times2} 2 \xrightarrow{+1} 3$. These end at different values (4 vs 3, so no loop yet). However, one finds $1 \xrightarrow{+1} 2 \xrightarrow{+1} 3$ and also $1 \xrightarrow{\times2} 2 \xrightarrow{+1} 3$. Those reach the same state “3”. Hence there is a loop: $(+1)+(+1)$ versus $(\times2)+(+1)$ gives the relation $1+1+1 = 2*1+1$, or simply $3 = 2+1$. That’s trivial in arithmetic (3 = 3), but in the formal generator sense it’s the first nontrivial loop (because the path sequences differ). Continuing this process systematically would enumerate all independent arithmetic identities (though it quickly becomes unmanageable by brute force, as identities become more complex). Nonetheless, this demonstrates how one can discover loops by search.

Computational group theory software like GAP or Magma can sometimes handle infinite groups given by a presentation if there are sufficient clues (for example, they can often tell if a given presentation defines a small finite group or an extension of known groups). One can input the Baumslag–Solitar presentation to GAP and ask for its lower central series or abelian invariants, etc., which helps classify its geometry (e.g., seeing a $\mathbb{Z}^2$ quotient indicates a Euclidean sector).

Lastly, computational experiments often guide theoretical conjectures. By computing many examples of expression spaces or manifold surgeries, one might conjecture patterns – for example, a certain surgery always yields a large fundamental group, or certain expression relations always produce hyperbolicity. These can then be attacked with theoretical tools.

In summary, computational approaches play a supportive role in this research: constructing examples of arithmetic expression spaces (as groups or CW-complexes), verifying properties on those examples, and visualizing the geometric behavior of these spaces. They bridge the gap between the abstract theory (which can be rigorous but complex) and concrete instances (which can provide intuition and evidence). The combination of theoretical reasoning with computational exploration can lead to a deeper understanding of how arithmetic operations manifest as geometric movements, how loops (identities) shape the topology of the space, and how cutting/gluing operations alter that landscape in the world of 3-manifolds and beyond.


\end{document}
