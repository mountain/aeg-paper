\documentclass{article}
\usepackage{amsmath, amssymb, amsfonts}
\usepackage{geometry}
\geometry{margin=2.5cm}
\title{Embedding Small Knot Groups into the Three-Dimensional Arithmetic-Expression Group}
\author{Mingli Yuan, ChatGPT, Gemini}
\date{\today}

\begin{document}
\maketitle

\section{The extended arithmetic-expression group $\mathcal{G}$}
We consider the positive real line $(0,\infty)$ acted upon by three fundamental one-parameter flows:
\begin{align}
\oplus_{\mu} &: x \mapsto x+\mu, \quad \text{(Translation)}\\
\otimes_{\lambda} &: x \mapsto e^{\lambda}x, \quad \text{(Scaling)}\\
\uparrow_{\kappa} &: x \mapsto x^{e^{\kappa}}, \quad \text{(Exponentiation)}
\end{align}
where $(\mu,\lambda,\kappa)\in\mathbb{R}^{3}$ are the parameters. While these flows generally do not commute in their action on $(0,\infty)$, their parameters generate a solvable Lie group $\mathcal{G}$ defined on $\mathbb{R}^3$. The group law is given by:
\begin{equation}
\label{eq:group-law}
(\mu,\lambda,\kappa)*(\mu',\lambda',\kappa') = \bigl(e^{\lambda+\kappa}\mu'+\mu, \lambda+\lambda', \kappa+\kappa'\bigr),
\end{equation}
with identity $(0,0,0)$ and inverse $(\mu,\lambda,\kappa)^{-1}=(-\mu e^{-(\lambda+\kappa)},-\lambda,-\kappa)$. This group $\mathcal{G}$ acts on $(0,\infty)$.

\section{A normal form for $a^{m}b^{n}$}
We embed two Wirtinger generators into $\mathcal{G}$ by:
\begin{equation}
a \longmapsto (\alpha, p, q), \qquad
b \longmapsto (\beta, r, s).
\end{equation}
Let $u = e^{p+q}$ and $v = e^{r+s}$. Repeated application of the group law \eqref{eq:group-law} yields the general forms:
\begin{align}
a^{m} &= \left(\alpha \sum_{k=0}^{m-1} u^k, mp, mq\right) = \bigl(\alpha\Sigma_{m}(u), mp, mq\bigr), \\
b^{n} &= \left(\beta \sum_{k=0}^{n-1} v^k, nr, ns\right) = \bigl(\beta\Sigma_{n}(v), nr, ns\bigr),
\end{align}
where $\Sigma_{k}(z)=1+z+\dots+z^{k-1} = \frac{z^k-1}{z-1}$ for $z \neq 1$.
The composition $a^m b^n$ is then:
\begin{equation}
\label{eq:am-bn}
a^{m}b^{n}=\bigl(u^m \beta\Sigma_{n}(v) + \alpha\Sigma_{m}(u), mp+nr, mq+ns\bigr).
\end{equation}

To force the word $a^{m}b^{n}$ to be trivial (equal to the identity $(0,0,0)$), we need its components to be zero:
\begin{align}
&\text{\textit{Multiplicative conditions:}} & mp+nr=0, & \quad mq+ns=0, \label{eq:mult} \tag{M} \\
&\text{\textit{Translational condition:}} & u^m \beta\Sigma_{n}(v) + \alpha\Sigma_{m}(u)=0. \label{eq:trans} \tag{T}
\end{align}

\section{Trefoil $3_1$ (relation $a^{2}b^{3}=e$)}
Here $m=2, n=3$. Condition~\eqref{eq:mult} gives $2p+3r=0$ and $2q+3s=0$. This implies $r+s = -\frac{2}{3}(p+q)$, so $v = e^{r+s} = u^{-2/3}$.
With $\beta=1$, the translational constraint \eqref{eq:trans} is $u^2 \Sigma_3(v) + \alpha \Sigma_2(u) = 0$, which means $\alpha(1+u) = -u^2 \Sigma_3(u^{-2/3})$.
The term $N = u^2 \Sigma_3(u^{-2/3}) = u^2(1+u^{-2/3}+u^{-4/3}) = u^2 + u^{4/3} + u^{2/3}$.
To reveal the Alexander polynomial, we factor $N$.
Let $t=u^{1/3}$. Then $N = t^6 + t^4 + t^2 = t^2(t^4+t^2+1)$.
Let $\Delta_{3_1}(t) = t^2 - t + 1$ be the Alexander polynomial for the trefoil knot.
Using the factorization $t^4+t^2+1 = (t^2-t+1)(t^2+t+1) = \Delta_{3_1}(t)(t^2+t+1)$, we have:
\begin{equation*}
    N = t^2 \Delta_{3_1}(t) (t^2+t+1).
\end{equation*}
Substituting back $t=u^{1/3}$:
\begin{equation*}
    N = u^{2/3} \Delta_{3_1}(u^{1/3}) (u^{2/3} + u^{1/3} + 1).
\end{equation*}
Thus, the constraint $\alpha(1+u) = -N$ gives:
\begin{equation}
\alpha(u) = - \frac{u^{2/3} (u^{2/3} + u^{1/3} + 1) \, \Delta_{3_1}(u^{1/3})}{u+1}.
\end{equation}
Unlike the figure-eight knot case, the constraint does not simply set the Alexander polynomial to zero. Instead, $\Delta_{3_1}(u^{1/3})$ appears as a factor in the expression determining the required translational shift $\alpha(u)$. Choosing any $u=e^{p+q} \in (0,\infty), u\ne1$, determines suitable $p,q$. Then $r=-2p/3, s=-2q/3$ and $\alpha(u)$ are determined, defining an embedding $a=(\alpha,p,q)$, $b=(1,r,s)$ such that $a^2 b^3 = e$.

\section{Figure-eight $4_1$ (relation $w = abbbaBAAB = e$)}
Let $A=a^{-1}$ and $B=b^{-1}$. The relation word $w = abbbaBAAB$ has exponent sum zero for both $a$ and $b$. Thus, the multiplicative part of its image in $\mathcal{G}$ is automatically zero.
Choose the embedding $a \mapsto (0,\lambda,\kappa)$ ($\alpha=0$) and $b \mapsto (1,0,0)$ ($\beta=1, r=s=0$). Then $u=e^{\lambda+\kappa}$, $v=1$, $A=(0,-\lambda,-\kappa)$, $B=(-1,0,0)$.
The image of the word $w$ in $\mathcal{G}$ is calculated to be:
\begin{equation}
w \longmapsto (3u-u^2-1, 0, 0).
\end{equation}
Setting this equal to the identity $(0,0,0)$ requires:
\begin{equation}
u^2-3u+1=0, \qquad \text{where } u=e^{\lambda+\kappa}.
\end{equation}
The familiar Alexander polynomial $\Delta_{4_1}(t)=t^{2}-3t+1$ therefore reappears as the sole constraint $\Delta_{4_1}(u)=0$.

\section{Cinquefoil $5_1$ (relation $a^{2}b^{5}=e$)}
Here $m=2, n=5$. Conditions~\eqref{eq:mult} give $2p+5r=0$ and $2q+5s=0$, hence $v=u^{-2/5}$.
Fixing $\beta=1$, the translational constraint \eqref{eq:trans} is $u^2 \Sigma_5(v) + \alpha \Sigma_2(u) = 0$, so $\alpha(1+u) = -u^2 \Sigma_5(u^{-2/5})$.
Let $E = u^2 \Sigma_5(u^{-2/5}) = u^2(1+u^{-2/5}+u^{-4/5}+u^{-6/5}+u^{-8/5}) = u^{2}+u^{8/5}+u^{6/5}+u^{4/5}+u^{2/5}$.
To reveal the Alexander polynomial, let $t=u^{1/5}$. Then $E = t^{10}+t^8+t^6+t^4+t^2 = t^2(t^8+t^6+t^4+t^2+1)$.
Let $\Delta_{5_1}(t) = t^4 - t^3 + t^2 - t + 1$ be the standard Alexander polynomial for the cinquefoil knot (which is also the cyclotomic polynomial $\Phi_{10}(t)$).
The polynomial part $P(t) = t^8+t^6+t^4+t^2+1$ factors as $P(t) = (t^4+t^3+t^2+t+1) \Delta_{5_1}(t)$.
Therefore:
\begin{equation*}
    E = t^2 (t^4+t^3+t^2+t+1) \Delta_{5_1}(t).
\end{equation*}
Substituting back $t=u^{1/5}$:
\begin{equation*}
    E = u^{2/5} (u^{4/5}+u^{3/5}+u^{2/5}+u^{1/5}+1) \, \Delta_{5_1}(u^{1/5}).
\end{equation*}
The constraint $\alpha(1+u) = -E$ gives:
\begin{equation}
\alpha(u) = - \frac{u^{2/5} (u^{4/5}+u^{3/5}+u^{2/5}+u^{1/5}+1) \, \Delta_{5_1}(u^{1/5})}{u+1}.
\end{equation}
Similar to the trefoil case, the Alexander polynomial $\Delta_{5_1}(u^{1/5})$ emerges as a factor determining $\alpha(u)$. Choosing any $u>0$, $u\neq1$ determines $p,q$. Then $r=-2p/5, s=-2q/5$ and $\alpha(u)$ are determined, defining the embedding $a=(\alpha,p,q)$, $b=(1,r,s)$ such that $a^2 b^5 = e$.

\section{Summary}
\begin{center}
% Table uses \Delta notation and sum for brevity
\begin{tabular}{c|c|c}
Knot & Multiplicative constraints & Translational solution constraint \\ \hline
$3_1$ ($a^2 b^3=e$) & $2p+3r=0, 2q+3s=0$ ($v=u^{-2/3}$) & $\alpha(u) = - \frac{u^{2/3} (u^{2/3} + u^{1/3} + 1) \, \Delta_{3_1}(u^{1/3})}{u+1}$ (given $\beta=1$) \\
$4_1$ ($w=e$) & automatic (exponent sums are 0) & $\Delta_{4_1}(u)=0$ (given specific $\alpha=0, \beta=1, r=s=0$) \\
$5_1$ ($a^2 b^5=e$)& $2p+5r=0, 2q+5s=0$ ($v=u^{-2/5}$) & $\alpha(u) = - \frac{u^{2/5} (\sum_{k=0}^4 u^{k/5}) \, \Delta_{5_1}(u^{1/5})}{u+1}$ (given $\beta=1$)
\end{tabular}
\end{center}

\section{Discussion}
The three-parameter group $\mathcal{G}$ provides a uniform algebraic-geometric setting for representing small knot groups:
\begin{itemize}
    \item For relations with zero exponent sums (like the figure-eight knot using $w$), the multiplicative constraints are automatically satisfied. The translational constraint often relates directly to the Alexander polynomial evaluated at $u=e^{\lambda+\kappa}$, resulting in $\Delta_K(u)=0$.
    \item For relations of torus-knot type like $a^m b^n = e$, non-trivial multiplicative constraints link the parameters of $a$ and $b$. The translational constraint typically requires a specific non-zero $\alpha$ (or $\beta$) value. The Alexander polynomial $\Delta_K$ (evaluated at a fractional power of $u$) appears as a factor within the expression determining this translational component, rather than directly yielding the constraint.
    \item The framework naturally handles $a^m b^n$ relations. Knot groups with more generators could potentially be embedded by extending this solvable group structure.
\end{itemize}
This suggests that the geometry associated with $\mathcal{G}$ captures essential algebraic information related to knot groups, particularly information encoded in the Alexander polynomial, while potentially offering parameters ($\alpha, \beta$, and the choice of $u$ off the constraint variety) to encode finer invariants.

\end{document}