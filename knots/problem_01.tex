\documentclass[12pt, a4paper]{article}
\usepackage{amsmath}
\usepackage{amssymb} % 用于数学符号 (如果需要)
\usepackage[margin=1in]{geometry} % 设置页边距
\usepackage[UTF8]{ctex} % 支持中文

\title{关于 $4_1$ 纽结群三种表示的同构映射问题}
% \author{请替换为你的名字或留空}
\date{\today} % 或指定日期 May 2, 2025

\begin{document}

\maketitle

\section{引言}
$4_1$ 纽结(八字结)的补空间基本群 $G = \pi_1(S^3 \setminus 4_1)$ 是纽结理论和低维拓扑中一个被广泛研究的对象。该群拥有多种不同的有限表示。理解这些不同表示之间的显式同构映射对于连接不同的理论框架和转换研究结果至关重要。我们在尝试计算确定这些同构映射时遇到了困难,希望寻求相关领域的专家建议。

\section{三种 $4_1$ 纽结群的表示}
我们关注以下三种都定义了(或被认为定义了)同一个群 $G$ 的表示方法:

\begin{enumerate}
    \item \textbf{HNN 扩张表示 ($G_{HNN}$):} 源于已知的 $4_1$ 纽结补空间的纤维丛结构,其纤维 $F$ 为穿孔环面 ($\pi_1(F) \cong F_2 = \langle u, v \rangle$)。
    \[
    G_{HNN} = \langle u, v, t \mid t^{-1}ut = h_*(u), \ t^{-1}vt = h_*(v) \rangle
    \]
    其中 $h_*: F_2 \to F_2$ 是 monodromy 自同构,满足 $h_*(u)=uv$ 和 $h_*(v)=vuv$。

    \item \textbf{标准的二生成元、单关系子表示 ($G_{Std}$):} 文献中常见的表示,很可能由 Wirtinger 表示化简得到。
    \[
    G_{Std} = \langle x, y \mid xy^{-1}x^{-1}y = yx^{-1}y^{-1}x \rangle
    \]
    其关系子可写作 $R_{std}(x,y) = xy^{-1}x^{-1}y x^{-1} y x y^{-1} = 1$。

    \item \textbf{来自 \texttt{snappy} 的表示 ($G_R$):} 我们从 \texttt{snappy} 软件包的输出中获得了一个表示。
    \[
    G_R = \langle a, b \mid abbbaBAAB = 1 \rangle
    \]
    其中 $A=a^{-1}$ 且 $B=b^{-1}$。对应的关系子为 $R_1(a,b) = ab^3ab^{-1}a^{-2}b^{-1}$。
\end{enumerate}

\section{问题描述:寻找显式同构映射}
我们的主要目标是确定这些表示之间的显式同构映射。具体而言,需要找到一组生成元如何能表示为另一组生成元的字 (words)。例如,找到映射 $\phi_{HNN \to Std}$ 和 $\phi_{Std \to HNN}$ 使得:
\begin{itemize}
    \item $x = \phi_{HNN \to Std}(u, v, t)$ 且 $y = \phi_{HNN \to Std}(u, v, t)$
    \item $u = \phi_{Std \to HNN}(x, y)$, $v = \phi_{Std \to HNN}(x, y)$, $t = \phi_{Std \to HNN}(x, y)$
\end{itemize}
类似地,我们也希望找到涉及 $G_R$ 的映射关系。

\section{计算中遇到的困难}
我们尝试使用 SageMath 调用 GAP 来自动计算这些同构映射。核心函数是 GAP 的 \texttt{IsomorphismFpGroup(G1, G2)}。

这些尝试持续失败,主要是在比较 HNN 表示 $G_{HNN}$ 与 $G_{Std}$ 或 $G_R$ 时:
\begin{itemize}
    \item 函数 \texttt{IsomorphismFpGroup(GHNN\_g, GR\_g\_std)} (比较 HNN 和标准二生成元表示) 报告错误 \texttt{Error, no 1st choice method found}。
    \item 即使在调用同构检查之前,先对两个群表示分别使用了 GAP 的 \texttt{SimplifiedFpGroup} 函数进行化简,同样的错误依然出现。
    \item 在比较 $G_{HNN}$ 与来自 \texttt{snappy} 的表示 $G_R$ 时,也遇到了类似的失败。
    \item 错误处理中调用的 \texttt{AreIsomorphicFpGroup} 函数似乎也未能明确给出结果,或者在主错误发生前就已中断。
\end{itemize}
这些结果表明,对于 $4_1$ 这个特定的群,GAP 的标准算法在比较 HNN 结构表示与二生成元、单关系子结构表示时,遇到了计算上的困难。

\section{具体请教的问题}
鉴于上述困难,我们希望就以下问题寻求您的专业见解:
\begin{enumerate}
    \item $G_{HNN}$ 与 $G_{Std}$ (或者 $G_R$) 之间的**显式同构映射**是已知的吗?如果已知,能否指点我们在哪些文献或资源中可能找到这些具体的映射关系或推导过程?
    \item 从计算群论的角度来看,比较 HNN 表示 (如 $G_{HNN}$) 与标准的二生成元、单关系子表示 (如 $G_{Std}$) 的同构,对于 GAP 的 \texttt{IsomorphismFpGroup} 等标准算法是否存在已知的难点?是否有推荐的替代计算策略或工具?
    \item 来自 \texttt{snappy} 的表示 $G_R = \langle a, b \mid abbbaBAAB=1 \rangle$ 是否具有已知的特殊性质?它与 $G_{Std}$ 或 $G_{HNN}$ 是否有已知的、可能不那么直接的联系或简化方法,可以解释其计算行为?(备注:我们发现一个特定的算术解释对此表示有效,但对 $G_{Std}$ 无效,这增加了我们对 $G_R$ 特殊性的好奇。)
\end{enumerate}

任何关于这些问题的指导都将对我们的研究非常有帮助。

非常感谢!

\end{document}