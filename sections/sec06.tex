% Optional: A brief introductory paragraph for the chapter, setting the stage.
% e.g., This chapter delves into the concrete application of AEG principles
% to the figure-eight knot ($4_1$), a prime example of a hyperbolic knot.
% We aim to construct arithmetic expression spaces that not only faithfully
% represent its knot group but also exhibit non-trivial geometric features,
% particularly in their zero-loci, linking them to algebraic ideals.

\subsection{The challenge of representing $F_2$ in HNN arithmetization for the $4_1$ knot}
\label{sec:hnn_f2_challenge}
% - Briefly recap the problem from notes/note_04.tex regarding the commutativity
%   issue with simple additive arithmetizations for the fiber group F_2 = <u,v>.
% - Motivate the need for a new geometric and arithmetic framework.

\subsection{An $(\alpha, \beta)$ parameter space with modulo $(1, \phi)$ structure}
\label{sec:alpha_beta_space_modulo}
% - Introduce the 2D Euclidean space for parameters \alpha and \beta.
% - Detail the modulo 1 operation for \alpha and modulo \phi for \beta.
% - Define the resulting quotient space, T^2_{(1,\phi)}.

\subsection{The dense singularity and the ``punctured torus'' topology}
\label{sec:dense_singularity_topology}
% - Explain the concept of the dense equivalence class of (0,0) due to
%   the irrationality of \phi.
% - Describe this as a "topological singularity."
% - Discuss its impact on the manifold structure and the idea of a
%   "torus with a dense singularity removed" or "punctured torus."
% - Elaborate on the necessary point-set topology considerations.

\subsection{Non-commutative arithmetic realization of $u$ and $v$}
\label{sec:non_commutative_uv_R2}
% - Define the explicit actions of u and v as non-commuting transformations
%   on the (\alpha, \beta) pairs in the universal cover \mathbb{R}^2.
% - Provide the mathematical formulation of these operations.
% - Prove or demonstrate their non-commutativity.

\subsection{Arithmetization of the monodromy $t$ and satisfaction of HNN relations}
\label{sec:monodromy_hnn_relations}
% - Define the action of the monodromy operator \mathcal{E}(t) on the
%   (\alpha, \beta) space or its derived values.
% - Incorporate the AEG parameter t_{var} = \phi^{\pm 2} (or related value)
%   associated with the $4_1$ knot.
% - Demonstrate how the HNN relations (e.g., t^{-1}ut = uv, t^{-1}vt = vuv
%   for the $4_1$ knot) are satisfied within this arithmetic framework.

\subsection{Geometric interpretation: towards a hyperbolic expression space}
\label{sec:hyperbolic_expression_space}
% - Discuss how this new arithmetization framework for u, v, and t
%   relates to the hyperbolic geometry of the $4_1$ knot's fiber surface.
% - Argue or conjecture that the expression space accommodating these
%   F_2 operations (i.e., the paths of computation) is inherently hyperbolic.
% - Link this back to the "torus with a singularity removed" concept, possibly
%   seeing its universal cover (where u,v act) as a model for \mathbb{H}^2.

\subsection{Valuation of expressions, zero-loci}
\label{sec:valuation_zeroloci_ideals}
% - Explain how final arithmetic values are obtained from the (\alpha, \beta) states
%   within this modulo framework.
% - Discuss the expected structure of zero-loci in this $E_{4_1}^{(HNN)}$ space.
%   (Acknowledge if it might still be "trivial" in some respects and set
%   the stage for exploring other presentations for richer zero-loci).
% - Introduce or further develop the idea that zero-loci in AEG can be
%   understood as geometric manifestations of algebraic ideals.
% - Briefly touch upon how this specific $4_1$ HNN arithmetization relates to
%   broader AEG concepts like torsion and ACS, if applicable.
% - (Optional: This section could also briefly introduce the motivation for
%   looking at other $4_1$ presentations in a subsequent section/chapter
%   if this chapter focuses solely on the HNN breakthrough first.)

% If you plan to discuss other presentations of $4_1$ and their potential
% for different AEG spaces *within this same chapter*, you might add further
% sections like:
% \section{Exploring Alternative Presentations of $G(4_1)$ for Non-Trivial AEG Spaces}
% \label{sec:alternative_presentations_4_1}
% % - Discuss Wirtinger, G_R (from snappy) etc.
% % - Outline strategies for their arithmetization aiming for complex zero-loci.

% \section{Comparative Analysis of AEG Spaces from Different $4_1$ Presentations}
% \label{sec:comparative_aeg_4_1}
% % - Compare geometric features, zero-loci, and how invariants like \Delta(t) appear.
