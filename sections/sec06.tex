In the preceding section, we introduced the Accumulative Commutative Space (ACS) as an abstract parameter plane essential for quantifying global arithmetic torsion. While powerful for global analysis, the ACS lacks an intrinsic mechanism to describe the \textit{local}, non-commutative dynamics of arithmetic operations.

This section bridges that gap by endowing the ACS with a rich differential structure. We construct a three-dimensional \textbf{Arithmetic Contact Manifold}, whose two-dimensional base is precisely a local, differential version of the ACS. This framework provides the geometric engine for the arithmetic flow, unifying the global, commutative picture of torsion with the local, non-commutative nature of operations. It is here that the flow equation from Section~\ref{sec:flow_equation} finds its most natural home, and we develop the differential calculus necessary for its study.

\subsection{Core definitions and the contact structure}\label{subsec:core_definitions}

We begin by establishing our geometric space. Consider a 3-dimensional manifold with coordinates $(u,v,a) \in \mathbb{R}^3$, where the base plane spanned by $(u,v)$ is understood as the differential realization of the ACS. The geometry is built upon two fundamental 1-forms, defined using constant real parameters $\mu$ and $\lambda$:
\begin{equation}\label{eq:contact_forms_def}
\omega := \mu\,du + \lambda a\,dv, \qquad \alpha := da - \omega.
\end{equation}
The 1-form $\alpha$ is central to our construction. Its primary role is to define a ``horizontal'' plane at each point, known as the contact distribution. To see how it works, let's consider its action on an arbitrary vector field $X = x_u \partial_u + x_v \partial_v + x_a \partial_a$. Recalling that $da(X) = x_a$, $du(X) = x_u$, and $dv(X) = x_v$, the action is given by:
\[
\alpha(X) = da(X) - \mu\,du(X) - \lambda a\,dv(X) = x_a - \mu x_u - \lambda a x_v.
\]
The set of all vectors $X$ for which $\alpha(X)=0$ constitutes this horizontal plane.

Throughout this work, we use two types of differentials in parallel: (i) the standard de Rham exterior derivative $d$, which is always nilpotent ($d^2=0$); and (ii) the \textbf{expression differential} $\delta$, which is axiomatically defined in Section~\ref{ch:differential_calculus}.

\paragraph{The contact property.}
A key property of $\alpha$ is that it defines a contact structure on our 3D space. This is verified by checking if the 3-form $\alpha \wedge d\alpha$ is a volume form (i.e., non-zero everywhere). First, we compute the exterior derivative of $\omega$:
\[
d\omega = d(\mu\,du + \lambda a\,dv) = \lambda\,da \wedge dv.
\]
Then, the exterior derivative of $\alpha$ is simply $d\alpha = d(da - \omega) = -d\omega = -\lambda\,da \wedge dv$. Now, we can compute the wedge product:
\begin{align*}
\alpha \wedge d\alpha &= (da - \omega) \wedge (-\lambda\,da \wedge dv) \\
&= -da \wedge (\lambda\,da \wedge dv) + \omega \wedge (\lambda\,da \wedge dv) \\
&= 0 + (\mu\,du + \lambda a\,dv) \wedge (\lambda\,da \wedge dv) \\
&= \mu\lambda\,du \wedge da \wedge dv + \lambda^2 a\,\underbrace{dv \wedge da \wedge dv}_{=0} \\
&= \mu\lambda\,du \wedge da \wedge dv.
\end{align*}
Provided $\mu\lambda \neq 0$, this is a volume form, confirming that $\alpha$ is a \textbf{contact form}.

\paragraph{Normal form, Reeb field, and contact distribution.}
By introducing natural units $\tilde{u} = \mu u$ and $\tilde{v} = \lambda v$, the form $\alpha$ is reduced to its canonical form $\alpha_0 = da - d\tilde{u} - a\,d\tilde{v}$, which facilitates comparison with standard literature.

The Reeb vector field $R$, defined by $i_R d\alpha = 0$ and $\alpha(R) = 1$, is $R = -(1/\mu)\partial_u$. The contact distribution $\mathcal{H}$, as introduced earlier, is formally the kernel of $\alpha$:
\[
\mathcal{H} := \ker\alpha = \{X \in TM : \alpha(X) = 0\}.
\]
Viewing the tangent bundle as a composition of the $(u,v)$-base and the $a$-fiber, the form $\alpha = da - \omega$ intrinsically links the ``vertical'' change $da$ to the ``horizontal'' displacement defined by $\omega$.

\subsection{The geometry on the contact distribution \texorpdfstring{$\ker\alpha$}{ker(alpha)}} % Revised title for 6.2
\label{subsec:geometry_on_ker_alpha}

We now define two special vector fields that form a basis for the horizontal plane $\mathcal{H}$ at every point. These are the horizontal lifts of the base coordinate vectors, which we term the \textbf{expression directional derivatives}:
\begin{equation}\label{eq:directional_derivatives}
D_u := \partial_u + \mu\,\partial_a, \qquad D_v := \partial_v + \lambda a\,\partial_a.
\end{equation}
These fields are constructed specifically to be horizontal, a fact we can verify directly. For $D_u$, its coordinate components are $(x_u, x_v, x_a) = (1, 0, \mu)$. Applying the formula for $\alpha(X)$:
\[
\alpha(D_u) = x_a - \mu x_u - \lambda a x_v = \mu - \mu(1) - \lambda a(0) = 0.
\]
For $D_v$, its components are $(x_u, x_v, x_a) = (0, 1, \lambda a)$. Applying $\alpha$:
\[
\alpha(D_v) = x_a - \mu x_u - \lambda a x_v = \lambda a - \mu(0) - \lambda a(1) = 0.
\]
Since both $D_u$ and $D_v$ are annihilated by $\alpha$, and they are clearly linearly independent, they form a basis for the 2-dimensional contact distribution:
\[
\ker\alpha = \text{span}\{D_u, D_v\}.
\]

For any smooth scalar field $F(u,v,a)$, we define its expression differential $\delta F$ and the directional derivative $D_\theta$ as:
\begin{equation}\label{eq:3}\tag{3}
\delta F := (D_uF)\,du + (D_vF)\,dv, \qquad D_\theta := \cos\theta\,D_u + \sin\theta\,D_v.
\end{equation}
This construction effectively reduces the geometry from the three-dimensional space $(u,v,a)$ to the two-dimensional contact distribution $\mathcal{H}$.

\paragraph{Structural relation between d and $\delta$.}
For any scalar field $F$, the two differentials are related by a fundamental identity:
\[
\boxed{ \delta F = dF - (\partial_a F)\,\alpha }
\]
This identity can be interpreted as projecting $dF$ onto the horizontal distribution $\mathcal{H}$ by subtracting its vertical component along $\alpha$. Expanding this definition yields:
\[
\delta F = (F_u + \mu F_a)\,du + (F_v + \lambda a F_a)\,dv,
\]
which is consistent with Eq.~\eqref{eq:15}. In particular, we recover $\delta a = \omega$, $\delta u = du$, and $\delta v = dv$.

\subsection{Legendrian flow and rectification}

A curve $\gamma(s) = (u(s), v(s), a(s))$ is \textbf{Legendrian} if its tangent vector lies in the contact distribution, i.e., $\dot\gamma(s) \in \ker\alpha$. For such a curve, the evolution of $a$ is governed by the \textbf{flow equation}:
\begin{equation}\label{eq:4}\tag{4}
\frac{da}{ds} = D_\theta a = \mu\cos\theta + \lambda a\sin\theta,
\end{equation}
where $\theta$ parameterizes the angle of the tangent vector in the basis $\{D_u,D_v\}$. With respect to a given metric on the base manifold, this flow equation can be written in its Eikonal form:
\begin{equation}\label{eq:5}\tag{5}
\|\nabla a\| = \sqrt{\mu^2 + \lambda^2 a^2}.
\end{equation}
We introduce a \textbf{rectifying variable} $y$:
\begin{equation}\label{eq:6}\tag{6}
y = \arcsinh\left(\frac{\lambda a}{\mu}\right) \quad\Rightarrow\quad \|\nabla y\| = \lambda.
\end{equation}
This rectification transforms the non-linear velocity field for $a$ into a constant-speed flow for $y$, which is advantageous for geometric constructions and enhances numerical stability.

\paragraph{Non-commutativity and curvature.}
The commutator of the horizontal vector fields yields a purely vertical vector, reflecting the "curvature" of the contact distribution. This phenomenon can be described as a "vertical return":
\begin{equation}\label{eq:7}\tag{7}
[D_u,D_v] = \mu\lambda\,\partial_a,\qquad \delta^2F = \mu\lambda(\partial_a F)\,du\wedge dv,\qquad \delta^2 a = \mu\lambda\,du\wedge dv.
\end{equation}
The circulation-area formula provides a tool for quantifying mesh singularities and global topological constraints:
\begin{equation}\label{eq:8}\tag{8}
\oint_{\partial\Sigma}\omega = \iint_\Sigma d\omega = \mu\lambda\iint_\Sigma du\wedge dv.
\end{equation}

\paragraph{Compatibility with de Rham cohomology.}
The exterior derivative of $\omega$ is $d\omega = \lambda\,da\wedge dv$. By pulling this 2-form back to a section where $\alpha=0$ (i.e., substituting $da=\omega$), we obtain:
\[
(d\omega)^* = \lambda\,\omega\wedge dv = \mu\lambda\,du\wedge dv,
\]
which is identical to $\delta^2 a$. This demonstrates the precise collaborative relationship between $d$ and $\delta$.

\paragraph{Example: Basis Flows.}
Flowing along $D_u$ implies $\dot{a} = \mu \Rightarrow a(s) = a_0 + \mu s$. Flowing along $D_v$ implies $\dot{a} = \lambda a \Rightarrow a(s) = a_0 e^{\lambda s}$. Linear combinations of these generate the general flow given by Eq.~\eqref{eq:4}.

\subsection{Algebraic Structure and Classification of the AEG Lie Algebra}

The contact geometry framework developed previously not only provides a dynamical system for the arithmetic flow but also reveals a deep underlying algebraic structure. The horizontal vector fields \(D_u\), \(D_v\), and the emergent vertical field \(\partial_a\) form a real Lie algebra under the Lie bracket. Classifying this algebra is essential for understanding the fundamental nature of the interplay between addition and multiplication.

\subsubsection{Commutation Relations and Lie Algebra Verification}

Let us define the basis for our Lie algebra, \(\mathfrak{g}\), as:
\begin{itemize}
    \item \(e_1 = D_u = \partial_u + \mu\partial_a\)
    \item \(e_2 = D_v = \partial_v + \lambda a \partial_a\)
    \item \(e_3 = \partial_a\)
\end{itemize}
Their commutation relations are directly computed as:
\begin{align}
    [e_1, e_2] &= \mu\lambda e_3 \\
    [e_1, e_3] &= [\partial_u + \mu\partial_a, \partial_a] = 0 \\
    [e_2, e_3] &= [\partial_v + \lambda a \partial_a, \partial_a] = -\lambda e_3
\end{align}
As vector fields on a smooth manifold, they are guaranteed to satisfy the Jacobi identity, \([X, [Y, Z]] + [Y, [Z, X]] + [Z, [X, Y]] = 0\), thus confirming that \(\mathfrak{g}\) is a bona fide Lie algebra.

\subsubsection{Classification: Solvable but not Nilpotent}

To classify \(\mathfrak{g}\), we examine its derived and lower central series.
\begin{itemize}
    \item \textbf{Solvability:} The derived series is \(\mathcal{D}^1\mathfrak{g} = [\mathfrak{g}, \mathfrak{g}] = \text{span}\{e_3\}\) and \(\mathcal{D}^2\mathfrak{g} = [\mathcal{D}^1\mathfrak{g}, \mathcal{D}^1\mathfrak{g}] = \{0\}\). Since the series terminates at zero, the algebra is \textbf{solvable}.
    \item \textbf{Nilpotency:} The lower central series is \(\mathcal{L}^1\mathfrak{g} = [\mathfrak{g}, \mathfrak{g}] = \text{span}\{e_3\}\). However, the next term, \(\mathcal{L}^2\mathfrak{g} = [\mathfrak{g}, \mathcal{L}^1\mathfrak{g}]\), contains the non-zero bracket \([e_2, e_3] = -\lambda e_3\), which means \(\mathcal{L}^2\mathfrak{g} = \text{span}\{e_3\}\). The series stabilizes and never reaches zero, hence the algebra is \textbf{non-nilpotent}.
\end{itemize}
Thus, the AEG Lie algebra is a three-dimensional, solvable, non-nilpotent real Lie algebra.

\subsubsection{Deeper Structure: A Central Extension of \(\mathfrak{aff}(1)\)}

A more precise characterization comes from analyzing the algebra's internal structure.
\begin{itemize}
    \item \textbf{Center:} The center \(Z(\mathfrak{g})\) consists of elements that commute with all elements of \(\mathfrak{g}\). A direct calculation shows that \(Z(\mathfrak{g}) = \text{span}\{e_1 - \mu e_3\} = \text{span}\{\partial_u\}\).
    \item \textbf{Quotient Algebra:} The quotient algebra \(\mathfrak{g}/Z(\mathfrak{g})\) is a two-dimensional Lie algebra spanned by the cosets \(\bar{e}_2\) and \(\bar{e}_3\). Their bracket is \([\bar{e}_2, \bar{e}_3] = -\lambda \bar{e}_3\). This is precisely the Lie algebra of the 1D affine group, \(\mathfrak{aff}(1)\).
\end{itemize}
This reveals a crucial insight: \textbf{the AEG Lie algebra \(\mathfrak{g}\) is a central extension of the affine Lie algebra \(\mathfrak{aff}(1)\) by a one-dimensional center}.

Furthermore, this structure is "almost-abelian," as it contains both the 2D non-abelian subalgebra \(\text{span}\{e_2, e_3\} \cong \mathfrak{aff}(1)\) and the 2D abelian ideal \(\text{span}\{e_1, e_3\}\).

\subsubsection{Isomorphism Class and Significance}

For non-zero parameters \(\mu, \lambda\), we can normalize the basis via the linear transformation \(e'_2 = e_2 / \lambda\) and \(e'_3 = \mu e_3\). The commutation relations become:
\begin{equation}
    [e_1, e'_2] = e'_3, \quad [e_1, e'_3] = 0, \quad [e'_2, e'_3] = -e'_3
\end{equation}
This demonstrates that for any \(\mu\lambda \neq 0\), the resulting Lie algebra belongs to a \textbf{single isomorphism class} over \(\mathbb{R}\). The parameters \(\mu\) and \(\lambda\) only scale the basis, they do not change the intrinsic algebraic structure.

In conclusion, the contact form \(\alpha=da-\mu\,du-\lambda a\,dv\), which in natural units \((\tilde{u}=\mu u, \tilde{v}=\lambda v)\) takes the canonical shape \(da-d\tilde{u}-a\,d\tilde{v}\), provides the precise geometric stage for this "geometrized affine algebra." The fact that this geometry realizes a central extension of \(\mathfrak{aff}(1)\) confirms that the AEG framework captures the fundamental algebraic signature of arithmetic—a structure born from the interaction of a translation-like operation (\(D_u\)) with a value-dependent scaling operation (\(D_v\)).

\subsection{Computational spacetime duality: a de Rham and Poincar\'e-duality viewpoint}
\label{subsec:spacetime_duality}

A recurring theme in AEG is that the same non-commutative defect can be computed either from a
\emph{one-dimensional boundary object} (a history) or from a \emph{two-dimensional interior object}
(a filling).  In the ACS, global torsion admits a Stokes-type ``triple identity'' (Section~5); on the
contact manifold, the circulation--area relation \eqref{eq:8} provides its local differential analogue.
Both statements share the schematic form
\[
\text{(boundary holonomy)} \;=\; \text{(curvature flux through a spanning surface)}.
\]
This observation motivates a \textbf{computational spacetime duality}:
\begin{itemize}
\item \textbf{time-like objects:} $1$D evaluation histories (words/paths, i.e.\ ordered operation sequences);
\item \textbf{space-like objects:} $2$D canonical structures (rewriting/proof surfaces that relate histories);
\item \textbf{duality bridge:} torsion as a de Rham pairing (curvature flux equals boundary holonomy).
\end{itemize}

\paragraph{Adjoining $2$-cells to the path calculus.}
A threadlike (or alternating) expression path can be viewed as a word in the generators
$\oplus_{\mu}$ and $\otimes_{\lambda}$ (and inverses), hence as a $1$-morphism in a path groupoid.
However, non-commutativity is not adequately captured by ordinary topological homotopy of curves.
Instead, we \emph{adjoin explicit $2$-dimensional data}: the basic generating $2$-cell is the
\emph{interchange square}
\[
(\oplus_{\mu}\,\otimes_{\lambda}) \;\Rightarrow\; (\otimes_{\lambda}\,\oplus_{\mu}),
\]
whose discrete label is the one-step torsion constant
\[
\tau_{\mathrm{local}}
:= \bigl(x \oplus_{\mu}\otimes_{\lambda}\bigr) - \bigl(x\otimes_{\lambda}\oplus_{\mu}\bigr)
= \mu\,(e^{\lambda}-1),
\]
independent of the initial value $x$.
In the infinitesimal limit (arithmetic coordinates $(u,v)$), this label is replaced by the integral
of the torsion/curvature density over the $2$-cell.  Concretely, we take the $2$-form
\begin{equation}\label{eq:torsion_density_form}
\Omega \;:=\; \delta^{2}a \;=\; \mu\lambda\,du\wedge dv,
\end{equation}
which already appeared as the curvature of the horizontal calculus in \eqref{eq:7}.
A composite rewriting surface $\Sigma$ obtained by pasting interchange cells carries the additive label
\[
\mathrm{label}(\Sigma) \;:=\; \iint_{\Sigma}\Omega,
\]
so the existence of a $2$-morphism (a rewriting surface) between two histories becomes the
appropriate notion of ``arithmetic homotopy'' in this setting.

\paragraph{de Rham pairing: curvature flux equals boundary holonomy.}
The contact geometry provides a canonical potential $1$-form for the local density \eqref{eq:torsion_density_form}.
Indeed, on the horizontal section $\alpha=0$ (equivalently $da=\omega$), we have
\[
(d\omega)^{*} = \mu\lambda\,du\wedge dv = \Omega,
\]
so $\Omega$ is the pulled-back curvature of the connection defined by $\alpha$.
Consequently, for any oriented surface $\Sigma$ in the $(u,v)$-base with boundary $\partial\Sigma$,
Stokes' theorem yields the de Rham pairing identity
\begin{equation}\label{eq:de_rham_pairing_local}
\iint_{\Sigma}\Omega
\;=\;
\iint_{\Sigma}(d\omega)^{*}
\;=\;
\oint_{\partial\Sigma}\omega,
\end{equation}
which is precisely the circulation--area relation \eqref{eq:8}.
In the global ACS picture, the same structure reappears with the weighted forms
$\eta_{\mathrm{ACS}}:= e^{M}\,dA$ and $\Omega_{\mathrm{ACS}}:= d\eta_{\mathrm{ACS}} = e^{M}\,dM\wedge dA$,
so that torsion is again computed either from a boundary integral or an interior flux.

\paragraph{A filling-cost functional.}
Once $2$-morphisms are admitted, the natural quantitative object is a \emph{filling cost}:
for two histories $\gamma$ and $\gamma'$ with the same semantic endpoints, define
\begin{equation}\label{eq:filling_cost_def}
\mathrm{Fill}(\gamma \Rightarrow \gamma')
\;:=\;
\inf_{\Sigma:\ \partial\Sigma=\gamma' - \gamma}\;
\iint_{\Sigma}\Omega,
\end{equation}
where the infimum runs over admissible rewriting surfaces $\Sigma$ realizing $\gamma \Rightarrow \gamma'$.
In the special case $\gamma'=\bar{\gamma}$ (reversal of the operation order), the ACS triple identity
recovers a canonical filling and computes the same invariant as a weighted area.  The functional
\eqref{eq:filling_cost_def} is therefore a natural extension from ``torsion as area'' to ``torsion as minimal filling.''

\paragraph{Interpretation.}
The message of this subsection is organizational rather than terminological: AEG naturally carries
both $1$D ``history data'' (evaluation paths) and $2$D ``proof data'' (rewriting fillings), and the
torsion invariant can be computed from either side.  This provides a geometric template for
studying trade-offs between time-like costs (length/word length/energy of a history) and
space-like costs (area or minimal filling cost of canonicalization), with torsion as the
duality bridge.
