In the preceding section, we introduced the Accumulative Commutative Space (ACS) as an abstract parameter plane essential for quantifying global arithmetic torsion. While powerful for global analysis, the ACS lacks an intrinsic mechanism to describe the \textit{local}, non-commutative dynamics of arithmetic operations.

This section bridges that gap by endowing the ACS with a rich differential structure. We construct a three-dimensional \textbf{Arithmetic Contact Manifold}, whose two-dimensional base is precisely a local, differential version of the ACS. This framework provides the geometric engine for the arithmetic flow, unifying the global, commutative picture of torsion with the local, non-commutative nature of operations. It is here that the flow equation from Section~\ref{sec:flow_equation} finds its most natural home, and we develop the differential calculus necessary for its study.

\subsection{Core definitions and the contact structure}\label{subsec:core_definitions}

We begin by establishing our geometric space. Consider a 3-dimensional manifold with coordinates $(u,v,a) \in \mathbb{R}^3$, where the base plane spanned by $(u,v)$ is understood as the differential realization of the ACS. The geometry is built upon two fundamental 1-forms, defined using constant real parameters $\mu$ and $\lambda$:
\begin{equation}\label{eq:contact_forms_def}
\omega := \mu\,du + \lambda a\,dv, \qquad \alpha := da - \omega.
\end{equation}
The 1-form $\alpha$ is central to our construction. Its primary role is to define a ``horizontal'' plane at each point, known as the contact distribution. To see how it works, let's consider its action on an arbitrary vector field $X = x_u \partial_u + x_v \partial_v + x_a \partial_a$. Recalling that $da(X) = x_a$, $du(X) = x_u$, and $dv(X) = x_v$, the action is given by:
\[
\alpha(X) = da(X) - \mu\,du(X) - \lambda a\,dv(X) = x_a - \mu x_u - \lambda a x_v.
\]
The set of all vectors $X$ for which $\alpha(X)=0$ constitutes this horizontal plane.

Throughout this work, we use two types of differentials in parallel: (i) the standard de Rham exterior derivative $d$, which is always nilpotent ($d^2=0$); and (ii) the \textbf{expression differential} $\delta$, which is axiomatically defined in Section~\ref{ch:differential_calculus}.

\paragraph{The contact property.}
A key property of $\alpha$ is that it defines a contact structure on our 3D space. This is verified by checking if the 3-form $\alpha \wedge d\alpha$ is a volume form (i.e., non-zero everywhere). First, we compute the exterior derivative of $\omega$:
\[
d\omega = d(\mu\,du + \lambda a\,dv) = \lambda\,da \wedge dv.
\]
Then, the exterior derivative of $\alpha$ is simply $d\alpha = d(da - \omega) = -d\omega = -\lambda\,da \wedge dv$. Now, we can compute the wedge product:
\begin{align*}
\alpha \wedge d\alpha &= (da - \omega) \wedge (-\lambda\,da \wedge dv) \\
&= -da \wedge (\lambda\,da \wedge dv) + \omega \wedge (\lambda\,da \wedge dv) \\
&= 0 + (\mu\,du + \lambda a\,dv) \wedge (\lambda\,da \wedge dv) \\
&= \mu\lambda\,du \wedge da \wedge dv + \lambda^2 a\,\underbrace{dv \wedge da \wedge dv}_{=0} \\
&= \mu\lambda\,du \wedge da \wedge dv.
\end{align*}
Provided $\mu\lambda \neq 0$, this is a volume form, confirming that $\alpha$ is a \textbf{contact form}.

\paragraph{Normal form, Reeb field, and contact distribution.}
By introducing natural units $\tilde{u} = \mu u$ and $\tilde{v} = \lambda v$, the form $\alpha$ is reduced to its canonical form $\alpha_0 = da - d\tilde{u} - a\,d\tilde{v}$, which facilitates comparison with standard literature.

The Reeb vector field $R$, defined by $i_R d\alpha = 0$ and $\alpha(R) = 1$, is $R = -(1/\mu)\partial_u$. The contact distribution $\mathcal{H}$, as introduced earlier, is formally the kernel of $\alpha$:
\[
\mathcal{H} := \ker\alpha = \{X \in TM : \alpha(X) = 0\}.
\]
Viewing the tangent bundle as a composition of the $(u,v)$-base and the $a$-fiber, the form $\alpha = da - \omega$ intrinsically links the ``vertical'' change $da$ to the ``horizontal'' displacement defined by $\omega$.

\subsection{The geometry on the contact distribution \texorpdfstring{$\ker\alpha$}{ker(alpha)}} % Revised title for 6.2
\label{subsec:geometry_on_ker_alpha}

We now define two special vector fields that form a basis for the horizontal plane $\mathcal{H}$ at every point. These are the horizontal lifts of the base coordinate vectors, which we term the \textbf{expression directional derivatives}:
\begin{equation}\label{eq:directional_derivatives}
D_u := \partial_u + \mu\,\partial_a, \qquad D_v := \partial_v + \lambda a\,\partial_a.
\end{equation}
These fields are constructed specifically to be horizontal, a fact we can verify directly. For $D_u$, its coordinate components are $(x_u, x_v, x_a) = (1, 0, \mu)$. Applying the formula for $\alpha(X)$:
\[
\alpha(D_u) = x_a - \mu x_u - \lambda a x_v = \mu - \mu(1) - \lambda a(0) = 0.
\]
For $D_v$, its components are $(x_u, x_v, x_a) = (0, 1, \lambda a)$. Applying $\alpha$:
\[
\alpha(D_v) = x_a - \mu x_u - \lambda a x_v = \lambda a - \mu(0) - \lambda a(1) = 0.
\]
Since both $D_u$ and $D_v$ are annihilated by $\alpha$, and they are clearly linearly independent, they form a basis for the 2-dimensional contact distribution:
\[
\ker\alpha = \text{span}\{D_u, D_v\}.
\]

For any smooth scalar field $F(u,v,a)$, we define its expression differential $\delta F$ and the directional derivative $D_\theta$ as:
\begin{equation}\label{eq:3}\tag{3}
\delta F := (D_uF)\,du + (D_vF)\,dv, \qquad D_\theta := \cos\theta\,D_u + \sin\theta\,D_v.
\end{equation}
This construction effectively reduces the geometry from the three-dimensional space $(u,v,a)$ to the two-dimensional contact distribution $\mathcal{H}$.

\paragraph{Structural relation between d and $\delta$.}
For any scalar field $F$, the two differentials are related by a fundamental identity:
\[
\boxed{ \delta F = dF - (\partial_a F)\,\alpha }
\]
This identity can be interpreted as projecting $dF$ onto the horizontal distribution $\mathcal{H}$ by subtracting its vertical component along $\alpha$. Expanding this definition yields:
\[
\delta F = (F_u + \mu F_a)\,du + (F_v + \lambda a F_a)\,dv,
\]
which is consistent with Eq.~\eqref{eq:15}. In particular, we recover $\delta a = \omega$, $\delta u = du$, and $\delta v = dv$.

\subsection{Legendrian flow and rectification}

A curve $\gamma(s) = (u(s), v(s), a(s))$ is \textbf{Legendrian} if its tangent vector lies in the contact distribution, i.e., $\dot\gamma(s) \in \ker\alpha$. For such a curve, the evolution of $a$ is governed by the \textbf{flow equation}:
\begin{equation}\label{eq:4}\tag{4}
\frac{da}{ds} = D_\theta a = \mu\cos\theta + \lambda a\sin\theta,
\end{equation}
where $\theta$ parameterizes the angle of the tangent vector in the basis $\{D_u,D_v\}$. With respect to a given metric on the base manifold, this flow equation can be written in its Eikonal form:
\begin{equation}\label{eq:5}\tag{5}
\|\nabla a\| = \sqrt{\mu^2 + \lambda^2 a^2}.
\end{equation}
We introduce a \textbf{rectifying variable} $y$:
\begin{equation}\label{eq:6}\tag{6}
y = \arcsin\left(\frac{\lambda a}{\mu}\right) \quad\Rightarrow\quad \|\nabla y\| = \lambda.
\end{equation}
This rectification transforms the non-linear velocity field for $a$ into a constant-speed flow for $y$, which is advantageous for geometric constructions and enhances numerical stability.

\paragraph{Non-commutativity and curvature.}
The commutator of the horizontal vector fields yields a purely vertical vector, reflecting the "curvature" of the contact distribution. This phenomenon can be described as a "vertical return":
\begin{equation}\label{eq:7}\tag{7}
[D_u,D_v] = \mu\lambda\,\partial_a,\qquad \delta^2F = \mu\lambda(\partial_a F)\,du\wedge dv,\qquad \delta^2 a = \mu\lambda\,du\wedge dv.
\end{equation}
The circulation-area formula provides a tool for quantifying mesh singularities and global topological constraints:
\begin{equation}\label{eq:8}\tag{8}
\oint_{\partial\Sigma}\omega = \iint_\Sigma d\omega = \mu\lambda\iint_\Sigma du\wedge dv.
\end{equation}

\paragraph{Compatibility with de Rham cohomology.}
The exterior derivative of $\omega$ is $d\omega = \lambda\,da\wedge dv$. By pulling this 2-form back to a section where $\alpha=0$ (i.e., substituting $da=\omega$), we obtain:
\[
(d\omega)^* = \lambda\,\omega\wedge dv = \mu\lambda\,du\wedge dv,
\]
which is identical to $\delta^2 a$. This demonstrates the precise collaborative relationship between $d$ and $\delta$.

\paragraph{Example: Basis Flows.}
Flowing along $D_u$ implies $\dot{a} = \mu \Rightarrow a(s) = a_0 + \mu s$. Flowing along $D_v$ implies $\dot{a} = \lambda a \Rightarrow a(s) = a_0 e^{\lambda s}$. Linear combinations of these generate the general flow given by Eq.~\eqref{eq:4}.

\subsection{Classification and Significance of the AEG Lie Algebra}

The contact geometry framework developed in the preceding sections, particularly the horizontal vector fields \(D_u\) and \(D_v\), does more than provide a dynamical system for the arithmetic flow. Its algebraic structure is of fundamental importance. The vector fields, under the operation of the Lie bracket, form a Lie algebra. Classifying this algebra reveals the intrinsic nature of the structure generated by the interplay of addition and multiplication.

Let us define a basis for this Lie algebra, \(\mathfrak{g}\), using the vector fields central to our construction:
\begin{itemize}
    \item \(e_1 = D_u = \partial_u + \mu\partial_a\)
    \item \(e_2 = D_v = \partial_v + \lambda a \partial_a\)
    \item \(e_3 = \partial_a\)
\end{itemize}
From direct computation, their commutation relations are given by:
\begin{align}
    [e_1, e_2] &= \mu\lambda e_3 \\
    [e_1, e_3] &= [\partial_u + \mu\partial_a, \partial_a] = 0 \\
    [e_2, e_3] &= [\partial_v + \lambda a \partial_a, \partial_a] = -\lambda e_3
\end{align}

\subsubsection*{Verification of the Lie Algebra Structure}
For \(\mathfrak{g}\) to be a valid Lie algebra, the Jacobi identity, \([X, [Y, Z]] + [Y, [Z, X]] + [Z, [X, Y]] = 0\), must hold for all \(X, Y, Z \in \mathfrak{g}\). It is sufficient to verify this for the basis vectors \((e_1, e_2, e_3)\).

\begin{align*}
    [e_1, [e_2, e_3]] + [e_2, [e_3, e_1]] + [e_3, [e_1, e_2]] &= [e_1, -\lambda e_3] + [e_2, -[e_1, e_3]] + [e_3, \mu\lambda e_3] \\
    &= -\lambda [e_1, e_3] - [e_2, 0] + \mu\lambda [e_3, e_3] \\
    &= -\lambda(0) - 0 + \mu\lambda(0) \\
    &= 0
\end{align*}
The identity holds. This rigorously confirms that the kinematic structure of Arithmetic Expression Geometry is governed by a bona fide Lie algebra.

\subsubsection*{Classification}
To determine the position of this Lie algebra within the standard classification system, we examine its derived series and lower central series.

\paragraph{Solvability:} The derived series \(\mathcal{D}^i\mathfrak{g}\) is defined by \(\mathcal{D}^0\mathfrak{g} = \mathfrak{g}\) and \(\mathcal{D}^{i+1}\mathfrak{g} = [\mathcal{D}^i\mathfrak{g}, \mathcal{D}^i\mathfrak{g}]\).
\begin{itemize}
    \item \(\mathcal{D}^1\mathfrak{g} = [\mathfrak{g}, \mathfrak{g}] = \text{span}\{[e_1, e_2], [e_1, e_3], [e_2, e_3]\} = \text{span}\{e_3\}\).
    \item \(\mathcal{D}^2\mathfrak{g} = [\mathcal{D}^1\mathfrak{g}, \mathcal{D}^1\mathfrak{g}] = [\text{span}\{e_3\}, \text{span}\{e_3\}] = \{0\}\).
\end{itemize}
Since the derived series terminates at zero, \textbf{the Lie algebra is solvable}. This reflects a hierarchical nature of non-commutativity within the system.

\paragraph{Nilpotency:} The lower central series \(\mathcal{L}^i\mathfrak{g}\) is defined by \(\mathcal{L}^0\mathfrak{g} = \mathfrak{g}\) and \(\mathcal{L}^{i+1}\mathfrak{g} = [\mathfrak{g}, \mathcal{L}^i\mathfrak{g}]\).
\begin{itemize}
    \item \(\mathcal{L}^1\mathfrak{g} = [\mathfrak{g}, \mathfrak{g}] = \text{span}\{e_3\}\).
    \item \(\mathcal{L}^2\mathfrak{g} = [\mathfrak{g}, \mathcal{L}^1\mathfrak{g}] = [\mathfrak{g}, \text{span}\{e_3\}]\). This contains the non-zero bracket \([e_2, e_3] = -\lambda e_3\), thus \(\mathcal{L}^2\mathfrak{g} = \text{span}\{e_3\}\).
\end{itemize}
Since \(\mathcal{L}^2\mathfrak{g} = \mathcal{L}^1\mathfrak{g} \neq \{0\}\), the series becomes stable and never reaches zero. Therefore, \textbf{the Lie algebra is non-nilpotent}.

\subsubsection*{Significance}
The classification of the AEG algebra as a \textbf{three-dimensional, solvable, non-nilpotent real Lie algebra} is profoundly significant. The non-nilpotent character stems directly from the commutation relation \([D_v, \partial_a] = -\lambda\partial_a\), which reveals that the multiplicative flow \(D_v\) acts upon the emergent `value' field \(\partial_a\) and reproduces it. This self-referential action, inherent to scaling, contrasts sharply with the additive flow \(D_u\), whose `translation-invariant' nature is captured by \([D_u, \partial_a] = 0\).

This algebraic structure is closely related to the Lie algebra of the \textbf{1D affine group}, \(\mathfrak{aff}(1)\), which governs translations and scalings of the real line and serves as a prototype for solvable, non-nilpotent Lie algebras. The AEG algebra can thus be understood as a richer, three-dimensional realization of this fundamental affine structure, with \(D_u\) playing the role of translation and \(D_v\) a value-dependent scaling. This algebraic positioning solidifies the structural legitimacy of the AEG framework and opens avenues for leveraging the well-developed theory of solvable Lie algebras to further explore the geometry of arithmetic.

