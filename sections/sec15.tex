Giving

\[
    ds^2 = \frac{1}{y^2} (\frac{dx^2}{\mu^2} + \frac{dy^2}{\lambda^2})
\]

we calculate the geometric quantities.
We follow the notion in the text book\cite{Needham2021VisualDG}, and the first fundamental form is given by

\[
    ds^2 = A^2 dx^2 + B^2 dy^2
\]

where

\[
    A = \frac{1}{\mu y}, \quad B = \frac{1}{\lambda y}
\]

\subsection{Line element}\label{sec:line-element}

The line element is already given by above equation.

\subsection{Area element}\label{sec:area-element}

The area element is given by

\[
    dS = A B dx dy
\]

hence we have

\[
    dS = \frac{1}{\mu \lambda y^2} dx dy
\]

\subsection{Gauss curvature}\label{sec:curvature-calculations}


Gauss curvature $K$ is given by

\[
    K = - \frac{1}{A B} \left(\partial_y \left(\frac{\partial_y A}{B}\right) + \partial_x \left(\frac{\partial_x B}{A}\right)\right)
\]

so we have

\[
    K = - \mu \lambda y^2 \left(
          \partial_y \left(\lambda y \partial_y \left(\frac{1}{\mu y}\right)\right)
        + \partial_x \left(\mu y \partial_x \left(\frac{1}{\lambda y}\right)\right)
    \right)
\]

\[
    K = - \lambda^2 y^2 \left(
    \partial_y \left( y \partial_y \left(\frac{1}{y} \right)\right)
    \right)
\]

\[
    K = - \lambda^2 y^2 \frac{1}{y^2}
\]

\[
    K = - \lambda^2
\]

\subsection{Laplacian}\label{sec:laplacian-calculations}

Given a metric tensor
\[
    g =
    \begin{bmatrix}
        A^2 & 0\\
        0 & B^2
    \end{bmatrix},
\]
where \(A\) and \(B\) are functions of the coordinates (typically \(x\) and \(y\)), the Laplacian of a function \(f(x, y)\) can be derived from the general expression of the Laplace-Beltrami operator for a Riemannian manifold. The formula for the Laplacian \(\Delta f\) in such a setting, using the metric components \(g_{ij}\), is given by:
\[
    \Delta f = \frac{1}{\sqrt{|g|}} \partial_i \left( \sqrt{|g|} g^{ij} \partial_j f \right),
\]
where \(|g|\) is the determinant of the metric tensor \(g_{ij}\), \(g^{ij}\) are the components of the inverse metric tensor, and \(\partial_i\) denotes partial differentiation with respect to the \(i\)th coordinate.

Given the metric tensor, the determinant \(|g|\) is \(A^2B^2\). The inverse metric tensor \(g^{ij}\) is simply:
\[
    g^{ij} =
    \begin{bmatrix}
        \frac{1}{A^2} & 0\\
        0 & \frac{1}{B^2}
    \end{bmatrix}.
\]
Plugging these into the formula for the Laplacian, we get:
\[
    \Delta f = \frac{1}{A B} \left[ \partial_x \left( B A^{-1} \partial_x f \right) + \partial_y \left( A B^{-1} \partial_y f \right) \right],
\]

In our setting, \(A = \frac{1}{\mu y}\) and \(B = \frac{1}{\lambda y}\):

\[
    \Delta f = y^2 \left(\mu^2 \frac{\partial^2 f}{\partial x^2} + \lambda^2 \frac{\partial^2 f}{\partial y^2}\right)
\]

And for the function \(f = - \frac{x}{y}\), we have

\[
    \Delta f = - \frac{2 \lambda^2 x}{y} = 2 \lambda^2 f
\]

So, we reach the conclusion that the function \(f = - \frac{x}{y}\) is a eigenfunction of the Laplacian with eigenvalue \(2 \lambda^2\).