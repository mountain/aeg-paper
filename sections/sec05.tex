\subsection{Global arithmetic torsion and the accumulative commutative space}
\label{sec:global_torsion_acs_narrative_enhanced_lc}

The non-commutative nature of fundamental arithmetic operations, such as addition and multiplication, is the very source of arithmetic torsion. Consider the local effect: evaluating an expression like $x \oplus_\mu \otimes_\lambda$ versus $x \otimes_\lambda \oplus_\mu$. While both "path segments" originate from the same initial value $x$, their terminal points in the arithmetic expression space will differ if a non-zero local torsion $\tau_{\text{local}} = \nu(x \oplus_\mu \otimes_\lambda) - \nu(x \otimes_\lambda \oplus_\mu)$ exists. These two segments do not naturally form a closed loop; instead, they represent a "gap" or a "tear" caused by the order of operations. This inherent "openness" or "torn state" at the local level signifies a deviation from commutativity and poses a challenge for directly applying integral theorems like Green's or Stokes', which typically rely on closed paths or well-defined bounded regions within the space of action.

The crucial question then arises: how do these elemental "tears" accumulate along an extended, arbitrary arithmetic path $\gamma$? How can we quantify the total geometric impact of these accumulated deviations? This is the heart of the "torsion-area problem" which we previously highlighted with specific examples in the $\mathfrak{E}_1$ space (cf. Material 4, Section~\ref{subsec:gridsandtorsion}) and the local differential formulation $d\tau_{\text{local}} = \mu \lambda du dv$. We seek a general principle to measure this overall "deviation from commutativity" for any path $\gamma$.

Let us reflect further on the meaning of torsion. Global arithmetic torsion, which we will define algebraically as $\tau_{\text{alg}}(\gamma) = \nu(\gamma) - \nu(\bar{\gamma})$ (the difference between evaluating a path and its reversed-sequence counterpart), is precisely a measure of this accumulated "torn state" across the entire path. It quantifies how far the sequence of operations, taken as a whole, has strayed from a kind of "effective commutativity" that would see $\nu(\gamma)$ and $\nu(\bar{\gamma})$ be equal. This persistent difference, this "tear" magnified globally, intrinsically suggests the need for a different kind of space—a "commutative space"—wherein these path-dependent discrepancies can be consistently charted and their collective magnitude assessed. We need a framework where the comparison between $\gamma$ and $\bar{\gamma}$ can be geometrically realized by forming a closed boundary.

Thus, the \textbf{accumulative commutative space} (ACS) is born. This space, constructed by the \textit{commutative accumulation} of operational parameters ($A_\gamma = \sum \mu_k$ and $M_\gamma = \sum \lambda_k$) from a path $\gamma$, provides exactly such a commutative canvas. Critically, in this ACS, any path $\gamma$ and its reversed-sequence counterpart $\bar{\gamma}$ share common start $(0,0)$ and end points $(A_\gamma, M_\gamma)$. This allows them to genuinely form the boundary $\partial\Sigma_\gamma$ of a well-defined planar region $\Sigma_\gamma$. It is upon this stage that we can rigorously explore and establish the geometric nature of global arithmetic torsion.

\subsubsection*{Algebraic definition of global arithmetic torsion $\tau(\gamma)$: comparison via path reversal}

Let $\gamma$ be an arbitrary arithmetic path, representing an ordered sequence of operations $op_n \circ \dots \circ op_1$ acting on an initial value $x$. The local non-commutative effects accumulate along this path. To quantify the total, accumulated arithmetic torsion for the entire path $\gamma$, we compare its standard evaluation $\nu(\gamma)$ with the evaluation of its \textbf{reversed path, $\bar{\gamma}$}. The reversed path $\bar{\gamma}$ is defined by applying the \textit{literal sequence of operations} from $\gamma$ in the exact reverse order ($op_1 \circ op_2 \circ \dots \circ op_n$) to the same initial value $x$.

The \textbf{global arithmetic torsion, $\tau(\gamma)$}, is then defined through its \textbf{algebraic evaluation form}:
\begin{equation}
\tau_{\text{alg}}(\gamma) = \nu(\gamma) - \nu(\bar{\gamma})
\label{eq:T_alg_formal_final_narrative_enhanced_acs_AM_lc}
\end{equation}
A crucial property, verified for the operations $\oplus_\mu$ and $\otimes_\lambda$, is that $\tau_{\text{alg}}(\gamma)$ is independent of the initial value $x$, making it an intrinsic characteristic of the operational structure of path $\gamma$. The evaluations $\nu(\gamma)$ and $\nu(\bar{\gamma})$ occur within the \textbf{arithmetic expression space} (which may be modeled by, e.g., the hyperbolic $\mathfrak{E}_1$ space or more generally over $\mathbb{R}$ or $\mathbb{C}$).

\subsubsection*{Geometric representation of paths in the accumulative commutative space and the region $\Sigma_\gamma$}

The accumulative commutative space (ACS), whose conceptual motivation was just outlined, is a Euclidean plane. For any given arithmetic path $\gamma$, its representation in this space is constructed by accumulating the parameters of its constituent operations:
\begin{itemize}
    \item $A_\gamma = \sum \mu_k$: The total accumulated additive charge from all $\oplus_{\mu_k}$ operations in $\gamma$.
    \item $M_\gamma = \sum \lambda_k$: The total accumulated logarithmic multiplicative charge from all $\otimes_{\lambda_k}$ operations in $\gamma$.
\end{itemize}
When the sequence of operation parameters defining path $\gamma$ is plotted as a trajectory in this ACS, it charts a path starting from the origin $(0,0)$ to a final point $(A_\gamma, M_\gamma)$. Critically, the reversed path $\bar{\gamma}$, having the same set of operations, also maps from $(0,0)$ to the \textit{same} endpoint $(A_\gamma, M_\gamma)$, though typically via a different trajectory in the $(A,M)$ plane (which are the coordinates of the ACS). These two trajectories---one for $\gamma$ and one for $\bar{\gamma}$---naturally enclose a two-dimensional region, which we denote as $\Sigma_\gamma$.

\subsubsection*{Core conclusion: the triple identity of global arithmetic torsion}

Within this framework, our central finding, verified through numerous examples, is that the global arithmetic torsion $\tau(\gamma)$ possesses three equivalent formulations. This \textbf{triple identity} bridges its algebraic definition with concrete geometric measures in the accumulative commutative space:

\textbf{(I) Algebraic evaluation form} (as defined in Eq. \ref{eq:T_alg_formal_final_narrative_enhanced_acs_AM_lc}):
\[ \tau(\gamma) = \nu(\gamma) - \nu(\bar{\gamma}) \]

\textbf{(II) Geometric interior integral form in the accumulative commutative space:}
The torsion is precisely equal to the integral of the 2-form $e^M dM \wedge dA$ over the region $\Sigma_\gamma$:
\[ \tau(\gamma) = \iint_{\Sigma_\gamma} e^M dM \wedge dA \]
(The order $dM \wedge dA$ versus $dA \wedge dM$ depends on the chosen orientation of $\Sigma_\gamma$; the kernel $e^M$ reflects the exponential scaling inherent in multiplicative operations).

\textbf{(III) Geometric boundary integral form in the accumulative commutative space:}
By Green's (or Stokes') Theorem, this area integral is equivalent to a line integral of the 1-form $\omega = e^M dA$ over the oriented boundary $\partial \Sigma_\gamma$ of the region $\Sigma_\gamma$. If $\partial \Sigma_\gamma$ is oriented as traversing $\bar{\gamma}$ and then $-\gamma$ (the reverse of path $\gamma$):
\[ \tau(\gamma) = \oint_{\partial \Sigma_\gamma} e^M dA \]
In its discrete summation form, for paths composed of $\oplus_{\mu_k}$ and $\otimes_{\lambda_k}$ operations, this becomes:
\[ \tau(\gamma) = \left(\sum_{(\mu_k, M_k) \in \bar{\gamma}} \mu_k e^{M_k}\right) - \left(\sum_{(\mu_j, M_j) \in \gamma} \mu_j e^{M_j}\right) \]
where the sum is over the additive steps $\mu$ in each path, and $M$ is the accumulated logarithmic multiplicative charge at the point of that addition.

In summary, our main theorem states the equality of these three forms:
\begin{equation}
\tau_{\text{alg}}(\gamma) = \tau_{\text{int}}(\gamma) = \tau_{\text{bound}}(\gamma)
\label{eq:triple_identity_final_narrative_enhanced_acs_AM_lc}
\end{equation}

The establishment of this triple identity, particularly the equivalence $\tau_{\text{alg}}(\gamma) = \iint_{\Sigma_\gamma} e^M dM \wedge dA$, provides a definitive affirmative answer to our initial "torsion-area problem": global arithmetic torsion for an arbitrary path $\gamma$ can indeed be precisely quantified as an $e^M$-weighted area within the accumulative commutative space. This lends a concrete geometric meaning to an otherwise abstract algebraic difference, distinct from, yet inspired by, the initial observations in $\mathfrak{E}_1$.

While the accumulative commutative space (ACS), with its specific accumulation rules for $A$ and $M$, has proven effective for this geometric quantification, its introduction might initially appear as a purpose-built construct. However, the significance of this parameter space is greatly amplified by an independent algebraic perspective. As our broader research reveals (and as will be discussed further), this very $(A,M)$ coordinate system (which defines the ACS) also emerges naturally as the target space for group homomorphisms from $F_2$ (the free group of arithmetic operations) and related groups such as $BS(m,n)$. These homomorphisms map operational sequences to their net $(A,M)$ "charges." This dual role of the ACS—both as a geometric stage for quantifying torsion and as an algebraic target for characterizing process structure—makes our theoretical narrative more cohesive and interesting, highlighting the ACS as a truly pivotal element in the emerging geometry of arithmetic expressions.

\subsubsection*{Significance and interpretation: a Stokes-like theorem for arithmetic expressions}

This \textbf{triple identity} (Eq. \ref{eq:triple_identity_final_narrative_enhanced_acs_AM_lc}) is a cornerstone of our current understanding of global arithmetic torsion. It establishes a precise and verified equivalence between:
\begin{itemize}
    \item A purely algebraic quantity ($\tau_{\text{alg}}(\gamma)$) derived from evaluation differences within the (potentially non-Euclidean) arithmetic expression space (such as $\mathfrak{E}_1$).
    \item A geometric interior integral ($\tau_{\text{int}}(\gamma)$) representing an $e^M$-weighted area in the flat Euclidean accumulative commutative space.
    \item An equivalent geometric boundary integral ($\tau_{\text{bound}}(\gamma)$) also in the accumulative commutative space.
\end{itemize}
The dual-space model---distinguishing the arithmetic expression space (where operations are realized and $\nu(\gamma)$ is determined) from the accumulative commutative space (where torsion is geometrically quantified via parameter accumulation)---is pivotal. The $e^M$ kernel serves as the crucial bridge, translating the multiplicative scaling dynamics inherent in arithmetic operations (and potentially manifest in the arithmetic expression space) into the weighting of geometric elements in the accumulative commutative space.

This framework provides a concrete ``Stokes-like theorem for global arithmetic torsion,'' relating a global algebraic effect (the total torsion) to the integral of a local geometric density (in the accumulative commutative space) or a boundary sum. It forms a solid foundation for further investigations, including relating this accumulative commutative space integral back to potential curvature integrals in spaces like $\mathfrak{E}_1$, thereby advancing towards a more comprehensive arithmetic Gauss-Bonnet type theorem.

\subsection{$F_2$ homomorphisms and ideal-like structures}
\label{sec:acs_algebraic_significance_revised_AM_lc}

In the preceding section, the accumulative commutative space (ACS) was introduced as a crucial geometric stage. Its commutative nature—where coordinates $(A_\gamma, M_\gamma)$ are formed by the simple summation of operational parameters, irrespective of their order in a path $\gamma$—allowed any path $\gamma$ and its reversed-sequence counterpart $\bar{\gamma}$ to share common start and end points. This was instrumental in defining a bounded region $\Sigma_\gamma$, thereby enabling the geometric quantification of global arithmetic torsion $\tau(\gamma)$ as an $e^M$-weighted area. Having established its utility as a "commutative canvas" for geometric measurement, we now turn to explore the deeper algebraic structures associated with this ACS and its connection to the fundamental generative structure of arithmetic paths.

Arithmetic paths, in their most abstract and unconstrained form, are generated by a sequence of elementary operations. Let us consider two fundamental types of abstract generative processes: an "additive process" $X_A$, which intends to shift a value by a basic unit $\mu_0$, and a "logarithmic multiplicative process" $X_M$, which intends to scale a value by a factor $e^{\lambda_0}$ (thus shifting its logarithm by $\lambda_0$). The set of all possible finite sequences composed of these processes and their inverses ($X_A^{-1}$ for subtraction by $\mu_0$, $X_M^{-1}$ for division by $e^{\lambda_0}$ or scaling by $e^{-\lambda_0}$) forms the free group on two generators, $F_2 = \langle X_A, X_M \rangle$. It is crucial to note that at this stage, we are considering $F_2$ purely as a group of symbolic operational sequences or "generative structures." We are not yet concerned with the arithmetic evaluation $\nu(\gamma)$ of a path $\gamma \in F_2$ to a specific numerical outcome, but rather with the algebraic properties inherent in these sequences themselves as revealed by their cumulative operational parameters.

This perspective leads us very naturally to define a group homomorphism from the (generally non-commutative) free group $F_2$ to an abelian group representing the accumulated parameters in the accumulative commutative space. Let $\mathbb{A}_{(A,M)}$ denote this target abelian group (which defines the ACS), typically $\mathbb{Z}\mu_0 \times \mathbb{Z}\lambda_0$ if $\mu_0, \lambda_0$ are considered indivisible units associated with single applications of $X_A$ and $X_M$ respectively (or $\mathbb{R}^2$ for continuous parameters). We define the homomorphism $\Phi: F_2 \to \mathbb{A}_{(A,M)}$ by its action on the generators:
\begin{itemize}
    \item $\Phi(X_A) = (\mu_0, 0)$
    \item $\Phi(X_M) = (0, \lambda_0)$
    \item $\Phi(X_A^{-1}) = (-\mu_0, 0)$
    \item $\Phi(X_M^{-1}) = (0, -\lambda_0)$
\end{itemize}
For any path $\gamma = g_1 g_2 \dots g_k \in F_2$ (where $g_i \in \{X_A, X_M, X_A^{-1}, X_M^{-1}\}$), its image is $\Phi(\gamma) = \sum_{i=1}^k \Phi(g_i) = (A_\gamma, M_\gamma)$, where $A_\gamma$ is the total accumulated "additive charge" and $M_\gamma$ is the total accumulated "logarithmic multiplicative charge". This map $\Phi$ is indeed a group homomorphism because composition of paths in $F_2$ corresponds to vector addition of their $(A,M)$ images in the abelian group $\mathbb{A}_{(A,M)}$. This homomorphism effectively translates the non-commutative compositional structure of $F_2$ paths into the simpler, commutative world of their net operational "footprints" $(A_\gamma, M_\gamma)$ within the ACS.

We can further consider the component homomorphisms:
\begin{itemize}
    \item $\phi_A: F_2 \to \mathbb{Z}\mu_0$ mapping $\gamma \mapsto A_\gamma$.
    \item $\phi_M: F_2 \to \mathbb{Z}\lambda_0$ mapping $\gamma \mapsto M_\gamma$.
\end{itemize}
These are also group homomorphisms. A key insight arises when we consider the preimages of "ideal-like" structures (specifically, subgroups of the form $k\mathbb{Z}\mu_0$ or $k\mathbb{Z}\lambda_0$) from the target abelian groups back into $F_2$. For any integer $k$, let:
\begin{itemize}
    \item $K_{(k),A} = \phi_A^{-1}(k\mathbb{Z}\mu_0) = \{ \gamma \in F_2 \mid A_\gamma \text{ is an integer multiple of } k\mu_0 \}$
    \item $K_{(k),M} = \phi_M^{-1}(k\mathbb{Z}\lambda_0) = \{ \gamma \in F_2 \mid M_\gamma \text{ is an integer multiple of } k\lambda_0 \}$
\end{itemize}
Since $\phi_A$ and $\phi_M$ are group homomorphisms and their target groups are abelian (making all subgroups normal), $K_{(k),A}$ and $K_{(k),M}$ are normal subgroups of $F_2$. For instance, $K_{(k),A}$ comprises all operational sequences whose net additive effect (in terms of $\mu_0$ units) is "trivial modulo $k$." These subgroups thus impose an arithmetic classification upon the abstract generative paths in $F_2$.

Remarkably, the lattice structure of these normal subgroups in $F_2$ (ordered by inclusion) is perfectly compatible with the lattice structure of ideals in $\mathbb{Z}$ and, consequently, with the nesting of closed sets in the Zariski topology on $\text{Spec } \mathbb{Z}$. For example, considering the $A$-component (and assuming $\mu_0=1$ for simplicity, so $\phi_A: F_2 \to \mathbb{Z}$):
\[ K_{(n),A} \subseteq K_{(d),A} \iff n\mathbb{Z} \subseteq d\mathbb{Z} \iff d|n \]
And similarly for the $M$-component. This corresponds directly to the nesting of Zariski closed sets $V(k\mathbb{Z})$ (where $V(\cdot)$ here denotes the variety of an ideal in algebraic geometry):
\[ V(d\mathbb{Z}) \subseteq V(n\mathbb{Z}) \iff d|n \]
Thus, we have the chain:
\[ K_{(n),A} \subseteq K_{(d),A} \iff d|n \iff V(d\mathbb{Z}) \subseteq V(n\mathbb{Z}) \]
This demonstrates a profound structural correspondence: the way these specific normal subgroups $K_{(k),A}$ (or $K_{(k),M}$) are organized within $F_2$ faithfully mirrors a fundamental organizing principle of the Zariski topology on $\text{Spec } \mathbb{Z}$.

In this manner, the accumulative commutative space (ACS), initially introduced as a geometric tool for understanding global torsion as an area, reveals its deeper algebraic significance. It serves as a "character space" for $F_2$, where abstract operational sequences are mapped to quantitative "charges" $(A_\gamma, M_\gamma)$. These charges, in turn, allow us to define ideal-like normal subgroups within $F_2$ that resonate with classical algebraic and topological structures found in number theory and algebraic geometry. This uncovers an intrinsic arithmetic order within the purely generative framework of $F_2$, even before any notion of arithmetic evaluation $\nu(\gamma)$ is applied. This understanding forms a crucial foundation as we later consider how these structures behave when $F_2$ is "condensed" by imposing relations to form other groups, such as the Baumslag-Solitar groups.

