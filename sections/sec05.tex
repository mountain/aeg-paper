\subsection{Discrete total arithmetic torsion calculations}\label{subsec:discrete-calculations}

The foundational operations defining threadlike arithmetic expressions, namely addition (denoted $\addop{\mu}$) and multiplication (denoted $\multop{\lambda}$, representing multiplication by $e^{\lambda}$), exhibit a fundamental non-commutativity. This non-commutativity is the origin of \emph{arithmetic torsion}, a measure of the discrepancy arising from different sequential orderings of these operations. Our goal in this subsection is to develop a systematic, purely discrete method for quantifying the accumulated effect of this non-commutativity along a given threadlike expression path, and to relate this accumulated torsion directly to differences in the final numerical evaluation outcomes.

We begin by considering the simplest instance of non-commutativity: swapping a single additive step $\addop{\mu}$ and a single multiplicative step $\multop{\lambda}$. The difference in outcome, often referred to as the \emph{local torsion} or the commutator effect (relative to a starting value $x$), is given by:
\begin{equation}
\tau_{\text{local}}(x; \mu, \lambda) = (x \addop{\mu} \multop{\lambda}) - (x \multop{\lambda} \addop{\mu}) = ( (x+\mu)e^{\lambda} ) - ( xe^{\lambda}+\mu ) = \mu(e^{\lambda} - 1).
\label{eq:local_torsion}
\end{equation}
This quantity $\mu(e^{\lambda}-1)$ represents the elementary "error" or difference generated by exchanging the order of one adjacent pair of operations.

For longer expression paths involving multiple additions and multiplications, the total accumulated effect of non-commutativity is not merely the sum of local torsions from adjacent swaps. Crucially, the effect of swapping an operation $\addop{\mu_i}$ with a \emph{subsequent} operation $\multop{\lambda_j}$ ($j>i$) is influenced by all the intermediate multiplicative operations $\multop{\lambda_k}$ ($i<k<j$). These intermediate multiplications effectively \emph{amplify} the scale at which the non-commutativity between $\addop{\mu_i}$ and $\multop{\lambda_j}$ manifests. This amplification phenomenon is consistent with, and can be understood through, the analysis of perturbation propagation along such expression paths (as discussed in Section~\ref{subsec:alternating} and subsequent analyses).

To systematically quantify the total effect, we calculate the contribution of each additive step $\addop{\mu_i}$ to the overall discrepancy between two canonical evaluation orders. Let $\gamma$ represent a specific evaluation order (e.g., a stepwise multiply-then-add sequence like $\alpha = (\multop{\lambda_1} \addop{\mu_1}) \dots (\multop{\lambda_l} \addop{\mu_l})$) and let $\bar{\gamma}$ represent a reference order (e.g., the global add-then-multiply sequence corresponding to $(x+\sum \mu_k) \exp(\sum \lambda_k)$, or the global multiply-then-add sequence $x \exp(\sum \lambda_k) + \sum \mu_k$). The contribution $\tau_i$ of a single step $\addop{\mu_i}$ to the total difference $E(\gamma) - E(\bar{\gamma})$ arises from its interaction with all subsequent multiplicative steps $\multop{\lambda_k}$ for $k > i$. As derived through algebraic manipulation involving telescoping sums (detailed, for instance, in the discussions leading to this summary), this contribution is precisely:
\begin{equation}
\tau_i = \mu_i \left( \exp\left(\sum_{k=i+1}^{l} \lambda_k\right) - 1 \right) = \mu_i (e^{\Lambda_{>i}} - 1),
\label{eq:tau_i_contribution}
\end{equation}
where we define the sum of subsequent logarithmic scaling factors as $\Lambda_{>i} := \sum_{k=i+1}^{l} \lambda_k$. The term $e^{\Lambda_{>i}}$ represents the total scaling factor applied to $\mu_i$ due to all subsequent multiplications in one evaluation order ($\gamma$), while the $-1$ term reflects its baseline contribution (scaled by $e^0=1$) in the reference order ($\bar{\gamma}$) where it is added without prior scaling by subsequent $\lambda_k$.

The \emph{total discrete accumulated torsion} $\tau_{sum}$ for the path $\alpha$ is then defined as the sum of the contributions from all additive steps:
\begin{equation}
\tau_{sum}(\alpha) = \sum_{i=1}^{l} \tau_i = \sum_{i=1}^{l} \mu_i (e^{\Lambda_{>i}} - 1).
\label{eq:tau_sum_discrete}
\end{equation}
While this formula was initially derived considering paths with a specific structure (e.g., $(\otimes \oplus)^l$), numerical verification on examples (such as $x \oplus_1^3 \otimes_{\ln 2}^3$) suggests that the formula and its underlying logic remain valid for other structures (like $\oplus^n \otimes^m$) provided $\Lambda_{>i}$ is correctly interpreted as the sum of exponents of \emph{all} multiplicative steps occurring after the $i$-th additive step along the path.

The principal result of this discrete analysis is the direct identification of this algebraically computed sum $\tau_{sum}$ with the difference in the final numerical outcomes between the chosen evaluation orders $\gamma$ and $\bar{\gamma}$:
\begin{equation}
\nu(\gamma) - \nu(\bar{\gamma}) = \tau_{sum}(\alpha) = \sum_{i=1}^{l} \mu_i (e^{\Lambda_{>i}} - 1)
\label{eq:discrete_torsion_evaluation_equivalence}
\end{equation}
(The sign depends on the precise definition of $\gamma$ and $\bar{\gamma}$; the magnitude of the difference is given by $|\tau_{sum}|$). This equality has been consistently verified in concrete examples, such as comparing the evaluation of $(x+3)\times 8$ (as $E(\gamma)$) and $x\times 8 + 3$ (as $E(\bar{\gamma})$), where the difference $21$ exactly matches the value computed by the $\tau_{sum}$ formula.

\subsubsection*{Example 1: Additions followed by Multiplications}
Consider the expression $E_1 = ((x+1)+1+1) \times 2 \times 2 \times 2$, corresponding to the path $\alpha_1: x \addop{1} \addop{1} \addop{1} \multop{\ln 2} \multop{\ln 2} \multop{\ln 2}$.
\begin{itemize}
    \item Additive steps: $\mu_1=1, \mu_2=1, \mu_3=1$.
    \item Multiplicative steps: $\lambda_1=\ln 2, \lambda_2=\ln 2, \lambda_3=\ln 2$.
    \item For $\mu_1=1$ (first addition): Subsequent multiplications are $\lambda_1, \lambda_2, \lambda_3$. $\Lambda_{>1} = \ln 2 + \ln 2 + \ln 2 = 3\ln 2 = \ln 8$. Contribution $\tau_1 = 1(e^{\ln 8}-1) = 7$.
    \item For $\mu_2=1$ (second addition): Subsequent multiplications are $\lambda_1, \lambda_2, \lambda_3$. $\Lambda_{>2} = \ln 8$. Contribution $\tau_2 = 1(e^{\ln 8}-1) = 7$.
    \item For $\mu_3=1$ (third addition): Subsequent multiplications are $\lambda_1, \lambda_2, \lambda_3$. $\Lambda_{>3} = \ln 8$. Contribution $\tau_3 = 1(e^{\ln 8}-1) = 7$.
    \item Total discrete torsion: $\tau_{sum}(\alpha_1) = \tau_1 + \tau_2 + \tau_3 = 7+7+7 = 21$.
    \item Verification: Let $\gamma$ be the evaluation order corresponding to $(x+1+1+1) \times 8$, so $\nu({\gamma}) = (x+3) \times 8 = 8x+24$. Let $\bar{\gamma}$ be the order corresponding to $(x \times 8) + 1+1+1$, so $\nu({\bar{\gamma}}) = 8x+3$. The difference is $\nu({\gamma}) - \nu({\bar{\gamma}}) = (8x+24) - (8x+3) = 21$. This matches $\tau_{sum}(\alpha_1)$.
\end{itemize}

\subsubsection*{Example 2: Alternating Operations}
Consider the expression $E_2 = (((x \times 2) + 3) \times 4) + 5$, corresponding to the path $\alpha_2: x \multop{\ln 2} \addop{3} \multop{\ln 4} \addop{5}$.
\begin{itemize}
    \item Additive steps: $\mu_1=3$ (occurring after $\lambda_1$), $\mu_2=5$ (occurring after $\lambda_2$).
    \item Multiplicative steps: $\lambda_1=\ln 2, \lambda_2=\ln 4$.
    \item For $\mu_1=3$: The subsequent multiplication is $\lambda_2=\ln 4$. $\Lambda_{>1} = \ln 4$. Contribution $\tau_1 = 3(e^{\ln 4}-1) = 3(4-1)=9$.
    \item For $\mu_2=5$: There are no subsequent multiplications. $\Lambda_{>2} = 0$. Contribution $\tau_2 = 5(e^{0}-1) = 5(1-1)=0$.
    \item Total discrete torsion: $\tau_{sum}(\alpha_2) = \tau_1 + \tau_2 = 9+0 = 9$.
    \item Verification: Let $\gamma$ be the natural evaluation order $\nu({\gamma}) = 8x+17$ (as calculated previously). Let $\bar{\gamma}$ be the order where all multiplications happen first, then all additions: $\nu({\bar{\gamma}}) = (x \times 2 \times 4) + 3 + 5 = 8x+8$. The difference is $\nu({\gamma}) - \nu({\bar{\gamma}}) = (8x+17) - (8x+8) = 9$. This matches $\tau_{sum}(\alpha_2)$.
\end{itemize}

These examples illustrate the application of the discrete sum formula \eqref{eq:tau_sum_discrete} and confirm its consistency with the evaluation difference \eqref{eq:discrete_torsion_evaluation_equivalence} for different path structures. This establishes a solid, purely discrete framework for quantifying accumulated arithmetic torsion, independent of continuous geometric interpretations or integration techniques, and serves as a basis for further investigation.

This establishes a robust, purely discrete framework for quantifying accumulated arithmetic torsion based on the operational sequence of additions and multiplications. It provides a solid foundation, independent of continuous geometric interpretations or integration techniques, upon which further investigations into the geometric implications (like area correspondence via $\iint e^v du dv$) and the structure of a full Arithmetic Gauss--Bonnet theorem can be built.


\subsection{From discrete to continuous}\label{subsec:discrete-to-continuous}



