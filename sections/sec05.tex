\subsection{Global Arithmetic Torsion and its Threefold Geometric Equivalence}
\label{sec:global_torsion_formal}

\subsubsection*{From Local to Global: Quantifying the Overall Non-Commutative Effect in Arithmetic Paths}

Prior discussions (e.g., Section 2 of the foundational materials) have illuminated the local non-commutative nature of arithmetic operations, particularly addition ($\oplus_\mu: x \mapsto x+\mu$) and exponential multiplication ($\otimes_\lambda: x \mapsto x \cdot e^\lambda$). For instance, a local arithmetic torsion $\tau = \mu(e^\lambda-1)$ was identified for the elementary swap $x \oplus_\mu \otimes_\lambda - x \otimes_\lambda \oplus_\mu$. Building upon this, our recent work has focused on defining and understanding a measure for the \textbf{total, accumulated} arithmetic torsion for an entire, arbitrary arithmetic path $S$. This global measure aims to capture the overall impact of all non-commutative interactions within the sequence of operations defining $S$.

\subsubsection*{Algebraic Definition of Global Arithmetic Torsion $\mathcal{T}(S)$: Comparison via Path Reversal}

To quantify this global effect for a path $S$ (an ordered sequence of operations $op_n \circ \dots \circ op_1$ acting on an initial value $x$), we compare its standard evaluation $\nu(S)$ with the evaluation of its \textbf{Reversed Path, $S_{\text{rev}}$}. The Reversed Path $S_{\text{rev}}$ is defined by applying the \textit{literal sequence of operations} from $S$ in the exact reverse order ($op_1 \circ op_2 \circ \dots \circ op_n$) to the same initial value $x$.

The \textbf{Global Arithmetic Torsion, $\mathcal{T}(S)$}, is then defined through its \textbf{Algebraic Evaluation Form}:
\begin{equation}
\mathcal{T}_{\text{alg}}(S) = \nu(S) - \nu(S_{\text{rev}})
\label{eq:T_alg_formal}
\end{equation}
A crucial property, verified for the operations $\oplus_\mu$ and $\otimes_\lambda$, is that $\mathcal{T}_{\text{alg}}(S)$ is independent of the initial value $x$, making it an intrinsic characteristic of the operational structure of path $S$. The evaluations $\nu(S)$ and $\nu(S_{\text{rev}})$ occur within the \textbf{Arithmetic Expression Space} (which may be modeled by, e.g., the hyperbolic $\mathfrak{E}_1$ space or more generally over $\mathbb{R}$ or $\mathbb{C}$).

\subsubsection*{The Geometric Stage: The $(U,V)$ Reference Space and the Region $\Sigma_S$}

For a geometric interpretation of $\mathcal{T}_{\text{alg}}(S)$, we introduce an auxiliary Euclidean plane termed the \textbf{Reference Space} (or $(U,V)$ Parameter Space). Its coordinates are formed by accumulating the parameters of the operations within a path $S$:
\begin{itemize}
    \item $U = \sum \mu_k$: The total additive displacement, from all $\oplus_{\mu_k}$ operations in $S$.
    \item $V = \sum \lambda_k$: The total logarithmic multiplicative displacement, from all $\otimes_{\lambda_k}$ operations in $S$.
\end{itemize}
When the sequence of operation parameters defining path $S$ is plotted as a trajectory in this Reference Space, it charts a path from the origin $(0,0)$ to a final point $(U_{\text{total}}, V_{\text{total}})$. Critically, the Reversed Path $S_{\text{rev}}$, having the same set of operations, also maps from $(0,0)$ to the \textit{same} endpoint $(U_{\text{total}}, V_{\text{total}})$, though typically via a different trajectory. These two trajectories---one for $S$ and one for $S_{\text{rev}}$---naturally enclose a two-dimensional region $\Sigma_S$ in the flat $(U,V)$ Reference Space.

\subsubsection*{The Core Conclusion: The Triple Identity of Global Arithmetic Torsion}

Our central finding, verified through numerous examples involving sequences of $\oplus_1$ and $\otimes_{\ln 2}$ operations of varying lengths (3, 4, and 5 steps), is that the Global Arithmetic Torsion $\mathcal{T}(S)$ possesses three equivalent formulations:

\textbf{(I) Algebraic Evaluation Form} (as defined in Eq. \ref{eq:T_alg_formal}):
\[ \mathcal{T}(S) = \nu(S) - \nu(S_{\text{rev}}) \]

\textbf{(II) Geometric Interior Integral Form in the Reference Space:}
The torsion is equal to the integral of the 2-form $e^V dV \wedge dU$ over the region $\Sigma_S$:
\[ \mathcal{T}(S) = \iint_{\Sigma_S} e^V dV \wedge dU \]
(The order $dV \wedge dU$ versus $dU \wedge dV$ depends on the chosen orientation of $\Sigma_S$; the kernel is $e^V$). This form quantifies $\mathcal{T}(S)$ as an $e^V$-weighted area in the Reference Space. The factor $e^V$ reflects the exponential scaling effect inherent in the multiplicative operations, a theme also seen in discussions of perturbation propagation (e.g., Section 2.2 of the foundational materials).

\textbf{(III) Geometric Boundary Integral Form in the Reference Space:}
By application of Green's (or Stokes') Theorem, the area integral above is equivalent to a line integral of the 1-form $\omega = e^V dU$ over the oriented boundary $\partial \Sigma_S'$ of $\Sigma_S$. If $\partial \Sigma_S'$ is oriented as traversing $S_{\text{rev}}$ and then $-S$ (the reverse of path $S$):
\[ \mathcal{T}(S) = \oint_{\partial \Sigma_S'} e^V dU \]
In its discrete summation form, for paths composed of $\oplus_{\mu_k}$ and $\otimes_{\lambda_k}$ operations, this becomes:
\[ \mathcal{T}(S) = \left(\sum_{(\mu_k, V_k) \in S_{\text{rev}}} \mu_k e^{V_k}\right) - \left(\sum_{(\mu_j, V_j) \in S} \mu_j e^{V_j}\right) \]
where the sum is over the additive steps $\mu$ in each path, and $V$ is the accumulated logarithmic multiplicative parameter at the point of that addition.

In summary, our main theorem states the equality of these three forms:
\begin{equation}
\mathcal{T}_{\text{alg}}(S) = \mathcal{T}_{\text{int}}(S) = \mathcal{T}_{\text{bound}}(S)
\label{eq:triple_identity_formal}
\end{equation}

\subsection*{Significance and Interpretation: A Stokes-like Theorem for Arithmetic Expressions}

This \textbf{Triple Identity} (Eq. \ref{eq:triple_identity_formal}) is a cornerstone of our current understanding of arithmetic torsion. It establishes a precise and verified equivalence between:
\begin{itemize}
    \item A purely algebraic quantity ($\mathcal{T}_{\text{alg}}(S)$) derived from evaluation differences within the (potentially non-Euclidean) Arithmetic Expression Space.
    \item A geometric interior integral ($\mathcal{T}_{\text{int}}(S)$) representing an $e^V$-weighted area in the flat Euclidean Reference Space.
    \item An equivalent geometric boundary integral ($\mathcal{T}_{\text{bound}}(S)$) also in the Reference Space.
\end{itemize}
The dual-space model---distinguishing the Arithmetic Expression Space (for non-commutative evaluation) from the $(U,V)$ Reference Space (for geometric quantification of torsion via parameter accumulation)---is pivotal. The $e^V$ kernel serves as the crucial bridge, translating the multiplicative scaling dynamics of the Arithmetic Expression Space into the weighting of geometric elements in the Reference Space.

This framework provides a concrete ``Stokes-like theorem for Global Arithmetic Torsion,'' relating a global algebraic effect to the integral of a local geometric density (or a boundary sum). It forms a solid foundation for further investigations into a more comprehensive Arithmetic Gauss-Bonnet type theorem and the deeper geometric structures underlying arithmetic expressions.

