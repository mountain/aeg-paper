% \usepackage{amsmath}

\subsection{Flow equation}\label{sec:equation}

Consider an infinitesimal generating process on a Riemannian surface $M$ using two generators:
one for an additional action $\mu$ and the other for a multiplicative action $e^\lambda$.
These two generators are perpendicular.
This generation process produces an assignment $A: M \to R$ over the surface.

For any point with an assignment $a_0$, if we consider a movement of distance $\epsilon$ in a direction with angle $\theta$
over a time period of $\delta$, we can establish the following:

\[
    a_{\delta} = (a_0 + \mu \epsilon \cos \theta)e^{\lambda \epsilon \sin \theta}
\]

or

\[
    a_{\delta} = a_0 e^{\lambda \epsilon \sin \theta} + \mu \epsilon \cos \theta
\]

Both formula can be simplified to the same result:

\[
    a_{\delta} = a_0 + \epsilon (a_0 \lambda \sin \theta + \mu \cos \theta)
\]

Then, we have the following equation:

\[
    \frac{1}{\delta} (a_{\delta} - a_0) = \frac{\epsilon}{\delta} (\mu \cos \theta + x_0 \lambda \sin \theta)
\]

When both $\delta$ and $\epsilon$ are towards zero, we get $da / dt$, and hence

\[
    \frac{da}{dt} = u (\mu \cos \theta + a \lambda \sin \theta)
\]

Or, we can change it to another form

\begin{equation}
    \frac{da}{ds} = \mu \cos \theta + a \lambda \sin \theta\label{eq:flow}
\end{equation}

We name this equation~\eqref{eq:flow} as the flow equation.

The left side of this equation is governed by the distance structure, while the right side is governed by the angle structure.
This leads to below theorem.

\begin{theorem}
Isometries keep the flow equation\eqref{eq:flow}
\label{thm:isometry}
\end{theorem}

\begin{proof}
  An isometry keeps both the distance structure and the angle structure, so it keeps the flow equation.
\end{proof}

We can also get a direct formal solution of the flow equation~\eqref{eq:flow}(details in Appendix~\ref{sec:directformalsolution}).

\begin{equation}
   a = (a_0 + \frac{\mu}{\lambda} \cot \theta) e^{\lambda s \sin \theta} - \frac{\mu}{\lambda} \cot \theta\label{eq:solution}
\end{equation}

While the validity of this solution is still uncertain because we have not assumed any constraints on the local coordinate system,
it is still a useful starting point for further investigation.
For example, we can use this solution to derive (details in Appendix~\ref{sec:conformance}) the relationship between assignments at different vertices in a cell of a mesh grid.

Assuming the assignment at the center point of a cell is $a_0$, we move in different directions and obtain the following assignments:

\begin{itemize}
\item $\theta = 0$: $a_s = a_0 + \mu s$
\item $\theta = \frac{\pi}{2}$: $a_s = a_0 e^{\lambda s}$
\item $\theta = \pi$: $a_s = a_0 - \mu s$
\item $\theta = \frac{3 \pi}{2}$: $a_s = a_0 e^{- \lambda s} $
\end{itemize}

This result is straightforward, but it demonstrates that the infinitesimal generating process is consistent with the discrete mesh grid.

\subsection{The contour-gradient form of flow equation}\label{subsec:the-contour-gradient-form}

It is easy to derive the contour equation in the local coordinate

\begin{equation}
    \mu \cos \theta_c + a \lambda \sin \theta_c = 0\label{eq:contour}
\end{equation}

then we have

\begin{equation}
    \theta_c = - \arctan \frac{\mu}{a \lambda}\label{eq:contourangle}
\end{equation}

the contour and the gradient are perpendicular to each other

\begin{equation}
    \theta_g = \pm \frac{\pi}{2} - \arctan \frac{\mu}{a \lambda}\label{eq:gradientangle}
\end{equation}

then along $\theta_g$ we have

\begin{equation}
    \frac{da}{ds} = \mu \cos (\pm \frac{\pi}{2} - \arctan \frac{\mu}{a \lambda}) + a \lambda \sin (\pm \frac{\pi}{2} - \arctan \frac{\mu}{a \lambda})
    \label{eq:alonggradient}
\end{equation}

\begin{equation}
    \frac{da}{ds} = \pm \sqrt{\mu^2 + \lambda^2 a^2}\label{eq:grad}
\end{equation}

By introducing the right-hand rotation angle $\phi$ along the gradient direction, we can establish a local polar coordinate system based on the gradient and contour lines.
Then the growth rate of $a$ along the angle $\phi$ is

\begin{equation}
    \frac{da}{ds} = \mu \cos (\frac{\pi}{2} - \arctan \frac{\mu}{a \lambda} + \phi) + a \lambda \sin (\frac{\pi}{2} - \arctan \frac{\mu}{a \lambda} + \phi)
    \label{eq:fourfold}
\end{equation}

And the simplified equation is

\begin{equation}
    \frac{da}{ds} = \sqrt {\mu^2 + a^2 \lambda^2} \cos \phi\label{eq:contourgradient}
\end{equation}

The equation~\eqref{eq:contourgradient} is the flow equation in the contour-gradient coordinate system.

Equation~\eqref{eq:contourgradient} is solvable, and we get the relation between $a$ and $s$:

\begin{equation}\label{eq:rel_a_s}
    \tanh(\lambda s \cos \phi - c) = \frac{\lambda a}{\sqrt{\mu^2 + \lambda^2 a^2}}
\end{equation}

we can further simplify the equation to

\begin{equation}
  a = \pm \frac{\mu}{\lambda} \sinh(\lambda s \cos \phi - c)\label{eq:gradevo}
\end{equation}

Under the initial condition $a = a_0$ when $s = 0$, we can get the following equation:

\begin{equation}
    a = \pm \frac{\mu}{\lambda} \sinh(\lambda s \cos \phi - \arcsinh \frac{a_0 \lambda}{\mu})\label{eq:gradevo2}
\end{equation}

In this coordinate system, the additional line and the multiplicative line are:

\begin{equation}
    \phi = \arccos \frac{\mu}{\sqrt {\mu^2 + a^2 \lambda^2}} \label{eq:additionalline}
\end{equation}

\begin{equation}
    \phi = \arcsin \frac{\mu}{\sqrt {\mu^2 + a^2 \lambda^2}}\label {eq:mulitiplcativeline}
\end{equation}

\subsection{The existence theorems}\label{subsec:existence-theorems}

There are two important existence theorems related to the flow equation~\eqref{eq:flow}.

The first existence theorem states that if we have a Riemann surface $M$, then there exists a function $a$ on $M$ that satisfies the flow equation~\eqref{eq:flow}. This theorem is proved in a later section, and we will not go into the details of the proof here.

\begin{theorem}
    Giving a Riemann surface $M$, there exists a function $a$ on $M$ satisfying the flow equation~\eqref{eq:flow}.
    \label{prop:existence1st}
\end{theorem}

The second existence theorem is also crucial. It says that if we have a smooth surface $S$ and a function $a$ on $S$, then we can find a metric $g$ on $S$ that makes $a$ satisfy the flow equation~\eqref{eq:flow}.

\begin{theorem}(By Le Zhang)
    Giving an oriented metrizable smooth surface $S$, and a function $a$ on $S$, there exists a metric $g$ on $S$ that makes $a$
    satisfying the flow equation~\eqref{eq:flow}.
    \label{prop:existence2nd}
\end{theorem}

\begin{proof}
    For an area $U$ of point $p$ on surface $S$, there exists an isothermal coordinate that satisfies the following equation:
    $$ds^2 =e^{2\rho}(du^2 + dv^2)$$
    where $u$ and $v$ are the coordinates of $U$ and $\rho$ is the function of $u, v$ on $U$.
    We have
    $$ \omega_1 = e^{\rho} du \hspace{4em}  \omega_2 = e^{\rho} dv $$
    $$ \mathbf{e}_1 = e^{-\rho} \partial_u \hspace{3em}  \mathbf{e}_2 = e^{-\rho} \partial_v $$
    The gradient of $a$ in the isothermal coordinate is given by
    $$\nabla_a = \frac{\partial a}{\partial u} du + \frac{\partial a}{\partial v} dv$$
    Now we rewrite the flow equation~\eqref{eq:flow} as the following equation:
    $$da = \mu \cos \theta ds + a \lambda \sin \theta ds$$
    hence, we have
    $$e^{2\rho} = \frac{\gamma^2}{\mu^2 + a^2 \lambda^2}$$
     \qedhere
\end{proof}

\subsection{Flow and function}\label{subsec:flow-and-function}

William P. Thurston, a notable mathematician, presented a celebrated example in his work\cite{Thurston1994OnPA}, which demonstrated various approaches to comprehend the mathematical concept of derivative. The next section of this article aims to present novel insight into functions. Namely, the treatment of functions as flows will be discussed.

\begin{definition}\label{def:projection}
Given a function $k$ on the real domain $R$, we can introduce a mapping $l$ on the arithmetic expression space $H$ such that the following diagram commutes.

\begin{center}
    \begin{tikzcd}
        H && H \\
        R && R
        \arrow["l", from=1-1, to=1-3]
        \arrow["\nu"', from=1-1, to=2-1]
        \arrow["\nu", from=1-3, to=2-3]
        \arrow["k"', from=2-1, to=2-3]
    \end{tikzcd}
\end{center}

where $\nu$ is the evaluation function of the expression. Then we call the mapping $l$ is the promotion of the function $k$,
or function $k$ is the projection of the mapping $l$.
\end{definition}


