\subsection{Currying and path notation}\label{subsec:currying}

Currying is a basic technique in functional programming\cite{Reynolds1972DefinitionalIF},
which is used to transform a function with multiple arguments into a sequence of functions with one argument.
By currying a threadlike arithmetic expression, we can obtain a sequence of functions that operate on an operand, which is the leftmost leaf node.

We introduce the following notation for currying a threadlike arithmetic expression:
\begin{itemize}
    \item initial operand: the leftmost leaf node
    \item operator: $\oplus_y: x \mapsto x + y$
    \item operator: $\ominus_y: x \mapsto x - y$
    \item operator: $\otimes_y: x \mapsto x \cdot e^y$
    \item operator: $\oslash_y: x \mapsto x \cdot e^{-y}$
\end{itemize}

For example, the threadlike arithmetic expression $(((((1 \times 2) \times 2) + 1) \times 3) + 6)$ can be curried as

\[\oplus_6(\otimes_{\ln 3}(\oplus_1(\otimes_{\ln 2}(\otimes_{\ln 2}(1)))))\]

Suppose we have a series of operators $a_1, a_2, \cdots a_{n-1}, a_n$, we introduce a \emph{path notation}.

\[x a_1 a_2 \cdots a_{n-1} a_n \coloneqq a\_n( a_{n-1}( \cdots a_2( a_1(x) ) \cdots ) )\]

So, the above example can be written as

\[1 \otimes_{\ln 2} \otimes_{\ln 2} \oplus_1 \otimes_{\ln 3} \oplus_6 \]

If a path begins with a number, we refer to it as a \emph{bounded path}.
If it does not, we refer to it as a \emph{free path}, similar to the concept of vectors from the origin versus vectors at arbitrary points.
a bounded path results in a number, while a free path results in a function.

Now we will verify that the operators within a path are associative.

\begin{lemma}\label{lemma:associative}
    The operators within a path are associative, i.e. we have \[a [b c] = [a b] c\]
\end{lemma}

\begin{proof}
We use normal typeface to express the path notation, and bold typeface to express the function notation.

For a free path, follow the definition, we have
\[a [b c] = [b c](\mathbf{a}) = \mathbf{c}(\mathbf{b}(\mathbf{a}))\]
\[[a b] c = \mathbf{c}([a b]) = \mathbf{c}(\mathbf{b}(\mathbf{a}))\]

hence, we have
\[a [b c] = [a b] c\]
is hold for a free path.

For a bounded path, we have
\[x a [b c] = [b c](\mathbf{a}(x)) = \mathbf{c}(\mathbf{b}(\mathbf{a}(x)))\]
\[x [a b] c = \mathbf{c}([a b](x)) = \mathbf{c}(\mathbf{b}(\mathbf{a}(x)))\]

hence, we have
\[a [b c] = [a b] c\]
is hold for a bounded path.

\end{proof}

\begin{definition}\label{definition:concatenate}
    The concatenation of paths $p_1 \cdot p_2$ is defined as the composite of functions:
    \[p_1 \cdot p_2 \coloneqq p_2 \circ p_1 \]
\end{definition}

When a sequence of paths is concatenated, and only the first path can be bounded.
If the first path is bounded, the concatenated result is a bounded path.
Otherwise, the concatenated result is a free path.

\subsection{Alternating threadlike expressions}\label{subsec:alternating}

Now we can define alternating threadlike expressions, which were mentioned in Section~\ref{sec:expression}, using the path notion.

\begin{equation}\label{eq:alternative}
    \alpha = a_1 b_1 a_2 b_2 \cdots a_l b_l, a_i = \otimes_{\lambda_i}, b_i = \oplus_{\mu_i}, \lambda_i, \mu_i \in \mathbb{R}
\end{equation}

where $\bigoplus$ and $\bigotimes$ denote addition and multiplication, respectively,
and the expression is a zigzag of alternating addition and multiplication operations.
$\alpha$ is a free path, and we can bind a number to it.

Since $0$ is the identity element for addition and $1$ is the identity element for multiplication,
it is straightforward to see that any arithmetic expression can be converted into an alternating threadlike expression
by introducing more $0$ and $1$ into the original expression.
So alternating threadlike expression is a kind of canonical form.

We can derive a formula for perturbations in alternating threadlike expressions.

Let us define the left-to-right accumulated sum of $\lambda_i$ as $\check{\lambda}_i$, such that:
\begin{equation}
\check{\lambda}_i = \sum_{j=1}^i \lambda_j, \check{\lambda}_0 = 0\label{eq:accsumlr}
\end{equation}

Then we also have right-to-left accumulated sum of $\lambda_i$
\begin{equation}
\hat{\lambda}_i = \check{\lambda}_l - \check{\lambda}_{l - i}, \hat{\lambda}_0 = 0\label{eq:accsumrl}
\end{equation}

Expanding equation~\eqref{eq:alternative} using the distributive law and the above notion at point $\mu_0$, we obtain:
\begin{align}
\alpha(\mu_0) & = e^{\lambda_l}(\cdots (e^{\lambda_2} (e^{\lambda_1} \mu_0 + \mu_1) + \mu_2) \cdots) + \mu_l \\
& = e^{\hat{\lambda}_l} \mu_0 + e^{\hat{\lambda}_{l - 1}} \mu_1  + e^{\hat{\lambda}_{l - 2}} \mu_2 + \cdots + e^{\hat{\lambda}_1} \mu_{l - 1} + e^{\hat{\lambda}_0} \mu_l
\end{align}

Next, at the starting point $\mu_0$, we introduce a perturbation $\tilde{\mu}_0 = e^{\eta_0} \mu_0 + \epsilon_0$,
where $\eta_0$ and $\epsilon_0$ are the disturbance terms added by the summation and multiplication operations, respectively. Then, we have:
\begin{align}
\alpha(\tilde{\mu}_0) & = e^{\hat{\lambda}_l} (\tilde{\mu}_0) + e^{\hat{\lambda}_{l - 1}} \mu_1  + e^{\hat{\lambda}_{l - 2}} \mu_2 + \cdots + e^{\hat{\lambda}_1} \mu_{l - 1} + e^{\hat{\lambda}_0} \mu_l \\
& = \alpha(\mu_0) + e^{\hat{\lambda}_l} (\tilde{\mu}_0 - \mu_0)
\end{align}

As a result, purely from an arithmetic perspective, without the need for limits, we can derive the following meaningful ratio:
\begin{equation}
\frac{\alpha(\tilde{\mu}_0) - \alpha(\mu_0)}{\tilde{\mu}_0 - \mu_0} = e^{\hat{\lambda}_l} = e^{\check{\lambda}_l}\label{eq:ratio}
\end{equation}

Now we extend this relationship from the starting point $\mu_0$ to the entire process, we define the recursive formula

\[
w_i = e^{\lambda_i} w_{i-1} + \mu_i, w_0 = 0
\]

and then we have

\begin{equation}
\frac{\tilde{w}_i - w_i}{\tilde{\mu}_0 - \mu_0} = e^{\check{\lambda}_i}, i \in \{1, ..., l\}\label{eq:perturbation1}
\end{equation}

So, we have

\[
\tilde{w}_i - w_i = e^{\check{\lambda}_i} (\tilde{\mu}_0 - \mu_0)
\]

and hence

\begin{equation}
\tilde{w}_i - w_i = e^{\lambda_i}(\tilde{w}_{i - 1} - w_{i - 1})\label{eq:perturbation2}
\end{equation}

That means the perturbation along the path is controlled by the multiplication terms of $e^{\lambda_i}$.

\subsection{Generated structure, commutator and arithmetic torsion}\label{subsec:generated-structure}

In order to study mesh grids like the one described in subsection~\ref{subsec:meshgrid},
we need to investigate the algebraic structure of the threadlike arithmetic expressions that are generated.

For real number $\mathbb{R}$ and elements $\mu, \lambda \in \mathbb{R}$, we consider all the arithmetical expressions
that are freely generated from
\begin{itemize}
    \item initial operand: $0$
    \item operator: $\oplus_\mu: x \mapsto x + \mu$
    \item operator: $\ominus_\mu: x \mapsto x - \mu$
    \item operator: $\otimes_\lambda: x \mapsto x \cdot e^\lambda$
    \item operator: $\oslash_\lambda: x \mapsto x \cdot e^{- \lambda}$
\end{itemize}

We denote these expressions as $E(\mu, \lambda)$, where $\mu$ is the additional generator and $e^\lambda$ is the multiplicative generator.
In cases where the context is clear, we may omit $\mu$ and $\lambda$ from the index.
Our goal is not to study only a single $E(\mu, \lambda)$, but rather to use a family of $E(\mu, \lambda)$ to approach a continuous space.

Since $\oplus_\mu$ and $\ominus_\mu$ are mutually inverse operations, it follows that $\otimes_\lambda$ and $\oslash_\lambda$ are also mutually inverse. This means that $E(\mu, \lambda)$ forms a group.
An observation is that the commutator of this group is not equal to identity generally,
especially the commutator of the generators.

\begin{equation}
x \oplus_\mu \otimes_\lambda \ominus_\mu \oslash_\lambda - x = \mu(1 - e^{-\lambda})\label{eq:commutator1}
\end{equation}
\begin{equation}
x \otimes_\lambda \oplus_\mu \oslash_\lambda \ominus_\mu - x = - \mu(1 - e^{-\lambda})\label{eq:commutator2}
\end{equation}

Formula \ref{eq:commutator1} obey the right-hand rule, and formula \ref{eq:commutator2} obey the left-hand rule.

Or equivlently\footnote{Please reference section \ref{subsec:descartes-coordinate}, the equivlence here is refered to the same order of the infinitesimal}, we define below difference $\tau$ obey the right-hand rule:

\begin{equation}
\tau = x \oplus_\mu \otimes_\lambda - x \otimes_\lambda \oplus_\mu = \mu(e^\lambda - 1)\label{eq:torsion}
\end{equation}

These differences are constant, indicating a type of torsion in the generated group.
And torsion $\tau$ is specifically referred to as the arithmetic torsion.

We will reveal that $\tau$ is related to the curvature of the surface in later sections.

\subsection{Levels of Equality, Singularity, and Symmetry Problems}
\label{subsec:problems-on-equality-singularity-symmetries}

From theoretical and computational perspectives, it is useful to consider different levels of equality within structures generated freely from arithmetic operations. These levels reveal distinct stages of abstraction, from the most basic sequential representation to the final numerical or geometric realization. We can distinguish at least the following key levels:

\begin{enumerate}
    \item \textbf{Literal Equality}: \\
    This is the finest level of equality, judged directly by the sequence of operations (or its string representation). Two threadlike expressions are literally equal only if their sequences are identical. While this level might be too strict to reflect computational outcomes, it provides the richest \emph{base textures} without any imposed algebraic constraints, forming the foundation upon which more complex structures and spaces can be woven.

    \item \textbf{Operational/Syntactic Equality}: \\
    This level is based on fundamental rules governing the composition of operations, chief among them being \textbf{associativity} (as proven in Lemma~\ref{lem:associativity}). Associativity ensures that the result of composing a sequence of operations is well-defined, independent of the grouping (parenthesization). This endows our arithmetic expressions with at least a foundational \textbf{monoid} or \textbf{groupoid} structure, enabling the consistent definition of "paths" and their composition, which is essential for geometrization. It is crucial to note that the sequential path operations central to this framework (e.g., compositions of addition $\oplus_\mu$ and multiplication $\otimes_\lambda$) generally do \textbf{not} satisfy the standard \textbf{distributive law} at this level. This is evidenced by the non-zero arithmetic torsion $\tau$, a key feature indicating the non-commutative and non-distributive nature of these combined operations.

    \item \textbf{Relational/Algebraic Equality}: \\
    Building upon the operational structure above, specific contexts often introduce \textbf{additional, concrete algebraic relations}. These relations might stem from:
    \begin{itemize}
        \item The existence of inverses for the operations, leading to a group structure.
        * Geometric or topological constraints, such as identities that must be satisfied by relators in a knot group.
        * Arithmetic identities satisfied by parameters involved in the operations, such as a minimal polynomial $P(\lambda)=0$.
    \end{itemize}
    This level of equality is defined by the \textbf{specific presentation (generators and relations)} of the resulting algebraic structure, shaping the particular algebraic object of interest (e.g., a specific group). Formal tools from combinatorial group theory, such as \textbf{HNN extensions} and \textbf{amalgamated free products}, provide a precise framework for describing these algebraic structures arising from geometric, topological, or arithmetic constraints. Results like Britton's Lemma ensure that these relationally-defined structures are typically non-trivial.

    \item \textbf{Semantic Equality}: \\
    This is the coarsest level of equality, determined by the final \textbf{numerical evaluation} of the expression or by the \textbf{value of the assignment function $a$} in a geometric model. Two expressions might be distinct at the Relational/Algebraic level (representing different group elements) but still yield the same semantic value. This level directly corresponds to the computational output or the state within the geometric realization.
\end{enumerate}

\textbf{Hierarchy and Challenges}:

We can view these levels of equality as forming a lattice or hierarchy based on refinement, ordered by implication from finer to coarser levels:
$$ E_{\text{Literal}} \implies E_{\text{Op/Syn}} \implies E_{\text{Rel/Alg}} \implies E_{\text{Semantic}} $$
where $E_X$ denotes "equality at level X", and $\implies$ means "implies". A finer equality necessarily satisfies any coarser equality.

It is important, however, to distinguish this lattice of \emph{equality relations} (ordered by refinement) from the relationship between the corresponding \emph{algebraic structures}. For instance, in the context of HNN extensions relevant to fibered knots (corresponding to the Relational/Algebraic level, e.g., the knot group $\pi_1(K)$), the structure typically contains the base group (corresponding to the finer Operational/Syntactic level, e.g., $\pi_1(F)$) as a (normal) \textbf{subgroup}. That is, the algebraic structure defined by the coarser ("higher" in the lattice) equality relation embeds the structure defined by the finer ("lower") relation. This is not a contradiction; it highlights that a coarser equivalence relation (identifying more pairs) can define a larger, more complex algebraic structure which contains the structure defined by the finer relation as a component.

Significant theoretical challenges arise from the "distance" between these different levels, particularly between the Relational/Algebraic structure and its Semantic/Geometric realization. For example:

\begin{itemize}
    \item \textbf{Role of Background Algebra}: While the AEG path operations themselves might be non-distributive, the discussion of Relational/Algebraic equality (e.g., involving $P(\lambda)=0$) relies on the parameters (like $\lambda$) residing within standard algebraic systems (fields, rings) where distributivity holds, allowing polynomials and minimal polynomials to be defined. These relations from standard algebra are then imposed as constraints on the non-distributive AEG operational structure.
    \item \textbf{Singularity Problem}: Certain literally or syntactically valid expressions might be semantically invalid (e.g., involving division by zero), especially when extending definitions from $\mathbb{Q}$ to $\mathbb{R}$ or $\mathbb{C}$. How do these semantic singularities manifest in the geometric model? Do they correspond to singularities in the space or specific behaviors of the assignment function $a$?
    \item \textbf{Symmetry Problem}: What is the relationship between the symmetries of the algebraic structure (e.g., automorphisms of the HNN group) and the symmetries of the final geometric space (e.g., isometries of the $\mathfrak{E}_1$ space)? How does the process of constructing the geometric space ("weaving the textures into a space") selectively realize or break the symmetries present at the algebraic level? Different levels of equality correspond to different "symmetry resources".
\end{itemize}

Exploring these problems is central to our research, aiming to develop a unified theoretical framework capable of accommodating these different structural levels and revealing the deep connections between arithmetic, algebra, geometry, and topology.

\emph{Foundational problem}: A careful reader may have noticed that the definition~\ref{def:arithmetic-expression} is based on rational numbers $\mathbb{Q}$.
Why can't we use real numbers $\mathbb{R}$ instead?
The answer is that syntactically valid expressions may not be semantically valid.
Dividing by zero can lead to invalid expressions, and the evaluation of the expression cannot be defined in this situation.
Therefore, in real numbers, an expression may be syntactically valid but semantically not valid,
and there is no algorithm that can decide whether an expression is semantically valid or not.
How can we bridge this gap and provide a continuous geometry space?
We will attempt to partially solve this problem in some special cases in section~\ref{sec:topology}.

\emph{Singular point problem}: We have a very strong intuition that semantically invalid expressions lead to singular points.
The way we discussed in complex analysis may be borrowed here: essential singularities and poles.

\emph{Symmetry and classification problem}: We conjecture that the equality lattice may not only play a role in the construction of a space, but also determine the symmetry of that space.
We can imagine that, at certain levels of the lattice, we weave syntactically generated substructures into points to form a space,
and the weaving process uses up some symmetrical resources, leaving the rest to form a symmetry on the space.
The structure within the total symmetry may provide us with a systematic way of constructing spaces, and allow us to classify spaces based on their symmetries.
