
From the symbolic trees of elementary arithmetic to the hyperbolic geometry of modular curves, the tension between
\emph{discrete algebra} and \emph{continuous geometry} has been a recurring theme throughout modern mathematics.
This paper proposes that the evaluation process of arithmetic expressions itself carries enough structure to be treated
as a genuine geometric object.

\subsection{Background and motivation}

An arithmetic expression is usually regarded as a finite syntactic object: a string or a tree built from numbers and the
operations $+,-,\times,\div$. Once evaluated, it collapses to a single real number, and the intermediate steps are
forgotten. In classical analysis and numerical computation, this intermediate structure is treated as an implementation
detail rather than a primary object of study.

In this work we take the opposite stance: \emph{the evaluation dynamics of arithmetic expressions}---the way values
propagate through a tree, and the way addition and multiplication interleave along a path---is elevated to the status of
a geometric phenomenon. Concretely, we show that certain families of expressions (threadlike expressions) can be
encoded as paths in a two–dimensional manifold, and that the failure of addition and multiplication to commute induces
a measurable geometric torsion on this manifold.

The guiding question is:

\begin{quote}
Can the evaluation dynamics of arithmetic expressions be realized as a geometric flow, in such a way that algebraic
features (such as non-commutativity and evaluation order) are reflected as geometric invariants (such as area, curvature
or contact torsion)?
\end{quote}

\subsection{A higher–level perspective: echoes of 1960s–70s mathematical physics}

Conceptually, this project sits in a lineage of ideas that became prominent in the 1960s and 1970s in mathematical
physics, where \emph{discrete} or \emph{combinatorial} data was systematically lifted into \emph{geometric} and
\emph{analytic} structures.

A few analogies are particularly relevant:

\begin{itemize}
  \item In geometric mechanics, Hamilton–Jacobi and eikonal equations reinterpret the evolution of a system as a
  flow of level sets of an action function on a manifold. Our flow equation for arithmetic assignment plays a similar
  role: it governs how the ``value'' $a$ propagates over a surface driven by additive and multiplicative generators.

  \item In gauge theory and index theory, non-commuting operators and group actions are encoded as curvature and
  characteristic classes on bundles. Here, the non-commutativity of ``add then multiply'' versus ``multiply then add''
  manifests as an \emph{arithmetic torsion}, which we relate to a geometric area element.

  \item In contact and symplectic geometry, which matured in the same period, a contact form ties together a base
  space and a distinguished ``vertical'' direction, and the dynamics of a system is realized as a Legendrian or
  Hamiltonian flow. Our arithmetic expression contact manifold plays an analogous role for evaluation dynamics:
  the base is a commutative parameter space, while the vertical coordinate records the assignment value.

  \item Finally, in lattice models and renormalization, discrete combinatorial structures are embedded into continuum
  geometric or field–theoretic frameworks. In a similar spirit, we embed discrete expression trees and paths into
  continuous manifolds (the arithmetic expression spaces), and study their scaling and torsion properties.
\end{itemize}

The present paper is not a work in mathematical physics; rather, it is consciously informed by this historical pattern:
\emph{start from a purely algebraic/combinatorial object, and seek a geometric enlargement in which its hidden
structure becomes visible as curvature, torsion, or contact geometry.}

\subsection{From expressions to flows}

We begin by formalizing arithmetic expressions via production rules over $\mathbb{Q}$ and by emphasizing a special
class of \emph{threadlike expressions}, in which all left children in the syntax tree are leaves.
These threadlike expressions admit a natural currying and path notation, and they can be viewed as analogues of
paths in homotopy theory. On this combinatorial side we identify a first invariant: the \emph{arithmetic torsion}, which
measures the discrepancy between the two basic compositions
\[
  x \xrightarrow{\;\oplus\mu\;} x+\mu \xrightarrow{\;\otimes\lambda\;} (x+\mu)e^{\lambda}
  \qquad\text{and}\qquad
  x \xrightarrow{\;\otimes\lambda\;} xe^{\lambda} \xrightarrow{\;\oplus\mu\;} xe^{\lambda}+\mu.
\]
The torsion is a non-zero constant depending on $\mu,\lambda$, indicating an intrinsic obstruction to distributivity
within this path calculus.

The central step is to show that these discrete propagations admit an infinitesimal counterpart.
We derive a flow equation
\[
  \frac{da}{ds} = \mu\cos\theta + a\lambda\sin\theta,
\]
which governs how the assignment value $a$ changes along a curve of arclength $s$ in a direction making angle
$\theta$ with two distinguished generating directions (addition and multiplication).
In coordinate–free form this becomes an eikonal equation
\[
  \|\nabla a\| = \sqrt{\mu^2 + \lambda^2 a^2},
\]
revealing the evaluation of expressions as a geometric wavefront propagation problem.

\subsection{Arithmetic expression spaces and torsion as area}

The flow equation leads naturally to the notion of an \emph{Arithmetic Expression Space (AES)}: a Riemannian
surface $M$ equipped with a scalar field $a:M\to\mathbb{R}$ satisfying the flow equation for fixed generators
$\mu,\lambda$.
We construct and analyse a concrete example, the \emph{first kind space} $\mathfrak{E}_1$, realized on the upper
half–plane endowed with a hyperbolic metric, where the simple formula
\[
  a(x,y) = -\frac{x}{y}
\]
simultaneously satisfies the flow equation and the Laplace eigenvalue equation $\Delta a = 2a$.
Within $\mathfrak{E}_1$ we identify addition–multiplication grids whose combinatorics reflect Baumslag–Solitar
groups of the form $\mathrm{BS}(2,1)$ and $\mathrm{BS}(1,2)$, and we show that arithmetic torsion along these grids
is proportional to hyperbolic area enclosed by evaluation paths.

To capture the global effect of non-commutativity along an arbitrary path, we introduce the
\emph{Accumulative Commutative Space (ACS)}.
This is a commutative parameter plane where each expression path $\gamma$ is mapped to the pair
$(A_\gamma,M_\gamma)$ of its accumulated additive and logarithmic multiplicative charges.
In this plane, a path and its reversed sequence share the same endpoints and enclose a region $\Sigma_\gamma$.
We prove a Stokes–type \emph{triple identity} asserting that the global arithmetic torsion
$\tau(\gamma)$ admits three equivalent descriptions: as an algebraic evaluation difference, as an $e^M$–weighted area
integral over $\Sigma_\gamma$, and as a boundary integral along $\partial\Sigma_\gamma$.

\subsection{Contact geometry and differential calculus for expressions}

While the ACS provides a flat commutative backdrop for measuring global torsion, it does not yet encode the local,
non-commutative dynamics.
To bridge this gap, we lift the ACS to a three-dimensional manifold with coordinates $(u,v,a)$ and introduce the
1–form
\[
  \alpha \;=\; da - (\mu\,du + \lambda a\,dv).
\]
We show that $\alpha$ is a contact form and that its kernel defines a horizontal distribution spanned by two vector fields
$D_u,D_v$.
Legendrian curves with respect to this contact structure carry precisely the same flow equation as AES–paths, so the
contact manifold can be regarded as a natural differential–geometric completion of the arithmetic expression space.

On this contact manifold we develop an \emph{expression differential} $\delta$, a horizontal counterpart of the de Rham
differential $d$ adapted to the arithmetic flow.
The operator $\delta$ satisfies a twisted Leibniz rule, its second iterate $\delta^2$ measures a curvature two–form
proportional to $\mu\lambda\,du\wedge dv$, and it is compatible with $d$ in the sense of an Ehresmann connection.

Finally, we introduce \emph{arithmetic holomorphic functions}, i.e.\ pairs $(f,g)$ satisfying a system of
Cauchy–Riemann–type equations written in terms of $D_u,D_v$.
This provides a first glimpse of an ``arithmetic function theory'' living on the contact structure, parallel in spirit to
classical complex analysis on $(u,v)$–planes but sensitive to the arithmetic torsion encoded in $\mu,\lambda$.

\subsection{Main contributions of this paper}

The contributions of this first paper can be summarized as follows:

\begin{itemize}
  \item We formalize arithmetic expressions via threadlike paths and identify a canonical notion of
  \emph{arithmetic torsion} arising from the non-commutativity of addition and multiplication.

  \item We derive a flow equation for assignment values, exhibit its Eikonal/Hamilton–Jacobi form, and use it to
  define the notion of an \emph{Arithmetic Expression Space (AES)}. We construct and analyse in detail the
  first kind space $\mathfrak{E}_1$ on the hyperbolic upper half–plane.

  \item We introduce the \emph{Accumulative Commutative Space (ACS)} and prove a triple identity relating global
  arithmetic torsion to a weighted area integral and a boundary integral in this commutative parameter plane.

  \item We construct an \emph{arithmetic expression contact structure} on a three–dimensional manifold, identify the
  associated Lie algebra as a solvable central extension of $\mathfrak{aff}(1)$, and show that Legendrian flows
  recover the arithmetic flow equation.

  \item We define a horizontal differential calculus adapted to this contact structure and formulate an initial notion
  of \emph{arithmetic holomorphic functions}, opening a path toward a geometric function theory for arithmetic
  expressions.
\end{itemize}

Throughout, the emphasis is not on solving classical problems, but on setting up a coherent geometric framework in
which new invariants of arithmetic expressions can be formulated and studied.

\subsection{Structure of the paper}

Section~\ref{sec:concepts} develops the basic combinatorial and algebraic framework: production rules for arithmetic
expressions, path notation, alternating threadlike expressions, arithmetic torsion, and a hierarchy of equality levels.
Section~\ref{sec:flowequation} derives the flow equation, presents its Lie–algebraic and coordinate–free forms, and
establishes consistency with discrete generating processes.
Section~\ref{sec:firstkind} constructs the first kind arithmetic expression space $\mathfrak{E}_1$ on the hyperbolic
upper half–plane, analyses its grid structures, and relates arithmetic torsion to hyperbolic area.

Section~\ref{sec:acspace} introduces the accumulative commutative space and proves the triple identity for global
arithmetic torsion, linking algebraic evaluation, area integrals, and boundary integrals.
Section~\ref{sec:ecstructure} lifts the ACS to a three–dimensional contact manifold, identifies the associated Lie
algebra, and interprets arithmetic flows as Legendrian curves.
Section~\ref{ch:differential_calculus} develops a differential calculus on the contact structure, including the expression
differential and its curvature.
Section~\ref{ch:holomorphic} introduces arithmetic holomorphic functions and outlines several open directions.

The overall aim is to show that arithmetic expressions, when viewed through this geometric lens, exhibit a rich internal
structure that parallels, in a discrete–arithmetic setting, many of the themes that emerged in geometric mechanics and
mathematical physics in the latter half of the twentieth century.
