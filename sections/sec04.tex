
In this section, we will present two analytic examples due to Le that are equivalent and belong to the class of spaces
called the first kind arithmetic expression space $\mathfrak{K}_1$.

\subsection{Example space I}\label{subsec:exmp1}

Consider the upper half plane ${\mathcal{H}: (x, y) \ | \ y > 0}$ equipped with an inner product and metrics defined as follows:

$$
\mathbf{a} \cdot \mathbf{b} = \begin{bmatrix} a_x & a_y \end{bmatrix} \begin{bmatrix} \frac{1}{y^2} & 0 \\ 0 & \frac{1}{y^2} \end{bmatrix} \begin{bmatrix} b_x \ b_y \end{bmatrix}
$$

and

$$
ds^2 = \frac{1}{y^2} (dx^2 + dy^2)
$$

We also consider an assignment function $A$ defined on $\mathcal{H}$ as follows\footnote{This analytic example is provided by Le Zhang, and the geometry interpretation is given by Mingli Yuan}:

\begin{equation}
A = - \frac{x}{y}
\end{equation}

\begin{theorem}\footnote{The proof is originally by Le Zhang, and modified by Mingli Yuan}
For the assignment $A$ defined by above formula, $A$ satisfy the flow equation\eqref{eq:flow}
\end{theorem}

\begin{proof}
$$
da = d(-\frac{x}{y}) = \frac{xdy - ydx}{y^2} = -\frac{dx + ady}{y}
$$

and

$$
ds = \frac{\sqrt{dx^2 + dy^2}}{y}
$$

then

$$
\frac{da}{ds} = - \frac{dx + ady}{y} \frac{y}{\sqrt{dx^2 + dy^2}} = - \frac{dx + ady}{\sqrt{dx^2 + dy^2}}
$$

Considering that the local coordinate is given by $(-1, 0)$ and $(0, -1)$ under the right-hand rule, we have:

$$
\cos \theta = \frac{\begin{bmatrix} dx & dy \end{bmatrix} \begin{bmatrix} \frac{1}{y^2} & 0 \\ 0 & \frac{1}{y^2} \end{bmatrix} \begin{bmatrix} -1 \\ 0 \end{bmatrix}}{\sqrt{\begin{bmatrix} dx & dy \end{bmatrix} \begin{bmatrix} \frac{1}{y^2} & 0 \\ 0 & \frac{1}{y^2} \end{bmatrix} \begin{bmatrix} dx \\ dy \end{bmatrix}}\sqrt{\begin{bmatrix} -1 & 0 \end{bmatrix} \begin{bmatrix} \frac{1}{y^2} & 0 \\ 0 & \frac{1}{y^2} \end{bmatrix} \begin{bmatrix} -1 \\ 0 \end{bmatrix}}}
$$

hence

$$
\cos \theta = \frac{-dx}{\sqrt{dx^2 + dy^2}}
$$

and similarly

$$
\sin \theta = \frac{-dy}{\sqrt{dx^2 + dy^2}}
$$

then

$$
\frac{da}{ds} = \cos \theta + a \sin \theta
$$

\end{proof}

We can verify $A$ is an eigenfunction of the Laplacian

$$
\Delta A = - y^2 (\frac{\partial^2}{\partial x^2} A + \frac{\partial^2}{\partial y^2} A) = y^2 (\frac{1}{\partial y} (\frac{1}{\partial y} \frac{x}{y})) = 2 A
$$

\subsection{A horocycle-based coordinate system}\label{subsec:horocyclebased}

First, we introduce the horocycle-based coordinate system for hyperbolic surfaces.
It is a global coordinate system given by two orthogonal sets of circles: one set consists of horocycles that share the same ideal point,
while the other consists of geodesics that are perpendicular to the first set.

On the Poincaré disc $\mathcal{P}$, the blue horocycles are tangible at the ideal point $\Omega$, forming the first set of circles.
The green lines are geodesics that form the second set. The coordinates of point $P$ are given by the lengths of $OQ$ and $QP$,
where point $O$ is the origin and the length is measured using the metric of the hyperbolic surface.

The coordinates of $P$ are $(u,v)$, where the sign of the length is determined by the following rules:
\begin{itemize}
\item u: if $P$ is on the same side of $\Omega$, then $u$ is positive; otherwise, $u$ is negative
\item v: the sign of the length should satisfy the right-hand rule
\end{itemize}

\begin{figure}[ht]
\centering
\resizebox{0.5\textwidth}{!}{\includegraphics{images/11-horocyclebased}}
\caption{A horocycle-based coordinate system}\label{fig:horocyclecoord}
\end{figure}

We can equip it with an inner product

\[
\mathbf{a} \cdot \mathbf{b} = \begin{bmatrix} a_u & a_v \end{bmatrix} \begin{bmatrix} e^{-2v} & 0 \\ 0 & 1 \end{bmatrix} \begin{bmatrix} b_u \\ b_v \end{bmatrix}
\]

and the metrics

$$
ds^2 = e^{-2v} du^2 + dv^2
$$

Laplacian is given by\cite{Costa2001ADO}

$$
\Delta = e^{2v} \frac{\partial^2}{{\partial u}^2} + \frac{\partial^2}{{\partial v}^2} - \frac{\partial}{\partial v}
$$

\subsection{Example space II}\label{subsec:exmp2}

Giving the Poincaré disc $\mathcal{P}$ equipped with the above horocycle-based coordinate system,
we consider an assignment function $A$ defined on $\mathcal{P}$ as follows\footnote{This analytic example is provided by Le Zhang, and the geometry interpretation is given by Mingli Yuan}:

\begin{equation}
A = u e^{-v}
\end{equation}

\begin{theorem}
For the assignment $A$ defined by above formula, $A$ satisfy the flow equation\eqref{eq:flow}
\end{theorem}

\begin{figure}[ht]
\centering
\resizebox{0.8\textwidth}{!}{\includegraphics{images/12-proofbymapping}}
\caption{Mapping between two examples}\label{fig:mapping}
\end{figure}

\begin{proof}

If we introduce complex in the upper half plane model in last section \ref{subsec:exmp1}

$$
z = x + y i
$$

and stetup a Möbius transform between the upper half plane and current horocycle-based coordinate system:

$$
z \mapsto \frac{z-i}{z+i}
$$

This transform maps each horizontal lines in $\mathcal{H}$ into the horocycles sharing the same ideal point $\Omega = 1$ in $\mathcal{P}$,
also it maps each vertical geodesics in $\mathcal{H}$ into geodesics in $\mathcal{P}$ which are perpendicular to the above horocycles.

And rewrite the Möbius transform in the target coordinate, we get:

$$
\begin{cases}
x = u\\
y = e^v \\
\end{cases}
$$

This lead to

$$
A = -\frac{x}{y} = u e^{-v}
$$

And because of theorem \ref{thm:isometry} and Möbius transform is conformal, we can conclude that $A = u e^{-v}$ obey flow equation.

\end{proof}

We can verify $A$ is also an eigenfunction of the Laplacian
$$
\Delta A = e^{2v} \frac{\partial^2(u e^{-v})}{{\partial u}^2} + \frac{\partial^2(u e^{-v})}{{\partial v}^2} - \frac{\partial(u e^{-v})}{\partial v} = 2A
$$

From the above proof, we can see that the two assignment function $A$ arose from the same geometry setting, they are equivalent to each other.

\subsection{Generator independence}

Considering the upper half plane $\mathcal{B}$:
$$
\{\mathcal{B}: (x, y) | y > 0 \}
$$

equipped with an inner product and metrics as follows:

$$
\mathbf{a} \cdot \mathbf{b} = \begin{bmatrix} a_x & a_y \end{bmatrix} \begin{bmatrix} \frac{1}{\mu^2 y^2} & 0 \\ 0 & \frac{1}{\lambda^2 y^2} \end{bmatrix} \begin{bmatrix} b_x \\ b_y \end{bmatrix}
$$

$$
ds^2 = \frac{1}{y^2}(\frac{dx^2}{\mu^2} + \frac{dy^2}{\lambda^2})
$$

Whatever the choice of $\mu$ and $\lambda$, the assignment is given by

\begin{equation}
A = - \frac{x}{y}
\end{equation}

We can verify $A$ satisfying the flow equation\eqref{eq:flow}, and also it is generator independent.

\begin{theorem}
The above $A$ satisfying the flow equation\eqref{eq:flow}
\end{theorem}

\begin{proof}
$$
da = d(-\frac{x}{y}) = \frac{xdy - ydx}{y^2} = -\frac{dx + a dy}{y}
$$

Notice that

$$
ds = \frac{1}{y}\sqrt{\frac{dx^2}{\mu^2} + \frac{dy^2}{\lambda^2}}
$$

then

$$
\frac{da}{ds} = - \frac{dx + a dy}{y} \frac{y}{\sqrt{\frac{dx^2}{\mu^2} + \frac{dy^2}{\lambda^2}}} = \frac{dx + a dy}{\sqrt{\frac{dx^2}{\mu^2} + \frac{dy^2}{\lambda^2}}}
$$

Consider the local coordinate system given by $(-1, 0)$ and $(0, -1)$ according to the right-hand rule, we have

$$
\cos \theta = \frac{\begin{bmatrix} dx & dy \end{bmatrix} \begin{bmatrix} \frac{1}{\mu^2 y^2} & 0 \\ 0 & \frac{1}{\lambda^2 y^2} \end{bmatrix} \begin{bmatrix} -1 \\ 0 \end{bmatrix}}{\sqrt{\begin{bmatrix} dx & dy \end{bmatrix} \begin{bmatrix} \frac{1}{\mu^2 y^2} & 0 \\ 0 & \frac{1}{\lambda^2 y^2} \end{bmatrix} \begin{bmatrix} dx \\ dy \end{bmatrix}}\sqrt{\begin{bmatrix} -1 & 0 \end{bmatrix} \begin{bmatrix} \frac{1}{\mu^2 y^2} & 0 \\ 0 & \frac{1}{\lambda^2 y^2} \end{bmatrix} \begin{bmatrix} -1 \\ 0 \end{bmatrix}}}
$$

Therefore we have

$$
\cos \theta = \frac{-\frac{dx}{\mu}}{\sqrt{\frac{dx^2}{\mu^2} + \frac{dy^2}{\lambda^2}}}
$$

Similarly

$$
\sin \theta = \frac{-\frac{dy}{\lambda}}{\sqrt{\frac{dx^2}{\mu^2} + \frac{dy^2}{\lambda^2}}}
$$

Hence we have

$$
\frac{da}{ds} = \mu \cos \theta + a \lambda \sin \theta
$$

\end{proof}

\subsection{Area formula in the global level}

In Section \ref{subsec:descartes-coordinate}, we introduced an area formula applicable at a local level, as detailed in Equation \eqref{eq:area_formula2}. This section aims to extend that exploration, testing the formula's applicability at a global level.

\begin{figure}[ht]
    \centering
    \resizebox{0.8\textwidth}{!}{\includegraphics{images/17-area-formula}}
    \caption{Illustration of the Area Formula}\label{fig:area-formula}
\end{figure}

Referencing Figure \ref{fig:area-formula} above, we recall the arithmetic space defined in Section \ref{subsec:meshgrid}. Our goal is to examine the arithmetic torsion across various steps:

For a single step, the scenario unfolds as follows:
\begin{equation}
    x \times 2 + 1 - (x + 1) \times 2 = -1
\end{equation}

Extending this to two steps, we encounter a different situation:
\begin{equation}
    x \times 4 + 2 - (x + 2) \times 4 = -6
\end{equation}

And for three steps, the pattern continues:
\begin{equation}
    x \times 8 + 3 - (x + 3) \times 8 = -21
\end{equation}

The question arises: What is the origin of the numbers $-1$, $-6$, and $-21$? By conceptualizing the area of the quadrilateral $ABCD$ as a basic unit cell,
it becomes apparent that the number $1$ represents a single unit cell of $ABCD$.
Similarly, the number $6$ corresponds to the area of $AEFG$, comprising six unit cells;
and the number $21$ aligns with the area of $AHIJ$, encompassing twenty-one unit cells.
This understanding allows us to appreciate how these values emerge naturally from the structure of the arithmetic space,
especially we connect area with arithmetic torsion which is defined on edges of that enclosing area.

\subsection{Related problems}

\emph{Eigenfunction problem}: TODO.

\emph{Eigenfunction and classification problem}: TODO.

\newpage
